\section{Devtool}

\begin{frame}
  \frametitle{Overview}
  \begin{itemize}
    \item \code{Devtool} is a set of utilities to ease the integration
    and the development of OpenEmbedded recipes.
    \item It can be used to:
      \begin{itemize}
        \item Generate a recipe for a given upstream application.
        \item Modify an existing recipe and its associated sources.
        \item Upgrade an existing recipe to use a newer upstream
          application.
      \end{itemize}
    \item \code{Devtool} adds a new layer, automatically managed, in
      \code{$BUILDDIR/workspace/}.
    \item It then adds or appends recipes to this layer so that the
      recipes point to a local path for their sources. In
      \code{$BUILDDIR/workspace/sources/}.
      \begin{itemize}
        \item Local sources are managed by \code{git}.
        \item All modifications made locally should be committed.
      \end{itemize}
  \end{itemize}
\end{frame}

\begin{frame}
  \frametitle{\code{devtool} usage 1/3}
  There are three ways of creating a new \code{devtool} project:
  \begin{itemize}
    \item To create a new recipe:
      \code{devtool add <recipe> <fetchuri>}
      \begin{itemize}
        \item Where \code{recipe} is the recipe's name.
        \item \code{fetchuri} can be a local path or a remote {\em
          uri}.
      \end{itemize}
    \item To modify the source for an existing recipe: \code{devtool modify <recipe>}
    \item To upgrade a given recipe:
      \code{devtool upgrade -V <version> <recipe>}
      \begin{itemize}
        \item Where \code{version} is the new version of the upstream
          application.
      \end{itemize}
  \end{itemize}
\end{frame}

\begin{frame}
  \frametitle{\code{devtool} usage 2/3}
  Once a \code{devtool} project is started, commands can be issued:
  \begin{itemize}
    \item \code{devtool edit-recipe <recipe>}: edit \code{recipe} in a text
      editor (as defined by the \code{EDITOR} environment variable).
    \item \code{devtool build <recipe>}: build the given
      \code{recipe}.
    \item \code{devtool build-image <image>}: build \code{image} with
      the additional \code{devtool} recipes' packages.
  \end{itemize}
\end{frame}

\begin{frame}
  \frametitle{\code{devtool} usage 3/3}
  \begin{itemize}
    \item \code{devtool deploy-target <recipe> <target>}: upload the
      \code{recipe}'s packages on \code{target}, which is a live
      running target with an SSH server running (\code{user@address}).
    \item \code{devtool update-recipe <recipe>}: generate patches from
      git commits made locally.
    \item \code{devtool reset <recipe>}: remove \code{recipe} from the
      control of \code{devtool}. Standard layers and remote sources
      are used again as usual.
  \end{itemize}
\end{frame}
