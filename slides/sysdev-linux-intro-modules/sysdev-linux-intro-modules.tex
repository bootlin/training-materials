\subsection{Using kernel modules}

\begin{frame}
  \frametitle{Advantages of modules}
  \begin{columns}
    \column{0.7\textwidth}
      \begin{itemize}
      \item Modules make it easy to develop drivers without rebooting:
        load, test, unload, rebuild, load...
      \item Useful to keep the kernel image size to the minimum (essential
        in GNU/Linux distributions for PCs).
      \item Also useful to reduce boot time: you don't spend time
        initializing devices and kernel features that you only need later.
      \item Caution: once loaded, have full control and privileges in the
        system. No particular protection. That's why only the \code{root} user
        can load and unload modules.
      \item To increase security, possibility to allow only signed modules,
        or to disable module support entirely.
      \end{itemize}
    \column{0.3\textwidth}
      \includegraphics[width=\textwidth]{slides/sysdev-linux-intro-modules/modules-to-access-rootfs.pdf}
  \end{columns}
\end{frame}

\begin{frame}
  \frametitle{Module dependencies}
  \begin{itemize}
  \item Some kernel modules can depend on other modules,
    which need to be loaded first.
  \item Example: the \code{ubifs} module depends on the
    \code{ubi} and \code{mtd} modules.
  \item Dependencies are described
    both in \code{/lib/modules/<kernel-version>/modules.dep}
    and in \code{/lib/modules/<kernel-version>/modules.dep.bin} (binary hashed format)\\
    These files are generated when you run \code{make modules_install}.
  \end{itemize}
\end{frame}

\begin{frame}
  \frametitle{Kernel log}

  When a new module is loaded, related information is available in the
  kernel log.
  \begin{itemize}
  \item The kernel keeps its messages in a circular buffer (so that it
    doesn't consume more memory with many messages)
  \item Kernel log messages are available through the \code{dmesg}
    command ({\bf d}iagnostic {\bf mes}sa{\bf g}e)
  \item Kernel log messages are also displayed in the system console
    (console messages can be filtered by level using the
    \code{loglevel} kernel command line parameter,
     or completely disabled with the
    \code{quiet} parameter). Example:\\
    \code{console=ttyS0 root=/dev/mmcblk0p2 loglevel=5} 
  \item Note that you can write to the kernel log from user space too.
    That's useful when your device's serial console is being monitored
    for critical messages:
    \code{echo "<n>Debug info" > /dev/kmsg}
  \end{itemize}
\end{frame}

\begin{frame}
  \frametitle{Module utilities (1)}
  \code{<module_name>}: name of the module file without the trailing \code{.ko}\\
  \begin{itemize}
  \item \code{modinfo <module_name>} (for modules in \code{/lib/modules})\\
    \code{modinfo <module_path>.ko}\\
    Gets information about a module without loading it: parameters, license,
    description and dependencies.\\
  \item \code{sudo insmod <module_path>.ko}\\
    Tries to load the given module. The full path to the module object
    file must be given.
  \end{itemize}
\end{frame}

\begin{frame}[fragile]
  \frametitle{Understanding module loading issues}
  \begin{itemize}
  \item When loading a module fails, \code{insmod} often doesn't give
    you enough details!
  \item Details are often available in the kernel log.
  \item Example:\\
\scriptsize
\begin{verbatim}
$ sudo insmod ./intr_monitor.ko
insmod: error inserting './intr_monitor.ko': -1 Device or resource busy
$ dmesg
[17549774.552000] Failed to register handler for irq channel 2
\end{verbatim}
  \end{itemize}
\end{frame}

\begin{frame}
  \frametitle{Module utilities (2)}
  \begin{itemize}
  \item \code{sudo modprobe <module_name>}\\
    Most common usage of \code{modprobe}: tries to load all the
    modules the given module depends on, and then this module. Lots of
    other options are available. \code{modprobe} automatically looks in
    \code{/lib/modules/<version>/} for the object file corresponding
    to the given module name.
  \item \code{lsmod}\\
    Displays the list of loaded modules\\
    Compare its output with the contents of \code{/proc/modules}!
  \end{itemize}
\end{frame}

\begin{frame}
  \frametitle{Module utilities (3)}
  \begin{itemize}
  \item \code{sudo rmmod <module_name>}\\
    Tries to remove the given module.\\
    Will only be allowed if the module is no longer in use (for
    example, no more processes opening a device file)
  \item \code{sudo modprobe -r <module_name>}\\
    Tries to remove the given module and all dependent modules (which
    are no longer needed after removing the module)
  \end{itemize}
\end{frame}

\begin{frame}
  \frametitle{Passing parameters to modules}
  \begin{itemize}
  \item Find available parameters:\\
    \code{modinfo usb-storage}
  \item Through \code{insmod}:\\
    \code{sudo insmod ./usb-storage.ko delay_use=0}
  \item Through \code{modprobe}:\\
    Set parameters in \code{/etc/modprobe.conf} or in any file in \code{/etc/modprobe.d/}:\\
    \code{options usb-storage delay_use=0}
  \item Through the kernel command line, when the driver is built statically into the kernel:\\
    \code{usb-storage.delay_use=0}
    \begin{itemize}
    \item \code{usb-storage} is the {\em driver name}
    \item \code{delay_use} is the {\em driver parameter name}. It
      specifies a delay before accessing a USB storage device (useful for
      rotating devices).
    \item \code{0} is the {\em driver parameter value}
    \end{itemize}
  \end{itemize}
\end{frame}

\begin{frame}
  \frametitle{Check module parameter values}
  How to find/edit the current values for the parameters of a loaded module?
  \begin{itemize}
  \item Check \code{/sys/module/<name>/parameters}.
  \item There is one file per parameter, containing the parameter value.
  \item Also possible to change parameter values if these files have
        write permissions (depends on the module code).
  \item Example:\\
	\code{echo 0 > /sys/module/usb_storage/parameters/delay_use}
  \end{itemize}
\end{frame}

\begin{frame}
  \frametitle{Useful reading}
  \begin{columns}
    \column{0.7\textwidth}
    Linux Kernel in a Nutshell, Dec. 2006
    \begin{itemize}
    \item By Greg Kroah-Hartman, O'Reilly\\
      \url{http://www.kroah.com/lkn/}
    \item A good reference book and guide on configuring, compiling
      and managing the Linux kernel sources.
    \item Freely available on-line!\\
      Great companion to the printed book for easy electronic searches!\\
      Available as single PDF file on
      \url{https://bootlin.com/community/kernel/lkn/}
    \item Getting old but still containing useful content.
    \end{itemize}
    \column{0.3\textwidth}
    \includegraphics[width=\textwidth]{slides/sysdev-linux-intro-modules/linux-kernel-in-a-nutshell.jpg}
  \end{columns}
\end{frame}
