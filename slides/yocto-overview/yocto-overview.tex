\section{Yocto Project and Poky reference system overview}

\subsection{The Yocto Project overview}

\begin{frame}
  \frametitle{About}
  \begin{itemize}
  \item The Yocto Project is an open source collaboration project
        that allows to build custom embedded Linux-based systems.
  \item Established by the Linux Foundation in 2010 and still managed by
    one of its fellows: Richard Purdie.
  \end{itemize}
\end{frame}

\begin{frame}{Yocto: principle}
  \begin{center}
    \includegraphics[width=\textwidth]{slides/yocto-overview/yocto-principle.pdf}
  \end{center}
  \begin{itemize}
  \item Yocto always builds binary packages (a ``distribution'')
  \item The final root filesystem is generated from the package feed
  \item The \href{https://docs.yoctoproject.org/_images/YP-flow-diagram.png}
                 {big picture} is way more complex
  \end{itemize}
\end{frame}

\begin{frame}{Lexicon: \code{bitbake}}
  In Yocto / OpenEmbedded, the {\em build engine} is implemented by the
  \code{bitbake} program
  \begin{itemize}
    \item \code{bitbake} is a task scheduler, like \code{make}
    \item \code{bitbake} parses text files
      %called {\em metadata}
      to know what it has to build and how
    \item It is written in Python (need Python 3 on the development host)
  \end{itemize}
\end{frame}

\begin{frame}{Lexicon: recipes}
  \begin{columns}
    \column{0.6\textwidth}
    \begin{itemize}
      \item The main kind of text file parsed by \code{bitbake} is {\em recipes},
        each describing a specific software component
      \item Each {\em Recipe} describes how to fetch and build a software
        component: e.g. a program, a library or an image
      \item They have a specific syntax
      \item \code{bitbake} can be asked to build any recipe, building all its
        dependencies automatically beforehand
    \end{itemize}
    \column{0.4\textwidth}
    \begin{center}
      \includegraphics[width=0.9\textwidth]{slides/yocto-overview/recipe-dependencies.pdf}
    \end{center}
  \end{columns}
\end{frame}

\begin{frame}{Lexicon: tasks}
  \begin{columns}
    \column{0.6\textwidth}
    \begin{itemize}
      \item The build process implemented by a recipe is split in several
        {\em tasks}
      \item Each task performs a specific step in the build
      \item Examples: fetch, configure, compile, package
      \item Tasks can depend on other tasks (including on tasks of other
        recipes)
    \end{itemize}
    \column{0.4\textwidth}
    \begin{center}
      \includegraphics[width=1\textwidth]{slides/yocto-overview/recipe-dependencies-tasks.pdf}
    \end{center}
  \end{columns}
\end{frame}

\begin{frame}{Lexicon: metadata and layers}
  \begin{itemize}
    \item The input to \code{bitbake} is collectively called {\em metadata}
    \item Metadata includes {\em configuration files}, {\em recipes}, {\em
      classes} and {\em include files}
    \item Metadata is organized in {\em layers}, which can be composed to
      get various components
      \begin{itemize}
        \item A layer is a set of recipes, configurations files and classes
          matching a common purpose
      \begin{itemize}
        \item For Texas Instruments board support, the {\em meta-ti-bsp}
          layer is used
      \end{itemize}
      \item Multiple layers are used for a project, depending on the needs
      \end{itemize}
    \item {\em openembedded-core} is the core layer
      \begin{itemize}
        \item All other layers are built on top of openembedded-core
        \item It supports the ARM, MIPS (32 and 64 bits), PowerPC, RISC-V
          and x86 (32 and 64 bits) architectures
        \item It supports QEMU emulated machines for these architectures
      \end{itemize}
  \end{itemize}
\end{frame}

\begin{frame}{Lexicon: Poky}
  \begin{itemize}
  \item The word {\em Poky} has several meanings
  \item Poky is a git repository that is assembled from other git
    repositories: bitbake, openembedded-core, yocto-docs and meta-yocto
  \item poky is the {\em reference distro} provided by the Yocto Project
  \item meta-poky is the layer providing the poky reference distribution
  \end{itemize}
\end{frame}

\begin{frame}{The Yocto Project lexicon}
  \begin{center}
    \includegraphics[height=0.8\textheight]{slides/yocto-overview/yocto-project-overview.pdf}
  \end{center}
\end{frame}

\begin{frame}
  \frametitle{The Yocto Project lexicon}
  \begin{itemize}
    \item The Yocto Project is \textbf{not used as} a finite set of
          layers and tools.
    \item Instead, it provides a \textbf{common base} of tools and
          layers on top of which custom and specific layers are added,
          depending on your target.
  \end{itemize}
\end{frame}

\begin{frame}
  \frametitle{Example of a Yocto Project based BSP}
  \begin{itemize}
    \item To build images for a BeagleBone Black, we need:
    \begin{itemize}
      \item The Poky reference system, containing all common recipes
            and tools.
      \item The {\em meta-ti-bsp} layer, a set of Texas Instruments
            specific recipes.
    \end{itemize}
    \item All modifications are made in your own layer.  Editing Poky or
      any other third-party layer is a \textbf{no-go}!
    \item We will set up this environment in the lab.
  \end{itemize}
\end{frame}

\subsection{The Poky reference system overview}

\begin{frame}
  \frametitle{Getting the Poky reference system}
  \begin{itemize}
    \item All official projects part of the Yocto Project are
          available at \url{https://git.yoctoproject.org/}
    \item To download the Poky reference system: \\
          {\small
          \code{git clone -b scarthgap https://git.yoctoproject.org/git/poky}
          }
    \item A new version is released every 6 months, and maintained for 7 months
    \item \textbf{LTS} versions are maintained for 4 years, and announced before their release.
    \item Each release has a codename such as \code{kirkstone} or \code{honister},
	  corresponding to a release number.
		  \begin{itemize}
			  \item A summary can be found at \url{https://wiki.yoctoproject.org/wiki/Releases}
		  \end{itemize}
  \end{itemize}
\end{frame}

\begin{frame}{Poky}
  \begin{center}
    \includegraphics[height=0.8\textheight]{slides/yocto-overview/yocto-overview-poky.pdf}
  \end{center}
\end{frame}

\begin{frame}
  \frametitle{Poky source tree 1/2}
  \begin{description}[style=nextline]
  \item[bitbake/] Holds all scripts used by the \code{bitbake} command.
    Usually matches the stable release of the BitBake project.
  \item[documentation/] All documentation sources for the Yocto
    Project documentation. Can be used to generate nice PDFs.
  \item[meta/] Contains the OpenEmbedded-Core metadata.
  \item[meta-skeleton/] Contains template recipes for BSP and
    kernel development.
  \end{description}
\end{frame}

\begin{frame}
  \frametitle{Poky source tree 2/2}
  \begin{description}[style=nextline]
  \item[meta-poky/] Holds the configuration for the Poky
    reference distribution.
  \item[meta-yocto-bsp/] Configuration for the Yocto Project
    reference hardware board support package.
  \item[LICENSE] The license under which Poky is distributed (a mix of
    GPLv2 and MIT).
  \item[oe-init-build-env] Script to set up the OpenEmbedded build
    environment. It will create the build directory.
  \item[scripts/] Contains scripts used to set up the environment,
    development tools, and tools to flash the generated images on the
    target.
  \end{description}
\end{frame}

\begin{frame}
  \frametitle{Documentation}
  \begin{itemize}
    \item Documentation for the current sources, compiled as a "mega
      manual", is available at:
      \url{https://docs.yoctoproject.org/singleindex.html}
    \item Variables in particular are described in the variable
      glossary:
      \url{https://docs.yoctoproject.org/genindex.html}
  \end{itemize}
\end{frame}
