\section{Autotools usage}

\begin{frame}{Why do we need {\em autotools}?}
  \begin{itemize}
  \item {\bf Portability} accross Unix systems, architectures, Linux
    distributions
    \begin{itemize}
    \item Some C functions do not exist everywhere, or have different
      names or prototypes, can behave differently
    \item Header files can be organized differently
    \item All libraries may not be available everywhere
    \end{itemize}
  \item {\bf Standardized} build procedure
    \begin{itemize}
    \item Standard options
    \item Standard environment variables
    \item Standard behavior
    \end{itemize}
  \end{itemize}
\end{frame}

\begin{frame}{Using {\em autotools} based packages}
  \begin{itemize}
  \item The basic steps to build an {\em autotools} based software
    component are:
    \begin{enumerate}
    \item {\bf Configuration}\\
      \code{./configure}\\
      Will look at the available build environment, verify required
      dependencies, generate Makefiles.
    \item {\bf Compilation}\\
      \code{make}\\
      Actually builds the software component, using the generated
      Makefiles.
    \item {\bf Installation}\\
      \code{make install}\\
      Installs what has been built.
    \end{enumerate}
  \end{itemize}
\end{frame}

\begin{frame}{What is {\tt configure} doing?}

\end{frame}

\begin{frame}{Standard Makefile targets}
  \begin{itemize}
  \item \code{all}, builds everything. The default target.
  \item \code{install}, installs everything that should be installed.
  \item \code{install-strip}, same as \code{install}, but then strips
    debugging symbols
  \item \code{uninstall}
  \item \code{clean}, remove what was built
  \item \code{distclean}, same as \code{clean}, but also removes the
    generated {\em autotools} files
  \item \code{check}, run the test suite
  \item \code{installcheck}, check the installation
  \item \code{dist}, create a tarball
  \end{itemize}
\end{frame}

\begin{frame}{Standard filesystem hierarchy}
  \begin{itemize}
  \item {\bf prefix}, defaults to \code{/usr/local}
    \begin{itemize}
    \item {\bf exec-prefix}, defaults to {\em prefix}
      \begin{itemize}
      \item {\bf bindir}, for programs, defaults to {\em exec-prefix/bin}
      \item {\bf libdir}, for libraries, defaults to {\em exec-prefix/lib}
      \end{itemize}
    \end{itemize}
  \item {\bf includedir}, for headers, defaults to {\em prefix/include}
  \item {\bf datarootdir}, defaults to {\em prefix/share}
    \begin{itemize}
    \item {\bf datadir}, defaults to {\em datarootdir}
    \item {\bf mandir}, for man pages, defaults to {\em datarootdir/man}
    \item {\bf infodir}, for info documents, defaults to {\em datarootdir/info}
    \end{itemize}
  \item \code{./configure --prefix=~/sys/}
  \end{itemize}
\end{frame}

\begin{frame}{Standard configuration variables}

\end{frame}

\begin{frame}{Out of tree build}

\end{frame}

\begin{frame}{Cross-compilation}

\end{frame}

\begin{frame}{Diverted installation with DESTDIR}

\end{frame}

\begin{frame}{Analyzing issues}

\end{frame}

\begin{frame}{Overall process}

\end{frame}

\begin{frame}{Regenerating {\em autotools} files}

\end{frame}

\setuplabframe
{Usage of existing {\em autotools} projects}
{
  \begin{itemize}
  \item TODO
  \end{itemize}
}
