\section{Writing recipes - advanced}

\subsection{Extending a recipe}

\begin{frame}
  \frametitle{Introduction to recipe extensions}
  \begin{itemize}
    \item It is a good practice to avoid modifying recipes available
          in third party layers so it is easy to update.
    \item But it is sometimes useful to apply a custom patch or add a
      configuration file for example.
    \item The \code{bitbake} \emph{build engine} allows to modify a recipe by
          extending it.
    \item Multiple extensions can be applied to a recipe.
  \end{itemize}
\end{frame}

\begin{frame}
  \frametitle{Introduction to recipe extensions}
  \begin{itemize}
    \item Variable values can be modified
      \begin{itemize}
      \item Using operators to set, append or prepend
      \item Using overrides to append, prepend or remove
      \end{itemize}
    \item Tasks can be modified
      \begin{itemize}
      \item Using overrides to append or prepend task code
      \item Adding new tasks
      \end{itemize}
  \end{itemize}
\end{frame}

\begin{frame}{Extend a recipe}
  \begin{center}
    \includegraphics[width=\textwidth]{slides/yocto-recipe-advanced/yocto-recipe-advanced-extend.pdf}
  \end{center}
\end{frame}

\begin{frame}[fragile]
  \frametitle{Extend a recipe}
  \begin{itemize}
    \item The recipe extensions end in \code{.bbappend}
    \item Append files must have the same root name as the recipe they
          extend, but can also use wildcards.
    \begin{itemize}
      \item \code{example_0.1.bbappend} applies to
            \code{example_0.1.bb}
      \item {\usebeamercolor[fg]{code} \path{example_0.%.bbappend}} applies
        to \code{example_0.1.bb} and \code{example_0.2.bb} but not
        \code{example_1.0.bb}
        \item The {\usebeamercolor[fg]{code} \path{%}} works only just before the \code{.bbappend}
          suffix
    \end{itemize}
    \item Append files should be {\bf version specific}. If the recipe
      is updated to a newer version, the append files must also be
      updated.
    \item If adding new files, the path to their directory must
          be prepended to the \yoctovar{FILESEXTRAPATHS} variable.
    \begin{itemize}
      \item Files are looked up in paths referenced in
            \yoctovar{FILESEXTRAPATHS}, from left to right.
      \item Prepending a path makes sure it has priority over the recipe's
            one. This allows to override recipes' files.
    \end{itemize}
  \end{itemize}
\end{frame}

\begin{frame}[fragile]
  \frametitle{Append file example}
  \begin{block}{}
    \begin{minted}{sh}
FILESEXTRAPATHS:prepend := "${THISDIR}/files:"

SRC_URI += "file://defconfig \
            file://fix-memory-leak.patch \
            "
    \end{minted}
  \end{block}
\end{frame}

\begin{frame}[fragile]
  \frametitle{Modifying existing tasks}
  Tasks can be extended with \code{:prepend} or \code{:append}
  \begin{block}{}
    \begin{minted}{sh}
do_install:append() {
    install -d ${D}${sysconfdir}
    install -m 0644 hello.conf ${D}${sysconfdir}
}
    \end{minted}
  \end{block}
\end{frame}

\subsection{Providing virtual packages}

\begin{frame}
  \frametitle{Providing virtual packages}
  \begin{itemize}
    \item \code{bitbake} allows to use virtual names instead of the actual
          package name. We saw a use case with \emph{package
          variants}.
    \item The virtual name is specified through the \yoctovar{PROVIDES}
          variable.
    \item Several recipes can provide the same virtual name. Only one
          will be built and installed into the generated image.
    \item \code{PROVIDES = "virtual/kernel"}
  \end{itemize}
\end{frame}

\subsection{Classes}

\begin{frame}
  \frametitle{Introduction to classes}
  \begin{itemize}
    \item Classes provide an abstraction to common code, which can be
          re-used in multiple recipes.
    \item Common tasks do not have to be re-developed!
    \item Any metadata and task which can be put in a recipe can be
          used in a class.
    \item Classes extension is \code{.bbclass}
    \item Classes must be located in the \code{classes-recipe},
      \code{classes-global}, or \code{classes} folders of a layer.
  \end{itemize}
\end{frame}

\begin{frame}
  \frametitle{Using classes}
  \begin{itemize}
    \item Recipes can use the classes located in the \code{classes-recipe}
      folder:
    \begin{itemize}
      \item \code{inherit class1 class2 ...}
    \end{itemize}
    \item A recipe can inherit from multiple classes.
    \item Classes in \code{classes-global} can be inherited from configuration
      files with \yoctovar{INHERIT}:
    \begin{itemize}
      \item \code{INHERIT:append = " class1 class2"}
    \end{itemize}
    \item Classes in \code{INHERIT} will be used by every built recipe.
    \item Classes in the \code{classes} folder can be used with \code{inherit} or
      \yoctovar{INHERIT}, as their usage is not clearly defined.
  \end{itemize}
\end{frame}

\begin{frame}
  \frametitle{Common classes}
  \begin{itemize}
    \begin{itemize}
      \item \code{classes-global/base.bbclass}
      \item \code{classes-recipe/kernel.bbclass}
      \item \code{classes-recipe/autotools.bbclass}
      \item \code{classes-recipe/autotools-brokensep.bbclass}
      \item \code{classes-recipe/cmake.bbclass}
      \item \code{classes-recipe/meson.bbclass}
      \item \code{classes-recipe/native.bbclass}
      \item \code{classes-recipe/systemd.bbclass}
      \item \code{classes-recipe/update-rc.d.bbclass}
      \item \code{classes/useradd.bbclass}
      \item \dots
    \end{itemize}
  \end{itemize}
\end{frame}

\begin{frame}
  \frametitle{The base class}
  \begin{itemize}
    \item Every recipe inherits the \code{base} class automatically.
    \item Defines basic common tasks with a default implementation:
      \begin{itemize}
        \item \code{fetch}, \code{unpack}, \code{patch}
        \item \code{configure}, \code{compile}, \code{install}
        \item Utility tasks such as: \code{clean}, \code{listtasks}
      \end{itemize}
    \item Automatically applies patch files listed in \code{SRC_URI}
    \item Defines mirrors: \code{SOURCEFORGE_MIRROR},
      \code{DEBIAN_MIRROR}, \code{GNU_MIRROR}, \code{KERNELORG_MIRROR}\dots
    \item Defines \code{oe_runmake}, using \yoctovar{EXTRA_OEMAKE} to use
      custom arguments.
  \end{itemize}
\end{frame}

\begin{frame}[fragile]
  \frametitle{The kernel class}
  \begin{itemize}
    \item Used to build Linux kernels.
    \item Defines tasks to configure, compile and install a kernel and
          its modules.
    \item Automatically applies a \code{defconfig} listed in \yoctovar{SRC_URI}
      \begin{block}{}
        \begin{minted}{sh}
SRC_URI += "file://defconfig"
        \end{minted}
      \end{block}
    \item The kernel is divided into several packages: \code{kernel},
          \code{kernel-base}, \code{kernel-dev},
          \code{kernel-modules}\dots
    \item Automatically provides the virtual package
          \code{virtual/kernel}.
    \item Configuration variables are available:
    \begin{itemize}
      \item \yoctovar{KERNEL_IMAGETYPE}, defaults to \code{zImage}
      \item \yoctovar{KERNEL_EXTRA_ARGS}
      \item \yoctovar{INITRAMFS_IMAGE}
    \end{itemize}
  \end{itemize}
\end{frame}

\begin{frame}
  \frametitle{The autotools class}
  \begin{itemize}
    \item Defines tasks and metadata to handle applications using the
          autotools build system (autoconf, automake and libtool):
    \begin{itemize}
      \item \code{do_configure}: generates the configure script using
            \code{autoreconf} and loads it with standard arguments or
            cross-compilation.
      \item \code{do_compile}: runs \code{make}
      \item \code{do_install}: runs \code{make install}
    \end{itemize}
    \item Extra configuration parameters can be passed with
          \yoctovar{EXTRA_OECONF}.
    \item Compilation flags can be added thanks to the
          \yoctovar{EXTRA_OEMAKE} variable.
  \end{itemize}
\end{frame}

\begin{frame}[fragile]
  \frametitle{Example: use the autotools class}
  \begin{block}{}
    \begin{minted}[fontsize=\scriptsize]{sh}
DESCRIPTION = "Print a friendly, customizable greeting"
HOMEPAGE = "https://www.gnu.org/software/hello/"
SECTION = "examples"
LICENSE = "GPL-3.0-or-later"

SRC_URI = "${GNU_MIRROR}/hello/hello-${PV}.tar.gz"
SRC_URI[md5sum] = "67607d2616a0faaf5bc94c59dca7c3cb"
SRC_URI[sha256sum] = "ecbb7a2214196c57ff9340aa71458e1559abd38f6d8d169666846935df191ea7"
LIC_FILES_CHKSUM = "file://COPYING;md5=d32239bcb673463ab874e80d47fae504"

inherit autotools
    \end{minted}
  \end{block}
\end{frame}

\begin{frame}
  \frametitle{The useradd class}
  \begin{itemize}
    \item This class helps to add users to the resulting image.
    \item Adding custom users is required by many services to avoid
          running them as root.
    \item \yoctovar{USERADD_PACKAGES} must be defined when the \code{useradd} class
          is inherited. It defines the individual packages produced by the
          recipe that need users or groups to be added.
    \item Users and groups will be created before the packages using
          it perform their \code{do_install}.
    \item At least one of the two following variables must be set:
    \begin{itemize}
      \item \yoctovar{USERADD_PARAM}: parameters to pass to
            \code{useradd}.
      \item \yoctovar{GROUPADD_PARAM}: parameters to pass to
            \code{groupadd}.
    \end{itemize}
  \end{itemize}
\end{frame}

\begin{frame}[fragile]
  \frametitle{Example: use the useradd class}
  \begin{block}{}
    \begin{minted}[fontsize=\scriptsize]{sh}
DESCRIPTION = "useradd class usage example"
SECTION = "examples"
LICENSE = "MIT"

SRC_URI = "file://file0"
LIC_FILES_CHKSUM = "file://${COREBASE}/meta/files/common-licenses/MIT;md5=0835ade698e0bc..."

inherit useradd

USERADD_PACKAGES = "${PN}"
USERADD_PARAM:${PN} = "-u 1000 -d /home/user0 -s /bin/bash user0"

FILES:${PN} = "/home/user0/file0"

do_install() {
    install -d ${D}/home/user0/
    install -m 644 file0 ${D}/home/user0/
    chown user0:user0 ${D}/home/user0/file0
}
    \end{minted}
  \end{block}
\end{frame}

\begin{frame}
  \frametitle{The bin\_package class}
  \begin{itemize}
    \item In some cases you only need to install pre-built files into the
      generated root filesystem
      \begin{itemize}
        \item E.g.: firmware blobs
      \end{itemize}
    \item \code{bin_package.bbclass} simplifies this
      \begin{itemize}
        \item Disables \code{do_configure} and \code{do_compile}
        \item Provides a default \code{do_install} that copies whatever is
          in \yoctovar{S} (useful e.g. when extracting a pre-built RPM/DEB)
      \end{itemize}
    \item Additionally you probably need:
      \begin{itemize}
        \item Remember to set the \code{LICENSE} to \code{CLOSED} if applicable
        \item You probably should also \code{inherit allarch}
      \end{itemize}
  \end{itemize}
\end{frame}

\subsection{BitBake file inclusions}

\begin{frame}
  \frametitle{Locate files in the build system}
  \begin{itemize}
    \item Metadata can be shared using included files.
    \item \code{BitBake} uses the \yoctovar{BBPATH} to find the files to
      be included. It also looks into the current directory.
    \item Three keywords can be used to include files from recipes,
      classes or other configuration files:
      \begin{itemize}
        \item \code{inherit}
        \item \code{include}
        \item \code{require}
      \end{itemize}
  \end{itemize}
\end{frame}

\begin{frame}
  \frametitle{The \code{inherit} keyword}
  \begin{itemize}
    \item \code{inherit} can be used in recipes or classes, to inherit
      the functionalities of a class.
    \item To inherit the functionalities of the \code{kernel} class,
      use: \code{inherit kernel}
    \item \code{inherit} looks for files ending in \code{.bbclass}, in
      \code{classes} directories found in \yoctovar{BBPATH}.
    \item It is possible to include a class conditionally using a
      variable: \code{inherit ${FOO}}
    \item Inheriting in configuration files is based on the \yoctovar{INHERIT}
      variable instead:
      \begin{itemize}
        \item \code{INHERIT += "rm_work"}
        \item This inherits the class globally (i.e. for all recipes)
      \end{itemize}
  \end{itemize}
\end{frame}

\begin{frame}
  \frametitle{The \code{include} and \code{require} keywords}
  \begin{itemize}
    \item \code{include} and \code{require} can be used in all files,
      to insert the content of another file at that location.
    \item If the path specified on the \code{include} (or
      \code{require}) path is relative, \code{bitbake} will insert the first
      file found in \yoctovar{BBPATH}.
    \item \code{include} does not produce an error when a file cannot
      be found, whereas \code{require} raises a parsing error.
    \item To include a local file: \code{require ninvaders.inc}
    \item To include a file from another location (which could be
      in another layer): \code{require path/to/file.inc}
  \end{itemize}
\end{frame}

\subsection{More recipe debugging tools}

\begin{frame}[fragile]
  \frametitle{More recipe debugging tools}
  \begin{itemize}
    \item A development shell, exporting the full environment can be
      used to debug build failures:
      \begin{block}{}
        \begin{minted}{console}
$ bitbake -c devshell <recipe>
        \end{minted}
      \end{block}
    \item To understand what a change in a recipe implies, you can
      activate build history in \code{local.conf}:
      \begin{block}{}
        \begin{minted}{sh}
INHERIT += "buildhistory"
BUILDHISTORY_COMMIT = "1"
        \end{minted}
      \end{block}
      Then use the \code{buildhistory-diff} tool to examine
      differences between two builds.
      \begin{itemize}
        \item \code{buildhistory-diff}
      \end{itemize}
  \end{itemize}
\end{frame}

\subsection{Network usage}

\begin{frame}
  \frametitle{Source fetching}
  \begin{itemize}
    \item \code{bitbake} will look for files to retrieve at the following
      locations, in order:
      \begin{enumerate}
        \item \yoctovar{DL_DIR} (the local download directory).
        \item The \yoctovar{PREMIRRORS} locations.
        \item The upstream source, as defined in \yoctovar{SRC_URI}.
        \item The \yoctovar{MIRRORS} locations.
      \end{enumerate}
    \item If all the mirrors fail, the build will fail.
  \end{itemize}
\end{frame}

\begin{frame}[fragile]
  \frametitle{Mirror configuration in OpenEmbedded-Core}
  \code{meta/classes-global/mirrors.bbclass}
  \begin{block}{}
    \begin{minted}[fontsize=\tiny]{sh}
PREMIRRORS += "git://sourceware.org/git/glibc.git        https://downloads.yoctoproject.org/mirror/sources/ \
               git://sourceware.org/git/binutils-gdb.git https://downloads.yoctoproject.org/mirror/sources/"

MIRRORS += "\
svn://.*/.*     http://sources.openembedded.org/ \
git://.*/.*     http://sources.openembedded.org/ \
https?://.*/.*  http://sources.openembedded.org/ \
ftp://.*/.*     http://sources.openembedded.org/ \
...
"
    \end{minted}
  \end{block}
\end{frame}

\begin{frame}[fragile]
  \frametitle{Configuring the premirrors}
  \begin{itemize}
    \item It is easy to add a custom mirror to the \yoctovar{PREMIRRORS} by
      using the \code{own-mirrors} class (only one URL supported):
  \end{itemize}
  \begin{block}{}
    \begin{minted}{sh}
INHERIT += "own-mirrors"
SOURCE_MIRROR_URL = "http://example.com/my-mirror"
    \end{minted}
  \end{block}
  \begin{itemize}
    \item For a more complex setup,  prepend custom mirrors to the
      \yoctovar{PREMIRRORS} variable:
  \end{itemize}
  \begin{block}{}
    \begin{minted}{sh}
PREMIRRORS:prepend = "\
 git://.*/.*   http://example.com/my-mirror-for-git/ \
 svn://.*/.*   http://example.com/my-mirror-for-svn/ \
 http://.*/.*  http://www.yoctoproject.org/sources/  \
 https://.*/.* http://www.yoctoproject.org/sources/  "
    \end{minted}
  \end{block}
\end{frame}

\begin{frame}{Creating a local mirror}
  \begin{itemize}
    \item The download directory can be exposed on the network to create a
      local mirror
      \begin{itemize}
        \item Except for sources fetched via an SCM a tarball of the
          repository is needed, not the bare git repository that is created
          by default
        \item You can use \yoctovar{BB_GENERATE_MIRROR_TARBALLS = "1"} to
          generate tarballs of the git repositories in \yoctovar{DL_DIR}
      \end{itemize}
  \end{itemize}
\end{frame}

\begin{frame}{Forbidding network access}
  \begin{itemize}
    \item Since Kirkstone (4.0), network access is only enabled
          in the \code{do_fetch} task, to make sure no untraced sources
	  are fetched.
    \item You can also completely disable network access using
      \yoctovar{BB_NO_NETWORK = "1"}
      \begin{itemize}
      \item To download all the sources before disabling network access use
        \code{bitbake --runall=fetch core-image-minimal}
      \end{itemize}
    \item Or restrict \code{bitbake} to only download files from the
      \yoctovar{PREMIRRORS}, using \yoctovar{BB_FETCH_PREMIRRORONLY = "1"}
  \end{itemize}
\end{frame}
