\subsection{Control interfaces for the Network Stack}

\begin{frame}{Networking stack control path}
	\begin{itemize}
		\item The Networking stack is very highly configurable, at all levels :
		\item Controller and driver behaviour, through \code{ethtool}, \textit{e.g.} set the link speed
		\item Interface configuration, with \code{iproute2}, \textit{e.g.} configure the IP address
		\item System-wide configuration, \textit{e.g.} enable IP forwarding 
		\item Per-connection configuration, \textit{e.g.} select the TCP congestion-control algorithm
			\begin{itemize}
				\item The \code{setsockopts()} syscall is covered later in this training.
			\end{itemize}
	\end{itemize}
\end{frame}

\begin{frame}[fragile]{ioctl interface}
	\begin{itemize}
		\item The \code{ioctl} syscall is used to perform device-specific configuration
		\item \code{ioctl()} acts on a \textbf{file descriptor}.
			\begin{itemize}
				\item For hardware configuration, we usually use \code{ioctl} on \code{/dev/xxx} descriptors
			\end{itemize}
		\item We don't have any \code{fd} that corresponds to a specific \kstruct{net_device}
		\item Network admin \code{ioctl} uses a \code{fd} corresponding to a \textbf{socket} with unspecified family : \code{AF_UNSPEC}
		\item Any socket \textbf{fd} can be used for network ioctls.
	\end{itemize}
	\begin{block}{Example ioctl - Get interface name}
	\begin{minted}{c}
struct ifreq ifr;
ifr.ifr_ifindex = ifindex;
ioctl (fd, SIOCGIFNAME, &ifr);
	\end{minted}
	\end{block}
\end{frame}

\begin{frame}{ioctl API}
	\begin{itemize}
		\item Network-related \code{ioctl} have the \code{SIOC} prefix :
			\begin{itemize}
				\item \textit{e.g.} \ksym{SIOCGIFNAME} : Returns the name of an interface from its index
				\item \textit{e.g.} \ksym{SIOCADDMULTI} : Add to the multicast address list
				\item \textit{e.g.} \ksym{SIOCSHWTSTAMP} : Contigure hardware timestamping
			\end{itemize}
		\item Most of the \code{ioctl} API is now frozen, and maintained for compatibility
		\item Replaced with \code{Netlink}, which offers more flexibility
	\end{itemize}
\end{frame}

\begin{frame}{sysctl interface}
	\begin{itemize}
		\item The \textbf{sysctl} parameters are global, kernel-level parameters tunable at runtime
		\item \textbf{sysctl} is equivalent to writing into the corresponding files under \code{/proc/sys/}
		\item e.g. \code{systcl net.ipv4.ip_forward=1} is equivalent to \code{echo 1 > /proc/sys/net/ipv4/ip_forward}
		\item Values can be stored in \code{/etc/sysctl.d/*.conf}, and loaded with \code{sysctl -p}
		\item \textbf{sysctl} values are per-namespace, inheriting values from the \code{init_net}
		\item \code{net.core} : Core and net\_device level configuration
			\begin{itemize}
				\item \code{sysctl net.core.netdev_budget} : Displays the default NAPI budget
			\end{itemize}
		\item \code{net.ipv4} IPv4 and Layer 4 configuration
			\begin{itemize}
				\item \code{sysctl net.ipv4.ip_forward} : Allow IP forwarding (Router mode)
				\item \code{sysctl net.ipv4.tcp_fin_timeout} : Set the TCP connection timeout (even for IPv6)
			\end{itemize}

		\item \code{net.ipv6} IPv6 configuration

	\end{itemize}
\end{frame}

\begin{frame}[fragile]{Netlink interface}
	\begin{itemize}
		\item More flexible kernel to userspace communication mechanism, based on sockets
			\begin{minted}{c}
fd = socket(AF_NETLINK, SOCK_RAW, NETLINK_GENERIC);
			\end{minted}
		\item Allows easy extension of the userspace API without breaking compatibility
		\item User applications must open a \textbf{netlink socket} and send specially-formatted messages
		\item The socket can also be listened to for Kernel to userspace notificatins
		\item Netlink messages are grouped in \textbf{families}, grouping message types per class.
			\begin{itemize}
				\item routing, ethtool, 802.11, team, macsec, etc. 
			\end{itemize}
		\item Netlink messages have a well-defined and stable format, but extensible.
	\end{itemize}
\end{frame}

\begin{frame}{Netlink Classic vs Netlink Generic}
	\begin{itemize}
		\item Most netlink users today use \textbf{generic netlink}
		\item This replaces \textbf{classic netlink}, which has statically allocated familiy ID
		\item Generic Netlink (\code{genetlink}) allows dynamic family regstration 
			\begin{itemize}
				\item Allows easy implementation of custom families
				\item Families are looked-up by \textbf{name} (a string) instead of ID.
			\end{itemize}
		\item Example families :
			\begin{itemize}
				\item "ethtool": Ethtool commands, also called \textbf{ethnl}
				\item "wireguard": Wireguard tunneling configuration
				\item "nl80211": Wifi-configuration netlink commands
			\end{itemize}
	\end{itemize}
\end{frame}

\begin{frame}[fragile]{Netlink Messages}
	\begin{itemize}
		\item Transmission of \textbf{netlink messages} :
			\begin{itemize}
				\item A fixed-format Header begins the message
				\item The information is conveyed through TLV items : \textbf{Type}, \textbf{Length}, \textbf{Value}
			\end{itemize}
	\end{itemize}
	\begin{minted}{c}
struct nlmsghdr {
      __u32   nlmsg_len;      /* Length of message including headers */
      __u16   nlmsg_type;     /* Generic Netlink Family (subsystem) ID */
      __u16   nlmsg_flags;    /* Flags - request or dump */
      __u32   nlmsg_seq;      /* Sequence number */
      __u32   nlmsg_pid;      /* Port ID, set to 0 */
};
struct genlmsghdr {
      __u8    cmd;            /* Command, as defined by the Family */
      __u8    version;        /* Irrelevant, set to 1 */
      __u16   reserved;       /* Reserved, set to 0 */
};
/* TLV attributes follow... */
	\end{minted}
\end{frame}

\begin{frame}{Netlink commands}
	\begin{itemize}
		\item There are multiple types of Netlink requests based on the \code{nlmsg_flags}
		\item Commands may be used to Get or Set some kernel attributes 
		\item Netlink \code{Get} commands can target one or several objects
			\begin{itemize}
				\item A single object request is a \code{.doit()} request
					\begin{itemize}
						\item \code{ip link show eth0}
					\end{itemize}
				\item An object listing request is a \code{.dumpit()} request
					\begin{itemize}
						\item \code{ip link show}
					\end{itemize}
			\end{itemize}
		\item Netlink also exposes \textbf{multicast notifications}
		\item The message content is made of a set of pre-defined Attributes, based on the Command and Family
			\begin{itemize}
				\item \textit{e.g.} Command \code{ETHTOOL_MSG_LINKMODES_GET} for family "ethtool"
				\item Contains \code{ETHTOOL_A_LINKMODES_SPEED}
			\end{itemize}
	\end{itemize}
\end{frame}

\begin{frame}[fragile]{Netlink specifications}
	\begin{itemize}
		\item Message content used to be specified directly in the kernel uAPI headers
		\item Formats are now defined as \textbf{Netlink Specs} written in \code{YAML}
		\item Specifications are written per-family in \kfile{Documentation/netlink/specs}
	\end{itemize}
	\begin{block}{Documentation/netlink/specs/ethtool.yaml}
	\begin{minted}{yaml}
name: ethtool
protocol: genetlink-legacy
doc: Partial family for Ethtool Netlink.

definitions:
  - ...
attribute-sets:
  - ...
operations:
  - ...
	\end{minted}
	\end{block}
\end{frame}

\begin{frame}{Netlink specifications - 2}
	\begin{itemize}
		\item Netlink specs are used internally to \textbf{generate} the uAPI headers
			\begin{itemize}
				\item Generated in \kfile{include/uapi/linux/ethtool_netlink_generated.h}
				\item Included by \kfile{include/uapi/linux/ethtool_netlink.h}
			\end{itemize}
		\item When modifying the specs, headers can be regenerated with \code{${KDIR}/tools/net/ynl/ynl-regen.sh}
		\item The \textbf{ynl} tool included in the kernel's sources can be used to sent hand-crafted messages
			\begin{itemize}
				\item \code{make -C tools/net/ynl}
			\end{itemize}
		\item Uses the Netlink Specs to derive the format and family :
			\code{ynl --family ethtool --no-schema --do linkinfo-get \ }
			\code{--json '{"header" : { "dev-name" : "eth0"}}'}
	\end{itemize}
\end{frame}

\begin{frame}{Netlink monitor}
	\begin{itemize}
		\item \textbf{Netlink Monitoring} car refer to 2 distinct operations :
		\item One can listen to \textbf{netlink notifications}
			\begin{itemize}
				\item Emitted by the kernel upon configuration change
				\item Applications can listen for specific notifications (Address change, link up, etc.)
				\item e.g. \code{ip monitor}, \code{ethtool --monitor}, etc.
			\end{itemize}
		\item It is also possible to listen to \textbf{All Netlink Traffic}
			\begin{itemize}
				\item It includes All netlink messages, requests, replies and notifications
				\item Done through a dedicated virtual interface : \code{nlmon}
				\item e.g. \code{ip link add name nlmon0 type nlmon}
				\item Tools such as \code{tcpdump} and \code{wireshark} can be used on the nlmon interface
			\end{itemize}
		\item All these mechanisms still go through network namespaces
	\end{itemize}
\end{frame}

\begin{frame}{Configuration serialization in the kernel}
	\begin{itemize}
		\item Actions triggered by \code{ioctl} or \code{netlink} messages often need serialization
			\begin{itemize}
				\item Some actions impact multiple devices (e.g. netns removal)
				\item Actions may be performed on multiple CPUs concurrently
			\end{itemize}
		\item The main lock used to serialize the configuration is the \textbf{rtnl lock}
			\begin{itemize}
				\item Global \kstruct{mutex}, taken with \kfunc{rtnl_lock} and released with \kfunc{rtnl_unlock}
				\item \textbf{R}ou\textbf{T}ing \textbf{N}et\textbf{L}ink
			\end{itemize}
		\item \code{net_device.lock} : Mutex to protect \textit{some} of the \kstruct{net_device} fields
			\begin{itemize}
				\item Very recent feature, introduced in \code{v6.14}
			\end{itemize}
		\item The list of \kstruct{net_device} is protected by RCU
		\item All \kstruct{net_device} instance are reference-counted and reference-tracked
	\end{itemize}
\end{frame}

\begin{frame}{The RTNL lock}
	\begin{itemize}
		\item Sometimes the Network Stack's Big Kernel Lock
			\begin{itemize}
				\item Its scope is slowly getting removed, replaced with more specific locks
			\end{itemize}
		\item Serializes most NDOs that aren't on the datapath
			\begin{itemize}
				\item e.g. it does not protect \code{.ndo_start_xmit()}.
			\end{itemize}
		\item Also serializes most \kstruct{ethtool_ops}
		\item Protects some of the \kstruct{net_device} fields
		\item For now, RTNL is \textbf{not} per-namespace, it is global. This is being reworked.
		\item Functions that rely on the caller holding rtnl often use \kfunc{ASSERT_RTNL}
		\item It is a \code{mutex} :
			\begin{itemize}
				\item It is possible to sleep while holding rtnl
				\item rtnl cannot be used when sleeping is forbidden (e.g. interrupt and softirq context)
			\end{itemize}
	\end{itemize}
\end{frame}

\begin{frame}{Using Netlink in the kernel}
	\begin{itemize}
		\item A new family can be registered by registering a \kstruct{genl_family}
		\item This allows registering custom messages and associated handlers
			\begin{itemize}
				\item e.g the \href{https://elixir.bootlin.com/linux/v6.15.2/source/drivers/net/macsec.c\#L3360}{macsec family}
			\end{itemize}
		\item Existing families already provide layers of abstractions :
			\begin{itemize}
				\item The \kstruct{rtnl_link_ops} is used for virtual netdev types
				\item The \href{https://elixir.bootlin.com/linux/v6.15.2/source/net/ethtool}{ethnl} abstraction is used for ethtool commands
			\end{itemize}
	\end{itemize}
\end{frame}

\begin{frame}[fragile]{Netlink Attributes}
	\begin{itemize}
		\item As netlink is part of \textbf{userspace API}, all attribute values are heavily checked
		\item When declaring attributes, we can specify a \textbf{policy}
			\begin{itemize}
				\item Allows specifying a range of acceptable values
					\begin{minted}{c}
const struct nla_policy ethnl_linkmodes_set_policy[] = {
        [ETHTOOL_A_LINKMODES_LANES] = NLA_POLICY_RANGE(NLA_U32, 1, 8),
};
					\end{minted}
			\end{itemize}
		\item Handling messages is done in the genetlink \code{.doit()} callback
		\item Netlink attributes are represented as \kstruct{nlattr}
		\item Attributes are passed as an array, indexed by attribute id, usually named \code{tb}
		\item Helpers are provided to get attribute values, \textit{e.g.} \kfunc{nla_get_u32}
		\begin{minted}{c}
if (data[IFLA_MACSEC_WINDOW])
        secy->replay_window = nla_get_u32(data[IFLA_MACSEC_WINDOW]);	
		\end{minted}
	\end{itemize}
\end{frame}

\begin{frame}[fragile]{\kstruct{netlink_ext_ack}}
	\begin{itemize}
		\item \kstruct{netlink_ext_ack} allows reporting error messages to userspace
		\item It is included as part of the reply to netlink requests
		\item It can be found passed as a parameter to numerous internal kernel function
	\end{itemize}
	{\fontsize{9}{10}
	\begin{minted}{c}
int dsa_port_mst_enable(struct dsa_port *dp, bool on, struct netlink_ext_ack *extack)
{
        if (on && !dsa_port_supports_mst(dp)) {
                NL_SET_ERR_MSG_MOD(extack, "Hardware does not support MST");
                return -EINVAL;
        }

        return 0;
}
	\end{minted}
	}
\end{frame}

\begin{frame}{Userpace libraries}
	\begin{itemize}
		\item \href{https://www.netfilter.org/projects/libmnl/index.html}{libmnl}: Simple and lightweight library to access netlink
			\begin{itemize}
				\item Used by \code{nftables}, \code{iproute2} and \code{ethtool}
			\end{itemize}
		\item \href{https://www.infradead.org/~tgr/libnl/}{libnl}: Higher level of abstraction but bigger binary
			\begin{itemize}
				\item Useful for programs that "just work"
			\end{itemize}
	\end{itemize}
\end{frame}

\begin{frame}{Userspace tooling}
	\begin{itemize}
		\item \textbf{iproute2}
			\begin{itemize}
				\item \code{ip} : Configure and query interfaces, routing and tunnels
				\item \code{bridge} : Configures bridges (switching)
				\item \code{tc} : Configures traffic control policing, shaping and filtering
				\item \code{dcb} : Configures "Data Center Bridging" for traffic priorisation
				\item These tools use the \textbf{Netlink} API
			\end{itemize}
		\item \textbf{net-tools}
			\begin{itemize}
				\item Obsolete ! replaced by \textbf{iproute2}
				\item \code{ifconfig} : Configure interfaces (replaced by \code{ip})
				\item \code{brctl} : Configure bridges (replaced by \code{bridge})
			\end{itemize}
		\item \textbf{ethtool}
			\begin{itemize}
				\item Used to configure Ethernet devices.
				\item Can be compiled with \code{ioctl} or \code{netlink} support.
			\end{itemize}
	\end{itemize}
\end{frame}

\begin{frame}{Userspace tooling - 2}
	\begin{itemize}
		\item \textbf{NetworkManager}
			\begin{itemize}
				\item Automatic configuration of interfaces based on config files
				\item Handles IP assignment, low-level parameters
				\item Very featureful, but bigger binary
				\item Exposes a DBus API to interact with other software
				\item See \manpage{NetworkManager}{8}
			\end{itemize}
		\item \textbf{Connman}
			\begin{itemize}
				\item Alternative to NetworkManager, more embedded-oriented and lightweight
				\item Also uses DBus to communicate with other software
			\end{itemize}
		\item \textbf{Systemd-networkd}
			\begin{itemize}
				\item Network configuration tool provided by SystemD 
				\item \code{.network} files are used to describe interfaces
				\item See \manpage{systemd-networkd}{8}
			\end{itemize}
		\end{itemize}
\end{frame}

\begin{frame}{C library}
	\begin{itemize}
		\item The C library exposes a few helper functions to manipulate Network features
		\item \code{if_nametoindex}: Get the \code{ifindex} of a given interface
		\item \code{if_indextoname}: Get the name of an interface from its \code{index}
			\begin{itemize}
				\item Can use either netlink or \code{ioctl}
			\end{itemize}
		\item \code{inet_aton}, \code{inet_addr}: Convert IP address from string to binary
		\item \code{htons}, \code{ntohs}, \code{htonl}, \code{ntohl}: Endianness conversion
			\begin{itemize}
				\item On most protocol headers, data is sent in \textbf{big endian} format
				\item Also referred-to as Network Byte Order
				\item \textbf{h}ost \textbf{to} \textbf{n}etwork \textbf{s}ort / \textbf{l}ong
			\end{itemize}
	\end{itemize}
\end{frame}



% nothing in /dev
% ioctl
% netlink, principle
% netlink families
% Common userspace tools

