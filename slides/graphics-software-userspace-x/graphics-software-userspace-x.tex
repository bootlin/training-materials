\subsection{Userspace: X Window}

\begin{frame}{X11 protocol and architecture}
  \begin{itemize}
  \item X11 core protocol implemented by Xorg:
    \begin{itemize}
    \item Asynchronous packet-based system with different types:\\
    \code{Request}, \code{Reply}, \code{Event} and \code{Error} packets
    \item Can be used locally (UNIX socket) or over network (TCP/IP)
    \end{itemize}
  \item Exposes \textbf{drawables} for clients to transfer or draw pixel data to the server:
    \begin{itemize}
    \item \textbf{Windows}: area of the display buffer owned by the application\\
      \textit{without backing storage, must be redrawn when occluded}
    \item \textbf{Pixmaps}: off-display backing storage that can be copied to windows
    \end{itemize}
  \item Windows are represented as a tree:
    \begin{itemize}
    \item Starting with the root window created by X
    \item Top-level windows and sub-windows created by clients
    \end{itemize}
  \item A graphics context (GC) allows requesting basic drawing and font rendering
  \item The server provides \textbf{input events} to concerned clients:
    \begin{itemize}
    \item Mouse movements relative to window coordinates
    \item Translated key symbols a from raw keycodes
    \end{itemize}
  \end{itemize}
\end{frame}

\begin{frame}{X11 protocol extensions}
  \begin{itemize}
  \item X11 has evolved over time through \textbf{extensions} to its main protocol
    \begin{itemize}
    \item Additional interfaces for X clients, matching \textbf{new hardware features}
    \end{itemize}
  \item \textbf{XKB}: complex keyboard layouts
  \item \textbf{Xinput2}: touchpad, touchscreen and multi-touch support
  \item \textbf{XSHM}: shared client/server memory, avoiding extra transfers/copies\\
    \textit{not\ possible to operate via the network}
  \item \textbf{XRandR}: monitor configuration and hotplugging without server restart
  \item \textbf{Composite}: delegates window composition to compositing window managers
  \item \textbf{XRender}: 2D rendering API with with alpha composition, rasterization, transformations, filtering
  \item \textbf{Xv}: video output format conversion and scaling offload in-DDX\\
    \textit{involves buffer copies and lacks synchronization with window position}
  \end{itemize}
\end{frame}

\begin{frame}{Xorg architecture and acceleration}
  \begin{itemize}
  \item Xorg is divided between generic and hardware-specific parts
  \item \textbf{Device-Independent-X} (DIX) concerns:
    \begin{itemize}
    \item X11 protocol implementation, client coordination
    \item Main event loop and event dispatching
    \item Graphics operations logic, boilerplate and fallback implementations
    \end{itemize}
  \item \textbf{Device-Dependent-X} (DDX) concerns:
    \begin{itemize}
    \item Input drivers (\code{xf86-input-...}) to grab events from the kernel
    \item Video drivers (\code{xf86-video-...}) to provide mode setting and 2D acceleration
    \end{itemize}
  \item \textbf{EXA} provides a 2D acceleration architecture between DIX and DDX
    \begin{itemize}
    \item Efficient way for drivers to expose accelerated 2D operation primitives
    \item Replaced the XFree86 Acceleration Architecture (XAA)
    \item Reduces driver boilerplate
    \end{itemize}
  \item \textbf{Glamor} provides 2D acceleration for the DDX using OpenGL 3D rendering
  \end{itemize}
\end{frame}

\begin{frame}{Xorg drivers overview}
  \begin{itemize}
  \item Generic Xorg \textbf{input} drivers:
    \begin{itemize}
    \item \code{xf86-input-libinput}: using \code{libinput} to get input events
    \item \code{xf86-input-evdev}: using the \code{evdev} kernel interface directly (\textbf{deprecated})
    \end{itemize}
  \item Specific Xorg \textbf{input} drivers:
    \begin{itemize}
    \item \code{xf86-input-synaptics}: for laptop touchpads
    \item \code{xf86-input-wacom}: for Wacom drawing tablets
    \item Specific drivers are \textbf{deprecated} in favor of \code{xf86-input-libinput}
    \end{itemize}
  \item Generic Xorg \textbf{display} drivers:
    \begin{itemize}
    \item \code{xf86-video-modesetting}: for \textbf{DRM KMS}, can be accelerated using \textbf{glamor}
    \item \code{xf86-video-fbdev}: for the \textbf{fbdev} interface, without acceleration (legacy)
    \item \code{xf86-video-vesa}: for the \textbf{Video BIOS Extension} (VBE) framebuffer (x86)
    \end{itemize}
  \item Specific Xorg \textbf{display} drivers:
    \begin{itemize}
    \item \code{xf86-video-[intel,nouveau,amdgpu]}: profiting from 2D acceleration blocks
    \item Specific drivers are deprecated in favor of \code{xf86-video-modesetting} and \textbf{glamor}\\
      \textit{the trend is to accelerate everything via 3D rendering instead of 2D accelerators}
    \end{itemize}
  \end{itemize}
\end{frame}

\begin{frame}{X11 and OpenGL acceleration: GLX and DRI2}
  \begin{itemize}
  \item Before DRM render nodes, there was a single device for KMS and render\\
    \textit{correlates with the idea of a graphics card mixing both aspects}
  \item The X server owns the graphics device \textbf{exclusively}
  \item Clients using OpenGL need to access the device for rendering
  \item The \textbf{GLX} API was introduced to perform \textbf{indirect rendering}:
    \begin{enumerate}
    \item Integrating OpenGL with the X Window API
    \item Forwarding GL calls to the GL implementation via the X server (AIGLX)\\
      \textit{introducing latency and performance issues}
    \end{enumerate}
  \item \textbf{The Direct Rendering Infrastructure} (DRI/DRI2) was introduced next
    \begin{itemize}
    \item The X server allowed access through DRM magic/auth
    \item Buffers were shared via \textit{GEM flinks}
    \item Now using the standalone \textbf{render node} and \textbf{dma-buf} instead
    \item Still in place for coordination between render and the display server
    \end{itemize}
  \item GLX remained as a GL windowing API for X11 (deprecated by EGL)
  \end{itemize}
\end{frame}

\begin{frame}{X11 and OpenGL acceleration: GLX and DRI2 (illustrated)}
  \begin{center}
  \includegraphics[width=0.4\textwidth]{slides/graphics-software-userspace-x/dri-data-flow.png}\\
  \textit{Data flow in X11 for different types of clients}
  \end{center}
\end{frame}

\begin{frame}{Xorg usage, integration and configuration}
  \begin{itemize}
  \item Xorg can be started with the \code{startx} command (wrapping \code{xinit})
    \begin{itemize}
    \item Executes server script from \code{/etc/X11/xinit/xserverrc} or \code{$HOME/.xserverrc}
    \item Executes client script from \code{/etc/X11/xinit/xinitrc} or \code{$HOME/.xinitrc}
    \end{itemize}
  \item An X \textbf{display manager} offers a login interface (e.g. KDM, LightDM)
    \begin{itemize}
    \item Runs under a Xorg server, with its own dedicated user
    \item Starts Xorg for authenticated users from session files in \code{/usr/share/xsessions/}
    \end{itemize}
  \item Used to require \textbf{running the server as root} to access graphics devices\\
    \textit{in particular, necessary to become DRM master}
    \begin{itemize}
    \item The \code{systemd-logind} login manager lifts the restriction
    \item Opens the DRM KMS fd privileged and passes it to Xorg via IPC
    \item Xorg can then drop privileges: details in the \code{Xorg.wrap} man page
    \end{itemize}
  \item Xorg is \textbf{configured} (both DIX and DDX) from \code{/etc/X11/xorg.conf}
  \item The \code{DISPLAY} environment variable indicates which server connection to use\\
    \begin{itemize}
    \item Already set for X client applications and inherited
    \item \code{export DISPLAY=:0} useful to launch programs from other TTYs
    \end{itemize}
  \end{itemize}
\end{frame}

\begin{frame}{Xorg architecture: input to display roundtrip}
  \begin{minipage}{0.49\textwidth}
    \centering
    \includegraphics[width=0.9\textwidth]{slides/graphics-software-userspace-x/x-architecture-roundtrip.png}
  \end{minipage}
  \hfill
  \begin{minipage}{0.49\textwidth}
    \begin{enumerate}
    \item An input event is read from the kernel by the server
    \item The affected client is determined and receives the event
    \item The client changes something and issues a rendering request
    \item The server performs rendering (DDX) and notifies the compositor
    \item The compositor updates the damaged regions in the back-buffer
    \item The server updates the display buffer from the compositor buffer (page flip)
    \end{enumerate}
  \end{minipage}
\end{frame}

\begin{frame}{Major issues with X11}
  \begin{itemize}
  \item The X11 core protocol and paradigm soon caused \textbf{various issues}:
    \begin{itemize}
    \item Based on buffer \textbf{copies, transfers} and frequent \textbf{redraws}\\
      \textit{solved with XSHM and DRI2 extensions}
    \item Immediate-mode drawing, with intermediate states scanned out\\
      \textit{solved by drawing everything client-side instead}
    \item Lack of synchronization/feedback interface\\
      \textit{specified with the DRI3 and Present extensions}
    \item Everything's a window with X... but not in practice (screensavers, popups)\\
    \textit{specified with the DRI3 and Present extensions}
    \item Heavy packet-based protocol causing \textbf{latency issues}
    \item \textbf{Security} concerns regarding client input/output isolation
    \end{itemize}
  \item Because the core protocol did not evolve, \textbf{extensions proliferated}:
    \begin{itemize}
    \item Complicated server aspects got \textbf{delegated} through extensions
    \item Working around major design issues, not solving them in depth
    \item In the end, the server mostly coordinates between other components
    \end{itemize}
  \item \textbf{Client-side rendering} became more common (raster, operations, fonts, etc)
  \end{itemize}
\end{frame}

\begin{frame}{Xorg code structure and walkthrough}
  \begin{itemize}
  \item Xorg source code available at: \url{https://gitlab.freedesktop.org/xorg/xserver}
  \item DDX components:
    \begin{itemize}
    \item Code specific to the Linux kernel under: \code{hw/xfree86/os-support/linux/}
    \item Modesetting DRM KMS driver under: \code{hw/xfree86/drivers/modesetting/}
    \item fbdev core library under: \code{hw/xfree86/fbdevhw/}
    \item Glamor implementation under: \code{glamor/}
    \end{itemize}
  \item DIX components:
    \begin{itemize}
    \item System-level helpers under: \code{os/}
    \item Common framebuffer operations abstraction under: \code{fb/}
    \item EXA abstraction under: \code{exa/}
    \end{itemize}
  \item DRI2 components:
    \begin{itemize}
    \item DRI2 common code under: \code{hw/xfree86/dri2}
    \item Modesetting DRI2 glue under: \code{hw/xfree86/drivers/modesetting/dri2.c}
    \item GLX support under: \code{glx/}
    \end{itemize}
  \end{itemize}
\end{frame}

\begin{frame}[fragile]{Xorg debug and documentation}
  \begin{itemize}
  \item Xorg has a \textbf{logging system} for all its components:
    \begin{itemize}
    \item Written to a file at \code{/var/log/Xorg.0.log}, \code{-logfile} option
    \item Verbosity can be set with the \code{-logverbose} option (log level)
    \item Printed on the standard output (\code{stdout})
    \end{itemize}
  \item Xorg can be bound to any VT with the \code{vt} command line option\\
    \textit{useful for remote debugging, with a virtual controlling terminal}
  \item \textbf{Community} contact:
    \begin{itemize}
    \item Mailing list: \code{xorg@lists.freedesktop.org}
    \item IRC channel: \code{#xorg} and \code{#xorg-devel} on the OFTC network
    \end{itemize}
  \item \textbf{Documentation} resources:
    \begin{itemize}
    \item Online wiki of the project: \url{https://www.x.org/wiki/Documentation/}
    \item Man pages: \code{X}, \code{Xserver}, \code{Xorg}, \code{xorg.conf}, \code{xinit} and more!
    \item Extensions specification documents
    \end{itemize}
  \end{itemize}
\end{frame}
