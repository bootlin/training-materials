\section{Linux device and driver model}

\subsection{Introduction}

\begin{frame}{The need for a device model?}
  \begin{itemize}
  \item The Linux kernel runs on a wide range of architectures and
    hardware platforms, and therefore needs to {\bf maximize the
      reusability} of code between platforms.
  \item For example, we want the same {\em USB device driver} to be
    usable on a x86 PC, or an ARM platform, even though the USB
    controllers used on these platforms are different.
  \item This requires a clean organization of the code, with the {\em
      device drivers} separated from the {\em controller drivers}, the
    hardware description separated from the drivers themselves, etc.
  \item This is what the Linux kernel {\bf Device Model} allows, in
    addition to other advantages covered in this section.
  \end{itemize}
\end{frame}

\begin{frame}
  \frametitle{Kernel and Device Drivers}
  \begin{columns}
    \column{0.7\textwidth} In Linux, a driver is always interfacing
    with:
    \begin{itemize}
    \item a {\bf framework} that allows the driver to expose the
      hardware features in a generic way.
    \item a {\bf bus infrastructure}, part of the device model, to
      detect/communicate with the hardware.
    \end{itemize}
    This section focuses on the {\em bus infrastructure}, while {\em kernel
      frameworks} are covered later in this training.
    \column{0.3\textwidth}
    \includegraphics[height=0.8\textheight]{slides/kernel-device-model/driver-architecture.pdf}
  \end{columns}
\end{frame}

\begin{frame}
  \frametitle{Device Model data structures}
  \begin{itemize}
  \item The {\em device model} is organized around three main data
    structures:
    \begin{itemize}
    \item The \kstruct{bus_type} structure, which represents one type of bus
      (USB, PCI, I2C, etc.)
    \item The \kstruct{device_driver} structure, which represents one driver
      capable of handling certain devices on a certain bus.
    \item The \kstruct{device} structure, which represents one device
      connected to a bus
    \end{itemize}
  \item The kernel uses inheritance to create more specialized
    versions of \kstruct{device_driver} and \kstruct{device}
    for each bus subsystem.
  \item In order to explore the device model, we will
    \begin{itemize}
    \item First look at a popular bus that offers dynamic enumeration,
      the {\em USB bus}
    \item Continue by studying how buses that do not offer dynamic
      enumeration are handled.
    \end{itemize}
  \end{itemize}
\end{frame}

\begin{frame}
  \frametitle{Bus Drivers}
  \begin{itemize}
  \item The first component of the device model is the bus driver
    \begin{itemize}
    \item One bus driver for each type of bus: USB, PCI, SPI, MMC,
      I2C, etc.
    \end{itemize}
  \item It is responsible for
    \begin{itemize}
    \item Registering the bus type (\kstruct{bus_type})
    \item Allowing the registration of adapter drivers (USB
      controllers, I2C adapters, etc.), able to detect the
      connected devices (if possible), and providing a
      communication mechanism with the devices
    \item Allowing the registration of device drivers (USB devices,
      I2C devices, PCI devices, etc.), managing the devices
    \item Matching the device drivers against the devices detected by
      the adapter drivers.
    \item Provides an API to implement both adapter drivers and device drivers
    \item Defining driver and device specific structures, mainly
      \kstruct{usb_driver} and \kstruct{usb_interface}
    \end{itemize}
  \end{itemize}
\end{frame}


\begin{frame}
  \frametitle{sysfs}
  \begin{itemize}
  \item The bus, device, drivers, etc. structures are internal to the
    kernel
  \item The \code{sysfs} virtual filesystem offers a mechanism to
    export such information to user space
  \item Used for example by \code{udev} to provide automatic module loading,
    firmware loading, mounting of external media, etc.
  \item \code{sysfs} is usually mounted in \code{/sys}
    \begin{itemize}
    \item \code{/sys/bus/} contains the list of buses
    \item \code{/sys/devices/} contains the list of devices
    \item \code{/sys/class} enumerates devices by the framework they are
          registered to (\code{net}, \code{input}, \code{block}...),
          whatever bus they are connected to. Very useful!
    \end{itemize}
  \end{itemize}
\end{frame}

\subsection{Example of the USB bus}

\begin{frame}
\frametitle{Example: USB Bus 1/3}
  \begin{center}
    \includegraphics[height=0.8\textheight]{slides/kernel-device-model/usb-bus-hardware.pdf}
  \end{center}
\end{frame}

\begin{frame}
\frametitle{Example: USB Bus 2/3}
  \begin{center}
    \includegraphics[height=0.8\textheight]{slides/kernel-device-model/usb-bus.pdf}
  \end{center}
\end{frame}

\begin{frame}
  \frametitle{Example: USB Bus 3/3}
  \begin{itemize}
  \item Core infrastructure (bus driver)
    \begin{itemize}
    \item \kdir{drivers/usb/core}
    \item \kstruct{bus_type} is defined in
      \kfile{drivers/usb/core/driver.c} and registered in
      \kfile{drivers/usb/core/usb.c}
    \end{itemize}
  \item Adapter drivers
    \begin{itemize}
    \item \kdir{drivers/usb/host}
    \item For EHCI, UHCI, OHCI, XHCI, and their implementations on
      various systems (Microchip, IXP, Xilinx, OMAP, Samsung, PXA, etc.)
    \end{itemize}
  \item Device drivers
    \begin{itemize}
    \item Everywhere in the kernel tree, classified by their type
    (Example: \kdir{drivers/net/usb})
    \end{itemize}
  \end{itemize}
\end{frame}

\begin{frame}
  \frametitle{Example of Device Driver}
  \begin{columns}
  \column{0.75\textwidth}
  \begin{itemize}
  \item To illustrate how drivers are implemented to work with the
    device model, we will study the source code of a driver for a USB
    network card
    \begin{itemize}
    \item It is USB device, so it has to be a USB device driver
    \item It exposes a network device, so it has to be a network driver
    \item Most drivers rely on a bus infrastructure (here, USB) and
      register themselves in a framework (here, network)
    \end{itemize}
  \item We will only look at the device driver side, and not the
    adapter driver side
  \item The driver we will look at is \kfile{drivers/net/usb/rtl8150.c}
  \end{itemize}
  \column{0.25\textwidth}
      \includegraphics[width=\textwidth]{slides/kernel-device-model/usb-network.pdf}
  \end{columns}
\end{frame}

\begin{frame}[fragile]
  \frametitle{Device Identifiers}
  \begin{itemize}
  \item Defines the set of devices that this driver can manage, so
    that the USB core knows for which devices this driver should be
    used
  \item The \kfunc{MODULE_DEVICE_TABLE} macro allows \code{depmod}
    (run by \code{make modules_install}) to extract the relationship
    between device identifiers and drivers, so that drivers can be
    loaded automatically by \code{udev}.
    See \code{/lib/modules/$(uname -r)/modules.{alias,usbmap}}
  \end{itemize}
  \begin{block}{}
  \begin{minted}[fontsize=\footnotesize]{c}
static struct usb_device_id rtl8150_table[] = {
    { USB_DEVICE(VENDOR_ID_REALTEK, PRODUCT_ID_RTL8150) },
    { USB_DEVICE(VENDOR_ID_MELCO, PRODUCT_ID_LUAKTX) },
    { USB_DEVICE(VENDOR_ID_MICRONET, PRODUCT_ID_SP128AR) },
    { USB_DEVICE(VENDOR_ID_LONGSHINE, PRODUCT_ID_LCS8138TX) },
    [...]
    {}
};
MODULE_DEVICE_TABLE(usb, rtl8150_table);
  \end{minted}
  \end{block}
\end{frame}

\begin{frame}[fragile]
  \frametitle{Instantiation of usb\_driver}
  \begin{itemize}
  \item \kstruct{usb_driver} is a structure defined by the USB
    core. Each USB device driver must instantiate it, and register
    itself to the USB core using this structure
  \item This structure inherits from \kstruct{device_driver},
    which is defined by the device model.
    \begin{block}{}
  \begin{minted}{c}
static struct usb_driver rtl8150_driver = {
    .name = "rtl8150",
    .probe = rtl8150_probe,
    .disconnect = rtl8150_disconnect,
    .id_table = rtl8150_table,
    .suspend = rtl8150_suspend,
    .resume = rtl8150_resume
};
  \end{minted}
  \end{block}
  \end{itemize}
\end{frame}

\begin{frame}[fragile]
  \frametitle{Driver registration and unregistration}
  \begin{itemize}
  \small
  \item When the driver is loaded / unloaded, it must register /
    unregister itself to / from the USB core
  \item Done using \kfunc{usb_register} and \kfunc{usb_deregister},
    provided by the USB core.
    \begin{block}{}
\begin{minted}[fontsize=\scriptsize]{c}
static int __init usb_rtl8150_init(void)
{
    return usb_register(&rtl8150_driver);
}

static void __exit usb_rtl8150_exit(void)
{
    usb_deregister(&rtl8150_driver);
}

module_init(usb_rtl8150_init);
module_exit(usb_rtl8150_exit);
\end{minted}
\end{block}
\item All this code is actually replaced by a call to the \kfunc{module_usb_driver} macro:
    \begin{block}{}
\begin{minted}[fontsize=\scriptsize]{c}
module_usb_driver(rtl8150_driver);
\end{minted}
\end{block}
  \end{itemize}
\end{frame}

\begin{frame}
  \frametitle{At Initialization}
  \begin{itemize}
  \item The USB adapter driver that corresponds to the USB controller
    of the system registers itself to the USB core
  \item The \ksym{rtl8150} USB device driver registers itself to the USB core
    \begin{center}
      \includegraphics[height=0.4\textheight]{slides/kernel-device-model/usb-registering.pdf}
    \end{center}
  \item The USB core now knows the association between the
    vendor/product IDs of \ksym{rtl8150} and the \kstruct{usb_driver} structure
    of this driver
  \end{itemize}
\end{frame}

\begin{frame}
  \frametitle{When a device is detected}
  \begin{center}
    \includegraphics[width=\textwidth]{slides/kernel-device-model/usb-detection.pdf}
  \end{center}
\end{frame}

\begin{frame}
  \frametitle{Probe Method}
  \begin{itemize}
  \item Invoked {\bf for each device} bound to a driver
  \item The \code{probe()} method receives as argument a structure
    describing the device, usually specialized by the bus
    infrastructure (\kstruct{pci_dev}, \kstruct{usb_interface}, etc.)
  \item This function is responsible for
    \begin{itemize}
    \item Initializing the device, mapping I/O memory, registering the
      interrupt handlers. The bus infrastructure provides methods to
      get the addresses, interrupt numbers and other device-specific
      information.
    \item Registering the device to the proper kernel framework, for
      example the network infrastructure.
    \end{itemize}
  \end{itemize}
\end{frame}

\begin{frame}[fragile]
\frametitle{Example: probe() and disconnect() methods}
\begin{columns}
  \column[t]{0.5\textwidth}
    \begin{block}{}
    \begin{minted}[fontsize=\tiny]{c}
static int rtl8150_probe(struct usb_interface *intf,
    const struct usb_device_id *id)
{
    rtl8150_t *dev;
    struct net_device *netdev;

    netdev = alloc_etherdev(sizeof(rtl8150_t));
    [...]
    dev = netdev_priv(netdev);
    tasklet_init(&dev->tl, rx_fixup, (unsigned long)dev);
    spin_lock_init(&dev->rx_pool_lock);
    [...]
    netdev->netdev_ops = &rtl8150_netdev_ops;
    alloc_all_urbs(dev);
    [...]
    usb_set_intfdata(intf, dev);
    SET_NETDEV_DEV(netdev, &intf->dev);
    register_netdev(netdev);

    return 0;
}
    \end{minted}
    \end{block}
  \column[t]{0.5\textwidth}
    \begin{block}{}
    \begin{minted}[fontsize=\tiny]{c}
static void rtl8150_disconnect(struct usb_interface *intf)
{
        rtl8150_t *dev = usb_get_intfdata(intf);

        usb_set_intfdata(intf, NULL);
        if (dev) {
                set_bit(RTL8150_UNPLUG, &dev->flags);
                tasklet_kill(&dev->tl);
                unregister_netdev(dev->netdev);
                unlink_all_urbs(dev);
                free_all_urbs(dev);
                free_skb_pool(dev);
                if (dev->rx_skb)
                        dev_kfree_skb(dev->rx_skb);
                kfree(dev->intr_buff);
                free_netdev(dev->netdev);
        }
}
    \end{minted}
    \end{block}
\end{columns}
\vfill
Source: \kfile{drivers/net/usb/rtl8150.c}
\end{frame}

\begin{frame}
  \frametitle{The Model is Recursive}
  \begin{center}
    \includegraphics[height=0.8\textheight]{slides/kernel-device-model/recursive-model.pdf}
  \end{center}
\end{frame}

\subsection{Platform drivers}

\begin{frame}{Platform devices}
  \begin{itemize}
  \item Amongst the non-discoverable devices, a huge family are the
    devices that are directly part of a system-on-chip: UART
    controllers, Ethernet controllers, SPI or I2C controllers, graphic
    or audio devices, etc.
  \item In the Linux kernel, a special bus, called the {\bf platform
      bus} has been created to handle such devices.
  \item It supports {\bf platform drivers} that handle {\bf platform
      devices}.
  \item It works like any other bus (USB, PCI), except that devices
    are enumerated statically instead of being discovered dynamically.
  \end{itemize}
\end{frame}

\begin{frame}[fragile]
  \frametitle{Implementation of a Platform Driver (1)}
  The driver implements a \kstruct{platform_driver}
  structure (example taken from \kfile{drivers/tty/serial/imx.c},
  simplified)
  \begin{block}{}
  \begin{minted}[fontsize=\scriptsize]{c}
static struct platform_driver serial_imx_driver = {
        .probe          = serial_imx_probe,
        .remove         = serial_imx_remove,
        .id_table       = imx_uart_devtype,
        .driver         = {
                .name   = "imx-uart",
                .of_match_table = imx_uart_dt_ids,
                .pm     = &imx_serial_port_pm_ops,
        },
};
\end{minted}
\end{block}
\end{frame}


\begin{frame}[fragile]
  \frametitle{Implementation of a Platform Driver (2)}
  ... and registers its driver to the platform driver infrastructure
  \begin{block}{}
  \begin{minted}[fontsize=\scriptsize]{c}
static int __init imx_serial_init(void) {
    return platform_driver_register(&serial_imx_driver);
}

static void __exit imx_serial_cleanup(void) {
    platform_driver_unregister(&serial_imx_driver);
}

module_init(imx_serial_init);
module_exit(imx_serial_cleanup);
  \end{minted}
\end{block}
Most drivers actually use the \kfunc{module_platform_driver}
macro when they do nothing special in \code{init()} and \code{exit()} functions:
  \begin{block}{}
  \begin{minted}[fontsize=\scriptsize]{c}
module_platform_driver(serial_imx_driver);
  \end{minted}
\end{block}
\end{frame}

\begin{frame}[fragile]
  \frametitle{Platform device instantiation}
  \begin{itemize}
  \item As platform devices cannot be detected dynamically, they are
    defined statically
    \begin{itemize}
    \item Legacy way: by direct instantiation of \kstruct{platform_device}
      structures, as done on a few old ARM platforms. The device was
      part of a list, and the list of devices was added to the system
      during board initialization.
    \item Current way: by parsing an "external" description, like a
      \emph{device tree} on most embedded platforms today, from which
      \kstruct{platform_device} structures are created.
    \end{itemize}
  \end{itemize}
\end{frame}

\begin{frame}
  \frametitle{Using additional hardware resources}
  \begin{itemize}
  \item Regular DT descriptions contain many information, including
    phandles (pointers) towards additional hardware blocks or hardware
    details which cannot be discovered.
    \begin{itemize}
    \item Some of them are available through a generic array of resources,
      like addresses for the I/O registers and IRQ lines:
      \begin{itemize}
      \item Such information can be represented using \kstruct{resource},
        and an array of \kstruct{resource} is associated to each
        \kstruct{platform_device}.
      \end{itemize}
    \item Common information/dependencies are parsed by the relevant
      subsystems, like clocks, GPIOs, or DMA channels:
      \begin{itemize}
      \item Each subsystem is responsible of instantiating its
        components, and offering an API to retrieve these objects and
        use them from device drivers.
      \end{itemize}
    \item Specific information might be directly be retrieved by
      device drivers, through (expensive) direct DT lookups (old drivers
      use \kstruct{platform_data}).
    \end{itemize}
  \item All these methods allow the same driver to be used with
    multiple devices functioning similarly, but with different
    addresses, IRQs, etc.
  \end{itemize}
\end{frame}

\begin{frame}[fragile]
  \frametitle{Using Resources}
  \begin{itemize}
  \item The platform driver has access to the resources:
    \begin{block}{}
  \begin{minted}[fontsize=\footnotesize]{c}
res = platform_get_resource(pdev, IORESOURCE_MEM, 0);
base = ioremap(res->start, PAGE_SIZE);
sport->rxirq = platform_get_irq(pdev, 0);
  \end{minted}
  \end{block}
  \item As well as the various common dependencies through individual
    APIs:
    \begin{itemize}
    \item \kfunc{clk_get}
    \item \kfunc{gpio_request}
    \item \kfunc{dma_request_channel}
    \end{itemize}
  \end{itemize}
\end{frame}

\begin{frame}[fragile]
  \frametitle{Driver data}
  \begin{itemize}
  \item In addition to the per-device resources and information, drivers
    may require driver-specific information to behave slighlty
    differently when different flavors of an IP block are driven by the
    same driver.
  \item A \code{const void *data} pointer can be used to store
    per-compatible specificities:
    \begin{block}{}
      \begin{minted}[fontsize=\footnotesize]{c}
static const struct of_device_id marvell_nfc_of_ids[] = {
        {
                .compatible = "marvell,armada-8k-nand-controller",
                .data = &marvell_armada_8k_nfc_caps,
        },
};
  \end{minted}
  \end{block}
  \item Which can be retrieved in the probe with:
    \begin{block}{}
  \begin{minted}[fontsize=\footnotesize]{c}
        /* Get NAND controller capabilities */
        if (pdev->id_entry) /* legacy way */
                nfc->caps = (void *)pdev->id_entry->driver_data;
        else /* current way */
                nfc->caps = of_device_get_match_data(&pdev->dev);
  \end{minted}
  \end{block}
  \end{itemize}
\end{frame}
