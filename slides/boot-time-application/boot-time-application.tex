\section{Optimizing applications}

\begin{frame}
\frametitle{Measuring: strace}
\begin{itemize}
	\item Allows to trace all the system calls made by an
              application and its children.
	\item Useful to:
	\begin{itemize}
		\item Understand how time is spent in user space
		\item For example, easy to find file open attempts (\code{open()}),
		      file access (\code{read()}, \code{write()}), and
		      memory allocations (\code{mmap2()}). Can be done
		      without any access to source code!
		\item Find the biggest time consumers
		      (low hanging fruit)
		\item Find unnecessary work done in applications
		      and scripts. Example: opening the same file(s)
		      multiple times, or trying to open files that
		      do not exist.
	\end{itemize}
	\item Limitation: you can't trace the \code{init} process!
\end{itemize}
\end{frame}

\input{strace.tex}
\input{ltrace.tex}
\input{valgrind.tex}

\begin{frame}[fragile]
\frametitle{perf}
\begin{itemize}
	\item Uses hardware performance counters, much faster than Valgrind!
	\item Need a kernel with \kconfig{CONFIG_PERF_EVENTS} and \kconfig{CONFIG_HW_PERF_EVENTS}
	\item User space tool: \code{perf}. It is part of the kernel
		sources so it is always in sync with your kernel.
	\item Usage:
	\begin{block}{}
\begin{verbatim}
perf record /my/command
\end{verbatim}
	\end{block}
	\item Get the results with:
	\begin{block}{}
\begin{verbatim}
perf report
\end{verbatim}
	\end{block}
        \item Note: advice to run \code{perf} on a filesystem built with
              glibc. Didn't manage to compile \code{perf} on a Musl
              root filesystem (Buildroot 2021.02 status). Once again, glibc is
              recommended for debugging.
\end{itemize}
\end{frame}

\begin{frame}[fragile]
\frametitle{perf report output}
\begin{block}{}
\tiny
\begin{verbatim}
# To display the perf.data header info, please use --header/--header-only options.
#
#
# Total Lost Samples: 0
#
# Samples: 5K of event 'cycles'
# Event count (approx.): 1392529663
#
# Overhead  Command  Shared Object             Symbol
# ........  .......  ........................  ....................................
#
    10.72%  ffmpeg   [kernel.kallsyms]         [k] video_get_user
    10.60%  ffmpeg   [kernel.kallsyms]         [k] vector_swi
     4.76%  ffmpeg   libc-2.31.so              [.] ioctl
     4.22%  ffmpeg   [kernel.kallsyms]         [k] __se_sys_ioctl
     3.81%  ffmpeg   [kernel.kallsyms]         [k] __video_do_ioctl
     3.42%  ffmpeg   libavformat.so.58.45.100  [.] avformat_find_stream_info
     2.83%  ffmpeg   [kernel.kallsyms]         [k] video_usercopy
     2.70%  ffmpeg   libc-2.31.so              [.] cfree
     2.58%  ffmpeg   [kernel.kallsyms]         [k] __fget_light
     2.53%  ffmpeg   libpthread-2.31.so        [.] __errno_location
     2.40%  ffmpeg   [kernel.kallsyms]         [k] arm_copy_from_user
     2.26%  ffmpeg   [kernel.kallsyms]         [k] memset
     2.09%  ffmpeg   [kernel.kallsyms]         [k] mutex_unlock
     2.06%  ffmpeg   [kernel.kallsyms]         [k] v4l2_ioctl
     2.05%  ffmpeg   libavcodec.so.58.91.100   [.] av_init_packet
     1.95%  ffmpeg   libc-2.31.so              [.] memset
...
\end{verbatim}
\end{block}
\end{frame}

\setuplabframe
{Optimizing the application}
{
\begin{itemize}
\item Compile the video player with just the features needed at run
      time.
\item Trace and profile the video player with \code{strace}
\item Observe size and time savings
\end{itemize}
}

