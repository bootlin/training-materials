\subsection{Linux Kernel: TTY}

\begin{frame}{Linux TTY subsystem introduction}
  \begin{itemize}
  \item The \textbf{TTY} subsystem handles teletypewriters to send/receive characters
  \item Source code located at \code{drivers/tty} in Linux
  \item Supports physical instances (e.g. UART, RS-232) and virtual ones
  \item Virtual terminals/consoles (VTs/VCs) associate a distinct keyboard and display
    \begin{itemize}
    \item Many VTs are created by Linux, available under: \code{/dev/tty*}
    \item Only a single VT is active at a time, switched with \code{Ctrl + Alt + Fi}
    \item Display grabbed using \textbf{fbcon} from the \textbf{fbdev} subsystem
    \item Keyboard grabbed using the \textbf{input} subsystem
    \item Can be used to show kernel messages (\code{console=tty1} in the cmdline)
    \item Every program runs under a controlling tty (given by the \code{tty} command)
    \end{itemize}
  \item Pseudo-terminals also exist, for software-based I/O only
    \begin{itemize}
    \item Created by programs (e.g. terminal emulator) under: \code{/dev/pts/*}
    \item Unrelated to graphics topics
    \end{itemize}
  \end{itemize}
\end{frame}

\begin{frame}{Linux TTY (illustrated)}
  \begin{center}
    \includegraphics[width=0.6\textwidth]{slides/graphics-software-linux-tty/getty.jpg}\\
    \textit{Getty running on \code{tty1} of a GNU/Linux system}\\
  \end{center}
\end{frame}

\begin{frame}[fragile]{Virtual terminals and graphics}
  \begin{itemize}
  \item With VTs, the kernel is \textbf{already using} the display and keyboard (seat)!
  \item Display servers need to switch to \textbf{graphics mode} to release the display:
  \begin{minted}[fontsize=\small]{console}
ret = ioctl(tty_fd, KDSETMODE, KD_GRAPHICS);
  \end{minted}
  \item And disable \textbf{keyboard support} on the standard input:
  \begin{minted}[fontsize=\small]{console}
ret = ioctl(tty_fd, KDSKBMODE, K_OFF);
  \end{minted}
  \item The display device can then be used \textbf{exclusively}
  \item Input is no longer interpreted (e.g. \code{Ctrl-C} is ignored)
  \item Graphics and keyboard mode must be restored when leaving to keep the VT usable
  \item Current modes can be queried with:
  \begin{minted}[fontsize=\small]{console}
short mode, kbmode;
ret = ioctl(tty_fd, KDGETMODE, &mode);
ret = ioctl(tty_fd, KDGKBMODE, &kbmode);
  \end{minted}
  \item More details in the \code{console_ioctl} man page
  \end{itemize}
\end{frame}

\begin{frame}[fragile]{Virtual terminals switching and graphics}
  \begin{itemize}
  \item However, the user might still want to switch VTs!
  \item So the display device must be \textbf{released/reacquired} for VT switching
  \item UNIX signals are used to notify the application, configured with:
  \begin{minted}[fontsize=\small]{console}
struct vt_mode vt_mode = { 0 };
vt_mode.mode = VT_PROCESS;
vt_mode.relsig = SIGUSR1;
vt_mode.acqsig = SIGUSR2;
ret = ioctl(tty_fd, VT_SETMODE, &vt_mode);
  \end{minted}
  \item VT switching must be acknowledged for the other VT to take over:
  \begin{minted}[fontsize=\small]{console}
ret = ioctl(tty_fd, VT_RELDISP, VT_ACKACQ); /* when entering VT */
ret = ioctl(tty_fd, VT_RELDISP, 1); /* when leaving VT */
  \end{minted}
  \item Failure to acknowledge will cause a \textbf{system hang}
  \item May be delegated to a deamon such as \code{seatd}
  \end{itemize}
\end{frame}
