\subsection{Fastboot}
\begin{frame}
  \frametitle{Definition}
  \begin{itemize}
  \item Fastboot is a protocol to communicate bootloaders over
    USB
  \item It is very simple to implement, making it easy to port on
    both new devices and on host systems
  \item Accessible with the \code{fastboot} command
  \end{itemize}
\end{frame}

\begin{frame}
  \frametitle{The Fastboot protocol}
  \begin{itemize}
  \item It is very restricted, only 10 commands and defined in the
    protocol specifications
  \item It is synchronous and driven by the host
  \item Allows to:
    \begin{itemize}
      \item Transmit data
      \item Flash the various partitions of the device
      \item Get variables from the bootloader
      \item Control the boot sequence
    \end{itemize}
  \item Specified in \code{fastboot_protocol.txt} in either
    \begin{itemize}
      \item bootable/bootloader/legacy (up to Android 4.1)
      \item system/core/fastboot (since Android 4.3)
    \end{itemize}
  \end{itemize}
\end{frame}

\begin{frame}
  \frametitle{Session example}
  \begin{center}
    \includegraphics[height=0.8\textheight]{slides/android-bootloaders-fastboot/fastboot.pdf}
  \end{center}
\end{frame}

\begin{frame}[fragile]{Booting into the bootloader}
  \begin{itemize}
    \item On a typical Android device you can boot into the bootloader by:
      \begin{itemize}
         \item powering on while pressing various buttons, the
           combination depends on your particular device.
         \item using \code{adb reboot bootloader}
      \end{itemize}
    \item Once the device has booted into the bootloader you can use
      the \code{fastboot} command on the development machine to
      communicate with it
  \end{itemize}
\end{frame}

\begin{frame}{fastboot commands (1/3)}
Basic commands
  \begin{description}
    \item[devices\hfill] \hfill \\
      List devices attached that will accept fastboot commands
    \item[getvar\hfill] \hfill \\
      Get a variable
    \item[continue\hfill] \hfill \\
      Continue the boot process as usual
    \item[reboot\hfill] \hfill \\
      Reboot device
    \item[reboot-bootloader] \hfill \\
      Reboot back into bootloader
  \end{description}
\end{frame}

\begin{frame}{fastboot commands (2/3)}
Flashing commands
\begin{description}
    \item[erase \code{<partition>}] \hfill \\
      Erase \code{<partition>}
    \item[flash \code{<partition>}] \hfill \\
      Erase and program \code{<partition>} with \code{<partition>.img}
      of current product
    \item[flash \code{<partition> <filename>}] \hfill \\
      Erase and program \code{<partition>} with \code{<filename>}
    \item[flashall\hfill] \hfill \\
      Erase and program \code{boot.img, recovery.img and system.img}
      of current product and then reboot
  \end{description}
  Where \code{<partition>} is one of \code{boot}, \code{recovery},
  \code{system}, \code{userdata}, \code{cache} current product is
  \code{$ANDROID_PRODUCT_OUT} \\
  Note: the location and size of partitions is hard-coded in the
  bootloader
\end{frame}

\begin{frame}[fragile]{fastboot commands (3/3)}
Special commands
  \begin{description}
    \item[oem \hfill] \hfill \\ Device-specific operations
    \item[boot \code{<kernel> <ramdisk>}] \hfill \\
      Load and boot \code{kernel} and \code{ramdisk}
  \end{description}
  Example:
  \footnotesize
  \begin{verbatim}
  fastboot -c "kernel command line" boot zImage ramdisk.cpio.gz
  \end{verbatim}
  \normalsize
\end{frame}

\begin{frame}{fastboot variables}
  The \code{getvar} command should return values for at least these
  \begin{description}[leftmargin=*,widest=a]
    \item[version] Version of the protocol: 0.4 is the one documented
    \item[version-bootloader] Version string of the Bootloader
    \item[version-baseband] Version string of the Baseband Software
    \item[product] Name of the product
    \item[serialno] Product serial number
    \item[secure] If "yes" the bootloader requires signed images
  \end{description}
  Vendor-specific variables must begin with an upper-case letter.
  Variables beginning with a lower-case letter are reserved for the
  Fastboot specifications and their evolution.
\end{frame}

\begin{frame}{Android boot and recovery images}
  \begin{itemize}
    \item The files \code{boot.img} and \code{recovery.img} are
      created by the tool \code{mkbootimg} (code in
      \code{system/core/mkbootimg}
    \item They contain a compressed kernel, the kernel command line
      and, optionally, a ramdisk in the Linux standard compressed cpio
      format
    \item Most Android bootloaders can read and load these images into
      memory
    \item The format is defined in \code{bootimg.h}
  \end{itemize}
\end{frame}

\begin{frame}{Android boot and recovery images}
  \begin{center}
  \includegraphics[width=1.0\textwidth]{slides/android-bootloaders-fastboot/boot-image-format.pdf}
  \end{center}
\end{frame}

\begin{frame}{Boot sequence}
  \begin{center}
  \includegraphics[width=0.8\textwidth]{slides/android-bootloaders-fastboot/boot-sequence.pdf}
  \end{center}
\end{frame}

