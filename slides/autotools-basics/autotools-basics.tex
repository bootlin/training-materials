\section{Autotools basics}

\begin{frame}{{\tt configure.ac} language}
  \begin{itemize}
  \item Really a shell script
  \item Processed through the \code{m4} preprocessor
  \item Language called \code{m4sh}
  \item Shell script augmented with special constructs for portability:
    \begin{itemize}
    \item \code{AS_IF} instead of shell \code{if ... then .. fi}
    \item \code{AS_CASE} instead of shell \code{case ... esac}
    \item etc.
    \end{itemize}
  \item {\em autoconf} provides a large set of {\em m4} macros to do
    most of the usual tests
  \item Make sure to quote arguments to macro with \code{[]}
  \end{itemize}
\end{frame}

\begin{frame}[fragile]{Minimal {\tt configure.ac}}
  \begin{block}{configure.ac}
  \begin{minted}{bash}
AC_INIT([hello], [1.0])
AC_OUTPUT
  \end{minted}
  \end{block}
  \begin{itemize}
  \item \code{AC_INIT}
    \begin{itemize}
    \item Every configure script must call \code{AC_INIT} before doing
      anything else that produces output.
    \item Process any command-line arguments and perform
      initialization and verification.
    \item Prototype:\\
      \code{AC_INIT (package, version, [bug-report], [tarname], [url])}
    \end{itemize}
  \item \code{AC_OUTPUT}
    \begin{itemize}
    \item Every \code{configure.ac}, should finish by calling
      \code{AC_OUTPUT}.
    \item Generates and runs config.status, which in turn creates the
      makefiles and any other files resulting from configuration.
    \end{itemize}
  \end{itemize}
\end{frame}

\begin{frame}[fragile]{Minimal {\tt configure.ac} example}

  \begin{block}{}
    \begin{minted}[fontsize=\scriptsize]{console}
$ cat configure.ac
AC_INIT([hello], [1.0])
AC_OUTPUT
$ ls
configure.ac
$ autoreconf -i
$ ls
autom4te.cache  configure  configure.ac
$ ./configure
configure: creating ./config.status
$ ls
autom4te.cache  config.log    config.status
configure       configure.ac
$ wc -l configure
2390 configure
\end{minted}
\end{block}

\end{frame}

\begin{frame}{Additional basic macros}
  \begin{itemize}
  \item \code{AC_PREREQ}
    \begin{itemize}
    \item Verifies that a recent enough version of {\em autoconf} is used
    \item \code{AC_PREREQ([2.68])}
    \end{itemize}
  \item \code{AC_CONFIG_SRCDIR}
    \begin{itemize}
    \item Gives the path to one source file in your project
    \item Allows {\em autoconf} to check that it is really where it
      should be
    \item \code{AC_CONFIG_SRCDIR([hello.c])}
    \end{itemize}
  \item \code{AC_CONFIG_AUX_DIR}
    \begin{itemize}
    \item Tells {\em autoconf} to put the auxiliary build tools it
      requires in a different directory, rather than the one of
      \code{configure.ac}
    \item Useful to keep cleaner build directory
    \end{itemize}
  \end{itemize}
\end{frame}

\begin{frame}{Checking for basic programs}
  \begin{itemize}
  \item \code{AC_PROG_CC}, makes sure a C compiler is available
  \item \code{AC_PROG_CXX}, makes sure a C++ compiler is available
  \item \code{AC_PROG_AWK}, \code{AC_PROG_GREP}, \code{AC_PROG_LEX},
    \code{AC_PROG_YACC}, etc.
  \end{itemize}
\end{frame}

\begin{frame}[fragile]{Checking for basic programs: example}

\begin{block}{configure.ac}
\begin{verbatim}
AC_INIT([hello], [1.0])
AC_PROG_CC
AC_OUTPUT
\end{verbatim}
\end{block}

\begin{block}{}
\begin{minted}[fontsize=\scriptsize]{console}
$ ./configure
checking for gcc... gcc
checking whether the C compiler works... yes
checking for C compiler default output file name... a.out
checking for suffix of executables...
checking whether we are cross compiling... no
checking for suffix of object files... o
checking whether we are using the GNU C compiler... yes
checking whether gcc accepts -g... yes
checking for gcc option to accept ISO C89... none needed
configure: creating ./config.status
\end{minted}
\end{block}

\end{frame}

\begin{frame}{{\tt AC\_CONFIG\_FILES}}
  \begin{itemize}
  \item \code{AC_CONFIG_FILES (file..., [cmds], [init-cmds])}
  \item Make \code{AC_OUTPUT} create each file by copying an input
    \code{file} (by default \code{file.in}), substituting the {\em
      output variable values}.
  \item Typically used to turn the Makefile templates
    \code{Makefile.in} files into final \code{Makefile}.
  \item Example:\\
    \code{AC_CONFIG_FILES([Makefile src/Makefile])}
  \item \code{cmds} and \code{init-cmds} are rarely used, see the {\em
      autoconf} documentation for details.
  \end{itemize}
\end{frame}

\begin{frame}{Output variables}
  \begin{itemize}
  \item {\em autoconf} will replace \code{@variable@} constructs by
    the appropriate values in files listed in \code{AC_CONFIG_FILES}
  \item Long list of standard variables replaced by {\em autoconf}
  \item Additional shell variables declared in \code{configure.ac} can
    be replaced using \code{AC_SUBST}
  \end{itemize}
\end{frame}

\begin{frame}[fragile]{{\tt AC\_CONFIG\_FILES} example (1/2)}

\begin{block}{configure.ac}
\begin{verbatim}
AC_INIT([hello], [1.0])
AC_PROG_CC
FOO=42
AC_SUBST([FOO])
AC_CONFIG_FILES([testfile])
AC_OUTPUT
\end{verbatim}
\end{block}

\begin{block}{testfile.in}
\begin{verbatim}
abs_builddir = @abs_builddir@
CC = @CC@
FOO = @FOO@
\end{verbatim}
\end{block}

\end{frame}

\begin{frame}[fragile]{{\tt AC\_CONFIG\_FILES} example (2/2)}

\begin{block}{Executing \code{./configure}}
\begin{minted}[fontsize=\scriptsize]{console}
/tmp/foo$ ./configure
checking for gcc... gcc
checking whether the C compiler works... yes
checking for C compiler default output file name... a.out
checking for suffix of executables... 
checking whether we are cross compiling... no
checking for suffix of object files... o
checking whether we are using the GNU C compiler... yes
checking whether gcc accepts -g... yes
checking for gcc option to accept ISO C89... none needed
configure: creating ./config.status
config.status: creating testfile
\end{minted}
\end{block}

\begin{block}{Generated \code{testfile}}
\begin{verbatim}
abs_builddir = /tmp/foo
CC = gcc
FOO = 42
\end{verbatim}
\end{block}

\end{frame}

\begin{frame}[fragile]{{\tt configure.ac}: a shell script}

\begin{itemize}
\item It is possible to include normal shell constructs in
  \code{configure.ac}
\item Beware to not use {\em bashisms}: use only POSIX compatible
  constructs
\end{itemize}

\begin{columns}
\column{0.4\textwidth}

\begin{block}{configure.ac}
{\tiny
\begin{verbatim}
AC_INIT([hello], [1.0])
echo "The value of CC is $CC"
AC_PROG_CC
echo "The value of CC is now $CC"
FOO=42
AC_SUBST([FOO])
if test $FOO -eq 42 ; then
   echo "The value of FOO is correct!"
fi
AC_CONFIG_FILES([testfile])
AC_OUTPUT
\end{verbatim}}
\end{block}

\column{0.6\textwidth}

\begin{block}{Running \code{./configure}}
\begin{minted}[fontsize=\tiny]{console}
The value of CC is 
checking for gcc... gcc
checking whether the C compiler works... yes
checking for C compiler default output file name... a.out
checking for suffix of executables... 
checking whether we are cross compiling... no
checking for suffix of object files... o
checking whether we are using the GNU C compiler... yes
checking whether gcc accepts -g... yes
checking for gcc option to accept ISO C89... none needed
The value of CC is now gcc
The value of FOO is correct!
configure: creating ./config.status
config.status: creating testfile
\end{minted}
\end{block}

\end{columns}

\end{frame}