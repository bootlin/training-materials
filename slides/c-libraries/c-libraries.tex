\begin{frame}
  \frametitle{glibc}
  \begin{columns}
    \column{0.7\textwidth}
    \begin{itemize}
    \item License: LGPL
    \item C library from the GNU project
    \item Designed for performance, standards compliance and portability
    \item Found on all GNU / Linux host systems
    \item Of course, actively maintained
    \item Quite big for small embedded systems: approx 2.5 MB on ARM
      (version 2.9 - \code{libc}: 1.5 MB, \code{libm}: 750 KB)
    \item \url{http://www.gnu.org/software/libc/}
    \end{itemize}
    \column{0.3\textwidth}
    \includegraphics[width=\textwidth]{slides/c-libraries/glibc.png}
  \end{columns}
\end{frame}

\begin{frame}
  \frametitle{uClibc}
  \begin{itemize}
  \item License: LGPL
  \item Lightweight C library for small embedded systems
    \begin{itemize}
    \item High configurability: many features can be enabled or
      disabled through a menuconfig interface
    \item Works only with Linux/uClinux, works on most embedded
      architectures
    \item No guaranteed binary compatibility. May need to
      recompile applications when the library configuration changes.
    \item Focus on size rather than performance
    \item Small compile time
    \end{itemize}
  \item \url{http://www.uclibc.org/}
  \end{itemize}
\end{frame}

\begin{frame}
  \frametitle{uClibc (2)}
  \begin{itemize}
  \item Most of the applications compile with uClibc. This applies to
    all applications used in embedded systems.
  \item Size (arm): 4 times smaller than glibc!
    \begin{itemize}
    \item uClibc 0.9.30.1: approx. 600 KB (libuClibc: 460 KB, libm:
      96KB)
    \item glibc 2.9: approx 2.5 MB
    \end{itemize}
  \item Some features not available or limited: priority-inheritance
    mutexes, NPTL support is very new, fixed Name Service Switch
    functionality, etc.
  \item Used on a large number of production embedded products,
    including consumer electronic devices
  \item Supported by all commercial embedded Linux providers (proof of
        maturity).
  \item Warning: though some development is still happening,
        the maintainers have stopped making releases since May 2012.
        The project is in trouble.
\end{itemize}
\end{frame}

\begin{frame}
  \frametitle{eglibc}
  \begin{columns}
    \column{0.8\textwidth}
    \begin{itemize}
    \item {\em Embedded glibc}, LGPL license too
    \item Was a variant of glibc, better adapted
      to the needs of embedded systems, because of past
      disagreement with glibc maintainers.
    \item eglibc's goals included reduced footprint, configurable
      components, better support for cross-compilation and
      cross-testing.
    \item Could be built without support for NIS, locales, IPv6, and many
      other features.
    \item Fortunately for eglibc, the glibc maintainer has changed
      and its features are now included in glibc. Version 2.19
      (Feb. 2014) was the last release.
    \item \url{http://en.wikipedia.org/wiki/Embedded_GLIBC}
    \end{itemize}
    \column{0.2\textwidth}
    \includegraphics[width=\textwidth]{slides/c-libraries/eglibc.png}
  \end{columns}
\end{frame}

\begin{frame}
  \frametitle{Honey, I shrunk the programs!}
  \begin{itemize}
  \item Executable size comparison on ARM, tested with {\em eglibc}
  2.15 and {\em uClibc} 0.9.33.2
  \item Plain ``hello world'' program (stripped): \\
    \begin{tabular}{| c || c | c |} \hline
    helloworld & static & dynamic \\ \hline
    {\em uClibc} & 18kB & 2.5kB \\
    {\em uClibc} with Thumb-2 & 14kB & 2.4kB \\
    {\em eglibc} with Thumb-2 & 361kB & 2.7kB \\ \hline
    \end{tabular} \\
  \item Busybox (stripped): \\
    \begin{tabular}{| c || c | c |} \hline
    busybox & static & dynamic \\ \hline
    {\em uClibc} & 750kB & 603kB \\
    {\em uClibc} with Thumb-2 & 533kB & 439kB \\
    {\em eglibc} with Thumb-2 & 934kB & 444kB \\ \hline
    \end{tabular}
  \end{itemize}
\end{frame}

\begin{frame}
  \frametitle{Other smaller C libraries}
  \begin{itemize}
  \item Several other smaller C libraries have been developed, but
    none of them have the goal of allowing the compilation of large
    existing applications
  \item They need specially written programs and applications
  \item Choices:
    \begin{itemize}
    \item Dietlibc, \url{http://www.fefe.de/dietlibc/}. Approximately
      70 KB.
    \item Newlib, \url{http://sourceware.org/newlib/}
    \item Klibc, \url{http://www.kernel.org/pub/linux/libs/klibc/},
      designed for use in an {\em initramfs} or {\em initrd} at boot
      time.
    \end{itemize}
  \end{itemize}
\end{frame}
