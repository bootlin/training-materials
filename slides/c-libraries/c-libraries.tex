\begin{frame}
  \frametitle{glibc}
  \begin{columns}
    \column{0.7\textwidth}
    \begin{itemize}
    \item License: LGPL
    \item C library from the GNU project
    \item Designed for performance, standards compliance and portability
    \item Found on all GNU / Linux host systems
    \item Of course, actively maintained
    \item By default, quite big for small embedded systems: approx 2.5 MB on ARM
      (version 2.9 - \code{libc}: 1.5 MB, \code{libm}: 750 KB)
    \item But some features not needed in embedded systems can be
          configured out (merged from the old {\em eglibc} project).
    \item \url{http://www.gnu.org/software/libc/}
    \end{itemize}
    \column{0.3\textwidth}
    \includegraphics[width=\textwidth]{slides/c-libraries/glibc.png}
  \end{columns}
\end{frame}

\begin{frame}
  \frametitle{uClibc-ng (1)}
  \begin{itemize}
  \item \url{http://uclibc-ng.org/}
  \item A continuation of the old uClibc project
  \item License: LGPL
  \item Lightweight C library for small embedded systems
    \begin{itemize}
    \item High configurability: many features can be enabled or
      disabled through a menuconfig interface
    \item Supports most embedded architectures
    \item Supports no-MMU architectures (ARM Cortex-M, Blackfin, etc.)
    \item No guaranteed binary compatibility. May need to
      recompile applications when the library configuration changes.
    \item Focus on size rather than performance
    \item Small compile time
    \end{itemize}
  \end{itemize}
\end{frame}

\begin{frame}
  \frametitle{uClibc-ng (2)}
  \begin{itemize}
  \item Most of the applications compile with uClibc-ng. This applies to
    all applications used in embedded systems.
  \item Size (arm): 3.5 times smaller than glibc!
    \begin{itemize}
    \item uClibc-ng 1.0.14: approx. 716kB (libuClibc: 282kB, libm:
      73kB)
    \item glibc 2.22: approx 2.5 MB
    \end{itemize}
  \item Some features not available or limited: priority-inheritance
    mutexes, fixed Name Service Switch functionality, etc.
  \item Used on a large number of production embedded products,
    including consumer electronic devices
\end{itemize}
\end{frame}

\begin{frame}
  \frametitle{Honey, I shrunk the programs!}
  \begin{itemize}
  \item Executable size comparison on ARM, tested with {\em glibc}
  2.22 and {\em uClibc-ng} 1.0.14
  \item Plain ``hello world'' program (stripped): \\
    \begin{tabular}{| c || c | c |} \hline
    helloworld & static & dynamic \\ \hline
    {\em uClibc} & 33.4kB & 2.5kB \\
    {\em uClibc} with Thumb-2 & 25.4kB & 2.5kB \\
    {\em eglibc} with Thumb-2 & 479kB & 2.7kB \\ \hline
    \end{tabular} \\
  \item Busybox (stripped): \\
    \begin{tabular}{| c || c | c |} \hline
    busybox & static & dynamic \\ \hline
    {\em uClibc} & 818kB & 664kB \\
    {\em uClibc} with Thumb-2 & 602kB & 504kB \\
    {\em eglibc} with Thumb-2 & 1206kB & 503kB \\ \hline
    \end{tabular}
  \end{itemize}
\end{frame}

\begin{frame}
  \frametitle{musl C library}
  \url{http://www.musl-libc.org/}
  \begin{itemize}
  \item A lightweight, fast and simple library for embedded systems
  \item Created while uClibc's development was stalled
  \item In particular, great at making small static executables
  \item Permissive license (MIT)
  \item Compare features with other C libraries:
    \url{http://www.etalabs.net/compare_libcs.html}
  \item Supported by build systems such as Buildroot
  \end{itemize}
\end{frame}

\begin{frame}
  \frametitle{Other smaller C libraries}
  \begin{itemize}
  \item Several other smaller C libraries have been developed, but
    none of them have the goal of allowing the compilation of large
    existing applications
  \item They can run only relatively simple programs,
	typically to make very small static executables and run
	in very small root filesystems.
  \item Choices:
    \begin{itemize}
    \item Dietlibc, \url{http://fefe.de/dietlibc/}. Approximately
      70 KB.
    \item Newlib, \url{http://sourceware.org/newlib/}
    \item Klibc, \url{http://www.kernel.org/pub/linux/libs/klibc/},
      designed for use in an {\em initramfs} or {\em initrd} at boot
      time.
    \end{itemize}
  \end{itemize}
\end{frame}
