\begin{frame}
  \frametitle{Porting the Linux kernel}
  \begin{itemize}
  \item The Linux kernel supports a lot of different CPU architectures
  \item Each of them is maintained by a different group of
    contributors
    \begin{itemize}
    \item See the \kfile{MAINTAINERS} file for details
    \end{itemize}
  \item The organization of the source code and the methods to port
    the Linux kernel to a new board are therefore very
    architecture-dependent
    \begin{itemize}
    \item For example, some architectures use the Device Tree, some do
      not.
    \end{itemize}
  \item This presentation is mainly focused on the ARM (32-bit) architecture
  \end{itemize}
\end{frame}

\begin{frame}
  \frametitle{Architecture, CPU and Machine}
  \begin{itemize}
  \item In the source tree, each architecture has its own directory
    \begin{itemize}
    \item \kdir{arch/arm} for the ARM 32-bit architecture
    \item \kdir{arch/arm64} for the ARM 64-bit architecture
    \end{itemize}
  \item The \kdir{arch/arm} directory contains generic ARM code
    \begin{itemize}
    \item \kreldir{arch/arm}{boot}, \kreldir{arch/arm}{common},
      \kreldir{arch/arm}{configs}, \kreldir{arch/arm}{kernel},
      \kreldir{arch/arm}{lib}, \kreldir{arch/arm}{mm},
      \kreldir{arch/arm}{nwfpe}, \kreldir{arch/arm}{vfp},
      \kreldir{arch/arm}{tools} and several others.
    \end{itemize}
  \item And many directories for different SoC families
    \begin{itemize}
    \item \code{mach-*} directories: \kreldir{arch/arm}{mach-pxa} for PXA SoCs,
      \kreldir{arch/arm}{mach-imx} for Freescale iMX SoCs, etc. They
      essentially contain:
      \begin{itemize}
      \item a small SoC description file
      \item power management code
      \item SMP code
      \end{itemize}
    \end{itemize}
  \item Some SoC families share some code, in directories named
    \code{plat-*}
  \item Device Tree source files are in \kdir{arch/arm/boot/dts}
  \end{itemize}
\end{frame}

\begin{frame}
  \frametitle{Before the Device Tree and ARM cleanup}
  \begin{itemize}
  \item Until 2011, the ARM architecture wasn't using the Device Tree,
    and a large portion of the SoC support was located in
    \code{arch/arm/mach-<soc>}.
  \item Each board supported by the kernel was associated to an unique
    {\em machine ID}.
  \item The entire list of {\em machine ID} can be downloaded at
    \url{https://www.arm.linux.org.uk/developer/machines/download.php}
    and one could freely register an additional one.
  \item The Linux kernel was defining a {\em machine structure} for
    each board, which associates the {\em machine ID} with a set of
    information and callbacks.
  \item The bootloader had to pass the {\em machine ID} to the kernel
    in a specific ARM register.
  \end{itemize}
  This way, the kernel knew what board it was booting on,
  and which init callbacks it had to execute.
\end{frame}

\begin{frame}
  \frametitle{The Device Tree and the ARM cleanup}
  \begin{itemize}
  \item As the ARM architecture gained significantly in popularity,
    some major refactoring was needed.
  \item First, the Device Tree was introduced on ARM: instead of using
    C code to describe SoCs and boards, a specialized language is
    used.
  \item Second, many driver infrastructures were created to replace
    custom code in \code{arch/arm/mach-<soc>}:
    \begin{itemize}
    \item The common clock framework in \kdir{drivers/clk}
    \item The pinctrl subsystem in \kdir{drivers/pinctrl}
    \item The irqchip subsystem in \kdir{drivers/irqchip}
    \item The clocksource subsystem in \kdir{drivers/clocksource}
    \end{itemize}
  \item The amount of code in \code{mach-<soc>} has now significantly
    reduced.
  \end{itemize}
\end{frame}

\begin{frame}
  \frametitle{Adding the support for a new ARM board}
  Provided the SoC used on your board is supported by the Linux kernel:
  \begin{enumerate}
  \item Create a {\em Device Tree} file in \kdir{arch/arm/boot/dts},
    generally named \code{<soc-name>-<board-name>.dts}, and make it
    include the relevant SoC \code{.dtsi} file.
    \begin{itemize}
    \item Your Device Tree will describe all the SoC peripherals that
      are enabled, the pin muxing, as well as all the devices on the
      board.
    \end{itemize}
  \item Modify \kfile{arch/arm/boot/dts/Makefile} to make sure your
    Device Tree gets built as a {\em DTB} during the kernel build.
  \item Tweak an existing configuration that matches your SoC and save
    it as \code{<board-name>_defconfig} in \kdir{arch/arm/configs}
  \item If needed, develop the missing device drivers for the devices
    that are on your board outside the SoC.
  \end{enumerate}
\end{frame}

\begin{frame}
  \frametitle{Studying the Crystalfontz CFA-10036 platform}
  After using a platform based on the AM335x processor from Texas
  Instruments, let's study another platform Bootlin
  has worked on specifically.
  \begin{columns}
    \column{0.7\textwidth}
    \begin{itemize}
    \item Crystalfontz CFA-10036
    \item Uses the Freescale iMX28 SoC, from the MXS family.
    \item 128MB of RAM
    \item 1 serial port, 1 LED
    \item 1 I2C bus, equipped with an OLED display
    \item 1 SD-Card slot
    \end{itemize}
    \column{0.3\textwidth}
    \includegraphics[width=\textwidth]{slides/kernel-porting-content/crystalfontz.jpg}
  \end{columns}
\end{frame}

\begin{frame}[fragile]
  \frametitle{Crystalfontz CFA-10036 Device Tree, header}
  \begin{itemize}
  \item SPDX license tag\\
  \item Mandatory Device Tree language definition\\
    \begin{block}{} \mint[fontsize=\small]{perl}+/dts-v1/;+ \end{block}
  \item Include the \code{.dtsi} file describing the SoC\\
    \begin{block}{}
      \mint[fontsize=\small]{perl}+#include "imx28.dtsi"+
    \end{block}
  \item Start the root of the tree (named \code{/}) then describe the board
    \begin{itemize}
    \item A human-readable string to describe the machine (shown at boot time)\\
      \begin{block}{}
        \mint[fontsize=\small]{perl}+model = "Crystalfontz CFA-10036 Board";+
      \end{block}
    \item A list of {\em compatible} strings, from the most specific one
      to the most general one. Mandatory to execute the right SoC specific
      initializations and board specific code.\\
      \begin{block}{}
        \mint[fontsize=\small]{perl}+compatible = "crystalfontz,cfa10036", "fsl,imx28";+
      \end{block}
    \end{itemize}
  \end{itemize}
\end{frame}

\begin{frame}[fragile]
  \frametitle{Crystalfontz CFA-10036, backbone}
  \begin{itemize}
  \item Definition of the buses and peripherals
    \begin{block}{}
      \begin{minted}[fontsize=\small]{perl}
/ {
    /* Define here 'standalone' peripherals and internal buses */
    memory {
        device_type = "memory";
        reg = <0x40000000 0x8000000>; /* 128 MB */
    };
    apb@80000000 {
        apbh@80000000 {
            /* Define apbh peripherals here */
            apbx@80040000 {
                /* Define apbx peripherals here */
            };
        };
    };
};
/* Reference here existing nodes with their labels */
      \end{minted}
    \end{block}
  \end{itemize}
\end{frame}

\begin{frame}[fragile]
  \frametitle{Crystalfontz CFA-10036 Device Tree, enable already
    described devices}
  \begin{itemize}
  \item The CFA-10036 has one debug UART. It is described in the iMX28
    DTSI file, so the corresponding controller should be referenced in
    the board DTS and enabled:
    \begin{block}{}
      \begin{minted}[fontsize=\small]{perl}
&duart {
    pinctrl-names = "default";
    pinctrl-0 = <&duart_pins_b>;
    status = "okay";
};
      \end{minted}
    \end{block}
    \item It also features an USB port which is described in the SoC
      DTSI but needs to be enabled:
      \begin{block}{}
        \begin{minted}[fontsize=\small]{perl}
&usb0 {
    pinctrl-names = "default";
    pinctrl-0 = <&usb0_otg_cfa10036>;
    status = "okay";
};
        \end{minted}
      \end{block}
  \end{itemize}
\end{frame}

\begin{frame}[fragile]
  \frametitle{Crystalfontz CFA-10036 Device Tree, fully describe
    additional devices}
  \begin{itemize}
  \item The I2C bus with a Solomon SSD1306 OLED display connected on
    it must be described entirely at the location where it belongs:
    \begin{block}{}
      \begin{minted}[fontsize=\tiny]{perl}
apbc@80040000 {
    i2c0: i2c@18000 { /* This means physical offset 0x80058000 */
        reg = <0x18000 0x1000>;
        pinctrl-names = "default";
        pinctrl-0 = <&i2c0_pins_b>;
        status = "okay";
        clock-frequency = <400000>;

        ssd1306: oled@3c {
            compatible = "solomon,ssd1306fb-i2c";
            pinctrl-names = "default";
            pinctrl-0 = <&ssd1306_cfa10036>;
            reg = <0x3c>;
            reset-gpios = <&gpio2 7 0>;
            solomon,height = <32>;
            solomon,width = <128>;
            solomon,page-offset = <0>;
        };
    };
      \end{minted}
    \end{block}
  \item Mind the display's pin configuration that has not yet been
    described
  \end{itemize}
\end{frame}

\begin{frame}[fragile]
  \frametitle{Crystalfontz CFA-10036 Device Tree, LEDs}
  \begin{itemize}
  \item One LED is connected to this platform, let's describe it as well
    \begin{block}{}
      \begin{minted}[fontsize=\small]{perl}
/ {
    leds {
        compatible = "gpio-leds";
        pinctrl-names = "default";
        pinctrl-0 = <&led_pins_cfa10036>;

        power {
            gpios = <&gpio3 4 1>;
            default-state = "on";
        };
    };
      \end{minted}
    \end{block}
  \item Also mind the pin configuration that we can define at any place
  \end{itemize}
\end{frame}

\begin{frame}[fragile]
  \frametitle{Crystalfontz CFA-10036 Device Tree, muxing}
  \begin{itemize}
  \item Definition of a few pins that will be muxed as GPIO, for LEDs and reset.
    \begin{block}{}
      \begin{minted}[fontsize=\tiny]{perl}
&pinctrl {
    ssd1306_cfa10036: ssd1306-10036@0 {
        reg = <0>;
        fsl,pinmux-ids = <0x2073>; /* MX28_PAD_SSP0_D7__GPIO_2_7 */
        fsl,drive-strength = <0>;
        fsl,voltage = <1>;
        fsl,pull-up = <0>;
    };

    led_pins_cfa10036: leds-10036@0 {
        reg = <0>;
        fsl,pinmux-ids = <0x3043>; /* MX28_PAD_AUART1_RX__GPIO_3_4 */
        fsl,drive-strength = <0>;
        fsl,voltage = <1>;
        fsl,pull-up = <0>;
    };
};
      \end{minted}
    \end{block}
  \end{itemize}
\end{frame}

\begin{frame}[fragile]
  \frametitle{Crystalfontz CFA-10036 Device Tree, Breakout Boards}
  \begin{itemize}
  \item The CFA-10036 can be plugged in other breakout boards, and the
    device tree also allows us to describe this, using includes. For
    example, the CFA-10057:
    \begin{block}{}
      \mint[fontsize=\small]{perl}+#include "imx28-cfa10036.dts"+
    \end{block}
  \item This allows to have a layered description. This can also be
    done for boards that have a lot in common, like the BeagleBone and
    the BeagleBone Black, or the AT91 SAMA5D3-based boards.
  \end{itemize}
\end{frame}

\begin{frame}[fragile]
  \frametitle{Crystalfontz CFA-10036: build the DTB}
  \begin{itemize}
  \item To ensure that the Device Tree Blob gets built for this board
    Device Tree Source, one need to ensure it is listed in
    \kfile{arch/arm/boot/dts/Makefile}:
    \begin{block}{}
      \begin{minted}{make}
dtb-$(CONFIG_ARCH_MXS) +=
        imx28-cfa10036.dtb \
        imx28-cfa10037.dtb \
        imx28-cfa10049.dtb \
        imx28-cfa10055.dtb \
        imx28-cfa10056.dtb \
        imx28-cfa10057.dtb \
        imx28-cfa10058.dtb \
        imx28-evk.dtb
      \end{minted}
    \end{block}
  \end{itemize}
\end{frame}

\begin{frame}
  \frametitle{Understanding the SoC support}
  \begin{itemize}
  \item Let's consider another ARM platform here for the kernel side of
    the support: the Marvell Armada 370/XP.
  \item For this platform, the core of the SoC support is located in
    \kdir{arch/arm/mach-mvebu}
  \item The \krelfile{arch/arm/mach-mvebu}{board-v7.c} file (see code on the next slide)
    contains the "{\em entry point}" of the SoC definition, the
    \code{DT_MACHINE_START} .. \code{MACHINE_END} definition:
    \begin{itemize}
    \item Defines the list of platform compatible strings that will
      match this platform, in this case
      \code{marvell,armada-370-xp}. This allows the kernel to know
      which \code{DT_MACHINE} structure to use depending on the DTB
      that is passed at boot time.
    \item Defines various callbacks for the platform initialization,
      the most important one being the \code{.init_machine} callback,
      running initialization code for the associated SoC.
    \end{itemize}
  \end{itemize}
\end{frame}

\begin{frame}[fragile]
  \frametitle{{\tt arch/arm/mach-mvebu/board-v7.c} (Linux 5.3)}
  \begin{block}{}
    \begin{minted}[fontsize=\tiny]{c}
static void __init mvebu_dt_init(void)
{
        if (of_machine_is_compatible("marvell,armadaxp"))
                i2c_quirk();
}

static void __init armada_370_xp_dt_fixup(void)
{
#ifdef CONFIG_SMP
        smp_set_ops(smp_ops(armada_xp_smp_ops));
#endif
}

static const char * const armada_370_xp_dt_compat[] __initconst = {
        "marvell,armada-370-xp",
        NULL,
};

DT_MACHINE_START(ARMADA_370_XP_DT, "Marvell Armada 370/XP (Device Tree)")
        .l2c_aux_val    = 0,
        .l2c_aux_mask   = ~0,
        .init_machine   = mvebu_dt_init,
        .init_irq       = mvebu_init_irq,
        .restart        = mvebu_restart,
        .reserve        = mvebu_memblock_reserve,
        .dt_compat      = armada_370_xp_dt_compat,
        .dt_fixup       = armada_370_xp_dt_fixup,
MACHINE_END
  \end{minted}
 \end{block}
\end{frame}

\begin{frame}
  \frametitle{Components of the minimal SoC support}
  The minimal SoC support consists of
  \footnotesize
  \begin{itemize}
  \item An SoC {\em entry point} file,
    \kfile{arch/arm/mach-mvebu/board-v7.c}
  \item At least one SoC \code{.dtsi} DT and one board \code{.dts} DT,
    in \kdir{arch/arm/boot/dts}
  \item An interrupt controller driver,
    \kfile{drivers/irqchip/irq-armada-370-xp.c}
  \item A timer driver,
    \kfile{drivers/clocksource/timer-armada-370-xp.c}
  \item An earlyprintk implementation to get early messages from the
    console, \kfile{arch/arm/Kconfig.debug} and
    \kdir{arch/arm/include/debug}
  \item A serial port driver in \kdir{drivers/tty/serial}. For Armada
    370/XP, the 8250 driver \kdir{drivers/tty/serial/8250} is used.
  \end{itemize}
  \normalsize
  This allows to boot a minimal system up to user space, using a root
  filesystem in {\em initramfs}.
\end{frame}

\begin{frame}
  \frametitle{Extending the minimal SoC support}

  Once the minimal SoC support is in place, the following core
  components should be added:
  \begin{itemize}
  \item Support for the clocks. Usually requires some clock drivers,
    as well as DT representations of the clocks. See
    \kdir{drivers/clk/mvebu} for Armada 370/XP clock drivers.
  \item Support for pin muxing, through the {\em pinctrl}
    subsystem. See \kdir{drivers/pinctrl/mvebu} for the Armada 370/XP
    drivers.
  \item Support for GPIOs, through the {\em GPIO} subsystem. See
    \kfile{drivers/gpio/gpio-mvebu.c} for the Armada 370/XP GPIO
    driver.
  \item Support for SMP, through \kstruct{smp_operations}. See
    \kfile{arch/arm/mach-mvebu/platsmp.c}.
  \end{itemize}
\end{frame}

\begin{frame}
  \frametitle{Adding controller drivers}
  Once the core pieces of the SoC support have been implemented, the
  remaining part is to add drivers for the different hardware blocks:
  \begin{itemize}
  \item Ethernet controller driver, in \kfile{drivers/net/ethernet/marvell/mvneta.c}
  \item SATA controller driver, in \kfile{drivers/ata/sata_mv.c}
  \item I2C controller driver, in \kfile{drivers/i2c/busses/i2c-mv64xxx.c}
  \item SPI controller driver, in \kfile{drivers/spi/spi-orion.c}
  \item PCIe controller driver, in \kfile{drivers/pci/controller/pci-mvebu.c}
  \item USB controller driver, in \kfile{drivers/usb/host/ehci-orion.c}
  \item etc.
  \end{itemize}
\end{frame}

\begin{frame}
  \frametitle{Porting the Linux kernel: further reading}
  \begin{columns}
    \column{0.5\textwidth}
    \begin{itemize}
    \item Gregory Clement, Your newer ARM64 SoC Linux support check-list!\\
          \url{https://bit.ly/2r8lHnE}
    \item Thomas Petazzoni, Your new ARM SoC Linux support check-list!\\
          \url{https://bit.ly/2ivqtDD}
    \item Our technical presentations on various kernel subsystems:\\
          \url{https://bootlin.com/docs/}
    \end{itemize}
    \column{0.5\textwidth}
    \includegraphics[height=0.35\textheight]{slides/kernel-porting-content/arm64-soc-support-checklist.png}\\
    \includegraphics[height=0.35\textheight]{slides/kernel-porting-content/arm-soc-support-checklist.png}
  \end{columns}
\end{frame}
