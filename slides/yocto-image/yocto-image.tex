\section{Images}
\subsection{Introduction to images}

\begin{frame}
  \frametitle{Overview 1/3}
  \begin{itemize}
    \item An \code{image} is the top level recipe and is used
      alongside the \code{machine} definition.
    \item Whereas the \code{machine} describes the hardware used and
      its capabilities, the \code{image} is architecture agnostic and
      defines how the root filesystem is built, with what packages.
    \item By default, several images are provided in Poky:
      \begin{itemize}
        \item \code{meta*/recipes*/images/*.bb}
      \end{itemize}
  \end{itemize}
\end{frame}

\begin{frame}
  \frametitle{Overview 2/3}
  \begin{itemize}
    \item Here are a few common images:
      \begin{description}
        \item[core-image-base] Console-only image, with full support
          of the hardware.
        \item[core-image-minimal] Small image, capable of booting a
          device.
        \item[core-image-minimal-dev] Small image with extra tools,
          suitable for development.
        \item[core-image-x11] Image with basic X11 support.
        \item[core-image-weston] Image with basic Wayland support.
        \item[core-image-rt] Like \code{core-image-minimal} with real time
          tools and test suite.
      \end{description}
  \end{itemize}
\end{frame}

\begin{frame}
  \frametitle{Overview 3/3}
  \begin{itemize}
    \item An \code{image} is no more than a recipe.
    \item It has a description, a license (optional) and inherits the
      \code{core-image} class.
  \end{itemize}
\end{frame}

\begin{frame}
  \frametitle{Organization of an image recipe}
  Some special configuration variables are used to describe an image:
      \begin{itemize}
        \item \yoctovar{IMAGE_BASENAME}: The name of the output image files.
          Defaults to \code{${PN}}.
        \item \yoctovar{IMAGE_INSTALL}: List of packages and package groups to
          install in the generated image (only toplevel packages, dependencies unnecessary)
        \item \yoctovar{IMAGE_ROOTFS_SIZE}: The final root filesystem size.
        \item \yoctovar{IMAGE_FEATURES}: List of features to enable in the
          image (e.g. \code{allow-root-login}).
        \item \yoctovar{IMAGE_FSTYPES}: List of formats the OpenEmbedded build
          system will use to create images. Could be set in machine
          definitions too (machine dependent).
        \item \yoctovar{IMAGE_LINGUAS}: List of the locales to be supported in
          the image.
        \item \yoctovar{IMAGE_PKGTYPE}: Package type used by the build system.
          One of \code{deb}, \code{rpm}, \code{ipk} and \code{tar}.
        \item \yoctovar{IMAGE_POSTPROCESS_COMMAND}: Shell commands to run at
          post process.
        \item \yoctovar{EXTRA_IMAGEDEPENDS}: Recipes to be built with the image, but
          which do not install anything in the root filesystem
          (e.g. the bootloader).
      \end{itemize}
\end{frame}

\begin{frame}[fragile]
  \frametitle{Example of an image}
  \begin{block}{}
    \begin{minted}{sh}
SUMMARY = "Example image"
IMAGE_INSTALL = "packagegroup-core-boot dropbear ninvaders"
IMAGE_LINGUAS = " "

inherit core-image
    \end{minted}
  \end{block}
  Note: unlike other recipes, image recipes don't need to set
  \yoctovar{LICENSE}.
\end{frame}

\begin{frame}
  \frametitle{Root filesystem generation}
  \begin{itemize}
    \item Image generation overview:
      \begin{enumerate}
        \item An empty directory is created for the root filesystem.
        \item Packages from \yoctovar{IMAGE_INSTALL} are installed into it
          using the package manager.
        \item One or more images files are created, depending on the
          \yoctovar{IMAGE_FSTYPES} value.
      \end{enumerate}
    \item Root filesystem creation is specific to the \yoctovar{IMAGE_PKGTYPE}
      value. It should be defined in the image recipe, otherwise the
      first valid package type defined in \yoctovar{PACKAGE_CLASSES} is
      used.
    \item All the magic is done in
      \code{meta/classes-recipe/rootfs_${IMAGE_PKGTYPE}.bbclass}
  \end{itemize}
\end{frame}

\subsection{Image types}

\begin{frame}
  \frametitle{\code{IMAGE_FSTYPES}}
  \begin{itemize}
    \item Configures the resulting root filesystem image format.
    \item If more than one format is specified, one image per format
      will be generated.
    \item Image formats instructions are provided by
      \code{openembedded-core}, in
      \code{meta/classes-recipe/image_types.bbclass}
    \item Common image formats are: \code{ext2}, \code{ext3}, \code{ext4},
          \code{squashfs}, \code{squashfs-xz}, \code{cpio}, \code{jffs2},
          \code{ubifs}, \code{tar.bz2}, \code{tar.gz}\dots
  \end{itemize}
\end{frame}

\begin{frame}
  \frametitle{Creating an image type}
  \begin{itemize}
    \item If you have a particular layout on your storage (for example
      bootloader location on an SD card), you may want to create your
      own image type.
    \item This is done through a class that inherits from
      \code{image_types}.
    \item It has to define a function named \code{IMAGE_CMD:<type>}.
    \item Append it to \yoctovar{IMAGE_TYPES}
  \end{itemize}
\end{frame}

\begin{frame}
  \frametitle{Creating an image conversion type}
  \begin{itemize}
    \item Common conversion types are: \code{gz}, \code{bz2},
          \code{sha256sum}, \code{bmap}\dots
    \item This is done through a class that inherits from
      \code{image_types}.
    \item It has to define a function named \yoctovar{CONVERSION_CMD}\code{:<type>}.
    \item Append it to \code{CONVERSIONTYPES}
    \item Append valid combinations to \yoctovar{IMAGE_TYPES}
  \end{itemize}
\end{frame}

\begin{frame}[fragile]
  \frametitle{wic}
  \begin{itemize}
    \item \code{wic} is a tool that can create a flashable image from
      the compiled packages and artifacts.
    \item It can create partitions (but doesn't support raw flash partitions
          and filesystems)
    \item It can select which files are located in
      which partition through the use of plugins.
    \item The final image layout is described in a \code{.wks} or
      \code{.wks.in} file.
    \item It can be extended in any layer.
    \item Usage example:
      \begin{block}{}
        \begin{minted}{sh}
WKS_FILE = "imx-uboot-custom.wks.in"
IMAGE_FSTYPES = "wic.bmap wic"
        \end{minted}
      \end{block}
    \item Note:
      \href{https://docs.yoctoproject.org/dev-manual/bmaptool.html}{bmaptool} is
      an alternative to \code{dd}, skipping uninitialized contents in
      partitions.
  \end{itemize}
\end{frame}

\begin{frame}[fragile]
  \frametitle{imx-uboot-custom.wks.in}
      \begin{block}{}
        \fontsize{7}{7}\selectfont
        \begin{minted}{sh}
part u-boot --source rawcopy --sourceparams="file=imx-boot" --no-table --align ${IMX_BOOT_SEEK}
part /boot --source bootimg-partition --use-uuid --fstype=vfat --label boot --active --align 8192 --size 64
part / --source rootfs --use-uuid --fstype=ext4 --label root --exclude-path=home/ --exclude-path=opt/ --align 8192
part /home --source rootfs --rootfs-dir=${IMAGE_ROOTFS}/home --use-uuid --fstype=ext4 --label home --align 8192
part /opt --source rootfs --rootfs-dir=${IMAGE_ROOTFS}/opt --use-uuid --fstype=ext4 --label opt --align 8192

bootloader --ptable msdos
        \end{minted}
      \end{block}
  \begin{itemize}
  \item Copies \code{imx-boot} from \code{$}\yoctovar{DEPLOY_DIR_IMAGE} in the
    image, aligned on (and so at that offset) \code{${IMX_BOOT_SEEK}}.
  \item Creates a first partition, formatted in FAT32, with the files
    listed in the \yoctovar{IMAGE_BOOT_FILES} variable.
  \item Creates an \code{ext4} partition with the contents on the root
    filesystem, excluding the content of \code{/home} and \code{/opt}
  \item Creates two \code{ext4} partitions, one with the content of
    \code{/home}, the other one with the content of \code{/opt}, from
    the image root filesystem.
  \end{itemize}
\end{frame}

\subsection{Package groups}

\begin{frame}
  \frametitle{Overview}
  \begin{itemize}
    \item Package groups are a way to group packages by functionality or
      common purpose.
    \item Package groups are used in image recipes to help building
      the list of packages to install.
    \item A package group is yet another recipe.
      \begin{itemize}
        \item Using the \code{packagegroup} class.
        \item The generated binary packages do not install any file, but
          they require other packages.
      \end{itemize}
    \item Be careful about the \yoctovar{PACKAGE_ARCH} value:
      \begin{itemize}
      \item Set to the value \code{all} by default,
      \item Must be explicitly set to \code{${MACHINE_ARCH}} when there is a machine
        dependency.
      \end{itemize}
  \end{itemize}
\end{frame}

\begin{frame}
  \frametitle{Common package groups}
  \begin{itemize}
    \item \code{packagegroup-base}
      \begin{itemize}
        \item Adds many core packages to the image based on
          \yoctovar{MACHINE_FEATURES} and \yoctovar{DISTRO_FEATURES}
      \end{itemize}
    \item \code{packagegroup-core-boot}
    \item \code{packagegroup-core-buildessential}
    \item \code{packagegroup-core-nfs-client}
    \item \code{packagegroup-core-nfs-server}
    \item \code{packagegroup-core-tools-debug}
    \item \code{packagegroup-core-tools-profile}
  \end{itemize}
\end{frame}

\begin{frame}[fragile]
  \frametitle{Example}
  \code{./meta/recipes-core/packagegroups/packagegroup-core-tools-debug.bb}:
  \begin{block}{}
    \begin{minted}{sh}
SUMMARY = "Debugging tools"

inherit packagegroup

RDEPENDS:${PN} = "\
    gdb \
    gdbserver \
    strace"
    \end{minted}
  \end{block}
\end{frame}
