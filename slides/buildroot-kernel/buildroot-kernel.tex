\section{Managing the Linux kernel configuration}

\begin{frame}{Introduction}
  \begin{itemize}
  \item The Linux kernel itself uses {\em kconfig} to define its
    configuration
  \item Buildroot cannot replicate all Linux kernel configuration
    options in its \code{menuconfig}
  \item Defining the Linux kernel configuration therefore needs to be
    done in a special way.
  \item Note: while described with the example of the Linux kernel,
    this discussion is also valid for other packages using {\em
      kconfig}: \code{barebox}, \code{uclibc}, \code{busybox} and
    \code{u-boot}.
  \end{itemize}
\end{frame}

\begin{frame}{Defining the configuration}
  \begin{itemize}
  \item In the \code{Kernel} menu in \code{menuconfig}, after
    selecting the kernel version, you have two options to define the
    kernel configuration:
    \begin{itemize}
    \item \code{Use a defconfig}
      \begin{itemize}
      \item Will use a {\em defconfig} provided within the kernel
        sources
      \item Available in \code{arch/<ARCH>/configs} in the kernel
        sources
      \item Used unmodified by Buildroot
      \item Good starting point
      \end{itemize}
    \item \code{Use a custom config file}
      \begin{itemize}
      \item Allows to give the path to either a full \code{.config},
        or a minimal {\em defconfig}
      \item Usually what you will use, so that you can have a custom
        configuration
      \end{itemize}
    \item \code{Additional fragments}
      \begin{itemize}
      \item Also to pass a list of configuration file fragments.
      \item They can complement or override configuration options
        specified in a {\em defconfig} or a full configuration file.
      \end{itemize}
    \end{itemize}
  \end{itemize}
\end{frame}

\begin{frame}{Changing the configuration}
  \begin{itemize}
  \item Running one of the Linux kernel configuration interfaces:
    \begin{itemize}
    \item \code{make linux-menuconfig}
    \item \code{make linux-nconfig}
    \item \code{make linux-xconfig}
    \item \code{make linux-gconfig}
    \end{itemize}
  \item Will load either the defined kernel {\em defconfig} or custom
    configuration file, and start the corresponding Linux kernel
    configuration interface.
  \item Changes made are only made in
    \code{$(O)/build/linux-<version>/}, i.e. they are not preserved
    across a clean rebuild.
  \item To save them:
    \begin{itemize}
    \item \code{make linux-update-config}, to save a full config file
    \item \code{make linux-update-defconfig}, to save a minimal
      defconfig
    \item Only works if a {\em custom configuration file} is used
    \end{itemize}
  \end{itemize}
\end{frame}

\begin{frame}{Typical flow}
  \begin{enumerate}
  \item \code{make menuconfig}
    \begin{itemize}
    \item Start with a {\em defconfig} from the kernel, say \code{mvebu_v7_defconfig}
    \end{itemize}
  \item Run \code{make linux-menuconfig} to customize the
    configuration
  \item Do the build, test, tweak the configuration as needed.
  \item You cannot do \code{make linux-update-{config,defconfig}},
    since the Buildroot configuration points to a kernel {\em
      defconfig}
  \item \code{make menuconfig}
    \begin{itemize}
    \item Change to a custom configuration file. There's no need for
      the file to exist, it will be created by Buildroot.
    \end{itemize}
  \item \code{make linux-update-defconfig}
    \begin{itemize}
    \item Will create your custom configuration file, as a minimal
      {\em defconfig}
    \end{itemize}
  \end{enumerate}
\end{frame}
