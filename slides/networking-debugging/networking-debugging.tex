\section{Debugging and tracing the Network Stack}

\begin{frame}{Challenges}
	\begin{itemize}
		\item Latency issues : Can come from different locations
			\begin{itemize}
				\item The network itself
				\item Internal queueing and buffering
				\item Hardware and OS-level latencies
			\end{itemize}
		\item Throughput issues
			\begin{itemize}
				\item May depend on the traffic type
				\item May simply be a symptom : TCP retransmissions due to bad L1 link quality
			\end{itemize}
		\item Link issues
			\begin{itemize}
				\item May be hardware related
				\item The kernel can't only tell you what it knows
			\end{itemize}
	\end{itemize}
\end{frame}

\begin{frame}[fragile]{Monitoring traffic and drops}
	\begin{itemize}
		\item In case of large number of repeats or local drops
		\item dropwatch : monitor the in-kernel packet drops, see \manpage{dropwatch}{1}
			\begin{minted}{bash}
$ dropwatch -lkas
$ dropwatch> start
2 drops at ip6_mc_input+1a8 (0xffffffff83347ba8) [software]
			\end{minted}
		\item retis : eBPF based monitoring. monitor drops as well as \code{skb} lifetime
		\item See the \href{https://retis.readthedocs.io/en/stable/}{official documentation}
			\begin{minted}{bash}
retis collect -c skb-drop --stack
			\end{minted}
	\end{itemize}
\end{frame}

\begin{frame}{Monitoring traffic with a capture}
	\begin{itemize}
		\item Tools such as wireshark and tcpdump use \code{AF_PACKET} sockets for monitoring 
		\item Some hardware devices may include extra information
			\begin{itemize}
				\item \textit{e.g.} the \textbf{radiotap} headers on 802.11 frames
			\end{itemize}
		\item The monitoring happens between the \textbf{driver} and \code{tc}
		\item On the receive side, it happens before firewalling with netfilter
		\item Capture format is standardized with the \code{pcap} format
			\begin{itemize}
				\item frames can be replayed, or analyzed on another host
			\end{itemize}
	\end{itemize}
\end{frame}

\begin{frame}{Offloading}
	\begin{itemize}
		\item Offloading issues are hard to troubleshoot, as the host doesn't see them
		\item \code{wireshark} may tell you \textbf{receive checksumming} issues
		\item \code{ethtool -k eth0} shows you the \textbf{features}
		\item Hardware counters should be used for debugging :
			\begin{itemize}
				\item \code{ethtool -S eth0}
				\item \code{ethtool --phy-statistics eth0}
				\item \code{ethtool -S eth0 --groups eth-mac|eth-phy|eth-ctrl|rmon}
			\end{itemize}
		\item Some information may be available in \textbf{debugfs}
		\item \code{ip -s link show eth0} shows software counters
		\item \code{cat /proc/interrupts} indicate the hardware interrupt counters
		\item \code{cat /proc/softirq} indicate the softirq counters
	\end{itemize}
\end{frame}

\begin{frame}{Traffic generation}
	\begin{itemize}
		\item iperf3 : Throughput testing, fairly simple to use
		\item netperf : Made by kernel developers, similar to iperf3, more featureful
		\item scapy : Traffic generator written in python, craft arbitrary frames
		\item DPDK's \href{https://pktgen-dpdk.readthedocs.io/en/latest/}{pktgen}
			\begin{itemize}
				\item PKTgen uses \textbf{kernel bypass}, not supported on every platform
				\item Allows very fast packet crafting (10gbs)
				\item Useful for testing multi-flow setups, or HW offloading
			\end{itemize}
	\end{itemize}
\end{frame}

\begin{frame}{iperf3}
	\begin{itemize}
		\item Widely used traffic generator
		\item \code{iperf3 -s -D} : Start in server mode
		\item \code{iperf3 -c 192.168.1.1} : Start in client mode, default is TCP
		\item \code{iperf3 -c 192.168.1.1 -u -b 0} : UDP mode, with unlimited bandwidth
		\item \code{iperf3 -c 192.168.1.1 -u -b 0 -l 100} : UDP mode, small packets
		\item \code{iperf3 -c 192.168.1.1 -P 16} : Multi-flow mode
	\end{itemize}
\end{frame}

\begin{frame}[fragile]{scapy}
	\begin{itemize}
		\item Traffic generator written in python. See the \href{https://scapy.net/}{official website}
		\item Allows generating arbitrary traffic very easily
		\item Each header can be crafted, for high flexibility
		\item Very easily scriptable
	\end{itemize}
	\begin{minted}{python}
# IPv4 with ToS field varying between 1 and 4
sendp(Ether()/IP(dst="1.2.3.4",tos=(1,4)), iface="eth0")

# Raw ethernet frame
sendp(Ether(dst="00:51:82:11:22:02"), iface="eth0") 

# Send and wait for reply, simple ping implementation
packet = IP(dst="192.168.42.1", ttl=20)/ICMP()
reply = sr1(packet) 	\end{minted}
\end{frame}

\begin{frame}{Counters}
	\begin{itemize}
		\item Layer 4 counters : Maintained by the kernel.
			\begin{itemize}
				\item \code{netstat -s} or \code{cat /proc/net/netstat}
				\item More statistics in \code{/proc/net/stat}
			\end{itemize}
		\item Layer 3 counters : Provided by \code{ip -s link show}
		\item Layer 2 counters : Hardware-provided
		\item XDP programs can be custom-written to gather specific statistics
		\item \code{xdp-monitor -s} tracks XDP statistics such as the number of drops and redirects
	\end{itemize}
\end{frame}

\begin{frame}{profiling}
	\begin{itemize}
		\item Allows identifying the bottlenecks in software
		\item \code{perf} can be used : Rely on hardware and software counters
		\item \code{Flamegraphs} help identify software bottlenecks, in kernel and userspace
			\begin{itemize}
				\item Covered in our \href{https://bootlin.com/training/debugging/}{debugging training}
			\end{itemize}
		\item \code{ftrace} will generate timelines of events, to track \textbf{latencies}
		\item See \href{https://docs.kernel.org/trace/ftrace.html}{the ftrace documentation}
	\end{itemize}
\end{frame}

% Existing infrastructure
% Selftests
% Profiling and performance issues
% Traffic issues and drops
% TCPdump, wireshark
% Link issues
