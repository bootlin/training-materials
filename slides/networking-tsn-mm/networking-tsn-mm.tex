\subsection{Mac Merge and Frame Preemption}

\begin{frame}{Frame preemption}
	\begin{itemize}
		\item At low speeds, it takes a while to transmit a full frame (123 microseconds at 100Mbps)
		\item Frame Preemption allows stopping the emission of a non-important frame
		\item A high-priority frame can then be sent, followed by the rest of the original frame
		\item Frame fragments have a different \textbf{Start of Frame Delimiter}
			\begin{itemize}
				\item This requires both link-partners to support FP, which can be negociated
				\item Incompatible link-partners will consider fragments as malformed frames
			\end{itemize}
	\end{itemize}
\end{frame}

\begin{frame}{ethtool MAC merge}
	\begin{itemize}
		\item The 802.3 standard defines 2 virtual MACs used for Frame Preemption
		\item The \textbf{eMAC} is the \textbf{express MAC}, high-priority traffic is sent through it
		\item The \textbf{pMAC} is the \textbf{preemptible MAC}, low-priority traffic is sent through it
		\item \code{ethtool --set-mm eth0 pman-enabled on tx-enabled on}
		\item Requires hardware support
		\item Steering between \code{eMAC} and \code{pMAC} is done using \code{tc mqprio} or \code{tc taprio}
	\end{itemize}
\end{frame}

\begin{frame}[fragile]{Frame Preemption in mqprio and taprio}
	\begin{itemize}
		\item Can only be used if hardware supports it and \code{pMAC} is enabled
		\item The \code{fp} parameter indicates if the class is express (E) or preemptible (P)
	\end{itemize}
	\begin{block}{mqprio example}
		\begin{minted}{bash}
tc qdisc add dev eth0 handle 100: parent root mqprio num_tc 4 \
map 0 0 0 0 1 1 1 1 2 2 2 2 3 3 3 3 queues 1@0 1@1 1@2 1@3 \
fp P P P E \ # Class 3 is express, classes 0, 1 and 2 are preemptible
hw 1 # offload to hardware
		\end{minted}
	\end{block}
\end{frame}
% MM
% FP

