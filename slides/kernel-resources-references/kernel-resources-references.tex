\begin{frame}
  Linux Weekly News
  \frametitle{Kernel Development News}
  \begin{itemize}
  \item \url{http://lwn.net/}
  \item The weekly digest off all Linux and free software
    information sources
  \item In depth technical discussions about the kernel
  \item Subscribe to finance the editors (\$7 / month)
  \item Articles available for non subscribers after 1 week.
  \end{itemize}
\end{frame}

\begin{frame}
  \frametitle{Useful Reading (1)}
  \begin{columns}
    \column{0.7\textwidth}
    Essential Linux Device Drivers, April 2008
    \begin{itemize}
    \item \url{http://elinuxdd.com/}
    \item By Sreekrishnan Venkateswaran, an embedded IBM engineer
      with more than 10 years of experience
    \item Covers a wide range of topics not covered by LDD: serial
      drivers, input drivers, I2C, PCMCIA and Compact Flash, PCI,
      USB, video drivers, audio drivers, block drivers, network
      drivers, Bluetooth, IrDA, MTD, drivers in user space, kernel
      debugging, etc.
    \item \emph{Probably the most wide ranging and complete Linux
          device driver book I've read} -- Alan Cox
    \end{itemize}
    \column{0.3\textwidth}
    \includegraphics[width=\textwidth]{slides/kernel-resources-references/eldd.jpg}
  \end{columns}
\end{frame}

\begin{frame}
  \frametitle{Useful Reading (2)}
  \begin{columns}
    \column{0.7\textwidth}
    \begin{itemize}
    \item Linux Kernel Development, 3rd Edition, Jun 2010
      \begin{itemize}
      \item Robert Love, Novell Press
      \item \url{http://free-electrons.com/redir/lkd3-book.html}
      \item A very synthetic and pleasant way to learn about kernel
        subsystems (beyond the needs of device driver writers)
      \end{itemize}
    \item The Linux Programming Interface, Oct 2010
      \begin{itemize}
      \item Michael Kerrisk, No Starch Press
      \item \url{http://man7.org/tlpi/}
      \item A gold mine about the kernel interface and how to use it
      \end{itemize}
    \end{itemize}
    \column{0.3\textwidth}
    \begin{center}
      \includegraphics[height=0.4\textheight]{slides/kernel-resources-references/linux-kernel-development.jpg}\\
      \includegraphics[height=0.4\textheight]{slides/kernel-resources-references/linux-programming-interface.png}
    \end{center}
  \end{columns}
\end{frame}

\begin{frame}
  \frametitle{Useful Online Resources}
  \begin{itemize}
  \item Kernel documentation
    \begin{itemize}
    \item \url{https://kernel.org/doc/}
    \end{itemize}
  \item Linux kernel mailing list FAQ
    \begin{itemize}
    \item \url{http://vger.kernel.org/lkml/}
    \item Complete Linux kernel FAQ
    \item Read this before asking a question to the mailing list
    \end{itemize}
  \item Kernel Newbies
    \begin{itemize}
    \item \url{http://kernelnewbies.org/}
    \item Glossary, articles, presentations, HOWTOs, recommended
      reading, useful tools for people getting familiar with Linux
      kernel or driver development.
    \end{itemize}
  \item Kernel glossary
    \begin{itemize}
    \item \url{http://kernelnewbies.org/KernelGlossary}
    \end{itemize}
\end{itemize}

\end{frame}

\begin{frame}
  \frametitle{International Conferences}
  \begin{itemize}
  \item Embedded Linux Conference:
    \includegraphics[width=0.35\textwidth]{slides/kernel-resources-references/elc-logo.png}\\
    \url{http://embeddedlinuxconference.com/}
    \begin{itemize}
    \item Organized by the Linux Foundation in the USA (February-April)
          and in Europe (October-November)
    \item Very interesting kernel and user space topics for embedded
      systems developers.
    \item Presentation slides and videos freely available
    \end{itemize}
  \item Linux Plumbers: \url{http://linuxplumbersconf.org}
    \begin{itemize}
    \item Conference on the low-level plumbing of Linux: kernel,
      audio, power management, device management, multimedia, etc.
    \end{itemize}
  \item linux.conf.au: \url{http://linux.org.au/conf/}
    \begin{itemize}
    \item In Australia / New Zealand
    \item Features a few presentations by key kernel hackers.
    \end{itemize}
  \end{itemize}
\end{frame}

\begin{frame}
  \frametitle{Continue to learn after the course}
  Here are a few suggestions:
  \begin{itemize}
  \item Run your labs again on your own hardware. The Nunchuk lab should
        be rather straightforward, but the serial lab will be quite different
	if you use a different processor.
  \item Help with tasks keeping the kernel code clean and up-to-date:\\
	\url{http://kernelnewbies.org/KernelJanitors}
  \item Propose fixes for issues reported by the {\em Coccinelle} tool:\\
	\code{make coccicheck}
  \item Learn by reading the kernel code by yourself, ask questions and
	propose improvements.
  \item Implement and share drivers for your own hardware, of course!
  \end{itemize}
\end{frame}
