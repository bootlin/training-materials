\subsection{Display Devices}

\begin{frame}{Visual display technologies generalities}
  \begin{itemize}
  \item Pixel data is pushed from the display interface to a visible surface\\
  \textit{using a dedicated controller on the display device}
  \item Pixels are split into 3 color cells (R-G-B)
    \begin{itemize}
    \item The human eye naturally merges light from the 3 cells
    \end{itemize}
  \item Pixel frames are displayed as (physical) arrays of color cells
  \item Smooth sequences require at least 24 fps, more is usually best
  \end{itemize}~

  \begin{minipage}[b]{0.45\textwidth}
    \centering
    \includegraphics[height=6.5em]{slides/graphics-hardware-display-devices/pixel-array.jpg}\\
    \textit{\small Pixel color cells on a LCD TN panel}
  \end{minipage}
  \hfill
  \begin{minipage}[b]{0.45\textwidth}
    \centering
    \includegraphics[height=6.5em]{slides/graphics-hardware-display-devices/pixel-array-text.jpg}\\
    \textit{\small A pixel array displaying text}
  \end{minipage}
\end{frame}

\begin{frame}{CRT display technology}
  \begin{itemize}
  \item Color \textbf{cathode-ray tubes} (CRTs), since the 1950s:
    \begin{itemize}
    \item Using electron beams to excite a phosphorescent screen
    \item Beams are guided by magnetic deflection
    \item One beam for each color with increased intensity for increased luminosity
    \item High energy consumption
    \item High contrast, low response time (\(1-10~\mu s\))
    \item Other issues: monitor size, burn-in (screensavers), remnant magnetic field (degaussing), high voltages and magnetic fields
    \end{itemize}
  \end{itemize}

  \begin{center}
  \includegraphics[height=8em]{slides/graphics-hardware-display-devices/crt-color.png}
  \end{center}
\end{frame}

\begin{frame}{Plasma display panels technology}
  \begin{itemize}
  \item \textbf{Plasma display panels} (PDPs), since the 1990s-2000s:
    \begin{itemize}
    \item Using gas cells brought to plasma state to strike light-emitting phosphor
    \item Flat array of cells, scales to large surfaces
    \item Medium energy consumption (depends on luminance)
    \item High contrast, low response time (\(\leq 1~\mu s\))
    \item Other issues: burn-in
    \item Gradually being \textbf{deprecated} in favor of other flat-panel technologies
    \end{itemize}
  \end{itemize}
\end{frame}

\begin{frame}{LCD display technology}
  \begin{itemize}
  \item \textbf{Liquid crystal displays} (LCDs) using \textbf{Thin-film-transistors} (TFT):
    \begin{itemize}
    \item Using the electrically-controlled alignment of crystal structures to block light
    \item Does not emit light: needs an always-on backlight source (usually LEDs)
    \item Low energy consumption (depends on backlight)
    \item Medium to low contrast, medium response time (\(1-10~ms\))
    \item \textbf{Twisted nematic} (TN): limited color quality and viewing angles, since the 1980s
    \item \textbf{In-plane switching} (IPS): improved color and viewing angles, since the 2000s
    \end{itemize}
  \end{itemize}

  \begin{center}
  \includegraphics[height=6em]{slides/graphics-hardware-display-devices/lcd-ips-shape.jpg}\\
  \textit{\small Chevron shapes that improve the viewing angle on IPS LCDs}
  \end{center}
\end{frame}

\begin{frame}{OLED display technology}
  \begin{itemize}
  \item \textbf{Organic light-emitting diodes} (OLEDs), since 2010:
    \begin{itemize}
    \item Using organic compounds (carbon-based) to emit light as R-G-B LEDs
    \item Allows flat and flexible surfaces, with a large viewing angle
    \item Low energy consumption
    \item Very high contrast, low response time (\(1-10~\mu s\))
    \item Issues: burn-in, independent cells aging, affected by UV light
    \item Rapidly becoming \textbf{very popular} and used
    \end{itemize}
  \end{itemize}

  \begin{center}
  \includegraphics[height=8em]{slides/graphics-hardware-display-devices/oled-display.jpg}\\
  \textit{\small A flexible OLED display panel}
  \end{center}
\end{frame}

\begin{frame}{EPD display technology}
  \begin{itemize}
  \item \textbf{Electrophoretic displays} (EPDs), since the 2000s:
    \begin{itemize}
    \item Using black and white electrically-charged ink-like particles in oil\\
    \textit{e.g. positive charge for black and negative for white}
    \item Electric fields attract one or the other color with current flow\\
    \textit{the particles stay in place after they were moved}
    \item Using incident light, does not emit light itself
    \item Very low consumption (only for changes)
    \item Very high response time (\(100~ms\)) and ghosting
    \end{itemize}
  \end{itemize}~

  \begin{minipage}[b]{0.45\textwidth}
    \centering
    \includegraphics[height=6.5em]{slides/graphics-hardware-display-devices/e-reader.jpg}\\
    \textit{\small An e-reader with an EPD display}
  \end{minipage}
  \hfill
  \begin{minipage}[b]{0.45\textwidth}
    \centering
    \includegraphics[height=6.5em]{slides/graphics-hardware-display-devices/epd-detail.jpg}\\
    \textit{\small Detail of an EPD display}
  \end{minipage}
\end{frame}

\begin{frame}{Display panels integration and monitors}
  \begin{itemize}
  \item Panels come with a dedicated \textbf{controller} to:
    \begin{itemize}
    \item Decode pixels from the display interface
    \item Electrically control the color cells of the panel
    \end{itemize}
  \item Panels can be used \textbf{standalone}, usually with:
    \begin{itemize}
    \item A single simple display interface
    \item No standardized connector, weak and short cables
    \item Only native dimensions supported and little to no configuration
    \end{itemize}
  \item Or they can be \textbf{integrated in monitors}, usually with:
    \begin{itemize}
    \item More complex and multiple display interfaces
    \item Standardized connectors, external cables
    \item Configuration (buttons and UI overlay), multiple dimensions
    \end{itemize}
  \end{itemize}
\end{frame}

\begin{frame}{Display panels integration and monitors (illustrated)}
  \begin{center}
  \includegraphics[width=0.5\textwidth]{slides/graphics-hardware-display-devices/display-panel.jpg}\\
  \textit{\small A display panel used standalone with an embedded board}
  \end{center}
\end{frame}
