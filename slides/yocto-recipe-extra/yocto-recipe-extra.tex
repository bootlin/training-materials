\section{Writing recipes - going further}

\subsection{The per-recipe sysroot}

\begin{frame}{Sysroot}
  \begin{itemize}
    \item The {\em sysroot} is the the logical root directory for headers
      and libraries
    \item Where {\em gcc} looks for headers, and {\em ld} looks for
      libraries
    \item Contains:
      \begin{itemize}
        \item The kernel headers
        \item The C library and headers
        \item Other libraries and their headers
      \end{itemize}
  \end{itemize}
\end{frame}

\begin{frame}{Per-recipe sysroot}
  \begin{itemize}
    \item Instead of a global sysroot, \code{bitbake} implements a {\em
      per-recipe sysroot}
    \item Before the actual build, each recipe prepares its own sysroot
      \begin{itemize}
        \item Contains libraries and headers {\em only} for the recipes it
          \code{DEPENDS} on
        \item Ensures the configuration stage will not detect libraries not
          explicitly listed in \code{DEPENDS} but already built for other
          reasons
        \item \code{${WORKDIR}/recipe-sysroot} for target recipes
        \item \code{${WORKDIR}/recipe-sysroot-native} for native recipes
      \end{itemize}
    \item At the end of the build, each recipe produces its destination
      sysroot
      \begin{itemize}
        \item Its own slice of sysroot, with the libraries and headers it
          directly provides
        \item Used as input for other recipes to generate their
          recipe-sysroot
        \item \code{${WORKDIR}/sysroot-destdir}
      \end{itemize}
  \end{itemize}
\end{frame}

\begin{frame}{Per-recipe sysroot}
  \begin{center}
    \includegraphics[width=1.00\textwidth]{slides/yocto-recipe-extra/per-recipe-sysroot.pdf}
  \end{center}
\end{frame}

\begin{frame}{The complete sysroot}
  \begin{itemize}
  \item A complete sysroot is available:
    \begin{itemize}
      \item For each image
        \begin{itemize}
          \item In \code{${WORKDIR}/recipe-sysroot} just like any recipe
        \end{itemize}
      \item In the SDK
        \begin{itemize}
          \item Covered later
        \end{itemize}
    \end{itemize}
  \end{itemize}
\end{frame}

\subsection{Using Python code in metadata}

\begin{frame}
  \frametitle{Tasks in Python}
  \begin{itemize}
    \item Tasks can be written in Python when using the \code{python}
      keyword.
    \item Two modules are automatically imported:
      \begin{itemize}
        \item \code{bb}: to access \code{bitbake}'s internal functions.
        \item \code{os}: Python's operating system interfaces.
      \end{itemize}
    \item You can import other modules using the \code{import}
      keyword.
    \item Anonymous Python functions are executed during parsing.
      \item Short Python code snippets can be written inline with the {\tt
        \$\{@<python code>\}} syntax.
  \end{itemize}
\end{frame}

\begin{frame}
  \frametitle{Accessing the datastore with Python}
  \begin{itemize}
    \item The \code{d} variable is accessible within Python tasks.
    \item \code{d} represents the \code{bitbake} datastore (where variables are
      stored).
  \end{itemize}
  \begin{description}
    \item[\code{d.getVar("X", expand=False)}] Returns the value of
      \code{X}.
    \item[\code{d.setVar("X", "value")}] Set \code{X}.
    \item[\code{d.appendVar("X", "value")}] Append \code{value} to
      \code{X}.
    \item[\code{d.prependVar("X", "value")}] Prepend \code{value} to
      \code{X}.
    \item[\code{d.expand(expression)}] Expand variables in
      \code{expression}.
  \end{description}
\end{frame}

\begin{frame}[fragile]
  \frametitle{Python code examples}
  \begin{minted}{python}
# Anonymous function
python __anonymous() {
    if d.getVar("FOO", True) == "example":
        d.setVar("BAR", "Hello, World.")
}
# Task
python do_settime() {
    import time
    d.setVar("TIME", time.strftime('%Y%m%d', time.gmtime()))
}
  \end{minted}
  \begin{minted}{sh}
# Inline
do_install() {
    echo "Build OS: ${@os.uname()[0].lower()}"
}

  \end{minted}
\end{frame}

\subsection{Variable flags}

\begin{frame}[fragile]
  \frametitle{Variable flags}
  \begin{itemize}
    \item {\em Variable flags}, or {\em varflags}, are used to store extra
      information on tasks and variables.
    \item They are used to control task functionalities.
    \item A typical example:
  \begin{minted}{sh}
SRC_URI[md5sum] = "97b2c3fb082241ab5c56ab728522622b"
  \end{minted}
  \item See
    \href{https://docs.yoctoproject.org/bitbake/2.2/bitbake-user-manual/bitbake-user-manual-metadata.html#variable-flags}{the list of varflags supported by bitbake}.
  \item More varflags can be added freely.
  \end{itemize}
\end{frame}

\begin{frame}[fragile]
  \frametitle{Examples}
  \begin{itemize}
    \item \code{dirs}: directories that should be created before the task
      runs. The last one becomes the work directory for the task.
      \begin{minted}{sh}
do_compile[dirs] = "${B}"
      \end{minted}
    \item \code{noexec}: disable the execution of the task.
      \begin{minted}{sh}
do_settime[noexec] = "1"
      \end{minted}
    \item \code{nostamp}: do not create a {\em stamp} file when running the
      task. The task will always be executed.
      \begin{minted}{sh}
do_menuconfig[nostamp] = "1"
      \end{minted}
    \item \code{doc}: task documentation displayed by {\em listtasks}.
      \begin{minted}{sh}
do_settime[doc] = "Set the current time in ${TIME}"
      \end{minted}
    \item \code{depends}: add a dependency between specific tasks
      \begin{minted}{sh}
do_patch[depends] = "quilt-native:do_populate_sysroot"
      \end{minted}
  \end{itemize}
\end{frame}

\subsection{Packages features}

\begin{frame}
  \frametitle{Benefits}
  \begin{itemize}
    \item Features can be built depending on the needs.
    \item This allows to avoid compiling all features in a software
      component when only a few are required.
    \item A good example is \code{ConnMan}: Bluetooth support
      is built only if there is Bluetooth on the target.
    \item The \code{PACKAGECONFIG} variable is used to configure the
      build on a per feature granularity, for packages.
  \end{itemize}
\end{frame}

\begin{frame}
  \frametitle{\code{PACKAGECONFIG}}
  \begin{itemize}
    \item \code{PACKAGECONFIG} takes the list of features to enable.
    \item \code{PACKAGECONFIG[feature]} takes up to six arguments,
      separated by commas:
      \begin{enumerate}
        \item Argument used by the configuration task if the feature
          is enabled (\code{EXTRA_OECONF}).
        \item Argument added to \code{EXTRA_OECONF} if the feature is
          disabled.
        \item Additional build dependency (\code{DEPENDS}), if enabled.
        \item Additional runtime dependency (\code{RDEPENDS}), if enabled.
        \item Additional runtime recommendations (\code{RRECOMMENDS}), if enabled.
	\item Any conflicting \code{PACKAGECONFIG} settings for this feature.
      \end{enumerate}
    \item Unused arguments can be omitted or left blank.
  \end{itemize}
\end{frame}

\begin{frame}[fragile]
  \frametitle{Example: from \code{ConnMan}}
  \begin{block}{}
  \begin{minted}{sh}
PACKAGECONFIG ??= "wifi openvpn"

PACKAGECONFIG[wifi] = "--enable-wifi,                 \
                       --disable-wifi,                \
                       wpa-supplicant,                \
                       wpa-supplicant"
PACKAGECONFIG[bluez] = "--enable-bluetooth,           \
                        --disable-bluetooth,          \
                        bluez5,                       \
                        bluez5"
PACKAGECONFIG[openvpn] = "--enable-openvpn,           \
                          --disable-openvpn,          \
                          ,                           \
                          openvpn"
  \end{minted}
  \end{block}
\end{frame}

\begin{frame}[fragile]
  \frametitle{Enabling PACKAGECONFIG features}
  \begin{itemize}
    \item In a \code{.bbappend} of the recipe, just append to \code{PACKAGECONFIG}
      \begin{block}{}
        \begin{minted}{sh}
PACKAGECONFIG:append = " <feature>"
PACKAGECONFIG:append = " tui"
        \end{minted}
      \end{block}

    \item In a config file (e.g. distro conf)
      \begin{block}{}
        \begin{minted}{sh}
PACKAGECONFIG:append:pn-<recipename> = " <feature>"
PACKAGECONFIG:append:pn-gdb = " tui"
        \end{minted}
      \end{block}
  \end{itemize}
\end{frame}

\begin{frame}[fragile]
  \frametitle{Inspecting available \code{PACKAGECONFIG} flags}
  \begin{itemize}
  \item \code{${POKY_DIR}/scripts/contrib/list-packageconfig-flags.py}
    shows the \code{PACKAGECONFIG} varflags available for each recipe:

      \begin{block}{}
        \begin{minted}[fontsize=\scriptsize]{sh}
$ ../poky/scripts/contrib/list-packageconfig-flags.py
RECIPE NAME    PACKAGECONFIG FLAGS
==================================
alsa-plugins   aaf jack libav maemo-plugin maemo-resource-manager pulseaudio samplerate speexdsp
connman        3g bluez client iptables l2tp nfc nftables openvpn pptp systemd tist vpnc wifi ...
gdb            babeltrace debuginfod python readline tui xz
...
        \end{minted}
      \end{block}

    \item The \code{-a} flag shows all the details:
      \begin{block}{}
        \begin{minted}[fontsize=\scriptsize]{sh}
$ ../poky/scripts/contrib/list-packageconfig-flags.py -a
connman-1.41
/home/murray/w/yocto-stm32-labs/poky/meta/recipes-connectivity/connman/connman_1.41.bb
PACKAGECONFIG wispr iptables client                   3g wifi                    bluez
PACKAGECONFIG[wifi] --enable-wifi, --disable-wifi, wpa-supplicant, wpa-supplicant
PACKAGECONFIG[bluez] --enable-bluetooth, --disable-bluetooth, bluez5, bluez5
PACKAGECONFIG[openvpn] --enable-openvpn --with-openvpn=${sbindir}/openvpn,--disable-openvpn,,openvpn
...
        \end{minted}
      \end{block}
  \end{itemize}
\end{frame}

\subsection{Conditional features}

\begin{frame}[fragile]
  \frametitle{Conditional features}
  \begin{itemize}
    \item Some values can be set dynamically, thanks to a set of
      functions:
    \item \code{bb.utils.contains(variable, checkval, trueval,
      falseval, d)}: if \code{checkval} is found in
      \code{variable}, \code{trueval} is returned; otherwise
      \code{falseval} is used.
    \item \code{bb.utils.filter(variable, checkvalues, d)}: returns
      all the words in the variable that are present in the
      checkvalues.
    \item Example:
      \begin{block}{}
      \fontsize{9}{9}\selectfont
      \begin{minted}{sh}
PACKAGECONFIG ??= "wispr iptables client\
                   ${@bb.utils.filter('DISTRO_FEATURES', '3g systemd wifi', d)} \
                   ${@bb.utils.contains('DISTRO_FEATURES', 'bluetooth', 'bluez', '', d)} \
"
      \end{minted}
      \end{block}
  \end{itemize}
\end{frame}

\subsection{Package splitting}

\begin{frame}{Package splitting}
  \begin{center}
    \includegraphics[width=0.95\textwidth]{slides/yocto-recipe-extra/splitting-packages.pdf}
  \end{center}
\end{frame}

\begin{frame}{Package splitting}
  \begin{itemize}
    \item \code{do_install} copies {\em all} files in the \code{D} directory
      (\tt \$\{WORKDIR\}/image).
    \item \code{do_package} splits files in several packages in \tt
      \$\{WORKDIR\}/packages-split
      \begin{itemize}
        \item based on the \code{PACKAGES} and \code{FILES} variables.
      \end{itemize}
    \item \code{do_package_write_rpm} generates RPM packages
  \end{itemize}
\end{frame}

\begin{frame}[fragile]
  \frametitle{\code{PACKAGES}}
  \begin{itemize}
    \item \code{PACKAGES} lists the packages to be built:
    \begin{block}{}
    \begin{minted}[fontsize=\small]{sh}
PACKAGES = "${PN}-src ${PN}-dbg ${PN}-staticdev ${PN}-dev \
    ${PN}-doc ${PN}-locale ${PACKAGE_BEFORE_PN} ${PN}"
    \end{minted}
    \end{block}
    \item More packages can be added to the default list
      \begin{itemize}
        \item Useful when a single remote repository provides multiple
          binaries or libraries.
        \item The order matters. \code{PACKAGE_BEFORE_PN} allows to pick
          files normally included in the default package in another.
      \end{itemize}
    \item \code{PACKAGES_DYNAMIC} allows to check dependencies with
      optional packages are satisfied.
    \item \code{ALLOW_EMPTY} allows to produce a package even if it is empty.
    \item To prevent configuration files from being overwritten during the
      Package Management System update process, use \code{CONFFILES}.
  \end{itemize}
\end{frame}

\begin{frame}[fragile]
  \frametitle{\code{FILES}}
  \begin{itemize}
    \item For each package a \code{FILES} variable lists the files to
      include.
    \item It must be package specific (e.g. with \code{:${PN}}, \code{:${PN}-dev}, dots).
    \item Defaults from \code{meta/conf/bitbake.conf}:
    \begin{block}{}
    \begin{minted}[fontsize=\small]{sh}
FILES:${PN}-dev = \
    "${includedir} ${FILES_SOLIBSDEV} ${libdir}/*.la \
     ${libdir}/*.o ${libdir}/pkgconfig ${datadir}/pkgconfig \
     ${datadir}/aclocal ${base_libdir}/*.o \
     ${libdir}/${BPN}/*.la ${base_libdir}/*.la \
     ${libdir}/cmake ${datadir}/cmake"
FILES:${PN}-dbg = \
    "/usr/lib/debug /usr/lib/debug-static \
     /usr/src/debug"
    \end{minted}
    \end{block}
  \end{itemize}
\end{frame}

\begin{frame}[fragile]
  \frametitle{\code{FILES}: the main package}
  \begin{itemize}
    \item The package named just \code{${PN}} is the one that gets
      installed in the root filesystem.
    \item In Poky, defaults to:
  \end{itemize}
  \begin{block}{}
    \begin{minted}[fontsize=\small]{sh}
FILES:${PN} = \
    "${bindir}/* ${sbindir}/* ${libexecdir}/* ${libdir}/lib*${SOLIBS} \
     ${sysconfdir} ${sharedstatedir} ${localstatedir} \
     ${base_bindir}/* ${base_sbindir}/* \
     ${base_libdir}/*${SOLIBS} \
     ${base_prefix}/lib/udev/rules.d ${prefix}/lib/udev/rules.d \
     ${datadir}/${BPN} ${libdir}/${BPN}/* \
     ${datadir}/pixmaps ${datadir}/applications \
     ${datadir}/idl ${datadir}/omf ${datadir}/sounds \
     ${libdir}/bonobo/servers"
    \end{minted}
  \end{block}
\end{frame}

\begin{frame}[fragile]
  \frametitle{Example}
  \begin{itemize}
    \item The \code{kexec tools} provides \code{kexec} and \code{kdump}:
    \begin{block}{}
    \begin{minted}{sh}
require kexec-tools.inc
export LDFLAGS = "-L${STAGING_LIBDIR}"
EXTRA_OECONF = " --with-zlib=yes"

SRC_URI[md5sum] = "b9f2a3ba0ba9c78625ee7a50532500d8"

PACKAGES =+ "kexec kdump"

FILES:kexec = "${sbindir}/kexec"
FILES:kdump = "${sbindir}/kdump"
    \end{minted}
    \end{block}
  \end{itemize}
\end{frame}

\begin{frame}[fragile]
  \frametitle{Inspecting packages}
  \code{oe-pkgdata-util} is a tool that can help inspecting
  packages:
  \begin{itemize}
  \item Which package is shipping a file:
    \begin{block}{}
      \begin{minted}[fontsize=\scriptsize]{sh}
$ oe-pkgdata-util find-path /bin/busybox
busybox: /bin/busybox
    \end{minted}
    \end{block}
  \item Which files are shipped by a package:
    \begin{block}{}
      \begin{minted}[fontsize=\scriptsize]{sh}
$ oe-pkgdata-util list-pkg-files busybox
busybox:
    /bin/busybox
    /bin/busybox.nosuid
    /bin/busybox.suid
    /bin/sh
    \end{minted}
    \end{block}
  \item Which recipe is creating a package:
    \begin{block}{}
      \begin{minted}[fontsize=\scriptsize]{sh}
$ oe-pkgdata-util lookup-recipe kdump
kexec-tools
$ oe-pkgdata-util lookup-recipe libtinfo5
ncurses
    \end{minted}
    \end{block}
\end{itemize}
\end{frame}

\subsection{Dependencies in detail}

\begin{frame}{\code{DEPENDS}}
  \begin{itemize}
    \item \code{DEPENDS} describes a build-time dependency
    \item Typical case: a program needs the library and headers files from
      a library to be configured and/or built
    \item In other words: it needs the library in its {\em sysroot}
    \item In \code{ninvaders.bb}, the line\\
      \code{DEPENDS = "ncurses"}\\
      creates a dependency
      \begin{itemize}
        \item Of \code{ninvaders.do_prepare_recipe_sysroot}
        \item On \code{ncurses.do_populate_sysroot}
      \end{itemize}
  \end{itemize}
\end{frame}

\begin{frame}{\code{RDEPENDS}}
  \begin{itemize}
    \item \code{RDEPENDS} describes a runtime dependency
    \item Typical case: a program uses another program at runtime via
      sockets, DBUS, etc, or simply executes it
    \item It does not need it at build time
    \item In \code{inetutils_2.4.bb}, the line\\
      \code{RDEPENDS:${PN}-ftpd += "xinetd"}\\
      creates a dependency
      \begin{itemize}
      \item Of \code{inetutils.do_build}
      \item On \code{xinetd.do_package_write_rpm}
      \end{itemize}
    \item And adds in the inetutils-ftpd RPM package a dependency on
      the xinetd RPM package
  \end{itemize}
\end{frame}

\begin{frame}{\code{RRECOMMENDS}}
  \begin{itemize}
    \item \code{RRECOMMENDS} is similar to \code{RDEPENDS}
    \item But if the dependency package is not built it will just be
      skipped instead of failing the build
    \item Typical cases:
      \begin{itemize}
        \item A package extends the features of a program, but its build
          has been disabled explicitly (e.g. via \code{BAD_RECOMMENDATIONS})
        \item Depending on a kernel module that might also be built-in in
          the kernel Image
      \end{itemize}
    \item In \code{watchdog_5.16.bb}, the line\\
      \code{RRECOMMENDS:${PN} += "kernel-module-softdog"}\\
      does nothing if the \code{softdog} kernel module is not built by
      the kernel (could be builtin)
  \end{itemize}
\end{frame}
