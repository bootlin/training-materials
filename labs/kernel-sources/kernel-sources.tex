\subchapter{Kernel source code}{Objective: Get familiar with the
  kernel source code}

After this lab, you will be able to:

\begin{itemize}

\item Get your own copy of the mainline Linux kernel source tree
\item Explore the sources in search for files, function headers or
  other kinds of information...
\item Browse the kernel sources with tools like \code{cscope} and LXR.
\end{itemize}

\section{Setup}

Go to the \code{$HOME/felabs/linux/modules} directory.

First, install required software packages:

\begin{verbatim}
sudo apt-get install git gitk git-email
\end{verbatim}

\section{Cloning the mainline Linux tree}

To begin working with the Linux kernel sources, we need to clone its
reference git tree, the one managed by Linus Torvalds.

The trouble is you have to download about 1.5 GB of data!

If you are running this command from home, or if you have very fast
access to the Internet at work (and if you are not 256 participants in the
training room), you can do it directly by connecting to
\url{http://git.kernel.org}:

{\small
\begin{verbatim}
git clone git://git.kernel.org/pub/scm/linux/kernel/git/torvalds/linux.git
\end{verbatim}
}

or if the network port for \code{git} is blocked by the corporate
firewall, you can use the \code{http} protocol as a less efficient
fallback:

{\small
\begin{verbatim}
git clone http://git.kernel.org/pub/scm/linux/kernel/git/torvalds/linux.git 
\end{verbatim}
}

If Internet access is not fast enough and if multiple people have to
share it, your instructor will give you a USB flash drive with a
\code{tar.xz} archive of a recently cloned Linux source tree.

You will just have to extract this archive in the current directory,
and then pull the most recent changes over the network:

\begin{verbatim}
tar Jxf linux-git.tar.xz
cd linux
git pull
\end{verbatim}

Of course, if you directly ran \code{git clone}, you won't have run 
\code{git pull} today. You may run \code{git pull} every morning though,
or at least every time you need an update of the upstream source tree.

\section{Get familiar with the sources}

As a Linux kernel user, you will very often need to find which file
implements a given function. So, it is useful to be familiar with
exploring the kernel sources.

\begin{enumerate}
\item Find the Linux logo image in the sources
\item Find who the maintainer of the 3C505 network driver is.
\item Find the declaration of the \code{platform_device_register()} function.
\end{enumerate}

Tip: if you need the \code{grep} command, we advise you to use \code{git
grep}. This command is similar, but much faster, doing the search only
on the files managed by git (ignoring git internal files and generated
files). 

\section{Use a kernel source indexing tool}

Now that you know how to do things in a manual way, let's use more
automated tools.

Try LXR (Linux Cross Reference) at \url{http://lxr.free-electrons.com}
and choose the Linux version closest to yours.

If you don't have Internet access, you can use \code{cscope} instead.

As in the previous section, use this tool to find where
the \code{platform_device_register()} is declared, implemented and
even used.
