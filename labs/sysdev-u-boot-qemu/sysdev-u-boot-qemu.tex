\subchapter{Bootloader - U-Boot}{Objectives: Compile and install the U-Boot bootloader,
  use basic U-Boot commands, set up TFTP communication with the development
  workstation.}

\section{Setup}

Go to the \code{$HOME/__SESSION_NAME__-labs/bootloader} directory.

Install the \code{qemu-system-arm} package. In this lab and in the
following ones, we will use a QEMU emulated ARM Vexpress Cortex A9 board.

\section{Configuring U-Boot}

Download U-Boot v2020.04.

First apply a patch that fixes the \code{editenv} command in U-Boot:
\begin{bashinput}
$ cd u-boot-2020.04
$ cat ../data/vexpress_flags_reset.patch | patch -p1
\end{bashinput}

Now configure U-Boot to support the ARM Vexpress Cortex A9 board
(\code{vexpress_ca9x4_defconfig}).

Get an understanding of U-Boot's configuration and compilation steps
by reading the \code{README} file, and specifically the {\em Building
the Software} section.

Basically, you need to specify the cross-compiler prefix
(the part before \code{gcc} in the cross-compiler executable name):
\bashcmd{$ export CROSS_COMPILE=arm-linux-}

Now that you have a valid initial configuration, run \code{make menuconfig}
to further edit your bootloader features:

\begin{itemize}
\item In the \code{Environment} submenu, we will configure U-Boot so
    that it stores its environment inside a file called {\em
    uboot.env} in a FAT filesystem on an MMC/SD card, as our emulated
    machine won't have flash storage:
    \begin{itemize}
    \item Unset \code{Environment in flash memory} (\projconfig{u-boot}{CONFIG_ENV_IS_IN_FLASH})
    \item Set \code{Environment is in a FAT filesystem} (\projconfig{u-boot}{CONFIG_ENV_IS_IN_FAT})
    \item Set \code{Name of the block device for the environment} (\projconfig{u-boot}{CONFIG_ENV_FAT_INTERFACE}): \code{mmc}
    \item \code{Device and partition for where to store the environment in
        FAT} (\projconfig{u-boot}{CONFIG_ENV_FAT_DEVICE_AND_PART}): \code{0:1}\newline
        The above two settings correspond to the arguments of the
        \code{fatload} command.
    \end{itemize}
\item Also add support for the \code{editenv} (\projconfig{u-boot}{CONFIG_CMD_EDITENV})
    and \code{bootd} (which can be abbreviated as \code{boot}, \projconfig{u-boot}{CONFIG_CMD_BOOTD})
    that are not present in the default configuration for our board.
\end{itemize}

In recent versions of U-Boot and for some boards, you will
need to have the Device Tree compiler:

\bashcmd{$ sudo apt install device-tree-compiler}

Finally, run \code{make}\footnote{You can speed up the
compiling by using the \code{-jX} option with \code{make}, where X
is the number of parallel jobs used for compiling. Twice the
number of CPU cores is a good value.}, which will build U-Boot.

This generates several binaries, including \code{u-boot} and
\code{u-boot.bin}.

\section{Testing U-Boot}

Still in U-Boot sources, test that U-Boot works:

\bashcmd{$ qemu-system-arm -M vexpress-a9 -m 128M -nographic -kernel u-boot}

\begin{itemize}
\item \code{-M}: emulated machine
\item \code{-m}: amount of memory in the emulated machine
\item \code{-kernel}: allows to load the binary directly in the emulated
      machine and run the machine with it. This way, you don't
      need a first stage bootloader. Of course, you don't
      have this with real hardware.
\end{itemize}

Press a key before the end of the timeout, to access the U-Boot prompt.

You can then type the \code{help} command, and explore the few commands
available.

Note: to exit QEMU, type \code{[Ctrl][a]} followed by \code{[h]}
to see available commands. One of them is \code{[Ctrl][a]} followed by
\code{[x]}, which allows to exit the emulator.

\section{SD card setup}

We now need to add an SD card image to the QEMU virtual machine,
in particular to get a way to store U-Boot's environment.

In later labs, we will also use such storage for other purposes
(to store the kernel and device tree, root filesystem and other
filesystems).

The commands that we are going to use will be further explained
during the {\em Block filesystems} lectures.

First, using the \code{dd} command, create a 1 GB file
filled with zeros, called \code{sd.img}:

\bashcmd{$ dd if=/dev/zero of=sd.img bs=1M count=1024}

This will be used by QEMU as an SD card disk image

Now, let's use the \code{cfdisk} command to create the partitions that
we are going to use:

\bashcmd{$ cfdisk sd.img}

If \code{cfdisk} asks you to \code{Select a label type}, choose
\code{dos}, as we don't really need a \code{gpt} partition table for
our labs.

In the \code{cfdisk} interface, create three primary partitions,
starting from the beginning, with the following properties:

\begin{itemize}
\item One partition, 64MB big, with the \code{FAT16} partition type.
      Mark this partition as bootable.

\item One partition, 8 MB big\footnote{For the needs of our system,
  the partition could even be much smaller, and 1 MB would be enough.
  However, with the 8 GB SD cards that we use in our labs, 8 MB will
  be the smallest partition that \code{cfdisk} will allow you to
  create.}, that will be used for the root filesystem. Due to the
  geometry of the device, the partition might be larger than 8 MB,
  but it does not matter. Keep the \code{Linux} type for the
  partition.

\item One partition, that fills the rest of the SD card image, that will be
  used for the data filesystem. Here also, keep the \code{Linux} type
  for the partition.
\end{itemize}

Press \code{Write} when you are done.

We will now use the {\em loop} driver\footnote{Once again, this will
be properly be explained during our {\em Block filesystems} lectures.}
to emulate block devices from this image and its partitions:

\bashcmd{$ sudo losetup -f --show --partscan sd.img}

\begin{itemize}
\item \code{-f}: finds a free loop device
\item \code{--show}: shows the loop device that it used
\item \code{--partscan}: scans the loop device for partitions
    and creates additional \code{/dev/loop<x>p<y>} block devices.
\end{itemize}

Last but not least, format the first partition as FAT16 with
a \code{boot} label:

\bashcmd{$ sudo mkfs.vfat -F 16 -n boot /dev/loop<x>p1}

The other partitions will be formated later.

Now, you can release the loop device:
\bashcmd{$ sudo losetup -d /dev/loop<x>}

\section{Testing U-Boot's environment}

Start QEMU again, but this time with the emulated SD card
(you can type the command in a single line):

\begin{bashcmd}
$ qemu-system-arm -M vexpress-a9 -m 128M -nographic \
	  	  -kernel u-boot-2020.04/u-boot \
		  -sd sd.img
\end{bashcmd}

Now, in the U-Boot prompt, make sure that you can set and store an environment variable:

\begin{bashinput}
$ setenv foo bar
$ saveenv
\end{bashinput}

Type \code{reset} which reboots the board, and then check that the
\code{foo} variable is still set:

\bashcmd{$ printenv foo}

\section{Setup networking between QEMU and the host}

To load a kernel in the next lab, we will setup networking between the QEMU emulated
machine and the host.

To do so, create a \code{qemu-myifup} script that will bring up a
network interface between QEMU and the host.  Here are its contents:

\begin{verbatim}
#!/bin/sh
/sbin/ip a add 192.168.0.1/24 dev $1
/sbin/ip link set $1 up
\end{verbatim}

Of course, make this script executable:
\bashcmd{$ chmod +x qemu-myifup}

As you can see, the host side will have the \code{192.168.0.1} IP
address. We will use \code{192.168.0.100} for the target side.
Of course, use a different IP address range if this conflicts with your
local network.

Then, you will need root privileges to run QEMU this time,
because of the need to bring up the network interface:

\begin{bashinput}
$ sudo qemu-system-arm -M vexpress-a9 -m 128M -nographic \
	-kernel u-boot-2020.04/u-boot \
	-sd sd.img \
	-net tap,script=./qemu-myifup -net nic
\end{bashinput}

Note the new net options:
\begin{itemize}
\item \code{-net tap}: creates a software network interface on the host side
\item \code{-net nic}: adds a network device to the emulated machine
\end{itemize}

On the host machine, using the \code{ip a} command, check that
there is now a \code{tap0} network interface with the expected IP
address.

On the U-Boot command line, you will have to configure the environment
variables for networking:

\begin{ubootinput}
=> setenv ipaddr 192.168.0.100
=> setenv serverip 192.168.0.1
=> saveenv
\end{ubootinput}

You can now test the connection to the host:
\ubootcmd{=> ping 192.168.0.1}

It should finish by:
\begin{verbatim}
host 192.168.0.1 is alive
\end{verbatim}

\section{tftp setup}

Install a {\em tftp} server on your host as explained in the slides.

Back in U-Boot, run \code{bdinfo}, which will allow you to find out that
RAM starts at \code{0x60000000}. Therefore, we will use the \code{0x61000000}
address to test {\em tftp}.

To test the TFTP connection, put a small text file in
the directory exported through TFTP on your development
workstation. Then, from U-Boot, do:

\ubootcmd{$ tftp 0x61000000 textfile.txt}

The \code{tftp} command should have downloaded the
\code{textfile.txt} file from your development workstation into
the board's memory at location \code{0x61000000}.

You can verify that the download was successful by dumping the
contents of the memory:

\ubootcmd{=> md 0x61000000}

\section{Rescue binary}

If you have trouble generating binaries that work properly, or later
make a mistake that causes you to lose your bootloader binary, you
will find a working version under \code{data/} in the current lab
directory.
