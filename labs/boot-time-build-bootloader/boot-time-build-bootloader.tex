\subchapter{Build the bootloader}{Objective: compile and install the
bootloader.}

After this lab, you will be able to compile U-Boot for your target
platform and run it from a micro SD card provided by your instructor.

\section{Setup}

Go to the \code{~/boot-time-labs/bootloader/u-boot/} directory.

Let's use the 2022.04 version:
\begin{verbatim}
git checkout v2022.04
\end{verbatim}

\section{Compiling environment}

As the previous Buildroot lab is probably not over yet, we will
use the cross-compiling toolchain provided by Ubuntu:

\begin{verbatim}
sudo apt install gcc-arm-linux-gnueabihf
export CROSS_COMPILE=arm-linux-gnueabihf-
\end{verbatim}

We will also need this package to compile U-Boot and later the
Linux kernel:
\begin{verbatim}
sudo apt install libssl-dev
\end{verbatim}

\section{Configuring U-Boot}

Let's use a ready-made U-Boot configuration for our hardware.

The \projdir{u-boot}{configs} directory normally contains one or several
configuration file(s) for each supported board. However, in our case,
we are going to use a more generic configuration file that supports all
TI AM335x based BeagleBone variants and the AM335x EVM board too:

\begin{verbatim}
make am335x_evm_defconfig
\end{verbatim}

\section{Compiling U-Boot}

Just run:
\begin{verbatim}
make
\end{verbatim}

or, to compile faster:

\begin{verbatim}
make -j 8
\end{verbatim}

This runs 8 compiler jobs in parallel (for example if you have 4 CPU
cores on your workstation... using more jobs than cores guarantees that
the CPUs and I/Os are always fully loaded, for optimum performance.

At the end, you have \code{MLO} and \code{u-boot.img} files that we will
put on a micro SD card for booting.

\input{../common/boneblack-sdcard-preparation.tex}

\section{Booting your new bootloader}

On a board in a normal state, there should be a bootloader on the on-board MMC
(eMMC) storage, and this will prevent you from using a bootloader on an
external SD card (unless you hold the \code{S2} button while powering
up your board, which is just suitable for exceptional needs).

Therefore, to override this behavior and use the external SD card,
instead, let's wipe out the \code{MLO} file on the eMMC.

Power up or reset your board, and in the U-Boot prompt, run:

\begin{verbatim}
fatls mmc 1
\end{verbatim}

You should see the \code{MLO} file in the list of files. Let's remove it
by issuing the below command (erasing the first 1MiB on the eMMC):

\begin{verbatim}
mmc dev 1
mmc erase 0 100000
\end{verbatim}

You can now see that \code{fatls mmc 1} no longer sees any file.

Now, copy your newly compiled \code{MLO} and \code{u-boot.img} files to
the SD card's \code{boot} partition, and after cleanly unmounting this
partition, insert the SD card into the board and reset it (you can now
use the \code{RESET} button on your LCD cape).

You should now see your board booting with your own \code{MLO} and U-Boot
binaries (the versions and compiled dates are shown in the console).
