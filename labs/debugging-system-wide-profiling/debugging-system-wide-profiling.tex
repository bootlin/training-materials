\subchapter
{System wide profiling and tracing}
{Objectives:
  \begin{itemize}
    \item System profiling with {\em trace-cmd} and {\em KernelShark}.
    \item (Bonus) System profiling with {\em perf} and FlameGraphs.
  \end{itemize}
}

\section{ftrace \& uprobes}

First of all, we will start a small program on the target using the following command:

\begin{bashinput}
$ mystery_program 1500 400 2 &
\end{bashinput}

In order to trace a full system, we can use ftrace. However, if we want to trace
the userspace, we'll need to add new tracepoints using uprobes. This can be done
manually with the uprobe sysfs interface or using \code{perf probe}.

Before starting to profile, we will compile our program to be instrumented.
On your development host, run:

\begin{bashinput}
$ cd /home/$USER/debugging-labs/nfsroot/root/system_profiling
$ make
\end{bashinput}

On the target, we will create a uprobe in the main function of the
\code{crc_random} program each time a crc is computed. First, let's list the line
numbers that are recognized by perf to add a uprobe:

\begin{bashinput}
$ cd /root/system_profiling
$ perf probe --source=. -x ./crc_random -L main
\end{bashinput}

{\em Notes:
 \begin{itemize}
  \item In order to be able to add such userspace probe, perf needs to access
symbols and source file
  \item If perf output is filled with misinterpreted ANSI escape sequences, you
  can append \code{| cat} to the command
 \end{itemize}}

Then, we can create a uprobe and capture the crc value using:

\begin{bashinput}
$ perf probe --source=. -x ./crc_random main:35 crc
\end{bashinput}

We can finally trace the application using \code{trace-cmd} with this event. We will
use the remote tracing capabilities of \code{trace-cmd} to do that. First, we will
start the \code{trace-cmd} server on the desktop platform:

\begin{bashinput}
trace-cmd listen -p 4567
\end{bashinput}

Then, on the target, run \code{trace-cmd} with the \code{-N} parameter to
specify the remote trace-cmd server. Interrupt it after about 50 CRC
computations in order to have a meaningful trace.

\begin{bashinput}
$ trace-cmd record -N 192.168.0.1:4567 -e probe_crc_random:main_L35 ./crc_random
^C
\end{bashinput}

{\em Note: if during the recording you encounter an \code{Unsupported file
version 7} error, it likely means that the trace-cmd tool on your
workstation is older than the one on the target, and does not understand
the default trace format used by the target. You can force trace-cmd on the
target to use an older version of the trace data format by also providing
\code{--file-version=6} on the recording command}

Then, using KernelShark on the host, analyze the trace:

\begin{bashinput}
$ sudo apt install kernelshark
$ kernelshark
\end{bashinput}

We can see that something is wrong, our process does not seem to compute crc at
a fixed rate. Let's trace the \code{sched_switch} events to see what is happening:

\begin{bashinput}
$ trace-cmd record -N 192.168.0.1:4567 -e sched_switch -e probe_crc_random:main_L35 ./crc_random
^C
\end{bashinput}

Reanalyze the traces with KernelShark and try to understand what is going on.

\section{(Bonus) System profiling with {\em perf} and FlameGraphs}

In order to profile the whole system, we are going to use perf and try to find
the function that takes most of the time executing.

First of all, we will run a global recording of functions and their backtrace on
(all CPUs) during 10 seconds using the following command on the target:

\begin{bashinput}
$ perf record -F 99 -g -- sleep 10
\end{bashinput}

This command will generate a \code{perf.data} file that can be used on a remote
computer for further analysis. Copy that file and fix the permissions using
\code{chown}:

\begin{bashinput}
$ cd /home/$USER/debugging-labs/buildroot/output/build/linux-%\workingkernel%/
$ sudo cp /home/$USER/debugging-labs/nfsroot/root/system_profiling/perf.data .
$ sudo chown $USER:$USER perf.data
\end{bashinput}

We will then use perf report to visualize the aquired data:

\begin{bashinput}
$ perf report --symfs=/home/$USER/debugging-labs/buildroot/output/staging/
	-k /home/$USER/debugging-labs/buildroot/output/build/linux-%\workingkernel%/vmlinux
\end{bashinput}

Another useful tool for performance analysis is flamegraphs. Latest perf
version includes a builtin support for flamegraphs but the template is not
available on debian so we will use another support provided by Brendan Gregg
scripts.

\begin{bashinput}
$ git clone https://github.com/brendangregg/FlameGraph.git
$ perf script | ./FlameGraph/stackcollapse-perf.pl | ./FlameGraph/flamegraph.pl > flame.html
\end{bashinput}

Using this flamegraph, analyze the system load.

{\em TIP: if the generated graph is not relevant/missing symbols, you may try to to all symbols
translation in the target before generating the graph on host, i.e use perf script on the target,
and bring back the result on host and provide it to Flamegraph scripts}

You can also generate a Flamegraph on previous labs: for example you can re-do a recording on
\code{png_convert} from the application profiling lab, generate the corresponding Flamegraph and confirm
your observations about most time-consuming functions.
