\subchapter{First compilation}{Get used to the build mechanism}

During this lab, you will:
\begin{itemize}
  \item Configure which system to build Android for
  \item Compile your first Android root filesystem
\end{itemize}

\section{Setup}

Stay in the \code{/home/<user>/felabs/android/aosp/android} directory.

\section{Compile a filesystem}

Now that \code{repo sync} is over, we will compile an Android system for the
emulator. First, be sure that the emulator is in your path. You can check
by running the \code{emulator} command in a terminal.\\

Now, run \code{source build/envsetup.sh}.

It contains many useful shell functions and aliases, such as \code{croot} to
go to the root directory of the Android source code or \code{lunch}, to select
the build options.\\

The target product for the emulator is {\it generic}, and we want to have an 
engineering build. To do this, run \code{lunch generic-eng}.\\

The build system will use the proper setup to build this target, so that we
only have \code{make} to run now.\\

Go grab (several cups of) coffee, and you will have the system compiled in three
images in the folder \code{out/target/product/generic}.

Run the emulator with these newly generated images and check the build version
in the Settings application in Android.
