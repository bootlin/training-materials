\subchapter{Use the Yocto Project SDK through Eclipse}{Build and use
the Yocto Project SDK with Eclipse}

During this lab, you will:
\begin{itemize}
  \item Integrate the Eclipse Yocto Project plugin
  \item Configure the plugin to work with the previously used Yocto project
  \item Develop and modify Poky from Eclipse
\end{itemize}

\section{Set up the environment}

First we need to set up the environment in order to be able to develop our
applications. We need Poky to build support for Eclipse (an IDE). Run:

\begin{verbatim}
bitbake meta-ide-support
\end{verbatim}

\section{Donwload Eclipse}

Download the Kepler version of Eclipse on the official website:
\url{http://www.eclipse.org/downloads/packages/eclipse-standard-432/keplersr2}.
Then uncompress the tarball and launch Eclipse. Ubuntu has a known bug, and
you need to run the following command from the extracted Eclipse directory to be
able to use it properly:
\begin{verbatim}
UBUNTU_MENUPROXY=0 ./eclipse
\end{verbatim}

\section{Install the Eclipse plugin}

First, you need to download a few packages to fulfill the Yocto
Eclipse plugin requirements. Open \code{Install New Software} in the
\code{Help} menu. Select the \code{Kepler -
http://download.eclipse.org/releases/kepler} repository and install:
\begin{itemize}
  \item LTTng - Linux Tracing Toolkit
  \item C/C++ Remote Launch
  \item Remote System Explorer End-user Runtime
  \item Remote System Explorer User Actions
  \item Target Management Terminal
  \item TCF Remote System Explorer add-in
  \item TCF Target Explorer
\end{itemize}

You can then either choose to download the already built Eclipse Yocto Project
plugin or to build your own by first downloading its source repository. We will
here download the latest plugin available on the Yocto Project website directly
from Eclipse.

First add the Yocto Project Eclipse Update site to the available software sites:
\url{http://downloads.yoctoproject.org/releases/eclipse-plugin/1.5.1/kepler}.
Then download:
\begin{itemize}
  \item Yocto Project ADT Plug-in
  \item Yocto Project BitBake Commander Plug-in
  \item Yocto Project Documentation plug-in
\end{itemize}

Finally, you need to set up the plugin itself. Open \code{Preferences} from the
"Window" menu. Then click on \code{Yocto Project ADT} in the preference dialog.
Choose \code{Build System Derived Toolchain}, set the \code{Toolchain Root
Location} to your build directory and the \code{Sysroot Location} to the
\code{tmp/sysroot/beaglebone} directory of your build directory. Then select the
\code{Target Architecture}. Apply and close, the plugin is set up!

\section{Build a Poky image from Eclipse}

The Eclipse Yocto Project plugin allows, in addition to compiling an application
to the right target architecture, to build a Poky image. We will here configure
Eclipse to build our exact previous BeagleBone image, but it is also possible to
create a build environment from scratch.

Select \code{New} in the \code{File} menu. Then double click on \code{New Yocto
Project} in the \code{Yocto Project Bitbake Commander} category. In the new
window, choose \code{Local} as the \code{Remote service provider} and
\code{local} as the connection name. Uncheck the \code{Clone from Yocto Project
Git Repository into new location}, this option is only used when starting a
vanilla project. The directory used by Eclipse will be \code{Location/Project
name}, you need to fill the form with this in mind: use the path of the
directory containing your previously used \code{poky} root directory and put
the name of this directory as the project name. Then click on \code{Finish}.
The new project is now visible in the \code{Project Explorer} panel.

In order to build a target image, you need to launch Hob, the graphical
interface for BitBake. You can find this tool under the \code{Project} menu. In
the Hob window, select the right target machine. You will then be able to select
our previously created image recipe. Have a look at the \code{Advanced
configuration} menu, and then start a build.

Once the build completed, you will see useful information about the image built
in the Hob window.

\section{Create a recipe}

The Yocto Project BitBake Commander Plug-in allows to fully manage the Yocto Project
from Eclipse, including creating a new recipe with a user friendly wizard. To
demonstrate this ability, we will create a new recipe for the wonderful Steam
Locomotive command (\code{sl}). The home page is located at:
\code{https://github.com/mtoyoda/sl}.

Select the \code{File → New → Other} menu. Double click on the \code{BitBake
Recipe} under the \code{Yocto Project BitBake Commander} category. Now you can
create the recipe by filling the form according to the previous labs we
followed.
