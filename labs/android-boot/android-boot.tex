\subchapter{Compile and boot an Android Kernel}{Learn how to build an Android
Kernel and test it}

After this lab, you will be able to:
\begin{itemize}
  \item Extract the kernel patchset from Android Kernel
  \item Apply it on a vanilla kernel
  \item Compile and boot a kernel for the emulator
\end{itemize}

\section{Setup}

Go to the \code{/home/<user>/felabs/android/kernel} directory.

\section{Compile a kernel for the goldfish platform}

In the standard Android source code we fetched, the kernel source is not included.
So the first thing to do is download the Android kernel sources.

\code{git clone https://android.googlesource.com/kernel/goldfish.git kernel}\\

Now that we have kernel sources, we need a toolchain to compile them. Fortunately,
Android provides one. First, add the toolchain to your path.

\code{export PATH=$(pwd)/prebuilt/linux-x86/toolchain/arm-eabi-4.4.3/bin:$PATH}

The platform for the emulator is called goldfish. Using Git, find the latest
kernel for this platform and switch to this branch with the command 

\code{git checkout <branch>}.

Now, configure the kernel for the target platform

\code{ARCH=arm make goldfish_defconfig}

Then, configure the kernel using the tool of your choice to add a custom suffix
to the kernel version and compile the kernel!

\code{ARCH=arm CROSS_COMPILE=arm-eabi- make}

A few minutes later, you can boot the emulator with the image found in
\code{arch/arm/boot}, and when it's booted, check the kernel version in the
Android Settings.

\section{Generate a patchset from Android Kernel}

Go to the \code{kernel} directory. That's where the kernel source code is. The
kernel however is the one from Android, and we need to fetch the source code
from the official kernel to get a point of reference to know which commits
differ. Git is great for that and offers everything we need to do so.

First, add the linux-stable repository from Greg Kroah-Hartmann to our
repository:\\

\code{git remote add stable git://git.kernel.org/pub/scm/linux/kernel/git/stable/linux-2.6-stable.git}\\

Now, we need to fetch the changes from this remote repository:

\code{git fetch stable}

A while later, you should have all the changes in your own repository, browsable
offline!

Now, open the main Makefile with your favorite editor and get the kernel version
that is in it. Now that we have it, we can generate the whole set of commits that
differ from the linux kernel by doing:

\code{git log v<kernel-version>..HEAD}

Find in that set of commits the one that adds the wakelocks.\\

Now, extract the set of patches with the \code{git format-patch} command.\\

Congrats! You now have all the patches to apply to a vanilla kernel to obtain a
full Android kernel.
