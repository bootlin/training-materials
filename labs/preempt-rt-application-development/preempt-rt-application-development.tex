\subchapter{Lab3: Optimizing an application for Real-Time}{}

During this lab, you will:
\begin{itemize}
  \item Analyse the source code of an application
  \item Find ways to improve the maximum latency
  \item Trace the program's execution using perf
\end{itemize}

\section{Compile and run the program}

In the \code{bootlin-labs-data} folder, you'll find a C program called \code{test_counter.c}
This program's goal is to count rising edges on a GPIO input pin.

The program stores the timestamp for each edge, ans stores it into a buffer.

Your goal is to implement the logic to compute the maximum latency during the
program's execution, and to try to improve it by following the best practises.

Using a signal generator, send a 10KHz square signal on pin 28 on connector 9.

Alternatively, apply the patch 0001-enable-spidev.patch file on your devicetree
to have the /dev/spidev0.0 device available, and use it to send pulses onto the
pin 28.

\subsection{Program configuration}

Pay attention to the initialization sequence for the program, and make sure you
use the right memory-management practises.

The program will run by default in \code{SCHED_OTHER}. Make sure that the scheduling
algorithm used is the right one.

Make sure that the timestamps are gathered using the right clock.

\subsection{Benchmarking}

Using \code{perf}, try to analyse the program. What can you say about the number
of pagefaults and their occurence ?

The GPIO controller on the beaglebone uses edge-triggered interrupts for event
reporting. Make sure these interrupts have the right priority.

