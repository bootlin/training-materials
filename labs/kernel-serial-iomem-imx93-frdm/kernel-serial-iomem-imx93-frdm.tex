\subchapter{Accessing I/O memory and ports}{Objective: read /
  write data from / to a hardware device}

Throughout the upcoming labs, we will implement a character driver
allowing to write data to additional CPU serial ports available on
the BeagleBone, and to read data from them.

After this lab, you will be able to:

\begin{itemize}
\item Add UART devices to the board device tree.
\item Access I/O registers to control the device and
      send first characters to it.
\end{itemize}

\section{Setup}

Go to your kernel source directory and continue working with the
\code{bootlin-labs} branch, this way we can keep the same custom Device
Tree we already created, and build on top of it.

\section{Add UART devices}

Before developing a driver for additional UARTS on the board, we
need to add the corresponding descriptions to the board Device Tree.

First, open the board design files and find the available connectors for LPUART3
and LPUART8. You should see that it is possible to connect LPUART3 to
GPIO\_IO14/15 and LPUART8 to GPIO\_IO12/13.

Using a new USB-serial cable with Female connectors, provided by your
instructor, connect your PC to LPUART3, after finding the correct pins in the
board user manual. The wire colors we typically use are:

\begin{itemize}
\item The blue wire should be connected \code{GND}.
\item The red wire (\code{TX}) should be connected to the board's \code{RX} pin.
\item The green wire (\code{RX}) should be connected to the board's \code{TX} pin.
\end{itemize}

{\scriptsize \sourcecode{labs/kernel-serial-iomem-imx93-frdm/uarts-pinctrl.dts}}

In your custom DTS file, reference again the pin controller node and add
declarations for the pin states needed for LPUART3 and LPUART8.

Then, declare the corresponding devices:

{\scriptsize \sourcecode{labs/kernel-serial-iomem-imx93-frdm/uarts.dts}}

This is a good example of how we can override definitions in the Device
Tree. \code{lpuart3} and \code{lpuart8} are already defined in
\kfile{arch/arm64/boot/dts/freescale/imx93.dtsi}. In the above code, we
just override a few properties and add the missing ones:

\begin{itemize}
\item \code{compatible}: use our driver instead of using the default one
      (\code{fsl,imx8ulp-lpuart}).
\item \code{status}: enable the device (was set to \code{disabled} in
      the original definition).
\item \code{pinctrl-names}, \code{pinctrl-0}: add pinmux settings
      (none were defined so far).
\end{itemize}

Compile and update your DTB.

\section{Operate a platform device driver}

Go to the \code{~/linux-kernel-labs/modules/nfsroot/root/serial/} directory.
You will find a \code{serial.c} file which already provides a platform
driver skeleton.

Add the code needed to match the driver with the devices which you have
just declared in the device tree.

Compile your module and load it on your target. Check the kernel log
messages, that should confirm that the \code{probe()} routine was
called\footnote{Don't be surprised if the \code{probe()} routine is
actually called twice! That's because we have declared two devices.
Even if we only connect a serial-to-USB dongle to one of them, both
of them are ready to be used!}.

\section{Create a device private structure}

In the same way as in the nunchuk lab, we now need to create a
structure that will hold device specific information and help
keeping pointers between logical and physical devices.

As the first things to store will be the base virtual address for
each device as well as one of their primary resource, the main clock feeding
each of them, let's declare this structure as follows:

\begin{verbatim}
struct serial_dev {
        void __iomem *regs;
        struct clk *clk;
};
\end{verbatim}

The first thing to do is allocating such a structure at the beginning
of the \code{probe()} routine. Let's do it with the \kfunc{devm_kzalloc}
function again as in the previous lab. Again, resource deallocation is
automatically taken care of when we use the \code{devm_} functions.

So, add the below line to your code:

\begin{verbatim}
struct serial_dev *serial;
...
serial = devm_kzalloc(&pdev->dev, sizeof(*serial), GFP_KERNEL);
if (!serial)
        return -ENOMEM;
\end{verbatim}

\section{Get a base virtual address for your device registers}

You can now get a virtual address for your device's base physical
address, by calling:

\begin{verbatim}
serial->regs = devm_platform_ioremap_resource(pdev, 0);
if (IS_ERR(serial->regs))
        return PTR_ERR(serial->regs);
\end{verbatim}

What's nice is that you won't ever have to release this resource,
neither in the \code{remove()} routine, nor if there are failures
in subsequent steps of the \code{probe()} routine.

Make sure that your updated driver compiles, loads and unloads well.

The second step very common step is to grab and enable the clock, otherwise the
registers will not be accessible:

\begin{verbatim}
serial->clk = devm_clk_get_enabled(&pdev->dev, "ipg");
if (IS_ERR(serial->clk)) {
	dev_err(&pdev->dev, "Failed to get/enable IPG clock\n");
	return PTR_ERR(serial->clk);
}
\end{verbatim}

Again, the resource is managed (thanks to the \code{devm_} prefix, so there is
nothing to disable nor free upon error/removal.

In order to verify the validity of the previous steps and check the
accessibility of the registers, read and display the version ID of the LPUART
controller by reading and displaying, in your probe function, the content of the
\code{VERID} register (at offset 0x00).

\section{Device initialization}

We are ready to configure a few registers which will allow us to enable the UART
devices. As the initialization is a bit tedious and brings little added value
for the purpose of this training, we give in a separated file a few
register/bitfields definitions, as well as a bit of logic to easily configure
all the registers at once.

Make sure to copy the content of \code{serial-snippet.c} (in the \code{serial}
folder) in your driver, and call \code{serial_init_controller()} in
your \code{probe()} function.

If you have a bit of spare time, you can look at chapter 60 of the i.MX93
processor reference manual for details about how to use the LPUART ports, to
understand better what we are doing here.

We are now ready to transmit characters over the serial ports!

\section{Standalone write routine}

Implement a C routine taking a pointer to a \code{serial_dev}
structure and one character as parameters, and writing
this character to the serial port, using the following steps:

\begin{enumerate}
\item Wait until the \ksym{LPUART_STAT_TDRE} bit gets set in the
  \ksym{LPUART_STAT} register. You can busy-wait for this condition to happen.
  In the busy-wait loop, you can call the \kfunc{cpu_relax} kernel function
  to ensure the compiler won't optimise away this loop.
\item Write the character to the \ksym{LPUART_DATA} register.
\end{enumerate}

Add a call to this routine from your module \code{probe()} function,
and recompile your module.

Open a new \code{picocom} instance on your new serial port (not the
serial console):

\begin{verbatim}
picocom -b 115200 /dev/ttyUSB0
\end{verbatim}

Load your module on the target. You should see the
corresponding character in the new \code{picocom} instance,
showing what was written to UART3.

You can also check that you also get the same character on UART8 (just connect
to the UART8 pins instead of the UART3 ones and reload the module).

\section{Driver sanity check}

Remove your module and try to load it again. If the second attempt to
load the module fails, it is probably because your driver doesn't
properly free the resources it allocated or registered, either at module
exit time, or after a failure during the module \code{probe()} function. Check
and fix your module code if you have such problems.
