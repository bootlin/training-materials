\subchapter{Using Sockets in userspace}
{Objective: Learn the basics of socket programming}

\section{Goals}
 
\begin{itemize}
\item Create a simple TCP client program
\item Use tcpdump to visualise traffic
\item Implement a simple packet dump program
\end{itemize}

\subsection{Network configuration}

In this lab, we'll setup a direct connection between the host and the target on the \textbf{192.168.42.0/24} subnet.

\begin{targetbashinput}
$ ip address flush lan0
$ ip link set lan0 up
$ ip address add 192.168.42.2/24 dev lan0
\end{targetbashinput}

\begin{hostbashinput}
$ ip address flush <iface>
$ ip link set <iface> up
$ ip address add 192.168.42.1/24 dev <iface>
\end{hostbashinput}

You can then run a simple \code{ping} test to make sure everything works :

\begin{hostbashinput}
$ ping 192.168.42.2
\end{hostbashinput}

\section{Simple TCP client}

Let's start simple by creating a simple TCP client program, that will run on the \textcolor{green}{Host}, and connect to a server running on the targe. We will use the \textbf{netcat} program on the target, which provides a very simple implementation of TCP and UDP servers and clients.

Go in the lab3 host-side directory :

\begin{hostbashinput}
$ cd /home/$USER/networking-labs/host/lab3
\end{hostbashinput}

In the \code{tcp_client.c} file, let's implement a very simple program to get familiar with the socket programming aspects.
The program will take 2 parameters :
\begin{itemize}
	\item The server's address, using the \textbf{dotted notation}
	\item The server's port
\end{itemize}

Start by creating your socket file descriptor, using the \textbf{socket} function. Use the \code{AF_xxx} family correspondig to the IPv4 address family. Use the \code{SOCK_xxx} type corresponding to TCP :

\code{int sockfd;}
\code{sockfd = socket(AF_???, SOCK_???, 0);}

We pass a 0 as the last paraemeter as we won't be using any flags.

Next, create an object representing an IPv4 address :
\code{struct sockaddr_in addr;}

You need to populate 3 fields within this struct :
\begin{itemize}
	\item \code{addr.sin_family} : Use the \code{AF_xxx} family corresponding to IPv4
	\item \code{addr.sin_port} : The server's port. You can use \code{atoi()} to convert the user-pased ASCII string to an integer. This field must be specified in \textbf{network byte order}, use \code{htons()} to convert the port in the right endianness.
	\item \code{addr.sin_addr} : The server's IP address. You can use \code{inet_aton} to convert from dotted notation to the 32bits value.
\end{itemize}

Next, let's connect to the server. The \code{connect()} call must be made, but it expects a generic \code{struct sockaddr} as a parameter, which is subclassed by \code{struct sockaddr_in}, so we need to cast it back to the parent class :

\code{connect(sockfd, (struct sockaddr *)&addr, sizeof(addr));}

Finally, write the string "hi !" into the socket, using \code{write()} or \code{send()}.

Don't forget to \code{close()} the socket before terminating the program.

You can compile it with :

\begin{hostbashinput}
$ make tcp_client
\end{hostbashinput}

Once your program looks good, you can start testing it !

Start the netcat TCP server on the Espressobin, listening on port 3000 :

\begin{targetbashinput}
$ nc -l -p 3000
\end{targetbashinput}

And test your client on the host machine :

\begin{hostbashinput}
$ ./tcp_client 192.164.42.2 3000
\end{hostbashinput}

You should see the string "hi !" printed on the Espressobin's console :)

\section{Visualize traffic with TCPdump}

Let's make sure that our data is indeed sent in TCP over IPv4. For that, we'll use the \textbf{tcpdump} tool.

First let's leave the netcat TCP server running in the background :

\begin{targetbashinput}
$ nc -l -k -p 3000 &
\end{targetbashinput}

Now run \textbf{tcpdump} on the Espressobin, listening on \code{lan0}, with the following flags :
\begin{itemize}
	\item \code{-n} : Print addresses instead of hostnames
	\item \code{-e} : Include Layer 2 information
\end{itemize}

\begin{targetbashinput}
$ tcpdump -n -e -i lan0
\end{targetbashinput}

Send the message to the server from the client :

\begin{hostbashinput}
$ ./tcp_client 192.164.42.2 3000
\end{hostbashinput}

You should see in the TCPDump output, on the target, the various packets exchanges during the establinshment of the TCP stream : SYN, SYN-ACK, ACK. Acknowledgments are represented with a \textbf{.} in the flags.
\begin{targetterminaloutput}
... Flags [S], ...
... Flags [S.], ...
... Flags [.], ...
\end{targetterminaloutput}

To dump the content of the packets, you can use :
\begin{targetbashinput}
$ tcpdump -XX -n -e -i lan0
\end{targetbashinput}

The Espressobin integrates a DSA switch, all traffic running on \code{lan0} goes through the \textbf{conduit interface} \code{eth0}. Try running the capture on \code{lan0} and \code{eth0}. Do you see any difference ?

\section{RAW socket: Implementing our own custom TCPDUMP}

TCPdump is based on the use of \code{AF_PACKET} sockets, which give access to raw Layer 2 frames. Let's write our very simple implementatin of tcpdump.

We will be running this program on the target, so move into the target-side of our lab folder :

\begin{hostbashinput}
$ cd /home/$USER/networking-labs/target/lab3
\end{hostbashinput}

Open the \code{monitor.c} file and let's start implementing it. Start by opening a new socket, with the \code{AF_PACKET} family, \code{SOCK_RAW} type and using the \code{htons(ETH_P_ALL)} protocol to listen to everything.

To listen on a given interface, we need to \code{bind} the socket to the given interface.

As a reminder, \code{bind()} takes 3 parameters :
\begin{itemize}
	\item \code{int sockfd} : The socket file descriptor
	\item \code{struct sockaddr *addr} : a sockaddr pointer. For \code{AF_PACKET}, we must use the \code{struct sockaddr_ll} subclass
	\item \code{socklen_t addrlen} : The size of our addres object, that is \code{sizeof(struct sockaddr_ll)}
\end{itemize}

The \code{stuct sockaddr_ll} is a generic structure representing a Layer 2 address :

\begin{verbatim}
struct sockaddr_ll {
    unsigned short sll_family;   /* Always AF_PACKET */
    unsigned short sll_protocol; /* Physical-layer protocol */
    int            sll_ifindex;  /* Interface number */
    unsigned short sll_hatype;   /* ARP hardware type */
    unsigned char  sll_pkttype;  /* Packet type */
    unsigned char  sll_halen;    /* Length of address */
    unsigned char  sll_addr[8];  /* Physical-layer address */
};
\end{verbatim}

We need to populate :
\begin{itemize}
	\item \code{sll_family} : \code{AF_PACKET}
	\item \code{sll_ifindex} : The interface index to listen to. Use \code{if_nametoindex()} to convert the interface name to its index
\end{itemize}

With these parameters constructed, call \code{bind()} on your socket.

You can now start reading traffic from your socket. To simplify a bit, let's read only the first 40 bytes of each packet.

In a \code{while} loop, call \code{recv} to read the first 40 bytes into a buffer, then display its content by calling the provided "hexdump" functin.

You can compile your code locally to test it :
\begin{hostbashinput}
$ make monitor
\end{hostbashinput}

To install it on your target, you have to update your linux image :

\begin{hostbashinput}
$ # clean your host-compiled monitor program :
$ make clean
$ Go in your buildroot folder
$ cd ../../buildroot
$ make lab3-rebuild all
\end{hostbashinput}

You can now re-flash your SDcard with the \code{dd} command, and reboot your Espressobin.

To run the monitor program :
\begin{targetbashinput}
$ cd lab3
$ ./monitor lan0
\end{targetbashinput}

