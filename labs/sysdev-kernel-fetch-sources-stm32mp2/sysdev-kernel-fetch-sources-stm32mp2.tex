\subchapter{Fetching Linux kernel sources}{Objective: learn how to fetch
the Linux kernel sources from git, from both the master and stable
branches.}

After this lab, you will be able to:
\begin{itemize}
\item Get the kernel sources from Stmicroelectronics git.
\end{itemize}

\section{Setup}

Create the \code{$HOME/__SESSION_NAME__-labs/kernel} directory and go into it.

Since the Linux kernel git repository is huge, our goal here is to start
downloading it right now, before starting the lectures about the Linux
kernel.

\section{Cloning the Linux tree}

To begin working with the Linux kernel sources, we need to clone its
reference git tree, the one managed by Linus Torvald.

However, this requires downloading more than 2.8 GB of data. If you
are running this command from home, or if you have very fast access to
the Internet at work (and if you are not 256 participants in the
training room), you can do it directly by connecting to
\url{https://git.kernel.org}:

\begin{verbatim}
    git clone https://github.com/torvalds/linux.git
\end{verbatim}

If Internet access is not fast enough and if multiple people have to
share it, your instructor will give you a USB flash drive with a
\code{tar.gz} archive of a recently cloned Linux source tree.

You will just have to extract this archive in the current directory,
and then pull the most recent changes over the network:

\begin{verbatim}
tar xf linux-git.tar.gz
cd linux
git checkout master
git pull
\end{verbatim}

Of course, if you directly ran
\code{git clone}, you won't have to run \code{git pull}, as
\code{git clone} already retrieved the latest changes. You may need to
run \code{git pull} in the future though, if you want to update a
newer Linux version.

We will now add the STMicroelectronics branch as remote, which contains
patches required for Linux to run properly on this board.

\begin{verbatim}
    git remote add st https://github.com/STMicroelectronics/linux
    git fetch st
\end{verbatim}
Now let's take a look at the branches provided by ST:
\begin{verbatim}
    git branch -r | grep st/
\end{verbatim}

For the labs we will use a specific branch of the Linux kernel ST, which is
the \code{v6.6-stm32mp-r1} branch.
\begin{verbatim}
$ git checkout v6.6-stm32mp-r1
\end{verbatim}

Now, let's continue the lectures. This will leave time for the commands
that you typed to complete their execution (if needed).

