\subchapter{Bootloader - TF-A and U-Boot}{Objectives: Set up serial
  communication, compile and install the U-Boot bootloader, use basic
  U-Boot commands, set up TFTP communication with the development
  workstation.}

As the bootloader is the first piece of software executed by a
hardware platform, the installation procedure of the bootloader is
very specific to the hardware platform. There are usually two cases:

\begin{itemize}

\item The processor offers nothing to ease the installation of the
  bootloader, in which case the JTAG has to be used to initialize
  flash storage and write the bootloader code to flash. Detailed
  knowledge of the hardware is of course required to perform these
  operations.

\item The processor offers a monitor, implemented in ROM, and through
  which access to the memories is made easier.

\end{itemize}

The STM32MP2 SoC falls into the second category. The monitor
integrated in the ROM reads the SD card to search for a valid
bootloader (the boot mode is actually configurable via a few input
pins). In case no bootloader is found, it will operate in a fallback
mode, that will allow to use an external tool to reflash some
executable through USB. Therefore, either by using an MMC/SD card or
this fallback mode, we can start up an STM32MP2-based board without
having anything installed on it.

\section{Setup}

Go to the \code{$HOME/__SESSION_NAME__-labs/bootloader} directory.

\section{Setting up serial communication with the board}

Plug the USB-C to USB-C cable on the Discovery board. There are
two USB-C ports on the board, choose the one labeled \code{USB PWR ST-LINK}.
This is a debug interface that exposes multiple debugging interfaces,
including a UART interface. When plugged in your computer, a serial
port should appear, {\tt \hosttty}.

You can also see this device appear by looking at the output of
\code{sudo dmesg}.

To communicate with the board through the serial port, install a
serial communication program, such as \code{picocom}:

\bashcmd{$ sudo apt install picocom}

If you run {\tt ls -l \hosttty}, you can also see that only
\code{root} and users belonging to the \code{dialout} group have
read and write access to the serial console. Therefore, you need
to add your user to the \code{dialout} group:

\bashcmd{$ sudo adduser $USER dialout}

{\bf Important}: for the group change to be effective, you have to
reboot your computer (at least on Ubuntu 24.04) and log in again.
A workaround is to run \code{newgrp dialout}, but it is not global.
You have to run it in each terminal.

Run {\tt picocom -b 115200 \hosttty}, to start serial
communication on {\tt \hosttty}, with a baudrate of 115200.
If you wish to exit \code{picocom}, press \code{[Ctrl][a]} followed by
\code{[Ctrl][x]}.

Don't be surprised if you don't get anything on the serial console yet,
even if you reset the board. That's because the SoC has nothing to boot
on yet. We will prepare a micro SD card to boot on in the next sections.

\section{TF-A and U-Boot relationship}

The boot process is done in two main stages. At power-on, the ROM code
(also called ROM monitor) executes the first-stage bootloader, known
as {\em fsbl}, from the internal SRAM. This stage is responsible for
initializing essential system components, including the DRAM.

Once the system is sufficiently initialized, the fsbl loads and
transfers control to the second-stage bootloader, known as {\em ssbl},
which is responsible for loading and launching the main operating
system (Linux).

In our setup, the {\em fsbl} is provided by TF-A BL2, and the {\em
ssbl} is U-Boot.

TF-A BL2 loads U-Boot from the Firmware Image Package ({\em FIP}),
which also includes other components such as the secure monitor (BL31)
and the OP-TEE secure OS. The FIP is generated as part of the TF-A
build process, which requires U-Boot and OP-TEE to be built
beforehand, so that it can be included in the package.

\section{U-Boot setup}

Download U-Boot from ST's Git repository:

\begin{bashinput}
$ git clone https://github.com/STMicroelectronics/u-boot.git
$ cd u-boot
$ git checkout v2023.10-stm32mp-r1
\end{bashinput}

Get an understanding of U-Boot's configuration and compilation steps
by reading the \code{README} file, and specifically the {\em Building
the Software} section.

Basically, you need to:

\begin{enumerate}

\item Specify the cross-compiler prefix
(the part before \code{gcc} in the cross-compiler executable name):
\bashcmd{$ export CROSS_COMPILE=aarch64-linux-}

\item Run \inlinebash{$ make <NAME>_defconfig}, where the list of
  available configurations can be found in the \code{configs/}
  directory. We will use the standard one (\code{stm32mp25}).

\item Now that you have a valid initial configuration, you can now run
  \inlinebash{$ make menuconfig} to further edit your bootloader
  features.

  \begin{itemize}
  \item In the \code{Environment} submenu, we will configure U-Boot so
    that it stores its environment inside a file called \code{uboot.env}
    in an ext4 filesystem:
    \begin{itemize}
    \item Disable \code{Environment is not stored}. We want changes to
      variables to be persistent across reboots
    \item Enable \code{Environment is in a EXT4 filesystem}. Disable
      all other options for environment storage (e.g. MMC, SPI, UBI,
      flash)
    \item The value for \code{Name of the block device for the
      environment} should be \code{mmc}
    \item The value for \code{Device and partition for where to store
      the environment in EXT4} should be \code{0:3}, which indicates
      we want to store the environment in the 3th partition of the
      first MMC device.
    \item The value for \code{Name of the EXT4 file to use for the
      environment} should be \code{/uboot.env}, which indicates the
      filename inside which the U-Boot environment will be stored
    \end{itemize}
  \end{itemize}

Install the following packages which should be needed to compile
U-Boot for your board:

\begin{bashinput}
$ sudo apt install libssl-dev device-tree-compiler swig \
       python3-dev python3-setuptools uuid-dev libgnutls28-dev
\end{bashinput}

\item Finally, run \bashcmd{make DEVICE_TREE=stm32mp257f-dk} which
  will build U-Boot
  \footnote{You can speed up the
  compiling by using the \code{-jX} option with \code{make}, where X
  is the number of parallel jobs used for compiling. Twice the
  number of CPU cores is a good value.}.
  The \code{DEVICE_TREE} variable specifies the specific
  Device Tree that describes our hardware board.

{\bf Note}: u-boot build may fail on your machine if you have a recent
version of python. Such issue is
\href{https://source.denx.de/u-boot/u-boot/-/commit/a63456b9191fae2fe49f4b121e025792022e3950}{already
fixed upstream}, but not in the version targeted for the training. To get the relevant fix,
you can cherry-pick the fix onto your local branch:

\code{git cherry-pick a63456b9191fae2fe49f4b121e025792022e3950}
\end{enumerate}

\section{OP-TEE setup}

OP-TEE is required on STM32MP2 platforms to provide a secure execution
environment during the boot process. It enables isolated execution of
sensitive operations such as key management and authentication.
Included in the FIP (Firmware Image Package), OP-TEE is essential for
booting on STM32MP2 platforms.

Get the ST-provided OP-TEE sources:

\begin{bashinput}
$ cd ..
$ git clone https://github.com/STMicroelectronics/optee_os.git
$ cd optee_os
$ git checkout 4.0.0-stm32mp-r1
\end{bashinput}

OP-TEE need some libraries to be built. You can install them with:
\begin{bashinput}
sudo apt install -y python3-pyelftools cmake python3-cryptography
\end{bashinput}

Then to build OP-TEE, several configuration parameters have to be
passed to the Makefile:

\begin{itemize}
\item Specify the cross-compiler prefix (the part before gcc in the
  cross-compiler executable name), either using the environment
  variable:\inlinebash{$ export CROSS_COMPILE=aarch64-linux-}, or just
  by adding it to the \code{make} commande line.
\item Specifies the cross-compiler prefix specifically for building
  the TEE core \inlinebash{$ export CROSS_COMPILE_core=aarch64-linux-}
\item Specifies the cross-compiler used for Trusted Applications (TAs)
  compiled for 64-bit ARM: \inlinebash{$ export
    CROSS_COMPILE_ta_arm64=aarch64-linux-}
\item Specifies which architectures to build user-mode TAs for :
  \inlinebash{$ export CFG_USER_TA_TARGETS=ta_arm64}
\item Disables support for the Trusted User Interface (TUI) framework:
  \inlinebash{$ export CFG_WITH_TUI=n}
\item Ensures that the OP-TEE core is built for 64-bit ARM
  architecture (AArch64).  It is required for platforms like STM32MP2,
  which are based on ARMv8 and operate in 64-bit mode: \inlinebash{$
    export CFG_ARM64_core=y}
\item Specify the output file : \inlinebash{$ export O=out}
\item Define the specific hardware target for the OP-TEE build
  \inlinebash{$ export PLATFORM=stm32mp2}
\end{itemize}

The full command line to build OP-TEE is therefore:

\begin{bashinput}
make O=out \
  CROSS_COMPILE=aarch64-linux- \
  CROSS_COMPILE_core=aarch64-linux- \
  CROSS_COMPILE_ta_arm64=aarch64-linux- \
  CFG_ARM64_core=y \
  CFG_USER_TA_TARGETS=ta_arm64 \
  PLATFORM=stm32mp2 \
  CFG_WITH_TUI=n \
  all
\end{bashinput}

Now that OP-TEE is built, we can move on to the next and final step:
TF-A.

\section{TF-A setup}

Get the ST-provided TF-A sources:

\begin{bashinput}
$ cd ..
$ git clone https://github.com/STMicroelectronics/arm-trusted-firmware.git
$ cd arm-trusted-firmware/
$ git checkout v2.10-stm32mp-r1
\end{bashinput}

We have to patch the TF-A sources to add more recent firmware code for
the DDR configuration:

\begin{bashinput}
$ git clone https://github.com/STMicroelectronics/stm32-ddr-phy-binary.git drivers/st/ddr/phy/firmware/bin
\end{bashinput}

Then to build TF-A, several configuration parameters have to be passed
to the Makefile:

\begin{itemize}
\item Specify the cross-compiler prefix (the part before gcc in the
  cross-compiler executable name), either using the environment
  variable:\inlinebash{$ export CROSS_COMPILE=aarch64-linux-}, or just
  by adding it to the \code{make} commande line.
\item The architecture has to be selected: \code{ARCH=aarch64}, as
  well as the major version of ARM architecture, here the Cortex A35
  is an ARMv8, so we need to use \code{ARM_ARCH_MAJOR=8}
\item The STM32MP2 platform is selected too with \code{PLAT=stm32mp2}
\item Sets the Secure Payload Dispatcher to OP-TEE: \code{SPD=opteed}
\item For this specific board, the device tree is generated and then
  needs to be specifed: \code{DTB_FILE_NAME=stm32mp257f-dk.dtb}
\item Specify the location of the OP-TEE BL32 image. Needed for
  trusted OS support:
  \code{BL32=../optee_os/out/core/tee-header_v2.bin}
\item Specify the additional OP-TEE image (paged memory region), which
  is part of the Trusted Execution Environment:
  \code{BL32_EXTRA1=../optee_os/out/core/tee-pager_v2.bin}
\item Specify the final part of the OP-TEE image (pageable region),
  which works with the pager image:
  \code{BL32_EXTRA2=../optee_os/out/core/tee-pageable_v2.bin}
\item Specify the Device Tree that will be used by U-Boot which is
  actually the Device Tree passed to U-Boot:
  \code{BL33_CFG=../u-boot/u-boot.dtb}
\item Specify that TF-A will be located on the SD card and therefore
  needs to have support for SD/MMC: \code{STM32MP_SDMMC=1}
\item Specify the location of the BL33 image, in our case U-Boot:
  \code{BL33=../u-boot/u-boot-nodtb.bin}
\item Select LPDDR4 memory configuration type:
  \code{STM32MP_LPDDR4_TYPE=1}
\end{itemize}

We can now generate the \code{all} and \code{fip} make targets with a
single (but long) command line:

\begin{bashinput}
$ make \
  CROSS_COMPILE=aarch64-linux- \
  PLAT=stm32mp2 ARM_ARCH_MAJOR=8 \
  ARCH=aarch64 \
  STM32MP_LPDDR4_TYPE=1 \
  DTB_FILE_NAME=stm32mp257f-dk.dtb \
  STM32MP_SDMMC=1 SPD=opteed \
  BL32=../optee_os/out/core/tee-header_v2.bin \
  BL32_EXTRA1=../optee_os/out/core/tee-pager_v2.bin \
  BL32_EXTRA2=../optee_os/out/core/tee-pageable_v2.bin \
  BL33=../u-boot/u-boot-nodtb.bin \
  BL33_CFG=../u-boot/u-boot.dtb \
  all fip
\end{bashinput}

At the end of the build, the important output files generated are
located in \code{build/stm32mp2/release/}. We will find there:

\begin{itemize}

\item \code{tf-a-stm32mp257f-dk.stm32}, which is TF-A BL2, serving as
  our first stage bootloader

\item \code{fip.bin}, which is the FIP image, which itself includes
  U-Boot and OP-TEE. This image will serve as the second stage
  bootloader.

\end{itemize}

\section{Flashing the bootloaders}

The ROM monitor in the STM32MP2 boot chain is the first piece of code
executed by the Cortex-A35 processor in secure mode. It selects the
boot source based on BOOT pins, OTP configuration, and TAMP registers,
and optionally performs authentication and decryption of the First
Stage Bootloader (FSBL). Once a valid FSBL (typically Trusted
Firmware-A BL2) is located, it is loaded into SYSRAM and executed. The
FSBL then initializes key system components such as the clock tree and
DDR controller, and proceeds to load the Secure Monitor (BL31), the
Secure OS (OP-TEE), and the Second Stage Bootloader (SSBL), which is
U-Boot. After all required components are securely loaded into DDR
memory, control is transferred to U-Boot, which takes over as the SSBL
to continue system initialization and load the Linux kernel.

In our case, all bootloader stages will be stored on the SD card, for
which a specific partitioning is necessary:

\begin{verbatim}
Number  Start    End        Size       File system  Name    Flags
1      34s      512s       479s                    fsbl
2      513s     8704s      8192s                   fip
3      8705s    129537s    120833s                 bootfs
\end{verbatim}

On your workstation, plug in the SD card your instructor gave you. Type
the \code{sudo dmesg} command to see which device is used by your
workstation. In case the device is \code{/dev/mmcblk0}, you will see
something like

\begin{verbatim}
[46939.425299] mmc0: new high speed SDHC card at address 0007
[46939.427947] mmcblk0: mmc0:0007 SD16G 14.5 GiB
\end{verbatim}

The device file name may be different (such as \code{/dev/sdb}
if the card reader is connected to a USB bus (either internally
or using a USB card reader).

In the following instructions, we will assume that your SD card is
seen as \code{/dev/mmcblk0} by your PC workstation.

Type the \code{mount} command to check your currently mounted
partitions. If SD partitions are mounted, unmount them:

\bashcmd{$ sudo umount /dev/mmcblk0p*}

We will erase the existing partition table and partition contents
by simply zero-ing the first 128 MiB of the SD card:

\bashcmd{$ sudo dd if=/dev/zero of=/dev/mmcblk0 bs=1M count=128}

Now, let's use the \code{parted} command to create the partitions that
we are going to use:

\bashcmd{$ sudo parted /dev/mmcblk0}

The ROM monitor handles {\em GPT} partition tables, let's create one:

\begin{verbatim}
(parted) mklabel gpt
\end{verbatim}

Then, the 4 partitions are created with:
\begin{verbatim}
(parted) mkpart fsbl 0% 512s
(parted) mkpart fip 513s 8704s
(parted) mkpart bootfs 8705s 129537s
\end{verbatim}

You can verify everything looks right with:

\begin{verbatim}
(parted) print
Model: SD SA08G (sd/mmc)
Disk /dev/mmcblk0: 15278080s
Sector size (logical/physical): 512B/512B
Partition Table: gpt
Disk Flags:

Number  Start    End        Size       File system  Name    Flags
 1      34s      512s       479s                    fsbl
 2      513s     8704s      8192s                   fip
 3      8705s    129537s    120833s                 bootfs
\end{verbatim}

Once done, quit:
\begin{verbatim}
(parted) quit
\end{verbatim}

{\em Note: \code{parted} is definitely not very user friendly compared
to other tools to manipulate partitions (such as \code{cfdisk}), but
that's the only tool which supports assigning names to GPT partitions.
In your projects, you could use \code{gparted}, which is a more
friendly graphical front-end on top of \code{parted}.}

Now, format the boot partition as an ext4 filesystem. This is where
U-Boot saves its environment:
\bashcmd{$ sudo mkfs.ext4 -L boot -O ^metadata_csum /dev/mmcblk0p3}

The \code{-O ^metadata_csum} option allows to create the filesystem
without enabling metadata checksums, which U-Boot doesn't support yet.

Now write the TF-A binary in the \code{fsbl} partition:

\begin{bashinput}
$ sudo dd if=/build/stm32mp2/release/tf-a-stm32mp257f-dk.stm32 of=/dev/mmcblk0p1 bs=1M conv=fdatasync
\end{bashinput}

Then flash the {\em fip} partition with the Firmware Image Package
containing U-Boot, the BL31 monitor and OP-TEE:

\begin{bashinput}
$ sudo dd if=build/stm32mp2/release/fip.bin of=/dev/mmcblk0p2 bs=1M conv=fdatasync
\end{bashinput}

\section{Testing the bootloaders}

Insert the SD card in the board slot. You can now power-up the board
by connecting the USB-C cable to the board, in CN6, \code{PWR_IN} and
to your PC at the other end. Check that it boots your new bootloaders.
You can verify this by checking the build dates:

\begin{verbatim}
  NOTICE:  CPU: STM32MP257FAK Rev.Y
  NOTICE:  Model: STMicroelectronics STM32MP257F-DK Discovery Board
  NOTICE:  Board: MB1605 Var1.0 Rev.C-01
  NOTICE:  BL2: v2.10-stm32mp2-r1.0(release):custom(75f8c318)
  NOTICE:  BL2: Built : 10:31:22, Apr  4 2025
  NOTICE:  BL2: Booting BL31
  NOTICE:  BL31: v2.10-stm32mp2-r1.0(release):custom(75f8c318)
  NOTICE:  BL31: Built : 10:31:22, Apr  4 2025
  I/TC: Early console on UART#2
  I/TC: 
  I/TC: Embedded DTB found
  I/TC: OP-TEE version: 4.0.0-stm32mp-r1-dev (gcc version 13.3.0 (crosstool-NG 1.27.0.20_329bb4d)) #7 Fri Apr  4 08:30:42 UTC 2025 aarch64
  I/TC: WARNING: This OP-TEE configuration might be insecure!
  I/TC: WARNING: Please check https://optee.readthedocs.io/en/latest/architecture/porting_guidelines.html
  I/TC: Primary CPU initializing
  I/TC: WARNING: All debug access are allowed
  I/TC: Override the OTP 124: 0 to 0x18db6
  I/TC: WARNING: Embeds insecure stm32mp_provisioning driver
  I/TC: UART console (non-secure)
  I/TC: PMIC STPMIC REFID:2.@ V1.1
  I/TC: Platform stm32mp2: flavor 257F_DK - DT stm32mp257f-dk.dts
  I/TC: OP-TEE ST profile: secure_and_system_services
  [    0.000000] SCP-firmware 2.13.0-intree-optee-os-4.0.0-stm32mp-r1-dev
  [    0.000000] 
  [    0.000000] [FWK] Module initialization complete!
  I/TC: Primary CPU switching to normal world boot
  I/TC: Reserved shared memory is disabled
  I/TC: Dynamic shared memory is enabled
  I/TC: Normal World virtualization support is disabled
  I/TC: Asynchronous notifications are enabled
  
  
  U-Boot 2023.10-stm32mp-r1 (Apr 03 2025 - 15:40:02 +0200)
  
  CPU: STM32MP257FAK Rev.Y
  Model: STMicroelectronics STM32MP257F-DK Discovery Board
  Board: stm32mp2 (st,stm32mp257f-dk)
  Board: MB1605 Var1.0 Rev.C-01
  DRAM:  4 GiB
  optee optee: OP-TEE: revision 4.0 (d9c4df97)
  I/TC: Reserved shared memory is disabled
  I/TC: Dynamic shared memory is enabled
  I/TC: Normal World virtualization support is disabled
  I/TC: Asynchronous notifications are enabled
  Core:  432 devices, 41 uclasses, devicetree: board
  WDT:   Started watchdog with servicing every 1000ms (32s timeout)
  NAND:  0 MiB
  MMC:   STM32 SD/MMC: 0, STM32 SD/MMC: 1
  Loading Environment from MMC... OK
  In:    serial
  Out:   serial
  Err:   serial
  Net:   eth0: eth1@482c0000
  FWU metadata read failed
  No EFI system partition
  No EFI system partition
  Failed to persist EFI variables
  Hit any key to stop autoboot:  0 
  Boot over mmc0!
  switch to partitions #0, OK
  mmc0 is current device
  STM32MP>  
\end{verbatim}

In U-Boot, type the \code{help} command, and explore the few commands
available.

\subsection{Adding a new command to the U-Boot shell}

Check whether the \code{config} command is available. This command
allows to dump the configuration settings U-Boot was compiled from.

If it's not, go back to U-Boot's configuration and enable it.

Re-run the build of U-Boot, and then re-run the build of TF-A so that
a new version of the \code{fip.bin} with the updated U-Boot is
generated.

Update the \code{fip} partition on the SD card with the new
\code{fip.bin} image and test that the command is now available and
works as expected.

\subsection{Playing with the U-Boot environment}

Display the U-Boot environment using \code{printenv}.

Set a new U-Boot variable \code{foo} to a value of your choice, using
\code{setenv}, and verify it has been set. Reset the board, and check
if \code{foo} is still defined: it should not.

Now repeat this process, but before resetting the board, use
\code{saveenv}. After the reset, check the \code{foo} variable is
still defined.

Now reset the environment to its default settings using \code{env
  default -a}, and save these changes using \code{saveenv}.

\section{Setting up networking}

The next step is to configure U-boot and your workstation to let your
board download files, such as the kernel image and Device Tree Binary
(DTB), using the TFTP protocol through a network connection.

With a network cable, connect the Ethernet port of
your board to the one of your computer. If your computer already has a
wired connection to the network, your instructor will provide you with
a USB Ethernet adapter. A new network interface should appear on your
Linux system.

\subsection{Network configuration on the target}

Let's configure networking in U-Boot:

\begin{itemize}
  \item \code{ipaddr}: IP address of the board
  \item \code{serverip}: IP address of the PC host
\end{itemize}

\begin{ubootinput}
=> setenv ipaddr 192.168.0.100
=> setenv serverip 192.168.0.1
\end{ubootinput}

Of course, make sure that this address belongs to a separate network
segment from the one of the main company network.

To make these settings permanent, save the environment:

\begin{ubootinput}
=> saveenv
\end{ubootinput}

\subsection{Network configuration on the PC host}

To configure your network interface on the workstation side, we need
to know the name of the network interface connected to your board.

Find the name of this interface by typing:
\bashcmd{=> ip a}

The network interface name is likely to be
\code{enxxx}\footnote{Following the {\em Predictable Network Interface
Names} convention:
\url{https://www.freedesktop.org/wiki/Software/systemd/PredictableNetworkInterfaceNames/}}.
If you have a pluggable Ethernet device, it's easy to identify as it's
the one that shows up after pluging in the device.

Then, instead of configuring the host IP address from NetworkManager's
graphical interface, let's do it through its command line interface,
which is so much easier to use:

\bashcmd{$ nmcli con add type ethernet ifname en... ip4 192.168.0.1/24}

\section{Setting up the TFTP server}

Let's install a TFTP server on your development workstation:

\begin{verbatim}
sudo apt install tftpd-hpa
\end{verbatim}

You can then test the TFTP connection. First, put a small text file in
the directory exported through TFTP on your development
workstation. Then, from U-Boot, do:

\begin{ubootinput}
=> tftp %\zimageboardaddr% textfile.txt
\end{ubootinput}

In case the download fails, make sure your host interface is correctly
configured and if a firewall is enabled make sure it does not filter out our
requests:
\begin{verbatim}
sudo ufw allow from 192.168.0.100
\end{verbatim}

Otherwise, the \code{tftp} command should have downloaded the
\code{textfile.txt} file from your development workstation into
the board's memory at location {\tt \zimageboardaddr}\footnote{
This location is part of the board DRAM. If you want
to check where this value comes from, you can check the SoC
datasheet at
\url{https://www.st.com/resource/en/reference_manual/dm00327659.pdf}.
It's a big document (more than 4,000 pages). In this document, look
for \code{Memory organization} and you will find the SoC memory map.
You will see that the address range for the memory controller
({\em DDRC})
starts at the address we are looking for.
You can also try with other values in the RAM address range.}.

You can verify that the download was successful by dumping the
contents of the memory:

\begin{ubootinput}
=> md %\zimageboardaddr%
\end{ubootinput}

We will see in the next labs how to use U-Boot to download, flash and
boot a kernel.

\section{Rescue binaries}

If you have trouble generating binaries that work properly, or later
make a mistake that causes you to lose your bootloader binaries, you
will find working versions under \code{data/} in the current lab
directory.
