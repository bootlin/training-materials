\subchapter{Bootloader optimizations}{Reduce bootloader execution time}

In this lab, we will run the final stage of boot time reduction:
\begin{itemize}
\item Improving the efficiency of the bootloader by optimizing its
usage
\item Recompiling the bootloader with the minimum set of options,
and even completely skip the second stage of the bootloader.
\end{itemize}

\section{Optimizing U-Boot usage}

By following the indications given in the lectures, start by optimizing
the way U-Boot is used.

At last, you can start by eliminating the infamous 2-second boot delay, something
you've surely been longing to do.

\section{Recompiling the bootloader}

It's now time to eliminate useless features in U-Boot. Go to
\code{~/boot-time-labs/bootloader/u-boot/} and run \code{make
menuconfig} to unselect features that we don't need in our system.

For the moment, don't touch the \code{SPL / TPL} options, as we will try
to use U-Boot's Falcon mode at the end.

In the same way you did when you reduced the kernel configuration,
do the changes {\bf progressively}, and even make backup copies of your
intermediate configurations (\code{.config} file). You will be glad you
did when you break U-Boot.

Once you have reached the minimum set of features, please measure boot
time and fill the below table:

\begin{tabular}{| l | l | r |}
  \hline
  Step & Duration & Description \\
  \hline
  \hline
  U-Boot SPL & & Between \code{U-Boot SPL 2021.04} and \code{U-Boot 2021.04} \\
  \hline
  U-Boot & & Between \code{U-Boot 2021.04} and \code{Starting kernel} \\
  \hline
  Kernel + Init scripts & & Between \code{Starting kernel} and \code{Starting ffmpeg} \\
  \hline
  Application & & Between \code{Starting ffmpeg} and \code{First frame decoded} \\
  \hline
  \hline
  Total & & \\
  \hline
\end{tabular}

\section{Using faster storage}

A last minute surprise: your instructor will give you new SD cards with
faster read performance, at least as fast as the Beagle Bone Black seems
to be able to go.

Why on earth didn't we use such SD cards right from the start of our
labs?

It's because slower storage acts as a magnifying glass (or as a slow
motion device) making it easier to observe elapsed time and the benefits
of our optimizations. If the storage was lightning fast, it would be
harder to appreciate speedups due to a small initramfs, for example.

So, edit the partition table of your new SD card, and create the
first partition in the same way as when you prepared your original SD
card. Then, copy the files over.

You can now go ahead and make tests again, and fill the table with your
latest results:

\begin{tabular}{| l | l | r |}
  \hline
  Step & Duration & Description \\
  \hline
  \hline
  U-Boot SPL & & Between \code{U-Boot SPL 2021.04} and \code{U-Boot 2021.04} \\
  \hline
  U-Boot & & Between \code{U-Boot 2021.04} and \code{Starting kernel} \\
  \hline
  Kernel + Init scripts & & Between \code{Starting kernel} and \code{Starting ffmpeg} \\
  \hline
  Application & & Between \code{Starting ffmpeg} and \code{First frame decoded} \\
  \hline
  \hline
  Total & & \\
  \hline
\end{tabular}

\section{Using U-Boot's {\em Falcon} mode}

It's now time to try U-Boot's capability to directly load the
Linux kernel from its first stage (SPL), instead of loading U-Boot.

What follows is based on U-Boot's own documentation in its sources:
\begin{itemize}
\item \projfile{u-boot}{doc/README.falcon} (generic details)
\item \projfile{u-boot}{board/ti/am335x/README} (specific details for boards with
      the am335x SoC)
\end{itemize}

The first thing to do is to generate a \code{uImage} file for the kernel
binary. This image file contains information that U-Boot uses to know a
few things about the kernel binary, most importantly the final load
address, but also the type of file (binary, script, environment file),
the target architecture and whether the binary is compressed or not.

This is called a {\em legacy image} for U-Boot. As you already know,
U-Boot can now boot a \code{zImage} file, but according to the Falcon
mode documentation, it does need a \code{uImage} file for SPL loading.

So, let's generate this file:
\begin{verbatim}
sudo apt install u-boot-tools
cd ~/boot-time-labs/kernel/linux/
make uImage LOADADDR=80008000
\end{verbatim}

Copy this \code{uImage} file to your SD card boot partition.

To save time, we are also going to recompile U-Boot without support for
loading the environment in the SPL file. Our own tests showed that this
saves about 250 ms!

So, in U-Boot's \code{menuconfig} file, go to the \code{SPL / TPL}
menu and unselect \code{Support an environment}. Compile U-Boot again
and copy the \code{u-boot.img} and \code{MLO} files to the boot
partition too.

Now, let's run the final preparation step. We will set the
\code{bootargs} environment variable, load the kernel and DTB, and
use U-Boot's \code{spl export} command to prepare a ready to boot record
with the DTB contents, the \code{bootargs}, the kernel loading addreses
and other information that Linux would need to boot. Note that the
U-Boot SPL will still load the \code{uImage} file from the FAT
filesystem in the first partition of the SD card.

In the below command, you'll see that we can use U-Boot's ready made
\code{loadaddr} and \code{fdtaddr} variables for addresses where to load
the kernel and DTB. At least this works with U-Boot for our board.

\begin{verbatim}
load mmc 0:1 ${loadaddr} uImage
load mmc 0:1 ${fdtaddr} dtb
setenv bootargs console=ttyO0,115200n8 rdinit=/playvideo
spl export fdt ${loadaddr} - ${fdtaddr}
\end{verbatim}

You can then see that \code{spl export} prepared everything to boot the
Linux kernel with the provided DTB, but didn't do it. At the end, it
tells you where the exported data were stored in RAM:

\begin{verbatim}
## Booting kernel from Legacy Image at 82000000 ...
   Image Name:   Linux-5.1.2-00001-gee667fd2c4d3
   Created:      2019-05-27  14:48:08 UTC
   Image Type:   ARM Linux Kernel Image (uncompressed)
   Data Size:    4664952 Bytes = 4.4 MiB
   Load Address: 80008000
   Entry Point:  80008000
   Verifying Checksum ... OK
## Flattened Device Tree blob at 88000000
   Booting using the fdt blob at 0x88000000
   Loading Kernel Image ... OK
   Loading Device Tree to 8ffee000, end 8ffffe57 ... OK
subcommand not supported
subcommand not supported
   Loading Device Tree to 8ffd9000, end 8ffede57 ... OK
Argument image is now in RAM: 0x8ffd9000
WARN: FDT size > CMD_SPL_WRITE_SIZE
\end{verbatim}

The last thing to do is to store such information in an \code{args} file
in the FAT partition on the MMC, using the starting RAM address provided
above and its size (\code{0x8ffede57 - 0x8ffd9000}):

\begin{verbatim}
fatwrite mmc 0:1 0x8ffd9000 args 1de57
\end{verbatim}

You're ready to go and reboot your board with the SD card inside.
You should not longer see the U-Boot second stage being loaded, but just
the SPL and the kernel.

If this doesn't work yet, please ask your instructor for advice and help.

When it works, update your table again:

\begin{tabular}{| l | l | r |}
  \hline
  Step & Duration & Description \\
  \hline
  \hline
  U-Boot SPL & & Between \code{U-Boot SPL 2021.04} and \code{Starting kernel} \\
  \hline
  Kernel + Init scripts & & Between \code{Starting kernel} and \code{Starting ffmpeg} \\
  \hline
  Application & & Between \code{Starting ffmpeg} and \code{First frame decoded} \\
  \hline
  \hline
  Total & & \\
  \hline
\end{tabular}

\section{Going further}

There are several things we can do to try to further optimize things:

\begin{itemize}
\item As our storage is now faster, it can be interesting to explore the
various kernel compression schemes again. The optimum solution may be a
different one.
\item Look for a solution to eliminate the delay detecting the USB
webcam.
\item If you don't manage to get rid of this delay, at least take
advantage of this spare time to show signs of life on the screen, by
implementing a splashscreen. You can even implement an animation.
One thing you can do is use BusyBox's \code{fbsplash} tool, to first
show an image on the framebuffer, and then even show a progress bar
(knowing how much time you have to wait for the camera to be ready).
\end{itemize}

