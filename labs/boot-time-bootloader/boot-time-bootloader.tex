\subchapter{Bootloader optimizations}{Reduce bootloader execution time}

In this lab, we will run the final stage of boot time reduction:
\begin{itemize}
\item Improving the efficiency of the bootloader by using faster storage
\item Configuring the bootloader so that it's second stage can be
skipped ({\em Falcon mode} in U-Boot).
\end{itemize}

\section{Optimizing U-Boot usage}

Here, we won't try to optimize U-Boot, because we are ultimately
going to skip its second stage. We could, but we don't need to.

\section{Using faster storage}

A last minute surprise: your instructor will give you new SD cards with
faster read performance, at least as fast as the Beagle Bone Black seems
to be able to go.

Why on earth didn't we use such SD cards right from the start of our
labs?

It's because slower storage acts as a magnifying glass (or as a slow
motion device) making it easier to observe elapsed time and the benefits
of our optimizations. If the storage was lightning fast, it would be
harder to appreciate speedups due to a small initramfs, for example.

So, edit the partition table of your new SD card, and create the
first partition in the same way as when you prepared your original SD
card. Then, copy the files over.

You can now go ahead and make tests again, and fill the
\code{~/__SESSION_NAME__-labs/results/faster-storage.ods} spreadsheet:
    
\includegraphics[width=\textwidth]{labs/boot-time-bootloader/faster-storage.png}

\section{Using U-Boot's {\em Falcon} mode}

It's now time to try U-Boot's capability to directly load the
Linux kernel from its first stage (SPL), instead of loading U-Boot.

What follows is based on U-Boot's own documentation in its sources:
\begin{itemize}
\item \projfile{u-boot}{doc/README.falcon} (generic details)
\item \projfile{u-boot}{board/ti/am335x/README} (specific details for boards with
      the am335x SoC)
\end{itemize}

The first thing to do is to generate a \code{uImage} file for the kernel
binary. This image file contains information that U-Boot uses to know a
few things about the kernel binary, most importantly the final load
address, but also the type of file (binary, script, environment file),
the target architecture and whether the binary is compressed or not.

This is called a {\em legacy image} for U-Boot. As you already know,
U-Boot can now boot a \code{zImage} file, but according to the Falcon
mode documentation, it does need a \code{uImage} file for SPL loading.

So, let's generate this file:
\begin{bashinput}
cd ~/%\training%-%\board%-labs/kernel/linux/
make uImage LOADADDR=80008000
\end{bashinput}

Copy this \code{uImage} file to your SD card boot partition.

We also need a few features to be enabled in the U-Boot SPL. So start
U-Boot's \code{make menuconfig}:
\begin{itemize}
\item In the \code{SPL / TPL} menu,
      \begin{itemize}
      \item Enable {\em Support SPL loading and booting of Legacy images}
	    (\code{CONFIG_SPL_LEGACY_IMAGE_SUPPORT=y}, now replaced by
	    \projconfig{u-boot}{CONFIG_SPL_LEGACY_IMAGE_FORMAT} in the latest
	    versions of U-Boot).
            That's needed to support loading the legacy \code{uImage} file.
      \item Also make sure \projconfigval{u-boot}{CONFIG_SPL_OS_BOOT}{y}
            ({\em Activate Falcon Mode}).
            That's the case for our current configuration, but configurations
            for other boards may not have this by default.
      \item Set \projconfigval{u-boot}{CONFIG_CMD_SPL_WRITE_SIZE}{0x20000} to have
	    enough space for the \code{spl export} output. This is actually
	    only to avoid a warning, as this setting is only used for
	    Falcon booting from NAND and NOR flash.
      \end{itemize}
\end{itemize}

Also modify the \projfunc{u-boot}{spl_start_uboot} function in the
\projfile{u-boot}{board/ti/am335x/board.c}, to remove the block
of lines under \code{#ifdef CONFIG_SPL_ENV_SUPPORT}. There seems to be
issues loading the environment from FAT, and loading an environment file
from a file is expensive anyway. Note that in the current U-Boot code,
we cannot disable \projconfig{u-boot}{CONFIG_SPL_ENV_SUPPORT}, because
it breaks some driver code.

Look at the rest of the \projfile{u-boot}{board/ti/am335x/board.c} file:
we can still fall back to starting U-Boot if we send a \code{c} key
in the serial console.

Compile U-Boot again and copy the \code{u-boot.img} and \code{MLO}
files to the boot partition.

Now, let's run the final preparation step. We will set the
\code{bootargs} environment variable, load the kernel and DTB, and
use U-Boot's \code{spl export} command to prepare a ready to boot record
with the DTB contents, the \code{bootargs}, the kernel loading addreses
and other information that Linux would need to boot. Note that the
U-Boot SPL will still load the \code{uImage} file from the FAT
filesystem in the first partition of the SD card.

In the below command, you'll see that we can use U-Boot's ready made
\code{loadaddr} and \code{fdtaddr} variables for addresses where to load
the kernel and DTB. At least this works with U-Boot for our board.

\begin{verbatim}
load mmc 0:1 ${loadaddr} uImage
load mmc 0:1 ${fdtaddr} dtb
setenv bootargs console=ttyS0,115200n8 rdinit=/playvideo lpj=4980736
spl export fdt ${loadaddr} - ${fdtaddr}
\end{verbatim}

You can then see that \code{spl export} prepared everything to boot the
Linux kernel with the provided DTB, but didn't do it. At the end, it
tells you where the exported data were stored in RAM:

\begin{verbatim}
## Booting kernel from Legacy Image at 82000000 ...
   Image Name:   Linux-<version>
   Created:      2022-04-29  17:47:56 UTC
   Image Type:   ARM Linux Kernel Image (uncompressed)
   Data Size:    3008496 Bytes = 2.9 MiB
   Load Address: 80008000
   Entry Point:  80008000
   Verifying Checksum ... OK
## Flattened Device Tree blob at 88000000
   Booting using the fdt blob at 0x88000000
   Loading Kernel Image
   Loading Device Tree to 8ffec000, end 8ffffce6 ... OK
subcommand not supported
subcommand not supported
   Loading Device Tree to 8ffd5000, end 8ffebce6 ... OK
Argument image is now in RAM: 0x8ffd5000
\end{verbatim}

The last thing to do is to store such information in an \code{args} file
in the FAT partition on the MMC. Through the \code{fdtargsaddr} and
\code{fdtargsaddr} environment variables, we know where to copy the
data from and their size:

\begin{verbatim}
fatwrite mmc 0:1 ${fdtargsaddr} args ${fdtargslen}
\end{verbatim}

You're ready to go and reboot your board with the SD card inside.
You should not longer see the U-Boot second stage being loaded, but just
the SPL and the kernel.

If this doesn't work yet, please ask your instructor for advice and help.

You can also try to send \code{c} characters at reset time to make
the SPL fall back to loading U-Boot.

When it works, fill the \code{~/__SESSION_NAME__-labs/results/final-results.ods}
spreadsheet:

\includegraphics[width=\textwidth]{labs/boot-time-bootloader/final-results.png}

\section{Going further}

There are several things we can do to try to further optimize things:

\begin{itemize}
\item Remove the warnings and information messages from the SPL
      to save a bit of time.
\item Reduce the features of the U-Boot SPL to make it load faster.
\item Make the SPL load the \code{uImage} file directly from raw MMC,
      so that there is no FAT intermediate layer.
\item As our storage is now faster, it can be interesting to explore the
      various kernel compression schemes again. The optimum solution may be a
      different one.
\item Look for a solution to eliminate the delay detecting the USB
      webcam.
\item If you don't manage to get rid of this delay, an idea is to modify
      \code{ffmpeg} to wait for the video device once it's ready to
      access it. This way, you start loading and initializing
      \code{ffmpeg} right away, and the video can be played as soon as
      the video device is ready. Parallelism always helps!
\end{itemize}

