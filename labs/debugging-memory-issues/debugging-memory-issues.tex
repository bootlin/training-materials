\subchapter
{Debugging memory issues}
{Objectives:
  \begin{itemize}
    \item Memory leak and misbehavior detection with {\em valgrind} and
            {\em vgdb}.
    \item Visualizing application heap using {\em Massif}.
  \end{itemize}
}

\section{valgrind \& vgdb}

Go into the \code{valgrind folder} and compile \code{valgrind.c} with debugging
information using:

\begin{bashinput}
$ cd /home/<user>/debugging-labs/nfsroot/root/valgrind
$ ${CROSS_COMPILE}-gcc -g3 valgrind.c -o test_valgrind
\end{bashinput}

Then run it on the target. Do you notice any problem ? Does it run correctly ?
Even though there is no segfault, an application might have some memory leaks
or even out-of-bounds accesses, uninitialized memory, etc.

Now, run the command again with valgrind using the following command:

\begin{bashinput}
$ valgrind --leak-check=full ./test_valgrind
\end{bashinput}

You'll see various errors found by valgrind
\begin{itemize}
  \item Invalid memory write
  \item Uninitialized memory
  \item Memory leaks
\end{itemize}

In order to pinpoint exactly each error and be able to disable with gdb, vgdb
can be used. We will do that remotely on the host using gdb-multiarch. First, we
need to run valgrind with vgdb enabled on the target:

\begin{bashinput}
$ cd /root/valgrind
$ valgrind --vgdb-error=0 --leak-check=full ./test_valgrind
\end{bashinput}

Then, in order to do remote debugging, we also need to run vgdb in listen mode.
Start another terminal in SSH on the target and run the following command:

\begin{bashinput}
$ vgdb --port=1234
\end{bashinput}

On the computer side, start gdb-multiarch and give it the \code{test_valgrind}
binary which will allow to detect the architecture and read symbols:
\begin{bashinput}
$ gdb-multiarch ./test_valgrind
\end{bashinput}

Finally, we'll need to connect to vgdb using the following gdb command:
\begin{bashinput}
(gdb) target remote <target_ip>:1234
\end{bashinput}

You will then be able to debug each error using gdb and valgrind will interrupt
the program each time it detects an error. Try to solve all the problems that
were encountered by valgrind.

NOTE: The backtrace for leaks is not shown on the target because all libraries
are stripped and thus do not have any debugging symbols anymore. This leads to
the impossibility to use the dwarf information for backtracing.

\section{Massif}

Massif is really helpful to understand what is going on the memory allocation
side of an application. Compile the \code{heap_profile} example that we did provide
using the following command:

\begin{bashinput}
$ cd /home/<user>/debugging-labs/nfsroot/root/heap_profile
$ ${CROSS_COMPILE}-gcc -g3 heap_profile.c -o heap_profile
\end{bashinput}

Once compile, on the target run it under massif using the following command:

\begin{bashinput}
$ cd /root/heap_profile
$ valgrind --tool=massif --time-unit=B ./heap_profile
\end{bashinput}

NOTE: we use \code{--time-unit=B} to set the X axis to be based on the allocated
size.

Once done, a \code{massif.out.<pid>} file will be created. This file can be 
displayed with \code{ms_print}. Based on the result, can you answer the
following questions:
\begin{itemize}
  \item What is the peak allocation size of this program ?
  \item How much memory was allocated during the program lifetime ?
  \item Do we have memory leaks at the end of execution ?
\end{itemize}

Note: \code{heaptrack} is not available on buildroot but is available on debian.
You can try the same experience using heaptrack on your computer and visualizing
the results with \code{heaptrack_gui}.