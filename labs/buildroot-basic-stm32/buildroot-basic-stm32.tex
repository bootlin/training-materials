\subchapter
{Basic Buildroot usage}
{Objectives:
  \begin{itemize}
  \item Get Buildroot
  \item Configure a minimal system with Buildroot for the STM32MP1 Discovery Kit 1
  \item Do the build
  \item Prepare the STM32MP1 Discovery Kit 1 for usage
  \item Flash and test the generated system
  \end{itemize}
}

\section{Setup}

Go to the \code{$HOME/__SESSION_NAME__-labs/} directory.

As specified in the Buildroot
manual\footnote{\url{https://buildroot.org/downloads/manual/manual.html\#requirement-mandatory}},
Buildroot requires a few packages to be installed on your
machine. Let's install them using Ubuntu's package manager:

\begin{bashinput}
sudo apt install sed make binutils gcc g++ bash patch \
  gzip bzip2 perl tar cpio python3 unzip rsync wget libncurses-dev
\end{bashinput}

\section{Download Buildroot}

Since we're going to do Buildroot development, let's clone the
Buildroot source code from its Git repository:

\begin{bashinput}
git clone https://gitlab.com/buildroot.org/buildroot.git
\end{bashinput}

Go into the newly created \code{buildroot} directory.

We're going to start a branch from the {\em 2025.02.6} Buildroot
release, with which this training has been tested.

\begin{bashinput}
git checkout -b bootlin 2025.02.6
\end{bashinput}

\section{Configuring Buildroot}

If you look under \code{configs/}, you will see that there is a file
named \code{stm32mp157a_dk1_defconfig}, which is a ready-to-use Buildroot
configuration file to build a system for the STM32MP1 Discovery Kit 1
platform. However, since we want to learn about Buildroot, we'll start
our own configuration from scratch!

Start the Buildroot configuration utility:

\begin{bashinput}
make menuconfig
\end{bashinput}

Of course, you're free to try out the other configuration utilities
\code{nconfig}, \code{xconfig} or \code{gconfig}.

Now, let's do the configuration:

\begin{itemize}

\item \code{Target Options} menu

  \begin{itemize}

  \item The STM32MP1 is an ARM based processor, so select \code{ARM
      (little endian)} as the target architecture.

  \item According to the STM32MP1 Discovery Kit 1 website at
    \url{https://www.st.com/en/evaluation-tools/stm32mp157a-dk1.html},
    the STM32MP157 SoC is based on a dual ARM Cortex-A7 core. So
    select \code{cortex-A7} as the \code{Target Architecture Variant}.

  \item On ARM two {\em Application Binary Interfaces} are available:
    \code{EABI} and \code{EABIhf}. Unless you have backward
    compatibility concerns with pre-built binaries, \code{EABIhf} is
    more efficient, so make this choice as the \code{Target ABI}
    (which should already be the default anyway).

  \item The other parameters can be left to their default value:
    \code{ELF} is the only available \code{Target Binary Format},
    \code{VFPv4} is the appropriate setting for the \code{Floating
      Point Strategy} and using the \code{ARM} instruction set is also
    a good default (we could use the \code{Thumb-2} instruction set
    for slightly more compact code).

  \end{itemize}

\item We don't have anything special to change in the
  \code{Build options} menu, but take nonetheless this opportunity to
  visit this menu, and look at the available options. Each option has
  a help text that tells you more about the option.

\item \code{Toolchain} menu

  \begin{itemize}

  \item By default, Buildroot builds its own toolchain. This takes
    quite a bit of time, and for \code{ARMv7} platforms, there is a
    pre-built toolchain provided by ARM. We'll use it through the
    {\em external toolchain} mechanism of Buildroot. Select
    \code{External toolchain} as the \code{Toolchain type}. Do not
    hesitate however to look at the available options when you select
    \code{Buildroot toolchain} as the \code{Toolchain type}.

  \item Select \code{Bootlin toolchains} as the \code{Toolchain}. It
    will automatically select the \code{armv7-eabihf glibc
      bleeding-edge 2024.05-1} variant, which is fine for our
    needs. Buildroot can either use pre-defined toolchains such as the
    ones provided by ARM, or custom toolchains (either downloaded from
    a given location, or pre-installed on your machine).

  \end{itemize}

\item \code{System configuration} menu

  \begin{itemize}

  \item For our basic system, we don't need a lot of custom {\em
      system configuration} for the moment. So take some time to look
    at the available options, and put some custom values for the
    \code{System hostname}, \code{System banner} and \code{Root
      password}.

  \end{itemize}

\item \code{Kernel} menu

  \begin{itemize}

  \item We obviously need a Linux kernel to run on our platform, so
    enable the \code{Linux kernel} option.

  \item By default, the most recent Linux kernel version available at
    the time of the Buildroot release is used. In our case, we want to
    use a specific version, to make sure our build is reproducible. So
    select \code{Custom version} as the \code{Kernel version}, and
    enter \code{6.12.47} in the \code{Kernel version} text field that
    appears.

  \item Now, we need to define which kernel configuration to use. The
    Linux kernel comes with a number of pre-defined configurations for
    ARM 32-bit platforms, in
    \url{https://git.kernel.org/cgit/linux/kernel/git/torvalds/linux.git/tree/arch/arm/configs/?id=v6.12}. However,
    the only configuration that supports STM32MP1 is
    \code{multi_v7_defconfig}, which has support for all ARMv7
    platforms, not just the STM32MP1, making it a very large
    configuration, with a long build time and a large kernel image.

    So instead, we will use our own custom Linux kernel configuration,
    which we are providing in
    \code{$HOME/__SESSION_NAME__-labs/buildroot-basic/linux-stm32mp1.config}.

    Create the directory \code{board/bootlin/stm32mp1} in the Buildroot
    source tree. We will use it to store all the files that are
    directly specific to our project. Copy the Linux kernel
    configuration file to this location.

    Then, tell Buildroot to use it as the Linux kernel configuration
    file: choose \code{Using a custom (def)config fil} in \code{Kernel
      configuration}, and set \code{Configuration file path} to
    \code{board/bootlin/stm32mp1/linux-stm32mp1.config}.

  \item The \code{Kernel binary format} is the next option. Since we
    are going to use a recent U-Boot bootloader, we'll keep the
    default of the \code{zImage} format.

  \item On ARM, all modern platforms now use the {\em Device Tree} to
    describe the hardware. The STM32MP1 Discovery kit 1 is in this
    situation, so you'll have to enable the \code{Build a Device Tree
    Blob}
    option. At \url{https://git.kernel.org/cgit/linux/kernel/git/torvalds/linux.git/tree/arch/arm/boot/dts/?id=v6.12},
    you can see the list of all Device Tree files available in the
    6.12 Linux kernel (note: the Device Tree files for boards use
    the \code{.dts} extension). The one for the STM32MP1 Discovery kit
    1 is \code{stm32mp157a-dk1.dts} located in the \code{st/}
    subdirectory. Even if talking about Device Tree is beyond the
    scope of this training, feel free to have a look at this file to
    see what it contains. Back in Buildroot, enable \code{Build a
    Device Tree Blob (DTB)} and type \code{st/stm32mp157a-dk1} as
    the \code{In-tree Device Tree Source file names}.

  \item The kernel configuration for this platform requires having
    OpenSSL available on the host machine. To avoid depending on the
    OpenSSL development files installed by your host machine Linux
    distribution, Buildroot can build its own version: just enable the
    \code{Needs host OpenSSL} option.

  \end{itemize}

\item \code{Target packages} menu. This is probably the most important
  menu, as this is the one where you can select amongst the 3000+
  available Buildroot packages which ones should be built and
  installed in your system. For our basic system, enabling
  \code{BusyBox} is sufficient and is already enabled by default, but
  feel free to explore the available packages. We'll have the
  opportunity to enable some more packages in the next labs.

\item \code{Filesystem images} menu. For now, keep only the \code{tar
    the root filesystem} option enabled. We'll take care separately of
  flashing the root filesystem on the SD card.

\item \code{Bootloaders} menu. The STM32MP1 platform supports two
  different booting possibilities: the {\em basic} boot strategy,
  where U-Boot implements both the first stage and second stage
  bootloader, and the {\em trusted} boot strategy, where TF-A replaces
  U-Boot as the first stage bootloader. For the sake of simplicity in
  this initial configuration, we will use the {\em basic} boot
  strategy.

  \begin{itemize}

  \item We'll use the most popular ARM bootloader, {\em U-Boot}, so
    enable it in the configuration.

  \item Select \code{Kconfig} as the \code{Build system}. U-Boot is
    transitioning from a situation where all the hardware platforms
    were described in C header files to a system where U-Boot re-uses
    the Linux kernel configuration logic. Since we are going to use a
    recent enough U-Boot version, we are going to use the latter,
    called {\em Kconfig}.

  \item Use the custom version of U-Boot \code{2025.07}.

  \item Look at
    \url{https://gitlab.denx.de/u-boot/u-boot/-/tree/v2025.07/configs}
    to identify the available U-Boot configurations. For many STM32MP1
    platforms, U-Boot has a single configuration called
    \code{stm32mp15_basic_defconfig}, which can then be given the exact
    hardware platform to support using a Device Tree. So we need to
    use \code{stm32mp15_basic} as \code{Board defconfig} and
    \code{DEVICE_TREE=st/stm32mp157a-dk1} as \code{Custom make options}

  \item U-Boot on STM32MP1 is split in two parts: the first stage
    bootloader called \code{u-boot-spl.stm32} and the second stage
    bootloader called \code{u-boot.img}. So, select \code{u-boot.img}
    as the \code{U-Boot binary format}, enable \code{Install U-Boot
      SPL binary image} and use \code{u-boot-spl.stm32} as the
    \code{U-Boot SPL binary image name}.

  \item The U-Boot build with need the Python interpreter and the {\em
    pylibfdt} library, so we need to enable the option \code{U-Boot
    needs pylibfdt}. If you wonder how you can know these details: you
    adjust these options after seeing build failures due to missing
    dependencies.

  \end{itemize}

\end{itemize}

You're now done with the configuration!

\section{Building}

You could simply run \code{make}, but since we would like to keep a
log of the build, we'll redirect both the standard and error outputs
to a file, as well as the terminal by using the \code{tee} command:

\begin{bashinput}
make 2>&1 | tee build.log
\end{bashinput}

While the build is on-going, please go through the following sections
to prepare what will be needed to test the build results.

\input{../common/stm32-prepare.tex}

\section{Preparing the SD card}

In order for the system to boot, we need to prepare the SD card with a
specific partition scheme, described by a {\em GPT} partition table:

\begin{verbatim}
Number  Start     End       Size      File system  Name   Flags
 1      1024kiB   2048kiB   1024kiB                fsbl1
 2      2048kiB   3072kiB   1024kiB                fsbl2
 3      3072kiB   5120kiB   2048kiB                ssbl
 4      5120kiB   21504kiB  16384kiB               boot
 5      21504kiB  87040kiB  65536kiB               root
\end{verbatim}

Plug the SD card your instructor gave you on your workstation. Type
the \code{dmesg} command to see which device is used by your
workstation. In case the device is \code{/dev/mmcblk0}, you will see
something like

\begin{verbatim}
[46939.425299] mmc0: new high speed SDHC card at address 0007
[46939.427947] mmcblk0: mmc0:0007 SD16G 14.5 GiB
\end{verbatim}

The device file name may be different (such as \code{/dev/sdb})
if the card reader is connected to a USB bus (either internally
or using a USB card reader).

In the following instructions, we will assume that your SD card is
seen as \code{/dev/mmcblk0} by your PC workstation.

Type the \code{mount} command to check your currently mounted
partitions. If SD partitions are mounted, unmount them:

\bashcmd{$ sudo umount /dev/mmcblk0p*}

Then, clear possible SD card contents remaining from previous use
(only the first few megabytes matter):

\bashcmd{$ sudo dd if=/dev/zero of=/dev/mmcblk0 bs=1M count=32}

Now, let's use the \code{parted} command to create the partitions that
we are going to use:

\bashcmd{$ sudo parted /dev/mmcblk0}

Specify the type of partition table, here {\em GPT}:

\begin{verbatim}
(parted) mklabel gpt
\end{verbatim}

Then, the 4 partitions are created with:
\begin{verbatim}
(parted) mkpart fsbl1 0% 4095s
(parted) mkpart fsbl2 4096s 6143s
(parted) mkpart ssbl 6144s 10239s
(parted) mkpart boot 10240s 43007s
(parted) mkpart root 43008s 174079s
\end{verbatim}

Let's mark the 4th partition as the boot partition:

\begin{verbatim}
(parted) toggle 4 boot
\end{verbatim}

You can check that the overall partitioning looks correct:

\begin{verbatim}
(parted) unit kiB
(parted) print
Number  Start     End       Size      File system  Name   Flags
 1      1024kiB   2048kiB   1024kiB                fsbl1
 2      2048kiB   3072kiB   1024kiB                fsbl2
 3      3072kiB   5120kiB   2048kiB                ssbl
 4      5120kiB   21504kiB  16384kiB               boot   boot, esp
 5      21504kiB  87040kiB  65536kiB               root
\end{verbatim}

Now everything should be ready. Hopefully by that time the Buildroot
build should have completed. If not, wait a little bit more.

Format the {\em boot} partition as a FAT filesystem:

\begin{bashinput}
sudo mkfs.vfat -F 32 -n boot /dev/mmcblk0p4
\end{bashinput}

Format the {\em root} partition as an ext4 filesystem:

\begin{bashinput}
sudo mkfs.ext4 -L rootfs -E nodiscard /dev/mmcblk0p5
\end{bashinput}

Eject and re-insert the SD card, so that your Linux system
automatically mounts the {\em boot} and {\em root} partitions.

\section{Flash the system}

Once Buildroot has finished building the system, it's time to put it
on the SD card:

\begin{itemize}

\item Flash the first stage bootloader, U-Boot SPL, in both the
  \code{fsbl1} and \code{fsbl2} partitions:

\begin{bashinput}
$ sudo dd if=output/images/u-boot-spl.stm32 of=/dev/mmcblk0p1 bs=1M conv=fdatasync
$ sudo dd if=output/images/u-boot-spl.stm32 of=/dev/mmcblk0p2 bs=1M conv=fdatasync
\end{bashinput}

\item Flash the second stage bootloader, U-Boot, in the \code{ssbl}
  partition:

\begin{bashinput}
$ sudo dd if=output/images/u-boot.img of=/dev/mmcblk0p3 bs=1M conv=fdatasync
\end{bashinput}

\item Copy the Linux kernel image and Device Tree Blob to the FAT filesystem:

\begin{bashinput}
$ cp output/images/zImage output/images/stm32mp157a-dk1.dtb /media/$USER/boot
\end{bashinput}

\item Create a file named \code{extlinux/extlinux.conf} in the
  \code{boot} partition. This file should contain the following lines:

{\small
\begin{fileinput}
label buildroot
  kernel /zImage
  devicetree /stm32mp157a-dk1.dtb
  append console=ttySTM0,115200 root=/dev/mmcblk0p5 rootwait
\end{fileinput}
}

These lines teach the U-Boot bootloader how to load the Linux kernel
image and the Device Tree, before booting the kernel. It uses a
standard U-Boot mechanism called {\em distro boot command}, see
\url{https://source.denx.de/u-boot/u-boot/-/raw/master/doc/README.distro}
for more details.

\item Extract the \code{rootfs.tar} file to the \code{rootfs}
  partition of the SD card, using:\\
  \inlinebash{sudo tar -C /media/$USER/rootfs/ -xf output/images/rootfs.tar}.

\end{itemize}

Cleanly unmount the two SD card partitions, and eject the SD card.

\section{Boot the system}

Insert the SD card in the STM32MP1 DK1 board and power on the board
(or reset the board if it's already powered on).

You should see your system booting. Make sure that the U-Boot SPL and
U-Boot version and build dates match with the current date. Do the
same check for the Linux kernel.

Login as \code{root} on the board, and explore the system. Run
\code{ps} to see which processes are running, and look at what
Buildroot has generated in \code{/bin}, \code{/lib}, \code{/usr} and
\code{/etc}.

\section{Explore the build log}

Back to your build machine, since we redirected the build output to a
file called \code{build.log}, we can now have a look at it to see what
happened. Since the Buildroot build is quite verbose, Buildroot prints
before each important step a message prefixed by the \code{>>>}
sign. So to get an overall idea of what the build did, you can run:

\begin{bashinput}
grep ">>>" build.log
\end{bashinput}

You see the different packages between downloaded, extracted, patched,
configured, built and installed.

Feel free to explore the \code{output/} directory as well.
