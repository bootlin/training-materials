\subchapter{Kernel source code}{Objective: Get familiar with the
  kernel source code}

After this lab, you will be able to:

\begin{itemize}

\item Create a branch based on a remote tree to explore a particular
      stable kernel version (from the \code{stable} kernel tree).
\item Explore the sources and search for files, function headers or
  other kinds of information\ldots
\item Browse the kernel sources with a tool like Elixir.
\end{itemize}

\section{Choose a particular stable version}

Let's work with a particular stable version of the Linux kernel.
It would have been more logical to do this in the previous lab, but we
wanted to get back to lectures while the \code{fetch} command was
running.

First, let's get the list of branches on our \code{stable} remote tree:

\begin{verbatim}
cd ~/__SESSION_NAME__-labs/src/linux
git branch -a
\end{verbatim}

As we will do our labs with the Linux \workingkernel\ stable branch, the remote branch
we are interested in is \texttt{remotes/stable/linux-\workingkernel.y}.

First, execute the following command to check which version you
currently have:

\begin{verbatim}
make kernelversion
\end{verbatim}

You can also open the \code{Makefile} and look at the beginning of it
to check this information.

Now, let's create a local branch starting from that remote branch:

\texttt{git checkout -b bootlin-labs stable/linux-\workingkernel.y}

Check the version again using the \code{make kernelversion} command
to make sure you now have a \workingkernel.y version.

\if\defstring{\labboard}{imx93-frdm}
Before exploring the sources, we will apply a patch to add the device
tree for the \code{i.MX93-FRDM} board in \code{arch/arm64/boot/dts
freescale}, as it is not yet present in the Linux sources.
\begin{verbatim}
git apply ../data/0001-imx93-frdm.patch
\end{verbatim}
After that, modify the \code{Makefile} in the \code{freescale} directory to build the device tree for the \code{i.MX93-FRDM} board.
\fi

\section{Exploring the sources manually}

As a Linux kernel user, you will very often need to find which file
implements a given function. So, it is useful to be familiar with
exploring the kernel sources.

\begin{enumerate}
\item Find the Linux logo image in the sources\footnote{Look for
      files in \code{logo} in their name. It's an
      opportunity to practise with the \code{find} command.}.
\item Find who the maintainer of the MVNETA network driver is.
\item Find the declaration of the \kfunc{platform_device_register} function.
\end{enumerate}

Tip: if you need the \code{grep} command, we advise you to use \code{git
grep}. This command is similar, but much faster, doing the search only
on the files managed by git (ignoring git internal files and generated
files).

\section{Use a kernel source indexing tool}

Now that you know how to do things in a manual way, let's use more
automated tools.

Try Elixir at \url{https://elixir.bootlin.com}
and choose the Linux version closest to yours.

As in the previous section, use this tool to find where
the \kfunc{platform_device_register} function is declared, implemented and
even used.
