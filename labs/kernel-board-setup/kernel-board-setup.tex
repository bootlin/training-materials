\subchapter{Board setup}{Objective: setup communication
with the board and configure the bootloader.}

After this lab, you will be able to:
\begin{itemize}
\item Access the board through its serial line
\item Create a bootable micro-SD card to boot the 
      board with
\item Configure the U-boot bootloader and a tftp server
      on your workstation to download files through tftp
\end{itemize}

\section{Getting familiar with the board}

\begin{center}
\includegraphics[width=8cm]{labs/kernel-board-setup/beaglebone-black-connectors.jpg}
\end{center}

Take some time to read about the board features and connectors on
\url{http://circuitco.com/support/index.php?title=BeagleBoneBlack}. The
above image was taken from this page.

Then download the board System Reference Manual found at 
\url{https://github.com/CircuitCo/BeagleBone-Black/blob/master/BBB_SRM.pdf?raw=true}
\footnote{There is a link to this manual on the above page}.
This is the ultimate reference about the board, giving all the details
about the design of the board and the components which were chosen.
You don't have to start reading this document now but you will need it
during the practical labs.

Last but not least, download the Technical Reference Manual (TRM) for
the TI AM3359 SoC, available on \url{http://www.ti.com/product/am3359}.
This document is more than 4700 pages big (20 MB)! You will need it
too during the practical labs. 

Don't hesitate to share your questions with the instructor.

\section{Setting up serial communication with the board}

The Beaglebone serial connector is exported on the 6 pins close to one
of the 48 pins headers. Using your special USB to Serial adaptor provided
by your instructor, connect the ground wire (blue) to the pin closest
to the power supply connector (let's call it pin 1), and the \code{TX} (red)
and \code{RX} (green) wires to the pins 4 (\code{RX}) and 5 (\code{TX}).

\footnote{See
\url{https://www.olimex.com/Products/Components/Cables/USB-Serial-Cable/USB-Serial-Cable-F/}
for details about the USB to Serial adaptor that we are using.} 

You always should make sure that you connect the \code{TX} pin of the cable
to the \code{RX} pin of the board, and vice versa, whatever the board and
cables that you use.

Once the USB to Serial connector is plugged in, a new serial port
should appear: \code{/dev/ttyUSB0}.  You can also see this device
appear by looking at the output of \code{dmesg}.

To communicate with the board through the serial port, install a
serial communication program, such as \code{picocom}:

\begin{verbatim}
sudo apt-get install picocom
\end{verbatim}

If you run \code{ls -l /dev/ttyUSB0}, you can also see that only
\code{root} and users belonging to the \code{dialout} group have
read and write access to this file. Therefore, you need to add your user
to the \code{dialout} group:

\begin{verbatim}
sudo adduser $USER dialout
\end{verbatim}

You now need to log out and log in again to make the new group
visible everywhere.

Now, you can run \code{picocom -b 115200 /dev/ttyUSB0}, to start serial
communication on \code{/dev/ttyUSB0}, with a baudrate of \code{115200}. If
you wish to exit \code{picocom}, press \code{[Ctrl][a]} followed by
\code{[Ctrl][x]}.

There should be nothing on the serial line so far, as the board is not
powered up yet.

Before booting your board, make sure that there is no micro-SD card
in the corresponding slot.

It is now time to power up your board by pluging in the mini-USB
cable supplied by your instructor (with your PC or a USB power supply at the
other end of the cable).

See what messages you get on the serial line.

\section{Prepare a bootable micro-SD card}

Using a micro-SD card is a convenient way to control the way
the board boots, whatever the initial contents of the on-board MMC
storage.

A micro-SD card also makes it easy to modify its contents by connecting
it to your workstation.

In order to boot from a micro-SD card, the CPU romcode needs the card
to be partitioned and formated in a particular way.

Connect the micro-SD card provided by your instructor to your
workstation:

\begin{itemize}
\item Either using a direct SD slot if available.
      In this case, the card should be seen as \code{/dev/mmcblk0} by
      your computer (check the \code{dmesg} command).
\item Either using a card reader provided by your instructor too.
      In this case, the card should be seen as \code{/dev/sdb}, or
      \code{/dev/sdc}, etc.
\end{itemize}

Now, run the \code{mount} command to check for mounted SD card
partitions. Umount them with command such as \code{sudo umount
/dev/mmcblk0p1} or \code{sudo umount /dev/sdb1}, depending on how
the system sees the media card device.

Now type the below command to partition the micro-SD card
(we assume that the card is seen as \code{/dev/mmcblk0}):

\begin{verbatim}
sudo sfdisk --in-order --Linux --unit M /dev/mmcblk0 << EOF
1,48,0xE,*
,,,-
EOF
\end{verbatim}

Now, let's format the first partition in FAT format:

\begin{verbatim}
sudo mkfs.vfat -F 16 /dev/mmcblk0p1 -n boot
\end{verbatim}

Remove the card and insert it again. It should automatically be mounted
on \code{/media/boot} \footnote{Or on \code{/media/<user>/boot} if you
are using Ubuntu 12.10 or later.}.

Now, copy the first stage bootloader (\code{MLO}: {\em Mmc LOad}) and the main
bootloader (\code{u-boot.cmd}) to your micro-SD card:

\begin{verbatim}
cd $HOME/felabs/linux/bootloader
cp beaglebone-black/MLO /media/boot/
cp beaglebone-black/u-boot.img /media/boot
sudo umount /media/boot
\end{verbatim}

\section{Boot the board}

Force the board to boot on the external card. To do this, you will
have to remove the power supply, press and hold the \code{boot switch}
button, insert the power supply, and then release this button.

Note that from now on, the board will always boot on the external 
MMC, until you remove the power. You only have do this manipulation
when you power on the board for the first time.

You should then see U-boot 2013.10-rc3 start.

Press a key in the \code{picocom} terminal to stop the U-boot
countdown. You should then see the U-Boot prompt:
\begin{verbatim}
U-Boot>
\end{verbatim}

You can now use U-Boot. Run the \code{help} command to see the available
commands.

\section{Setting up Ethernet communication}

The next step is to configure U-boot and your workstation to let your
board download files, such as the kernel image and Device Tree Binary
(DTB), using the TFTP protocol through an Ethernet cable.

To start with, install a TFTP server on your development workstation:

\begin{verbatim}
sudo apt-get install tftpd-hpa
\end{verbatim}

With a network cable, connect the Ethernet port of your board to the
one of your computer. If your computer already has a wired connection
to the network, your instructor will provide you with a USB Ethernet
adapter. A new network interface, probably \code{eth1} or \code{eth2},
should appear on your Linux system.

To configure your network interface on the workstation side, click on
the \code{Network Manager} tasklet on your desktop, and select
\code{Edit Connections}.

\begin{center}
\includegraphics[width=8cm]{labs/kernel-board-setup/network-config-1.png}
\end{center}

Select the new wired network connection:

\begin{center}
\includegraphics[width=8cm]{labs/kernel-board-setup/network-config-2.png}
\end{center}

In the \code{IPv4 Settings} tab, make the interface use a static IP
address, like \code{192.168.0.1} (of course, make sure that this address
belongs to a separate network segment from the one of the main company
network). You will also need to specify the local network mask
(\emph{netmask}, often \code{255.255.255.0}). You can keep the
\code{Gateway} field empty (don't click put the cursor inside the
corresponding text box, otherwise it will ask for a legal value)
or set it to \code{0.0.0.0}:

\begin{center}
\includegraphics[width=8cm]{labs/kernel-board-setup/network-config-3.png}
\end{center}

Now, it's time to configure networking on U-Boot's side.

Back to the U-Boot command line, set the below environment variables:

\begin{verbatim}
setenv ipaddr 192.168.0.100
setenv serverip 192.168.0.1
\end{verbatim}

Save these settings to the MMC card:
\footnote{The U-boot environment settings are stored in some free space
between the master boot record (512 bytes, containing the partition
tables and other stuff), and the beginning of the first partition (often
at \code{32256}). This is why you won't find any related file in the
first partition.}

\begin{verbatim}
saveenv
\end{verbatim}

You can then test the TFTP connection.  First, put a small text
file in \code{/var/lib/tftpboot}. Then, from U-Boot, do:

\begin{verbatim}
tftp 0x81000000 textfile.txt
\end{verbatim}

{\bf Caution: known issue in Ubuntu 12.04 and later}: if this command
doesn't work, you may have you have to stop the server and start it
again every time you boot your workstation:

\begin{verbatim}
/etc/init.d/tftpd-hpa restart
\end{verbatim}

The \code{tftp} command should command should have downloaded the
\code{textfile.txt} file from your development workstation into the
board's memory at location \code{0x81000000} (this location is part of
the board DRAM). You can verify that the download was successful by
dumping the contents of the memory:

\begin{verbatim}
md 0x81000000
\end{verbatim}

We are now ready to load and boot a Linux kernel!
