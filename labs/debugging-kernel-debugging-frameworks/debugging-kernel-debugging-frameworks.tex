\subchapter
{Kernel debugging: built-in testing}
{Objectives:
  \begin{itemize}
    \item Debugging locks and sleeps mistakes using {\em PROVE\_LOCKING} and {\em
    DEBUG\_ATOMIC\_SLEEP} options.
    \item Find a module memory leak using {\em kmemleak}.
  \end{itemize}
}

\section{Locking and sleeps problems}

\kconfig{CONFIG_PROVE_LOCKING} and \kconfig{CONFIG_DEBUG_ATOMIC_SLEEP} have been
enabled in the provided kernel image.
First, compile the module on your development host using the following command line:

\begin{bashinput}
$ cd /home/$USER/__SESSION_NAME__-labs/nfsroot/root/locking
$ export CROSS_COMPILE=/home/$USER/__SESSION_NAME__-labs/buildroot/output/host/bin/%\crosscompile%
$ export ARCH=%\arch%
$ export KDIR=/home/$USER/__SESSION_NAME__-labs/buildroot/output/build/linux-%\workingkernel%/
$ make
\end{bashinput}

On the target, load the \code{locking.ko} module and look at the output in dmesg:

\begin{bashinput}
# cd /root/locking
# insmod locking_test.ko
# dmesg
\end{bashinput}

Once analyzed, unload the module. Try to understand and fix all the problems that
have been reported by the \code{lockdep} system.

\section{Kmemleak}

The provided kernel image contains kmemleak but it is disabled by default to
avoid having a large overhead. In order to enable it, reboot the target and enable
kmemleak by adding \code{kmemleak=on} on the command line. Interrupt U-Boot at
reboot and modify the \code{bootargs} variable:

\begin{bashinput}
=> env edit bootargs
=> <existing bootargs> kmemleak=on
=> boot
\end{bashinput}

Then compile the dummy test module on your development host:

\begin{bashinput}
$ cd /home/$USER/__SESSION_NAME__-labs/nfsroot/root/kmemleak
$ export CROSS_COMPILE=/home/$USER/__SESSION_NAME__-labs/buildroot/output/host/bin/%\crosscompile%
$ export ARCH=%\arch%
$ export KDIR=/home/$USER/__SESSION_NAME__-labs/buildroot/output/build/linux-%\workingkernel%/
$ make
\end{bashinput}

On the target, load the \code{kmemleak_test.ko} and trigger an immediate
kmemleak scan using:

\begin{bashinput}
# cd /root/kmemleak
# insmod kmemleak_test.ko
# rmmod kmemleak_test
# echo scan > /sys/kernel/debug/kmemleak
\end{bashinput}

Note that you might need to run the \code{scan} command several times
before it detects a leakage due to memory still containing references to
the leaked pointer. Soon after that, the kernel will report that some leaks
have been identified. Display them and analyze them using:

\begin{bashinput}
# cat /sys/kernel/debug/kmemleak
\end{bashinput}

You will see that the symbols addresses do not make sense. This is due to the
\code{kptr_restrict} configuration which must be change to allow displaying
pointer addresses. To do so, use the following command on the target:

\begin{bashinput}
# sysctl kernel.kptr_restrict=1
\end{bashinput}

You can use \code{addr2line} to identify the location in source code of the
lines that did cause the reports. You may need to substract module loading address:
you can guess the address by taking a look at \code{/proc/modules} while the module
is loaded.

You may also notice other memory leaks that are actually some real memory leaks
that did exist in the kernel version used for this training !

Once the lab is done, don't forget to remove \code{kmemleak=on} from your
kernel commandline.

