\subchapter{Describing Hardware Devices}
{Objective: learn how to describe hardware devices.}

\section{Goals}

Now that we covered the Device Tree theory, we can explore the list of
existing devices and make new ones available. In particular, we will
create a custom Device Tree to describe the few extensions we will make
to our board.

\section{Setup}

Stay in the \code{~/linux-kernel-labs/src/linux} directory, you should be in
the \code{bootlin-labs} branch.

Download a useful document sharing useful details about the Nunchuk
and its connector:\\
\url{https://bootlin.com/labs/doc/nunchuk.pdf}

\section{Create a custom device tree}

To let the Linux kernel handle a new device, we need to add a
description of this device in the board device tree.

As the i.MX93-FRDM device tree is provided by the kernel community,
and will continue to evolve on its own, we don't want to make changes
directly to the device tree file for this board.

The easiest way to customize the board DTS is to create a new DTS file
that includes the imx93-11x11-frdm one, and adds
its own definitions.

So, create a new
\code{arch/arm64/boot/dts/freescale/imx93-11x11-frdm-custom.dts} file in which
you just include the regular board DTS file. We will add further
definitions in the next sections.

\begin{verbatim}
// SPDX-License-Identifier: GPL-2.0
#include "imx93-11x11-frdm.dts"
\end{verbatim}
Modify the \kfile{arch/arm64/boot/dts/freescale/Makefile} file to add your custom
Device Tree, and then have it compiled with (\code{make dtbs}). Now,
copy the new DTB to the tftp server home directory, change the DTB file
name in the U-Boot configuration\footnote{Tip: you just need to run
\code{editenv bootcmd} and \code{saveenv}.}, and boot the board.

\section{Setting the board's model name}

Modify the custom Device Tree file to override the model name for your
system. Set the \code{model} property to \code{Training i.MX93-FRDM}. Don't hesitate to ask your instructor if you're not sure how.

Recompile the device tree, and reboot the board with it. You should see
the new model name in two different places:

\begin{itemize}
\item In the first kernel messages on the serial console.
\item In \code{/sys/firmware/devicetree/base/model}. This can be
      handy for a distribution to identify the device it's running on.
\end{itemize}

\section{Driving LEDs}

The i.MX93-FRDM board has a RGB LED on the side of the debug USB-C port.But it does not light up.

Start by looking at the different description files and look for a node
that would be defining the LED.

The LED is actually supposed to be triggered by a driver matching
the compatible \code{gpio-leds}. This is a generic driver which acts on
LEDs connected to GPIOs. But as you can observe, despite being part of
the in-use Device Tree, the LED remains off. The reason for that is the
absence of driver for this node: nothing actually drives the LED even
it is described. So you can start by recompiling your kernel with
\kconfigval{CONFIG_LEDS_CLASS}{y} and \kconfigval{CONFIG_LEDS_GPIO}{y}.

You should now see the LED right next to the debug connector bright in white. If
you look at the binding documents
\kfile{Documentation/devicetree/bindings/leds/common.yaml} and
\kfile{Documentation/devicetree/bindings/leds/leds-gpio.yaml}, you'll
notice we can tweak the \code{default-state} in order to change the
color of the LED.

You would need to modify the "official" DTS file in order to do that.
But because we do not want to impact the orginal file, we might instead
add a label to the \code{leds} container node. We
could then reference this new label in our custom DTS and overwrite the
\code{default-state} property of each LED subnode.

Reboot the board using the new DTS and observe the LED default states
change.

\section{Managing I2C buses and devices}

The next thing we want to do is connect a Nunchuk joystick
to an I2C bus on our board. The I2C bus is very frequently used
to connect all sorts of external devices. That's why we're covering
it here.

\subsection{Enabling an I2C bus}

As we can see in the board user manual, the i.MX93-FRDM board has
only one I2C bus available on its expansion connectors. This is LPI2C1,
which is connected to the 10-pin 2x5 2.54 mm connector. But if we look
at the design files, we can see that among all the GPIOs connected on the main
EXPI connector, \code{EXP_GPIO_IO02} and \code{EXP_GPIO_IO03} are physically
wired to the SoC \code{GPIO_IO02} and \code{GPIO_IO} pins, which can be routed
internally to the LPI2C4 bus by enabling the alternate mode 1 (see reference
manual \code{27.3.1.8 SW_MUX_CTL_PAD_GPIO_IO02 SW MUX Control Register}).

From a training perspective, it is more interesting to play with LPI2C4, as it
is disabled by default in the board's Device Tree.

Let's see which I2C buses are already enabled:

\begin{bashinput}
# i2cdetect -l
i2c-1	i2c       	44350000.i2c                    	I2C adapter
i2c-2	i2c       	42530000.i2c                    	I2C adapter
i2c-0	i2c       	44340000.i2c                    	I2C adapter

\end{bashinput}

Here you can see there are only three I2C controllers.

As the bus numbering scheme in Linux doesn't always match the one
in the datasheets, let's check the base addresse of these controllers:

\begin{bashinput}
# ls -l /sys/bus/i2c/devices/i2c-*
lrwxrwxrwx    1 root     root             0 Jan  1 00:08 /sys/bus/i2c/devices/i2c-0 -> ../../../devices/platform/soc@0/44000000.bus/44340000.i2c/i2c-0
lrwxrwxrwx    1 root     root             0 Jan  1 00:08 /sys/bus/i2c/devices/i2c-1 -> ../../../devices/platform/soc@0/44000000.bus/44350000.i2c/i2c-1
lrwxrwxrwx    1 root     root             0 Jan  1 00:08 /sys/bus/i2c/devices/i2c-2 -> ../../../devices/platform/soc@0/42000000.bus/42530000.i2c/i2c-2

\end{bashinput}

That's not completely straightforward, but you can suppose that:
\begin{itemize}
\item LPI2C0 is at address \code{0x44340000}
\item LPI2C1 is at address \code{0x44350000}
\item LPI2C2 is at address \code{0x42530000}
\end{itemize}

Now let's double check the addressing by looking at the imx 93 processor reference manual, downloaded earlier.
in the \code{Low Power Inter-Integrated Circuit (LPI2C)} and \code{LPI2C memory map} sections:

\begin{itemize}
\item LPI2C1 is at address \code{0x44340000}
\item LPI2C2 is at address \code{0x44350000}
\item LPI2C3 is at address \code{0x42530000}
\end{itemize}

So, we are lucky that \code{i2c-0} in Linux corresponds to I2C1 in the
datasheet, that \code{i2c-1} corresponds to IC I2C2, and that \code{i2c-2} corresponds to I2C3.

Fortunately, I2C4 is already defined in the DTS includes
used by the Device Tree for our board. In our case, that's in
\kfile{arch/arm64/boot/dts/freescale/imx93.dtsi}. Look by yourself in this
file, and you will find its definition, but with \code{status =
"disabled";}. This means that this I2C controller is not enabled yet,
and it's up to boards using it to do so.

Make a reference to this definition in your custom DTS and enable this
bus. Also configure it to function at 100 kHz. Reboot your board with
the update.

Back to the running system, we can now see that there is one more
I2C bus:

\begin{bashinput}
# i2cdetect -l
i2c-3	i2c       	42540000.i2c                    	I2C adapter
i2c-1	i2c       	44350000.i2c                    	I2C adapter
i2c-2	i2c       	42530000.i2c                    	I2C adapter
i2c-0	i2c       	44340000.i2c                    	I2C adapter

\end{bashinput}

Run the below command to confirm that the new bus has the same address
as in the datasheet (\code{0x42540000}):

\bashcmd{ls -l /sys/bus/i2c/devices/i2c-3}

\subsection{Prepare the I2C device DT description}

Before describing your nunchuk device, let's think about what will be
needed:
\begin{itemize}
\item The device node should follow a standard pattern.

  The node name should be \code{joystick@addr}, the convention for node
  names is \code{<device-type>@<addr>}.

\item We want to be able to fully identify the programming model.

  This is usually done using a unique compatible string. The compatible
  contains a vendor prefix and then a more specific string. We will use
  \code{nintendo,nunchuk}.

\item We need to identify how to reach the device.

  This is the \code{reg} property and we should set it to the I2C
  address of the nunchuk. You will find the I2C slave address of the
  Nunchuk on the nunckuk document that we have downloaded
  earlier\footnote{This I2C slave address is enforced by the device
    itself. You can't change it.}.

\item (Optional) There are two types of nunchuks.

  There are white and black nunchuks, which don't expect the same
  initialization flow. We could imagine a boolean property named
  \code{nintendo,alternate-init} which will change the initialization
  logic. See the nunchuk pdf for details about the alternate flow, but if you
  are attending an on-site session, mind the nunchuk you got behaves like a
  black, even though it looks very white\footnote{The Nunchuk copies we
  carry got a white cover even though their electronics behaves like the
  orginal black version.}.

\end{itemize}

Stopping here is sufficient as writing device-tree bindings is not
strictly required to continue the labs, but if you feel comfortable
you may want to write your own binding file, eg:
\begin{bashinput}
Documentation/devicetree/bindings/misc/nintendo,nunchuk.yaml
\end{bashinput}
Once you are confident with your bindings, you can even copy the
examples from the \code{wrong-nunchuk-examples.yaml} (in the
\code{nunchuk} labs folder) inside your bindings and verify they all
pass/fail as expected!
\begin{bashinput}
sudo apt install python3 python3-pip
pip3 install dtschema

make DT_SCHEMA_FILES=misc/nintendo,nunchuk.yaml dt_binding_check
\end{bashinput}

\subsection{Declare the Nunchuk device}

As a child node to the \code{LPI2C4} bus, now declare an I2C device
for the Nunchuk, following the above rules.

If you wrote an optional YAML binding, you can also double check your
node:
\begin{bashinput}
make DT_SCHEMA_FILES=misc/nintendo,nunchuk.yaml dtbs_check
\end{bashinput}

After updating the running Device Tree, explore
\code{/sys/firmware/devicetree}, where every subdirectory corresponds to
a DT node, and every file corresponds to a DT property. You can search
for presence of the new \code{joystick} node:

{\small
\begin{verbatim}
# find /sys/firmware/devicetree -name "*joystick*"
/sys/firmware/devicetree/base/soc@0/bus@42000000/i2c@42540000/joystick@52
\end{verbatim}
}

You can also check the whole structure of the loaded Device Tree, using
the Device Tree Compiler (\code{dtc}), which we put in the root
filesystem:
\begin{verbatim}
# dtc -I fs /sys/firmware/devicetree/base/ > /tmp/dts
# grep -B15 -A4 nunchuk /tmp/dts
\end{verbatim}

Once your new Device Tree seems correct, commit your changes. As you
modified a shared file and a custom file, it is good practice to commit
these changes in two different patches.
