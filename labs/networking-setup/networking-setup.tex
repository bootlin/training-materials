\subchapter{Setup the build environment}
{Objective: Configure our testing setup for the labs}

\section{Goals}

\begin{itemize}
\item Have an overview of the build setup
\item Run our first Buildroot build
\item Setup the Host machine
\end{itemize}

\section{Install the required packages}

You need some packages installed on your host machine :

\begin{hostbashinput}
$ sudo apt install build-essential git tcpdump wireshark iperf3 python3-scapy ethtool \
  clang linux-tools-common libbpf-dev pahole
\end{hostbashinput}

\section{Getting started with Buildroot}

For this training session, we will be running a pre-configured Linux OS image that
we will generate usnig \href{https://buildroot.org/}{Buildroot} version 2025.02

The Espressobin V7 is already well supported in Buildroot, the image we are using
contains a few extra setup options that will make the next labs easier to run :

\begin{itemize}
	\item The Linux Kernel version is v6.12.y, the latest LTS release
	\item Some networking related packages are pre-installed, namely :
		\begin{itemize}
			\item ethtool, for low-level network interface configuration (see \manpage{ethtool}{8})
			\item iperf3, for traffic generation (see \manpage{iperf3}{1})
			\item iproute2, replacing the busybox-based implementation, for network configuration (see \manpage{ip}{8})
			\item tcpdump, for traffic analysis (see \manpage{tcpdump}{8})
			\item Custom packages for the various labs.
		\end{itemize}
\end{itemize}

Our configuration also includes an overlay directory, which will allows us to very easily
install custom-made files into our rootfilesystem.

\section{Understanding the Lab materials}

In this training, we will be running some commands both on the Espressobin and the Host machine.

In order to quickly see where a given command should run, we have color-coded the instructions.

\subsection{Target commands}

Commands on a Yellow background are to be run on the Espressobin, also referred to as \textbf{Target} :

\begin{targetbashinput}
$ echo "Hello world, I am an Espressobin"
\end{targetbashinput}

For simplicity, the expected output of commands running on the Espressbin also appear on a yellow background :
\begin{targetterminaloutput}
Hello world, I am an Espressobin
\end{targetterminaloutput}

\subsection{Target commands}

Commands that needs to run on your \textbf{Host} machine are on a Green background :
\begin{hostbashinput}
$ echo "Hello world, I am the host machine"
\end{hostbashinput}

For simplicity, the expected output of commands running on the Espressbin also appear on a yellow background :
\begin{hostterminaloutput}
Hello world, I am the host machine
\end{hostterminaloutput}

\section{Setup the host machine for the neworking labs}

You can use a built-in Ethernet port n your host machine, provided that it is not in use, and that it is capable of 1Gbps speed. This can be checked with :

\begin{hostbashinput}
$ ethtool <host_iface>
        settings for enp0s20f0u1u1:
        Supported ports: [ TP	 MII ]
        Supported link modes:   10baseT/Half 10baseT/Full
                                100baseT/Half 100baseT/Full
                                1000baseT/Half 1000baseT/Full
        Supported pause frame use: No
        Supports auto-negotiation: Yes
        Supported FEC modes: Not reported
        ...
\end{hostbashinput}

If you do not have such an interface, you can use the provided USB to Ethernet adapter. It supports 1Gbps, but its driver doesn't report that.

We will be doing some manual re-configuration of the host interface. Regardless if you chose to use
the built-in interface, you need to make sure that NetworkManager will not try to re-configure
your interface, thus overriding your configuration. You can achieve that temporarily by running :

\begin{hostbashinput}
$ nmcli device set <iface> managed no
\end{hostbashinput}

This only temporary, as the interface will become managed again when your host machine reboots.

\section{Building our image}

Let's now build our Buildroot image :

\begin{hostbashinput}
$ cd /home/$USER/%\training%-%\board%-labs/
$ git clone https://github.com/bootlin/buildroot
$ cd buildroot
$ git checkout networking-training/2025.02.x
$ make globalscale_espressobin_networking_defconfig
$ make
\end{hostbashinput}

This should take a while :) The Buildroot image provided is also hosted on our github, you can take a look \href{https://github.com/bootlin/buildroot}{here} for more details.

\subsection{Preparing the Espressobin}

The Espressobin is powered by a 12V DC external PSU. Make sure that your PSU is rated for 2A, and has a center-positive Barrel Jack.

In addition, to access the debug serial console, you need to use a
micro-USB cable connected to the micro-USB port, near the Barrel Jack.

Once your micro-USB cable is connected, a \code{/dev/ttyUSB0} device
will apear on your PC. You can see this device appear by looking at
the output of \code{dmesg} on your workstation.

To communicate with the board through the serial port, install a
serial communication program, such as \code{picocom}:

\begin{hostbashinput}
sudo apt install picocom
\end{hostbashinput}

If you run \code{ls -l /dev/ttyUSB0}, you can also see that only
\code{root} and users belonging to the \code{dialout} group have read
and write access to this file. Therefore, you need to add your user to
the \code{dialout} group:

\begin{hostbashinput}
sudo adduser $USER dialout
\end{hostbashinput}

{\bf Important}: for the group change to be effective, you have to
{\em completely reboot} the system \footnote{As explained on
\url{https://askubuntu.com/questions/1045993/after-adding-a-group-logoutlogin-is-not-enough-in-18-04/}.}.
A workaround is to run \code{newgrp dialout}, but it is not global.
You have to run it in each terminal.

Now, you can run \code{picocom -b 115200 /dev/ttyUSB0}, to start
serial communication on \code{/dev/ttyUSB0}, with a baudrate of
\code{115200}. If you wish to exit \code{picocom}, press
\code{[Ctrl][a]} followed by \code{[Ctrl][x]}.

There should be nothing on the serial line so far, as the board is not
powered up yet.



\subsection{Flashing the SDCard}

You will now need to flash a sdcard with the 
\code{output/images/sdcard.img} file. Plug your sdcard on your computer and
check on which \code{/dev/sdX} it has been mounted (you can use the \code{dmesg}
command to check that). For instance, if the sdcard has been mounted on
\code{/dev/sde}, use the following command:

\begin{hostbashinput}
$ sudo dd if=output/images/sdcard.img of=/dev/sde
$ sync
\end{hostbashinput}

NOTE: Double-check that you are targeting the correct device before executing
the dd command!

Once flashed, insert the sdcard into the Espressobin.

