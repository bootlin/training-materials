\subchapter
{System status}
{Objectives:
  \begin{itemize}
    \item Observe running processes using {\em ps} and {\em top}.
    \item Check memory allocation and mapping with {\em procfs} and {\em pmap}.
    \item Monitor other resources usage using {\em iostat}, {\em vmstat} and {\em netstat}.
  \end{itemize}
}

\section{Observe system status}

In order to examine the platform, you can either execute commands through the
\code{picocom} terminal or open a SSH connection.

Now that the board is up and running, let's try to understand what is running
on this system. The provided image includes numerous tools to analyze the
system. Try to answer the following questions using the commands that were
presented during the course:

\begin{enumerate}
  \item How many CPU does this processor have?
  \item What are the memory maps used by the \code{dropbear} process?
  \item How much PSS memory is used by \code{dropbear}?
  \item What is the amount of memory available for applications on the system?
  \item How much unused memory is left on our system?
  \item Is there swapped memory?
  \item Is there an application using too much CPU?
  \item How much time is spent by CPU0 in system mode (kernel)?
  \item Is there some IOs ongoing with storage devices?
  \item How many Mbytes/s are transferred from/to the MMC card?
  \item What is the process generating transfers to the MMC?
  \item Which processor receives most of the interrupts?
  \item How many interrupts were received from the MMC controller?
\end{enumerate}

Once found, you can remove the files \code{/etc/init.d/S25stress-ng} and
\code{/etc/init.d/S26mmc-reader} and reboot to have a cleaner system.
