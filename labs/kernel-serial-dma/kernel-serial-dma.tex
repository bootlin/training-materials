\subchapter{DMA}{Objective: learn how to use the \code{dma-mapping} API
  to handle DMA buffers and coherency, as well as the \code{dmaengine}
  API to deal with DMA controllers through a generic abstraction}

During this lab, you will:

\begin{itemize}
\item Setup streaming mappings with the \code{dma} API
\item Configure a DMA controller with the \code{dmaengine} API
\item Configure the hardware to trigger DMA transfers
\item Wait for DMA completion
\end{itemize}

\section{Setup}

This lab is a continuation of all the previous {\em serial} labs. Use
the same kernel, environment and paths!

\section{Preparing the driver}

We will use DMA in the write path. As we will receive data from
userspace, we will need a bounce buffer, so we can create a second
buffer named \code{tx_buf} of the same size as \code{rx_buf} in our
\code{serial_dev} structure.

As we will also need the \code{resource} structure with the MMIO physical
addresses from outside of the \code{->probe()}, it might be relevant to save
the \code{resource} pointer used to derive the \code{miscdev} name into the
\code{serial_dev} structure.

Finally, the device-model \code{struct device *} contained in the
platform device will soon be very useful as well, so we can save it in
our \code{struct serial_dev *} object.

Before going further, re-compile and test your driver.

The \code{serial_write} callback and \code{serial_fops} can now be renamed \code{serial_write_pio} and \code{serial_fops_pio},
while we will implement a new callback named \code{serial_write_dma} and a new
set of file operations called \code{serial_fops_dma} which uses this callback
for \code{.write} and keeps the same values for other fields. This new set of
file operations should be used by default.

Let's now create two helpers supposed to initialize and cleanup our DMA
setup. We will call \code{serial_init_dma()} right before registering
the \code{misc} device. In the \code{->probe()} error path and in the
remove callback, we will call \code{serial_cleanup_dma()}. Make sure that errors
are handled correctly and returned to the caller. A special case should be
handled when no DMA channel is available (with the \code{-ENODEV} code returned)
in order to fallback to the the \code{serial_fops_pio} file operations.

\section{Prepare the DMA controller}

The OMAP UART controller can make use of an external DMA controller. On
the AM3359, it is actually wired to a DMA controller named EDMA. So we
will have to deal with the \code{dmaengine} API in order to prepare DMA
transfers on the controller side. The idea of this API is to fully
abstract the characteristics of the DMA controller.

In a complete driver we should probably use the helpers checking
capabilities. Let's just skip this part and assume the two IPs are
compatible and the addressing masks properly set to 32-bit.

We should at the very least request the DMA channel to use. Open the SoC
device tree and find the uart2 and uart4 nodes. Look for \code{dma}
channel properties and their names. While uart2 seem to be connected to
the DMA controller through two different channels (one for each
direction), uart4 is not. Hence, when requesting the channels with
\kfunc{dma_request_chan}, we must take care to check and return the error code
wrapped in the returned \kstruct{dma_chan} pointer. This can be done with the
\code{IS_ERR()} and \code{PTR_ERR()} macros. You may display the corresponding
error string with {\tt \%pe}! Also make sure that this case is correctly handled
both in the calling code to fallback to the \code{serial_fops_pio} file
operations and in the DMA cleanup function.

This channel will be used by all the \code{dmaengine} helpers, so better
save it in our \code{serial_dev} structure.
\begin{verbatim}
struct serial_dev {
        ...
        struct dma_chan *txchan;
};
\end{verbatim}

We can now configure the DMA controller with details about the upcoming
transfers:
\begin{itemize}
\item memory to device transfers
\item the source will be memory, we will map buffers when they come,
  there is no particular constraint on this side
\item the destination is the UART Tx FIFO, we will ask the DMA to
  transfer the bytes one after the other (hardware signaling already
  handles the internal ``flow'')
\item we shall not use the UART Tx FIFO directly, to be generic we shall use
  \kfunc{dma_map_resource} first (and save it in \code{serial_dev} to be able
  to unmap it later)
\end{itemize}

\begin{verbatim}
struct dma_slave_config txconf = {};

serial->fifo_dma_addr = dma_map_resource(dev, serial->res->start + UART_TX * 4,
                                         4, DMA_TO_DEVICE, 0);
if (dma_mapping_error(dev, serial->fifo_dma_addr)) ...

txconf.direction = DMA_MEM_TO_DEV;
txconf.dst_addr_width = DMA_SLAVE_BUSWIDTH_1_BYTE;
txconf.dst_addr = serial->fifo_dma_addr;
ret = dmaengine_slave_config(serial->txchan, &txconf);
if (ret) ...
\end{verbatim}

The cleanup helper should on its side call
\kfunc{dmaengine_terminate_sync} just to be sure no transfer is
ongoing, right before un-mapping the FIFO with \kfunc{dma_unmap_resource} and
releasing the DMA channel with \kfunc{dma_release_channel}.

It is time to recompile your driver and see if any header is missing...

\section{Prepare the UART controller}

On its side, the UART controller must assert some signals to drive the DMA
flow. We must enable the controlling logic on the Tx DMA channel, by enabling
\code{DMACTL} in mode 3. We also configure the UART to transmit all the bytes
as soon as they get in.

\begin{verbatim}
#define OMAP_UART_SCR_DMAMODE_CTL3 0x7
#define OMAP_UART_SCR_TX_TRIG_GRANU1 BIT(6)

/* Enable DMA */
reg_write(serial, OMAP_UART_SCR_DMAMODE_CTL3 | OMAP_UART_SCR_TX_TRIG_GRANU1,
          UART_OMAP_SCR);
\end{verbatim}

\section{Process user write requests}

It is now time to deal with user buffers again.

Before doing anything in the \code{write} hook, we shall fill-in the
\code{serial_dev} structure with:
\begin{itemize}
\item a \code{bool txongoing} flag to prevent concurrent uses of the same
  Tx DMA channel (would be possible by queuing new requests, but let's keep this
  implementation simple) while not holding any lock for the full duration of
  the operation.
\item a \code{struct completion txcomplete} object to asynchronously inform the
  write thread that the DMA transaction is over (very much like we did with the
  \code{waitqueue} in the interrupt lab). This object shall be initialized with
  \code{init_completion(&serial->txcomplete)}.
\end{itemize}

\begin{verbatim}
struct serial_dev {
        ...
        struct dma_chan *txchan;
        bool txongoing;
        struct completion txcomplete;
};
\end{verbatim}

In the write hook, we shall first check if the DMA channel has been
properly retrieved. If not, we should definitely fallback to the PIO
implementation.

Then, in order to simplify the code, we will no longer deal with
concurrent operations. In order to safely serialize writes, we can start
and end the write hook with something like:

\begin{verbatim}
/* Prevent concurrent Tx */
spin_lock_irqsave(&serial->lock, flags);
if (serial->txongoing) {
        spin_unlock_irqrestore(&serial->lock, flags);
        return -EBUSY;
}
serial->txongoing = true;
spin_unlock_irqrestore(&serial->lock, flags);

...

spin_lock_irqsave(&serial->lock, flags);
serial->txongoing = false;
spin_unlock_irqrestore(&serial->lock, flags);
\end{verbatim}

The first step in this \code{->write()} hook is to use \code{serial->tx_buf} as
bounce buffer by copying the user data using \kfunc{copy_from_user}. Let's
handle up to \code{SERIAL_BUFSIZE} bytes at a time. One can use \kfunc{min_t}
to derive the right amount of bytes to deal with.

Before mapping the buffer for DMA purposes, we need to handle one last
thing. The OMAP 8250 UART controller has a limitation, its internal circuitry
to trigger DMA transfers is a bit broken, and the first character needs to be
fed manually, so let's just save that first byte away:

\begin{verbatim}
char first;

/* OMAP 8250 UART quirk: need to write the first byte manually */
first = serial->tx_buf[0];
\end{verbatim}

Now we can remap the buffer. We have a single buffer so we can use
\kfunc{dma_map_single}. The output value is a \ksym{dma_addr_t}. Save this
value as we will reuse it. Also do not forget to check its validity with
\kfunc{dma_mapping_error}.

We now have all the missing information compared to the \code{serial_init_dma}
step, like the \ksym{dma_addr_t} of the buffer and its length. Let's create a
descriptor filled with all the default information known by the DMA controller
plus the additional details we can now provide:

\begin{verbatim}
struct dma_async_tx_descriptor *desc;

desc = dmaengine_prep_slave_single(serial->txchan, serial->dma_addr + 1,
                                   len - 1, DMA_MEM_TO_DEV,
                                   DMA_PREP_INTERRUPT | DMA_CTRL_ACK);
if (!desc) ...
\end{verbatim}

Mind the \code{+ 1} and \code{- 1} operations in the snippet above, they are
here to skip the first character which we will send manually.

We can now use the returned descriptor to register a callback. This callback
will just call \kfunc{complete} over the completion object. Which also means
this completion object could be re-initialized while we register the callback,
just in case.

The DMA transfer contained in the descriptor can now be queued into the DMA
controller queue:

\begin{verbatim}
dma_cookie_t cookie;

cookie = dmaengine_submit(desc);
ret = dma_submit_error(cookie);
if (ret) ...
\end{verbatim}

The transfer can be triggered. This is usually an operation that is only
required on the DMA controller side, but remember here we also need to trigger
it on the UART controller side:

\begin{verbatim}
dma_async_issue_pending(serial->txchan);
reg_write(serial, first, UART_TX);
\end{verbatim}

The transfer being asynchronous, it is finally required to wait for completion
with one of the \kfunc{wait_for_completion} variants, and to call
\kfunc{dma_unmap_single} right after it.

You can now test your driver. Writing a single character is of course not
relevant as our driver would just use the previous method to send it. Try with
a string instead!
