\subchapter{Input interface}{Objective: make the I2C device available
to userspace using the input subsystem.}

After this lab, you will be able to:

\begin{itemize}
\item Expose device events to userspace through an input interface,
      using the kernel based polling API for input devices
      (kernel space perspective)
\item Handle registration and allocation failures in a clean
      way.
\item Get more familiar with the usage of the input interface
      (user space perspective)
\end{itemize}

\section{Add polled input device support to the kernel}

The nunchuk doesn't have interrupts to notify the I2C master that 
its state has changed. Therefore, the only way to access device data
and detect changes is to regularly poll its registers, using the input
polling API described in the lectures.

Rebuild your kernel with static support for polled input device support
(\code{CONFIG_INPUT_POLLDEV=y}). With the default configuration, this
feature is available as a module, which is less convenient.

Update and reboot your kernel.

\section{Register an input interface}

The first thing to do is to add an input device to the system. Here are
the steps to do it:

\begin{itemize}
\item Declare a pointer to an \code{input_polled_dev} structure in the
      \code{probe} routine. You can call it \code{polled_input}.
      You can't use a global variable because your driver needs to be
      able to support multiple devices.
\item Allocate such a structure in the same function, using the
      \code{input_allocate_polled_device()} function. 
\item Also declare a pointer to an \code{input_dev} structure. You can 
      call it \code{We won't
      need to allocate it, because it is already part of the
      \code{input_polled_dev} structure, and allocated at the same time.
      We will use this as a shortcut to keep the code simple.
\item Still in the \code{probe()} function, add the input device to
      the system by calling \code{input_register_polled_device()};
\end{itemize}

At this stage, first make sure that your module compiles well (add
missing headers if needed).

\section{Handling probe failures}

In the code that you created, make sure that you handle failure
situations properly.

\begin{itemize}
\item Of course, test return values values properly and log 
      the causes of errors.
\item If the call to \code{input_register_polled_device()} fails,
      you must also free the \code{input_polled_dev} structure
      before returning an error. If you don't do that, you will create
      memory leaks in the kernel. In the general case, failure to
      release things that have been allocated or registered before
      can prevent you from reloading a module.
\end{itemize}

To implement this correctly without duplicating or creating ugly code,
it's recommended to use \code{goto} statements.

See {\em Chapter 7: Centralized exiting of functions} in
\code{Documentation/CodingStyle} for useful guidelines and an example.
Implement this in your driver. 


