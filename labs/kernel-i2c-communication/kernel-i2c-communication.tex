\subchapter{Using the I2C bus}
{Objective: Use the I2C bus to implement communication with the Nunchuk device}

\section{Goals}

After this lab, you will be able to:
\begin{itemize}
\item Find your driver and device in \code{/sys}.
\item Implement basic \code{probe()} and \code{remove()} driver
  functions and make sure that they are called when there is a
  device/driver match.
\item Access I2C device registers through the bus.
\end{itemize}

\section{Setup}

Stay in the \code{~/__SESSION_NAME__-labs/src/linux} directory for kernel and DTB
compiling (stay in the \ifdefstring{\labboard}{beagleplay}{\cod
{beagleplay-custom}}{} \ifdefstring{\labboard}{BealgeBone Black}
{bbb-custom}{} \ifdefstring{\labboard}{imx93-frdm}{bootlin-labs}{}
branch).

In a new terminal, go to
\code{~/__SESSION_NAME__-labs/modules/nfsroot/root/nunchuk/}.  This directory
contains a Makefile and an almost empty \code{nunchuk.c} file.

\section{Exploring /dev}

Start by exploring \code{/dev} on your target system. Here are a few
noteworthy device files that you will see:

\begin{itemize}
 \item {\em Terminal devices}: devices starting with \code{tty}.
       Terminals are user interfaces taking text as
       input and producing text as output, and are typically used by
       interactive shells. In particular, you will find
       \code{console} which matches the device specified through
       \code{console=} in the kernel command line. You will also find
       the {\tt \ttyname} device file.
 \item {MMC device(s) and partitions}: devices starting with
       \code{mmcblk}. You should here recognize the MMC device(s)
       on your system and the associated partitions.
 \item If you have a real board (not QEMU) and a USB stick, you could
       plug it in and if your kernel was built with USB host and mass
       storage support, you should see a new \code{sda} device appear,
       together with the \code{sda<n>} devices for its partitions.
\end{itemize}

Don't hesitate to explore \code{/dev} on your workstation too
and ask any questions to your instructor.

\section{Exploring /sys}

The next thing you can explore is the {\em Sysfs} filesystem.

A good place to start is \code{/sys/class}, which exposes devices
classified by the kernel frameworks which manage them.

For example, go to \code{/sys/class/net}, and you will see all the
networking interfaces on your system, whether they are internal,
external or virtual ones.

Find which subdirectory corresponds to the network connection
to your host system, and then check device properties such as:
\begin{itemize}
   \item \code{speed}: will show you whether this is a gigabit
         or hundred megabit interface.
   \item \code{address}: will show the device MAC address. No
	 need to get it from a complex command!
   \item \code{statistics/rx_bytes} will show you how many bytes
	 were received on this interface.
\end{itemize}

Don't hesitate to look for further interesting properties by yourself!

You can also check whether \code{/sys/class/thermal} exists and is not
empty on your system. That's the thermal framework, and it allows
to access temperature measures from the thermal sensors on your system.

Next, you can now explore all the buses (virtual or physical) available
on your system, by checking the contents of \code{/sys/bus}.

In particular, go to \code{/sys/bus/mmc/devices} to see all the
MMC devices on your system. Go inside the directory for the first device
and check several files (for example):

\begin{itemize}
\item \code{serial}: the serial number for your device.
\item \code{preferred_erase_size}: the preferred erase block for your
      device. It's recommended that partitions start at multiples of this
      size.
\item \code{name}: the product name for your device. You could display
      it in a user interface or log file, for example.
\item \code{date}: apparently the manufacturing date for the device.
\end{itemize}

Don't hesitate to spend more time exploring \code{/sys} on your system
and asking questions to your instructor.

\section{Implement a basic I2C driver for the Nunchuk}

It is now time to start writing the first building blocks of the I2C
driver for our Nunchuk.

Relying on explanations given during the lectures, fill the
\code{nunchuk.c} file to implement:

\begin{itemize}
\item \code{probe()} and \code{remove()} functions that will
      be called when a Nunchuk is found.
      For the moment, just put a call to \kfunc{pr_info} inside
      to confirm that these functions are called.
\item Initialize a \ksym{i2c_driver} structure, and register
      the i2c driver using it. Make sure that you use
      a \code{compatible} property that matches the one in the
      Device Tree.
\end{itemize}

You can now compile your module and reboot your board, to
boot with the updated DTB.

\section{Driver tests}

You can now load the \code{/root/nunchuk/nunchuk.ko} file.
You need to check that the \code{probe()} function gets called
then, and that the \code{remove()} function gets called too
when you remove the module.

List the contents of \code{/sys/bus/i2c/drivers/nunchuk}. You
should now see a symbolic link corresponding to our new device.

Also list \code{/sys/bus/i2c/devices/}. You should now see the
Nunchuk device, which can be recognized through its \code{0052} address.
Follow the link and you should see a symbolic link back to the Nunchuk
driver.

\section{Device initialization}

Now that we have checked that the \code{probe()} and \code{remove()}
functions are called, remove the messages that you added to trace the
execution of these functions.

The next step is to read the state of the nunchuk registers, to find out
whether buttons are pressed or not, for example.

Before being able to read nunchuk registers, the first thing to do is
to send initialization commands to it. That's also a nice way of making
sure I2C communication works as expected.

In the probe routine (run every time a matching device is found):

\begin{enumerate}
\item Using the I2C raw API (\kfunc{i2c_master_send} and
        \kfunc{i2c_master_recv}), send two bytes to the
        device: \code{0xf0} and \code{0x55}\footnote{
	The I2C messages to communicate with a wiimote
        extension are in the form: \code{<i2c_address> <register> }
        for reading and \code{<i2c_address> <register> <value>} for
        writing. The address, \code{0x52} is sent by the i2c framework
        so you only have to write the other bytes, the register
        address and if needed, the value you want to write. There are
        two ways to set up the communication. The first known way was
        with data encryption by writing \code{0x00} to register
        \code{0x40} of the nunchuk.  With this way, you have to
        decrypt each byte you read from the nunchuk (not so hard but
        something you have to do).  Unfortunately, such encryption
        doesn't work on third party nunchuks so you have to set up
        unencrypted communication by writing \code{0x55} to
        \code{0xf0} instead. This works across all brands of nunchuks
        (including Nintendo ones).}.
      Make sure you check the return value of the function you're
      using. This could reveal communication issues.  Using Elixir, find
      examples of how to handle failures properly using the same
      function.

      (Optional) If you defined a \code{nintendo,alternate-init}
      property, you may want to check it's presence in the device tree
      using \kfunc{device_property_read_bool}, and derive the right
      initialization bytes from it.

\item Let the CPU wait for 1 ms by using the \kfunc{udelay} routine.
      Let's use Elixir again to find the right C headers to include...

      The Elixir results are a bit confusing here, because
      \kfunc{udelay} is defined in \code{arch/<arch>/include/asm/delay.h} files,
      but not in an \code{include/linux/<file>.h>} that is normally used
      in kernel code.

      However, look at \kfile{include/linux/delay.h} and you will see
      that it includes \code{asm/delay.h} which corresponds to the
      specific headers for the current architecture. So you need to include
      \code{linux/delay.h}.

      {\bf General rule}: whenever the symbol you're looking
      for is defined in \code{arch/<arch>/include/asm/<file>.h}, you
      can include \code{linux/<file>.h} in your kernel code.

\item In the same way, send the \code{0xfb} and \code{0x00} bytes now.
      This completes the nunchuk initialization.
\end{enumerate}

Recompile and load the driver, and make sure you have no communication
errors.

\section{Read nunchuk registers}

As the nunchuk does not feature any type of external signaling nor any
internal bit to advertize a possible end-of-conversion status, the user
is required to regularly poll the registers, each read triggering the
next conversion. This leads to a specific situation: the first read
triggers the first conversion but returns some data which can be
considered garbage and safely discarded.

As a consequence, we will need to read the registers twice the first time!

To keep the code simple and readable, let's create a
\code{nunchuk_read_registers()} function to read the registers once.
In this function:

\begin{enumerate}
\item Start by putting a \code{10 ms} delay by calling
      \code{usleep_range(10000, 20000)}, guaranteed to sleep between 10 and 20
      ms.\footnote{That's better than using \kfunc{udelay} because it is not making
      an active wait, and instead lets the CPU run other tasks in the meantime.
      Moreover, this is better than using \kfunc{usleep} if your wait time is
      flexible because this function will try to group tasks wakeup rather than
      creating a specific timer to wake up that task.
      You'll find interesting details on how to sleep or wait in kernel
      code for specified durations in the kernel documentation:
      \kdochtml{timers/delay_sleep_functions}.}
      Such waiting time is needed to add time between the previous i2c
      operation and the next one.
\item Write \code{0x00} to the bus. That will allow us to read
      the device registers.
\item Add another \code{10 ms} delay.
\item Read 6 bytes from the device, still using the I2C raw API.
      Check the return value as usual.
\end{enumerate}

\section{Reading the state of the nunchuk buttons}

Back to the \code{probe()} function, call your new function twice.

After the second call, compute the states of the \code{Z} and \code{C}
buttons, which can be found in the sixth byte that you read.

As explained on
\url{https://bootlin.com/labs/doc/nunchuk.pdf}:

\begin{itemize}
\item \code{bit 0 == 0} means that \code{Z} is pressed.
\item \code{bit 0 == 1} means that \code{Z} is released.
\item \code{bit 1 == 0} means that \code{C} is pressed.
\item \code{bit 1 == 1} means that \code{C} is released.
\end{itemize}

Using boolean operators, write code that initializes a \code{zpressed}
integer variable, which value is \code{1} when the \code{Z} button is
pressed, and \code{0} otherwise. Create a similar \code{cpressed}
variable for the \code{C} button\footnote{You may use the \kfunc{BIT}
macro, which will make your life easier. See Elixir for details.}.

The last thing is to test the states of these new variables at the end
of the \code{probe()} function, and log a message to the console
when one of the buttons is pressed.

\section{Testing}

Compile your module, and reload it. No button presses should be
detected. Remove your module.

Now hold the \code{Z} button and reload and remove your module again:
\begin{verbatim}
insmod /root/nunchuk/nunchuk.ko; rmmod nunchuk
\end{verbatim}

You should now see the message confirming that the driver found
out that the \code{Z} button was held.

Do the same over and over again with various button states.

At this stage, we just made sure that we could read the state
of the device registers through the I2C bus. Of course, loading and
removing the module every time is not an acceptable way of
accessing such data. We will give the driver a proper {\em input}
interface in the next slides.
