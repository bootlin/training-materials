\subchapter{System Customization}{Learn how to build a customized Android}

After this lab, you will be able to:
\begin{itemize}
  \item Use the product configuration system
  \item Change the default wallpaper
  \item Add extra properties to the system
  \item Use the product overlays
\end{itemize}

\section{Setup a new product}

Using the \code{git checkout} command on the files you modified,
revert the changes you made to the beagleboneblack product and define
a new product named \textit{training} instead\footnote{Remember that
  all the source files we have were obtained from various git
  repositories. This allows to cancel changes or to get back to
  earlier versions.}.  This product will have all the attributes of
the beaglebone product for now, plus the extra packages we will add
along the labs.

Remember that you need to use \code{make installclean} when switching
from one product to another.

The system should compile and boot flawlessly on the BeagleBone, with
all the corrections we made earlier.

\section{Change the default wallpaper}

First, set up an empty overlay in your product directory.

The default wallpaper is located in \code{frameworks/base/core/res/res/drawable/}.
Use the overlay mechanism to replace the wallpaper by a custom one without
modifying the original source code.
