\subchapter{Ubuntu Linux installation}{Installing Linux on your
workstation}

Our training labs are done with the Ubuntu 12.04 distribution
(\url{http://www.ubuntu.com/}), one of the most popular GNU/Linux distributions. We
are going to use the Desktop edition.

These steps are meant to be executed before the training session. You can get
support by sending e-mail to \href{mailto:
support@free-electrons.com}{support@free-electrons.com}.

\section{Download Ubuntu}

Get Ubuntu Desktop 12.04 from
\url{http://www.ubuntu.com/desktop/get-ubuntu/download} and choose the
64 bit (amd64) version. Note that the 32 bit version is not supported.

{\bf Important note}: in our practical labs, we don't support Linux installations made
in a virtual machine (VMware, VirtualBox, etc.). It's because we will need to
access real hardware (serial port, USB, etc.), and this will be very difficult
to do through a virtual machine. Another reason is that some of our labs
require strong computing resources, and using a virtual machine could
cause people to spend much more time compiling software than if they they
didn't use a virtual machine, all this at the expense of time available
for learning and making experiments.

Follow the instructions on the download page to burn a cdrom or to
prepare a bootable USB disk.

\section{Freeing space on the hard drive}

Do some cleaning up on your hard drive. In order to install Ubuntu and do the
labs in good conditions, you will need at least 40 GB of free space.

\subsection{Defragmenting Windows partitions}
Now, defragment your Windows partitions (if Windows is installed on your
PC). This will allow to make contiguous disk space available for a separate
Linux partition.

\section{Install Ubuntu}

Once you have gathered enough contiguous disk space,
you can go ahead and install Ubuntu on your PC.

Follow the instructions given on
\url{http://www.ubuntu.com/download/desktop/install-desktop-long-term-support}.

We advise you to let the installing utility figure out the disk
partitions by itself. The default settings are fine for our training
labs. Just make sure that you allocate at least 40 GB of storage to
install Ubuntu.

\section{Configure network and Internet access}

Make sure you can access the network properly. Ubuntu automatically uses DHCP to
get an IP address from your network, so it usually just works flawlessly.

If your company requires to go through a proxy to connect to the Internet, you
can configure this through the \code{Network} application in the \code{System settings}
interface (usually available at the upper right corner of your screen).

\section{Configure package repositories}

Now, make sure the Ubuntu package repositories are properly enabled, by running
\code{sudo synaptic} in a terminal or \code{System Tools -> Administration ->
Synaptic Package Manager} from the desktop. Make sure that the \code{universe}
and \code{multiverse} repositories are all enabled in the \code{Settings ->
Repositories} menu.

You can also make these changes by hand by editing the
\code{/etc/apt/sources.list} file, and uncommenting the corresponding lines.
For your convenience, you should unselect the \code{cdrom} package source.
Most of the graphical tools in Linux are based on command-line tools, so there's
usually more than one way to configure something!

\section{Apply package updates}
In Synaptic, hit the \code{Reload} button, which will download the latest version of
the packages lists from the Ubuntu servers. This operation is the same as
running \code{sudo apt-get update} on the command line. Then, hit Mark all
upgrades and then \code{Apply}. This will do the same as \code{sudo
apt-get dist-upgrade} in the command line.

Depending on your network connection speed, this could take from several minutes
to approximately one hour.

Cleaning downloaded package update files can save hundreds of megabytes. This is
useful if free space is scarce.

Once this is done, remove downloaded package update files:
\begin{verbatim}
sudo apt-get clean
\end{verbatim}

Rebooting is needed after applying kernel updates, if there were any.
Please reboot your computer when you are done applying the updates.
