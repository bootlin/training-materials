\subchapter{XDP - Using eBPF}
{Objective: learn eBPF programming using the XDP hook}

\section{Goals}

 \begin{itemize}
 \item Compile our first eBPF program, and use it to drop everything
 \item Use eBPF maps through a userspace program
 \item Circumvent the hardware flaws of the armada3720 with XDP\_REDIRECT
 \end{itemize}

\section{Compiling and loading our first XDP program}

The Armada 3720 uses the \href{tod}{mvneta} driver, which has support for XDP upstream.

Let's first start by compiling and loading the simplest possible program, which
will accept any incoming frame.

Let's do this work in our \textbf{buildroot overlay} directory :

\begin{hostbashinput}
$ cd /home/$USER/networking-labs/buildroot/overlay/root
$ mkdir xdp
$ cd xdp
\end{hostbashinput}

Create a simple program named \code{xdp-pass.bpf.c} that always returns \code{XDP_PASS}, based on the following
skeleton :

\begin{verbatim}
#include <linux/bpf.h>
#include <bpf/bpf_helpers.h>

SEC("xdp")
int xdp_pass_prog(struct xdp_md *ctx)
{
	/* Your code here */
}

char _license[] SEC("license") = "GPL";
\end{verbatim}

Compile it to eBPF with \code{clang} :

\begin{hostbashinput}
clang -target bpf -g -O2 -c xdp-pass.bpf.c -o xdp-pass.bpf.o
\end{hostbashinput}

Finally, we need to re-generate our linux image, so that it now contains our XDP
program. We need to enable a few options in buildroot for that :

\begin{hostbashinput}
$ cd /home/$USER/networking-labs/buildroot
$ make menuconfig
\end{hostbashinput}

Search with '/' the option "bpftool", enable it by pressing space. Then look for "elfutils", and also enable it.
Move to the "Save" option, and hit "Quit" mutiple times until you get out of the menuconfig interface.

You can now re-generate the image :

\begin{hostbashinput}
$ make
\end{hostbashinput}

Now re-flash your micro SDcard with the \textbf{dd} command, as explained in the
setup lab.

Once your Espressbin has started, let's setup the network interface lan1


\section{Filtering}


Our first XDP test program will perform the simple task of dropping every packet
that arrives onto the lan0 interface. This is not as simple as it sounds, as the
Espressobin uses a DSA switch. This means that by default, the \code{eth0} port
receives frames from ALL ports. It distinguises between the frames from each port
by looking at the \textbf{DSA tag}. This will be covered in greater details in
the next section of the training.

Each frame that arrives into our interfaces has the following layout :

\begin{verbatim}
   .----.----.--------.--------.-----.----.---------
   | DA | SA | 0xdada | 0x0000 | DSA | ET | Payload ...
   '----'----'--------'--------'-----'----'---------
     6    6       2        2      4    2       N
\end{verbatim}

This layout is called "Extended DSA", and includes an 8-byte in-between the MAC
source address and the Ethertype.

The tag contains :
\begin{itemize}
	\item a 2-bytes etherype \code{0xdada}
	\item 2 bytes of zeroes
	\item A 4-bytes DSA tag with the value :
		\begin{verbatim}
		dsa_header[0] = (1 << 6) | tag_dev;
		dsa_header[1] = tag_port << 3;
		dsa_header[2] = 0;
		dsa_header[3] = 0;
		\end{verbatim}
\end{itemize}

In order to know the TAG id of a given interface, you need to look at the devicetree :
\kfile{arch/arm64/boot/dts/marvell/armada-3720-espressobin.dtsi\#L182}

The Tag id is set by the \code{reg} value.

If we summarize, we only need to get the 8-byte tag, and recover the \code{tag_port} with

\begin{verbatim}

    /* Skip the first 4 bytes of the EDSA tag*/

    /* Check byte 1 of the DSA tag */
    source_port = (dsa_header[1] >> 3) & 0x1f;

\end{verbatim}

Create a new file in \code{buildroot/overlay/root/xdp} named \code{xdp-filter.bpf.c}, and use
the following skeleton :

\begin{verbatim}
#include <linux/bpf.h>
#include <bpf/bpf_helpers.h>

SEC("xdp")
int xdp_filter_prog(struct xdp_md *ctx)
{
        void *data_end = (void *)(long)ctx->data_end;
	void *data = (void *)(long)ctx->data;

	/* Grab the Ethernet header */
	struct ethhdr *eth = data;

	/* Check the size */
	if (eth + 1 > data_end)
	        return XDP_DROP;

	/* Skip the 2 '0' bytes */
	/* Get the next 4 bytes and check the source_port */

        return XDP_PASS;
}
char _license[] SEC("license") = "GPL";
\end{verbatim}

Compile and load that program.

You should now be only able to use the LAN1 and WAN, the LAN0 interface should
drop any incoming packet.

%\section{Redirect to CPU}
%
%Armada 3720 has a known issue, the network interrupts aren't routed correctly and all network interrupts
%are handled by CPU core 0.
%
%We however have 2 cores available, it would be nice to be able to spread the load
%
% - Redirect to other CPU
%
%\kfile{tools/testing/selftests/bpf/progs/test_xdp_with_cpumap_helpers.c}
%
% - Maps : Simple firewall drop all udp traffic except ports we configure
