\subchapter{Filesystems - Block file systems}{Objective: configure and
  boot an embedded Linux system relying on block storage}

After this lab, you will be able to:
\begin{itemize}
\item Manage partitions on block storage.
\item Produce file system images.
\item Configure the kernel to use these file systems
\item Use the tmpfs file system to store temporary files
\end{itemize}

\section{Goals}

After doing the {\em A tiny embedded system} lab, we are going to copy
the filesystem contents to the emulated SD card. The storage will be
split into several partitions, and your QEMU emulated board will
be booted from this SD card, without using NFS anymore.

\section{Setup}

Throughout this lab, we will continue to use the root filesystem we
have created in the \code{$HOME/__SESSION_NAME__-labs/tinysystem/nfsroot}
directory, which we will progressively adapt to use block filesystems.

\section{Filesystem support in the kernel}

Recompile your kernel with support for SquashFS and ext4\footnote{Basic
configuration options for these filesystems will be sufficient. No need
for things like extended attributes.}.

Update your kernel image on the tftp server. We will only later copy
the kernel to our FAT partition.

Boot your board with this new kernel and on the NFS filesystem you
used in this previous lab.

Now, check the contents of \code{/proc/filesystems}. You should see
 that ext4 and SquashFS are now supported.

\section{Format the third partition}

We are going to format the third partition of the SD card image
with the ext4 filesystem, so that it can contain uploaded images.

Setup the loop device again:
\begin{verbatim}
sudo losetup -f --show --partscan sd.img
\end{verbatim}

And then format the third partition:

\begin{verbatim}
sudo mkfs.ext4 -L data /dev/loop<x>p3
\end{verbatim}

Now, mount this new partition on a directory on your host (you could
create the \code{/mnt/data} directory, for example) and move the contents of the
\code{/www/upload/files} directory (in your target root filesystem) into
it. The goal is to use the third partition of the SD card as the storage
for the uploaded images.

You can now unmount the partition and free the loop device:
\begin{verbatim}
sudo umount /mnt/data
sudo losetup -d /dev/loop<x>
\end{verbatim}

Now, restart QEMU and from the Linux command line and
mount this third partition on \code{/www/upload/files}.

Once this works, modify the startup scripts in your root filesystem
to do it automatically at boot time.

Reboot your target system again and with the mount command, check that
\code{/www/upload/files} is now a mount point for the third SD card
partition. Also make sure that you can still upload new images, and
that these images are listed in the web interface.

\section{Adding a tmpfs partition for log files}

For the moment, the upload script was storing its log file in
\code{/www/upload/files/upload.log}. To avoid seeing this log file in
the directory containing uploaded files, let's store it in
\code{/var/log} instead.

Add the \code{/var/log/} directory to your root filesystem and modify
the startup scripts to mount a \code{tmpfs} filesystem on this
directory. You can test your \code{tmpfs} mount command line on the
system before adding it to the startup script, in order to be sure
that it works properly.

Modify the \code{www/cgi-bin/upload.cfg} configuration file to store
the log file in \code{/var/log/upload.log}. You will lose your log
file each time you reboot your system, but that's OK in our
system. That's what \code{tmpfs} is for: temporary data that you don't need
to keep across system reboots.

Reboot your system and check that it works as expected.

\section{Making a SquashFS image}

We are going to store the root filesystem in a SquashFS filesystem in
the second partition of the SD card.

In order to create SquashFS images on your host, you need to install
the \code{squashfs-tools} package. Now create a SquashFS image of your
NFS root directory.

Setup the loop device again, and using the \code{dd} command,
copy the file system image to the second partition in the SD card
image. Release the loop device.

\section{Booting on the SquashFS partition}

In the U-boot shell, configure the kernel command line to use the
second partition of the SD card as the root file system. Also add the
\code{rootwait} boot argument, to wait for the SD card to be properly
initialized before trying to mount the root filesystem. Since the SD
cards are detected asynchronously by the kernel, the kernel might try
to mount the root filesystem too early without \code{rootwait}.

Check that your system still works. Congratulations if it does!

\section{Store the kernel image and DTB on the SD card}

Setup the loop device again, and mount the FAT partition
in the SD card image (for example on \code{/mnt/boot}).
Then copy the kernel image and Device Tree to it.

Unmount the FAT partition and release the loop device.

You now need to adjust the \code{bootcmd} of U-Boot so
that it loads the kernel and DTB from the SD card instead of loading
them from the network.

In U-boot, you can load a file from a FAT filesystem using a command
like

\begin{verbatim}
fatload mmc 0:1 0x61000000 filename
\end{verbatim}

Which will load the file named \code{filename} from the first
partition of the device handled by the first MMC controller to the
system memory at the address \code{0x61000000}.

Type \code{reset} in U-Boot to reboot the board and make
sure that your system still boots fine.
