\subchapter{Lab2: Advanced Yocto configuration}{Configure the build, customize the
	output images and use NFS}

During this lab, you will:
\begin{itemize}
  \item Customize the package selection
  \item Configure the build system
  \item Use the rootfs over NFS
\end{itemize}

\section{Set up the Ethernet communication and NFS on the board}

It isn't practical at all to reflash the root filesystem on the target
every time a change is made. Fortunately, it is possible to set up
networking between the development workstation and the target. Then,
workstation files can be accessed by the target through the network,
using NFS.

First we need to set the kernel boot arguments U-Boot will pass to the
Linux kernel at boot time. For that, edit the extlinux configuration file, in
the bootfs partition of the SD card and change the \code{APPEND} line to:

{\small
\begin{verbatim}
APPEND root=/dev/nfs rw console=${console},${baudrate} nfsroot=192.168.0.1:/nfs,vers=3,tcp ip=192.168.0.100:::::eth0
\end{verbatim}
}

\section{Set up the Ethernet communication on the workstation}

With a network cable, connect the Ethernet port of your board to the
one of your computer. If your computer already has a wired connection
to the network or does't have any Ethernet port, your instructor will
provide you an USB Ethernet adapter. A new network interface should
 appear on your Linux system.

Find the name of this interface by typing:
\begin{verbatim}
ip a
\end{verbatim}

The network interface name is likely to be
\code{enxxx}\footnote{Following the {\em Predictable Network Interface
Names} convention:
\url{https://www.freedesktop.org/wiki/Software/systemd/PredictableNetworkInterfaceNames/}}.
If you have a pluggable Ethernet device, it's easy to identify as it's
the one that shows up after pluging in the device.

Then, instead of configuring the host IP address from Network
Manager's graphical interface, let's do it through its command line
interface, which is so much easier to use:

\begin{verbatim}
nmcli con add type ethernet ifname en... ip4 192.168.0.1/24
\end{verbatim}

\section{Set up the NFS server on the workstation}

First install the NFS server on the training computer and create the root NFS
directory:
\begin{verbatim}
sudo apt install nfs-kernel-server
sudo mkdir -m 777 /nfs
\end{verbatim}

Then make sure this directory is used and exported by the NFS server by adding
the below line to the \code{/etc/exports} file:

\begin{verbatim}
/nfs *(rw,sync,no_root_squash,subtree_check)
\end{verbatim}

Finally, make the NFS server use the new configuration:
\begin{verbatim}
sudo exportfs -r
\end{verbatim}

\section{Add a package to the rootfs image}

You can add packages to be built by editing the local configuration file
\code{$BUILDDIR/conf/local.conf}. The \yoctovar{IMAGE_INSTALL} variable controls the
packages included into the output image.

To illustrate this, add the Dropbear SSH server to the list of enabled
packages.

Tip: do not overwrite the default enabled package list, but append the Dropbear
package instead.

\section{Boot with the updated rootfs}

First we need to put the rootfs under the NFS root directory so that it is
accessible by NFS clients. Simply uncompress the archived output image in the
previously created \code{/nfs} directory:
\begin{verbatim}
sudo tar xpf $BUILDDIR/tmp/deploy/images/beagleplay/\
  core-image-minimal-beagleplay.rootfs.tar.xz -C /nfs
\end{verbatim}

Then boot the board.

The Dropbear SSH server was enabled a few steps before, and should now be
running as a service on the BeaglePlay. You can test it by accessing the
board through SSH:
\begin{verbatim}
ssh root@192.168.0.100
\end{verbatim}

You should see the BeaglePlay command line!

\section{Choose a package variant}

Dependencies of a given package are explicitly defined in its recipe.
Some packages may need a specific library or piece of software but
others only depend on a functionality. As an example, the kernel
dependency is described by \code{virtual/kernel}.

To see which kernel is used, dry-run BitBake:
\begin{verbatim}
bitbake -vn virtual/kernel
\end{verbatim}

In our case, we can see the \code{linux-ti-staging} provides the
\code{virtual/kernel} functionality:
\small
\begin{verbatim}
NOTE: selecting linux-ti-staging to satisfy virtual/kernel due to PREFERRED_PROVIDERS
\end{verbatim}
\normalsize

We can force Yocto to select another \code{kernel} by explicitly
defining which one to use in our local configuration. Try switching
from \code{linux-ti-staging} to \code{linux-dummy} only using the
local configuration.

Then check the previous step worked by dry-running again BitBake.
\begin{verbatim}
bitbake -vn virtual/kernel
\end{verbatim}

Tip: you may need to define the more specific information here to be sure
it is the one used. The \yoctovar{MACHINE} variable can help here.

As this was only to show how to select a preferred provider for a
given package, you can now use \code{linux-ti-staging} again.

\section{BitBake tips}

BitBake is a powerful tool which can be used to execute specific commands. Here
is a list of some useful ones, used with the \code{virtual/kernel} package.

\begin{itemize}
  \item The Yocto recipes are divided into numerous tasks, you can print them
        by using: \code{bitbake -c listtasks virtual/kernel}.
  \item BitBake allows to call a specific task only (and its dependencies)
        with: \code{bitbake -c <task> virtual/kernel}. (\code{<task>} can be
        \code{menuconfig} here).
  \item You can force to rebuild a package by calling: \code{bitbake -f
        virtual/kernel}
  \item \code{world} is a special keyword for all packages. \code{bitbake
        --runall=fetch world} will download all packages sources (and their
        dependencies).
  \item You can get a list of locally available packages and their current
        version with: \\
        \code{bitbake -s}
  \item You can also find detailed information on available packages, their
        current version, dependencies or the contact information of the
        maintainer by visiting: \\
        \url{https://layers.openembedded.org/layerindex/branch/master/recipes/}
\end{itemize}

For detailed information, please run \code{bitbake -h}

\section{Going further}

If you have some time left, let's improve our setup to use TFTP, in
order to avoid having to reflash the SD card for every test.

First, install a TFTP server (package \code{tftpd-hpa}) on your system.

Then copy the Linux kernel image and Device Tree to the TFTP server home
directory (specified in \code{/etc/default/tftpd-hpa}) so that they are
made available by the TFTP server.

Then, in the U-Boot shell, change the \code{bootcmd} variable to load the
kernel image and the Device Tree over TFTP (replace \code{dtb} by
the actual Device Tree file for your board):

\small{
\begin{verbatim}
setenv ipaddr 192.168.0.100
setenv serverip 192.168.0.1
setenv bootcmd 'tftp ${loadaddr} Image; tftp ${fdtaddr} dtb; booti ${loadaddr} - ${fdtaddr}'
\end{verbatim}
}

Still in the U-Boot shell, set the \code{bootargs} specifying the kernel command line that
we previously set in \code{extlinux.conf}:

{\small
\begin{verbatim}
setenv bootargs root=/dev/nfs rw console=${console},${baudrate}
  nfsroot=${serverip}:/nfs,vers=3,tcp ip=${ipaddr}:::::eth0
\end{verbatim}
}

See the training materials of our {\em Embedded Linux system
  development} course for details!
