\subchapter{Lab8: Develop your application in the Poky SDK}{Generate and use
the Poky SDK}

During this lab, you will:
\begin{itemize}
  \item Build the Poky SDK
  \item Install the SDK
  \item Compile an application for your machine in the SDK
    environment
\end{itemize}

\section{Build the SDK}

Two SDKs are available, one only embedding a toolchain and the
other one allowing for application development. We will use the latter one
here.

\if\defstring{\labboard}{stm32mp1}
On \code{stm32mp1}, we first need to remove ST host tooling that we
don't need and that currently fails to build anyway.

So, remove all the lines touching \yoctovar{TOOLCHAIN_HOST_TASK} in
\code{meta-st-stm32mp/conf/machine/include/st-machine-common-stm32mp.inc}.
\fi

Build an SDK for the \code{bootlinlabs-image-minimal} image, with
the \code{populate_sdk} task.

Once the SDK is generated, a script will be available in
\code{tmp/deploy/sdk}.

\section{Install the SDK}

Open a new terminal to be sure that no extra environment variable is set.
We mean to show you how the SDK sets up a fully working environment.

Install the SDK in \code{$HOME/__SESSION_NAME__-labs/sdk} by executing the script
generated at the previous step.

{\footnotesize
\if\defstring{\labboard}{beaglebone}
\begin{verbatim}
$BUILDDIR/tmp/deploy/sdk/poky-glibc-x86_64-bootlinlabs-image-minimal-cortexa8hf-neon-toolchain-2.5.sh
\end{verbatim}
\fi
\if\defstring{\labboard}{stm32mp1}
\begin{verbatim}
$BUILDDIR/tmp/deploy/sdk/poky-glibc-x86_64-bootlinlabs-image-minimal-cortexa7t2hf-neon-vfpv4-bootlinlabs-toolchain-5.0.1.sh
\end{verbatim}
\fi
\if\defstring{\labboard}{beagleplay}
\begin{verbatim}
$BUILDDIR/tmp/deploy/sdk/poky-glibc-x86_64-bootlinlabs-image-minimal-aarch64-bootlinlabs-toolchain-5.0.4.sh
\end{verbatim}
\fi
}

\section{Set up the environment}

Go into the directory where you installed the SDK
(\code{$HOME/__SESSION_NAME__-labs/sdk}). Source the environment script:
\if\defstring{\labboard}{beaglebone}
\begin{verbatim}
source environment-setup-cortexa8hf-vfp-neon-poky-linux-gnueabi
\end{verbatim}
\fi
\if\defstring{\labboard}{stm32mp1}
\begin{verbatim}
source environment-setup-cortexa7t2hf-neon-vfpv4-poky-linux-gnueabi
\end{verbatim}
\fi
\if\defstring{\labboard}{beagleplay}
\begin{verbatim}
source environment-setup-aarch64-poky-linux
\end{verbatim}
\fi

Have a look at the exported environment variables:
\begin{verbatim}
env
\end{verbatim}

\section{Compile an application in the SDK}

Download the essential \code{Ctris} sources at
\url{https://download.mobatek.net/sources/ctris-0.42-1-src.tar.bz2}

Extract the source in the SDK:
\begin{verbatim}
tar xf ctris-0.42-1-src.tar.bz2
tar xf ctris-0.42.tar.bz2
cd ctris-0.42
\end{verbatim}

Then modify the Makefile, to make sure that the environment variables exported
by the SDK script are not overridden.

Try to compile the application. Just like nInvaders, ctris is also an old
program and won't build with a recent toolchain. You will face these
errors:

\begin{itemize}
  \item The ctris makefile uses the native compiler, not the cross compiler
  provided by the SDK; while you could fix it using \code{make -e} as done
  for nInvaders, try fixing it by editing the \code{Makefile} this time;
  hint: you don't need to write any code, just to delete two lines

  \item Building with a recent GCC will give the following error, not
  reported by older versions:
\begin{verbatim}
error: format not a string literal and no format arguments [-Werror=format-security]
\end{verbatim}
  Fix this by adding \code{-Wno-error=format-security} to \code{CFLAGS}

  \item As for nInvaders, you will see the \code{multiple definition of...}
  error; add the \code{-fcommon} flag to \yoctovar{CFLAGS} also for ctris

\end{itemize}

You can check the application was successfully compiled for the right
target by using the \code{file} command.  The \code{ctris} binary should be
an ELF \ifdefstring{\labboard}{beagleplay}{64}{32}-bit LSB
executable compiled for ARM.

Finally, you can copy the binary to the board, by using the \code{scp}
command. Then run it and play a bit to ensure it is working fine!
