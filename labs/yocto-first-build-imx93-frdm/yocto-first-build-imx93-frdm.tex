\subchapter{Lab1: First Yocto Project build}{Your first dive into Yocto
Project and its build mechanism}

During this lab, you will:
\begin{itemize}
  \item Set up an OpenEmbedded environment
  \item Configure the project and choose a target
  \item Build your first Poky image
\end{itemize}

\section{Setup}

Before starting this lab, make sure your home directory is not
encrypted using eCryptFS. OpenEmbedded cannot be used on top of an eCryptFS file
system due to limitations in file name lengths.

Install the required packages:
\begin{bashinput}
sudo apt install gawk wget git diffstat unzip texinfo gcc build-essential \
  chrpath socat cpio python3 python3-pip python3-pexpect xz-utils debianutils \
  iputils-ping python3-git python3-jinja2 python3-subunit zstd liblz4-tool \
  file locales libacl1
\end{bashinput}

\section{Avoiding unprivileged user namespace restrictions}

Ubuntu 24.04 added an apparmor policy preventing usage of unprivileged user
namespace restrictions to improve security. Unfortunately this prevents
bitbake from working, because it uses namespaces to forbid untracked
downloads outside of the \code{do_fetch} task. Ironically, bitbake does
this to improve security. This results in the following error message:

\begin{verbatim}
ERROR: PermissionError: [Errno 1] Operation not permitted
...
    with open("/proc/self/uid_map", "w") as f:
\end{verbatim}

To disable this apparmor restriction, run this command in a shell:

\begin{bashinput}
echo 0 | sudo tee /proc/sys/kernel/apparmor_restrict_unprivileged_userns
\end{bashinput}

You will need to run this command every time you reboot your machine.

For more information, including how to disable this restriction
persistently, see
\href{https://discourse.ubuntu.com/t/ubuntu-24-04-lts-noble-numbat-release-notes/39890#p-99950-unprivileged-user-namespace-restrictions}
     {the Ubuntu 24.04 Release Notes}.

\section{Download Yocto}

Go to the \code{$HOME/__SESSION_NAME__-labs/} directory.

Download the \code{scarthgap} version of Poky:
\begin{bashinput}
git clone https://git.yoctoproject.org/git/poky
cd $HOME/__SESSION_NAME__-labs/poky
git checkout -b scarthgap-5.0.4 scarthgap-5.0.4
\end{bashinput}

Return to your project root directory (\code{cd $HOME/__SESSION_NAME__-labs/})
and download the \code{meta-openembedded}, \code{meta-arm}, and \code{meta-freescale} layers:
\begin{bashinput}
git clone -b scarthgap https://github.com/openembedded/meta-openembedded.git
git clone -b scarthgap https://git.yoctoproject.org/git/meta-arm
git clone -b scarthgap https://github.com/Freescale/meta-freescale.git
\end{bashinput}

\section{Set up the build environment}

Check you're using Bash. This is the default shell when using Ubuntu.

Export all needed variables and set up the build directory:
\begin{bashinput}
cd $HOME/__SESSION_NAME__-labs
source poky/oe-init-build-env
\end{bashinput}

You must specify which machine is your target. By default it
is \code{qemu}. We need to build an image for an \code{imx93-11x11-lpddr4x-evk}.
Update the \yoctovar{MACHINE} configuration variable accordingly.

You must also accept the Freescale EULA by adding the following line to the
\code{local.conf} file: \code{ACCEPT_FSL_EULA = "1"}.

Also, if you need to save disk space on your computer you can add \code{INHERIT
+= "rm_work"} in the previous configuration file. This will remove the
package work directory once a package is built.

Don't forget to update the configuration to include all the layers you have just added. Edit the
layer configuration file (\code{$BUILDDIR/conf/bblayers.conf}) and append the
full path to the \code{meta-freescale},
\code{meta-arm/meta-arm}, \code{meta-arm/meta-arm-toolchain}, \code{meta-openembedded/meta-python},
\code{meta-openembedded/meta-oe}, and \code{meta-openembedded/meta-multimedia}
directories to the \code{BBLAYERS} variable.

\section{Build your first image}

Now that you're ready to start the compilation, simply run:
\begin{bashinput}
bitbake core-image-minimal
\end{bashinput}

Once the build finished, you will find the output images under
\code{$BUILDDIR/tmp/deploy/images/imx93-11x11-lpddr4x-evk}.

\section{Set up the SD card}

In this first lab we will use an SD card to store the bootloader, kernel and
root filesystem files. The SD card image has been generated and is
named \code{core-image-minimal-imx93-11x11-lpddr4x-evk.rootfs.wic.gz}.

Now flash the image with the following command:
\begin{bashinput}
gunzip -c core-image-minimal-imx93-11x11-lpddr4x-evk.rootfs.wic.gz | sudo dd of=/dev/mmcblkX conv=fdatasync bs=1M status=progress
\end{bashinput}

\section{Setting up serial communication with the board}
To set up serial communication with the board, simply connect the USB-C cable to the port labeled DEBUG. 

Once the cable is plugged in, a new serial port
should appear: \code{/dev/ttyACM0}.  You can also see this device
appear by looking at the output of \code{dmesg}.

To communicate with the board through the serial port, install a
serial communication program, such as \code{picocom}:

\begin{bashinput}
sudo apt install picocom
\end{bashinput}

If you run \code{ls -l /dev/ttyACM0}, you can also see that only
\code{root} and users belonging to the \code{dialout} group have
read and write access to this file. Therefore, you need to add your user
to the \code{dialout} group:

\begin{bashinput}
sudo adduser $USER dialout
\end{bashinput}

{\bf Important}: for the group change to be effective, in Ubuntu 18.04, you have to
{\em completely reboot} the system \footnote{As explained on
\url{https://askubuntu.com/questions/1045993/after-adding-a-group-logoutlogin-is-not-enough-in-18-04/}.}.
A workaround is to run \code{newgrp dialout}, but it is not global.
You have to run it in each terminal.

Now, you can run \code{picocom -b 115200 /dev/ttyACM0}, to start serial
communication on \code{/dev/ttyACM0}, with a baudrate of \code{115200}. If
you wish to exit \code{picocom}, press \code{[Ctrl][a]} followed by
\code{[Ctrl][x]}.

There should be nothing on the serial line so far, as the board is not
powered up yet.

\section{Boot}
Insert the SD card in the dedicated slot on the imx93-frdm. Make sure that switches 1 and 2 in the center of the board are set to ON, and switches 3 and 4 are set to OFF.

Plug in the power USB-C. You should see boot messages on the console.

Wait until the login prompt, then enter \code{root} as user.
Congratulations! The board has booted and you now have a shell.
