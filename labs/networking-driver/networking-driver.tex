\subchapter{Investigating low level behaviour}
{Objective: Configure our testing setup for the labs}

\section{Goals}

\begin{itemize}
\item Configure offloading and measure its impacts
\item Analyse traffic and drops
\item Use MQPrio to improve performances in specific setups
\item Use traffic generation tools
\end{itemize}

\section{Install the required packages}

\begin{hostbashinput}
$ sudo apt install iperf3 dropwatch
\end{hostbashinput}

\section{Traffic generation with iperf3}

Let's use \code{iperf3} to generate traffic between the Host and Target.

Run iperf3 in servermode (\code{-s}) and in the background (\code{-D})
\begin{targetbashinput}
$ ip link set lan1 up
$ ip address add 192.168.42.2/24 dev lan1
$ iperf3 -s -D
\end{targetbashinput}

Generate some traffic from your host :

\begin{hostbashinput}
$ iperf3 -c 192.168.42.2
\end{hostbashinput}

By default, TCP traffic is sent. Let's now try UDP, using 400 bytes datagrams

\begin{hostbashinput}
$ iperf3 -c 192.168.42.2 -u -b 0 -l 400
\end{hostbashinput}

The speed you are seeing is the speed at which the Host sends UDP to the Target.

To see the speed at which the target manages to receive the packets, you need to
run the \code{iperf3} server in the foreground :

\begin{targetbashinput}
$ killall iperf3
$ iperf3 -s
\end{targetbashinput}

Run the host-side iperf3 again. What do you see ?

\section{Analysing performances}

One way of investigating the ingress processing is by \textbf{profiling} the system.

Run the iperf3 UDP stream in the background, and start perf :

\begin{targetbashinput}
$ perf top
\end{targetbashinput}

Can you identify the bottleneck ?

\section{Using mqprio for traffic priorisation}

When using a tool such as MQPrio, the process is more involved as you will need
to classify your traffic, to indicate which packets are high-priority.

We will use VLANs for that purpose. We are going to create 2 flows between the
host and target :
\begin{itemize}
	\item An untagged flow, with low-priority traffic, on 192.168.42.0/24
	\item A tagged flow, with high-priority traffic, on 192.168.50.0/24
\end{itemize}

Create a VLAN interface on your host machine, with an \textbf{egress mapping} :

\begin{hostbashinput}
$ ip link add link <iface> name e.10 type vlan id 10 egress 0:0 1:1 2:2 3:3 4:4 5:5 6:6 7:7
$ ip address add 192.168.50.1/24 dev e.10
$ ip address add 192.168.42.1/24 dev <iface>
\end{hostbashinput}

Do the equivalent command on the Espressobin :
\begin{targetbashinput}
$ ip link add link lan1 name lan1.10 type vlan id 10 egress 0:0 1:1 2:2 3:3 4:4 5:5 6:6 7:7
$ ip address add 192.168.50.2/24 dev lan1.10
$ ip address add 192.168.42.2/24 dev lan1
\end{targetbashinput}

Don't forget to put your interfaces UP !

Now, run the \code{iperf3} server on your Espressbin, in the background :
\begin{targetbashinput}
$ iperf3 -s -D
\end{targetbashinput}

Let's now mark all traffic on \code{lan1.10} as priority 7 :
\begin{targetbashinput}
iptables -t mangle -A POSTROUTING -o lan1.100 -p udp -j CLASSIFY --set-class 0:7
iptables -t mangle -A POSTROUTING -o lan1.100 -p tcp -j CLASSIFY --set-class 0:7
\end{targetbashinput}

Finally, let's configure \textbf{mqprio} :
\begin{targetbashinput}
tc qdisc add dev eth0 parent root handle 100 mqprio num_tc 2 \
map 0 0 0 0 0 0 0 1 \
queues 7@0 1@7 hw 1 mode channel shaper \
bw_rlimit min_rate 0 0 max_rate 50Mbit 1000Mbit
\end{targetbashinput}

Now check that traffic on lan1 is rate-limited to 50Mbps and that traffic on lan1.100 goes at full speed !
