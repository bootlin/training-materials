\subchapter{Downloading kernel source code}{Get your own copy of the
mainline Linux kernel source tree}

\section{Setup}

Create the \code{$HOME/__SESSION_NAME__-labs/src} directory.

\section{Installing git packages}

First, let's install software packages that we will need
throughout the practical labs:

\begin{verbatim}
sudo apt install git gitk git-email
\end{verbatim}

\section{Git configuration}

After installing \code{git} on a new machine, the first thing to do is
to let \code{git} know about your name and e-mail address:

\begin{verbatim}
git config --global user.name 'My Name'
git config --global user.email me@mydomain.net
\end{verbatim}

Such information will be stored in commits. It is important
to configure it properly when the time comes to generate and
send patches, in particular.

It can also be particularly useful to display line numbers when using the
\code{git grep} command. This can be enabled by default with the following
configuration:

\begin{verbatim}
git config --global grep.lineNumber true
\end{verbatim}

\input{kernel-clone-master-and-stable.tex}
