\subchapter{Kernel - Cross-compiling}{Objective: Learn how to
cross-compile a kernel for an ARM target platform.}

After this lab, you will be able to:
\begin{itemize}
\item Checkout a stable version of the Linux kernel
\item Set up a cross-compiling environment
\item Cross compile the kernel for the \labboarddescription
\item Use U-Boot to download the kernel
\item Check that the kernel you compiled starts the system
\end{itemize}

\section{Setup}

Stay in the \code{$HOME/__SESSION_NAME__-labs/kernel} directory.

\section{Cross-compiling environment setup}

To cross-compile Linux, you need to have a cross-compiling
toolchain. We will use the cross-compiling toolchain that we
previously produced, so we just need to make it available in the PATH:

\begin{bashinput}
$ export PATH=$HOME/x-tools/aarch64-training-linux-musl/bin:$PATH
\end{bashinput}

Also, don't forget to either:

\begin{itemize}
\item Define the value of the \code{ARCH} and \code{CROSS_COMPILE}
  variables in your environment (using \code{export})
\item {\bf Or} specify them on the command line at every invocation of
  \code{make}, i.e.: \code{make ARCH=... CROSS_COMPILE=... <target>}
\end{itemize}

\section{Linux kernel configuration}

By running \code{make help}, look for the proper Makefile target to
configure the kernel for your processor.

The standard configuration for this kernel is actually \code{defconfig},
but this will generate a pretty big kernel with support for many other
SoCs. However, we can reduce it to compile faster and get a small
kernel.

So, apply this configuration , and then run \code{make menuconfig}.


\begin{itemize}

\item Disable \kconfig{CONFIG_GCC_PLUGINS} if it is set. This will skip
  building special {\em gcc} plugins, which would require extra dependencies
  for the build.
\item In the \code{System Type} menu, remove support for all the SoCs except
  the STM32MP257 ones. Don't forget to disable the TI ones too which are
  in a submenu.
\item Disable \kconfig{CONFIG_DRM}, which will skip support for many display
controller and GPU drivers.
\item In Device Drivers, disable the support of Mellanox \kconfig{CONFIG_NET_VENDOR_MELLANOX}
\item In Device Drivers -> Input device support -> Generic input layer, 
      disable the support of touch screan \kconfig{CONFIG_INPUT_TOUCHSCREEN}
\item In Device Driver enable \kconfig{CONFIG_DEVTMPFS} and \kconfig{CONFIG_DEBUG_FS}
\item We also need to enable the \kconfig{CONFIG_STMMAC_ETH}, \kconfig{CONFIG_STMMAC_PLATFORM}
      and \kconfig{CONFIG_DWMAC_STM32} options. This is required for the STM32MP2 SoC
\end{itemize}


Please note that this will definitely not build the smallest
and most optimized kernel for STM32MP2: \code{defconfig}
enables plenty of features and drivers that will not be useful on our
particular board.

\section{Cross compiling}

You're now ready to cross-compile your kernel. Simply run:

\bashcmd{$ make}

and wait a while for the kernel to compile. Don't forget to use
\code{make -j<n>} if you have multiple cores on your machine!

Look at the kernel build output to see which file contains
the kernel image.

Also look in the Device Tree Source directory to see which \code{.dtb}
files got compiled. Find which \code{.dtb} file corresponds to your board.


\section{Load and boot the kernel using U-Boot}

As we are going to boot the Linux kernel from U-Boot,
we need to set the \code{bootargs} environment corresponding
to the Linux kernel command line:

\begin{ubootinput}
=> setenv bootargs console=%\console%
=> saveenv
\end{ubootinput}
We will use TFTP to load the kernel image on the board:

\begin{itemize}

\item On your workstation, copy the {\tt Image} and DTB
(\texttt{\dtname}\texttt{.dtb}) to the directory exposed by the TFTP server.

\item On the target (in the U-Boot prompt), load {\tt Image} from
TFTP into RAM:
\begin{ubootinput}
=> tftp %\zimageboardaddr% Image%
\end{ubootinput}

\item Now, also load the DTB file into RAM:
\begin{ubootinput}
=> tftp %\dtbboardaddr% %\dtname%.dtb
\end{ubootinput}

\item Boot the kernel with its device tree:
\begin{ubootinput}
=> %booti \zimageboardaddr% - %\dtbboardaddr%
\end{ubootinput}

\end{itemize}

You should see Linux boot and finally panicking. This is expected: we
haven't provided a working root filesystem for our device yet.

You can now automate all this every time the board is booted or
reset. Reset the board, and customize \code{bootcmd}:

\begin{ubootinput}
=> setenv bootcmd 'tftp %\zimageboardaddr\ Image; tftp \dtbboardaddr\ {\dtname}.dtb; booti \zimageboardaddr\ - \dtbboardaddr'%
=> saveenv
\end{ubootinput}

\ifdefstring{\labboard}{beaglebone}
{{\bf Known issue}: with at least U-Boot 2023.04 and 2024.04 and with
USB networking, there is an issue running two \code{tftp} commands in a
row in \code{bootcmd}. To work around this issue before we get a chance
to fix it upstream, insert \code{sleep 0.1} between the two \code{tftp}
commands and everything will work as expected.}{}

Restart the board to make sure that booting the kernel is now automated.
