\subchapter{Backing up your lab files}{Objective: clean up and make an
archive of your lab directory}

\section{End of the training session}

Congratulations. You reached the end of the training session. You now
have plenty of working examples you created by yourself, and you can
build upon them to create more elaborate things.

In this last lab, we will create an archive of all the things you
created. We won't keep everything though, as there are lots of things
you can easily retrieve again.

\section{Create a lab archive}

Go to the directory containing your felabs directory:

\begin{verbatim}
cd $HOME
\end{verbatim}

Now, run a command that will do some clean up and then create an archive with the most important files:

\begin{itemize}
\item Kernel configuration files
\item Other source configuration files (BusyBox, Crosstool-ng...)
\item Kernel images
\item Toolchain
\item Other custom files
\end{itemize}

Here is the command:

\begin{verbatim}
./felabs/archive-labs
\end{verbatim}

At end end, you should have a \code{felabs-<user>.tar.lzma} archive
that you can copy to a USB flash drive, for example. This file should
only be a few hundreds of MB big.
