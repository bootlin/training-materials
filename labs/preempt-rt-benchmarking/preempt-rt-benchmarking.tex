\subchapter{Lab2: Testing and Benchmarking the system}{}

During this lab, you will:
\begin{itemize}
  \item Use benchmarking tools to measure the latencies of a system
  \item Use stressors to analyse various scenarios
  \item Determine suitable options to further improve latencies
\end{itemize}

\section{Determine the maximum latency}

First, we will need to install some benchmark and analysis tools on our system.

In the Buildroot \code{make menuconfig} interface, enable the following packages :

\begin{itemize}
	\item rt-tests
	\item powertop
	\item perf
	\item stress-ng
	\item iperf3
	\item fping
	\item scheduling utilites from the util-linux package
	\item python3
\end{itemize}

Re-build the image, and boot it on the Beaglebone.

Let's first start by establishing the baseline latency by simply running \code{cyclictest} :

\begin{bashinput}
cyclictest
\end{bashinput}

The goal of this lab is to try to lower the latency as much as possible while
the system is under various types of loads. This will allow us to get the
best running conditions for our applications.

Some tweaks will really be useful on SMP systems, so don't hesitate to test on
you own machine !

Here's a few leads :

\begin{itemize}
	\item Is the system SMP ?
	\item Do we need to isolate our task ?
	\item Try changing the scheduling policy
	\item Try changing the scheduling priority
	\item Investigate the various interrupts
	\item Take a look at the cpuidle and cpufreq configuration
	\item Are there any NMIs ?
\end{itemize}

Stress the system using several tools :

\begin{itemize}
	\item hackbench
	\item stress-ng
	\item iperf3
	\item fping
\end{itemize}

Some kernel options can also be useful :
\begin{itemize}
	\item \code{CONFIG_TRACE_HWLAT}
\end{itemize}
