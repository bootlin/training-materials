\subchapter{Git}{Objective: use the basic Git features}

After this lab, you will be able to:

\begin{itemize}
\item Clone a Git repository
\item Explore the history of a Git repository
\item Make changes in your own branch
\item Generate the patches corresponding to your own branch
\end{itemize}

\section{Setup}

Go to \code{/home/<user>/felabs/linux/git/}

This lab assumes that you already installed git software and cloned
the Linus Torvalds' git tree. See our {\em Kernel source code} lab for
details (\url{http://free-electrons.com/doc/training/linux-kernel/}).

\section{Configuring Git}

Configure your name and email address in git with \code{git config}.

\section{Clone a repository}

We already cloned Linus Torvalds' git tree, but it is useful to know
how to do it again. Go to \url{http://git.kernel.org} and make sure you know
how to find the \code{git://} URL of his Linux tree.

Cloning downloaded quite a lot of data, but then at the end, we have
the full history of the Linux kernel (since the kernel developers
started to use Git, around kernel 2.6.12). We can access and explore
this history offline.

\section{Exploring the history}

With \code{git log}, look at the list of changes that have been made on the scheduler.

With \code{git log}, look at the list of changes and their associated
patches, that have been made on the ATMEL serial driver
(\code{drivers/tty/serial/atmel_serial.c}) between the versions 3.0
and 3.1 of the kernel.

With \code{git diff}, look at the differences between \code{fs/jffs2/}
(which contains the JFFS2 filesystem driver) in 3.0 and 3.1.

With \code{gitk}, look at the full history of the UBIFS filesystem (in
\code{fs/ubifs/}).

On the {\em gitweb} interface of Linus Torvalds tree, available at
\url{http://git.kernel.org/?p=linux/kernel/git/torvalds/linux.git},
search all commits that have been done by Free Electrons (hint: use
the search engine by author).

\section{Make your changes}

Create your own branch with \code{git branch} and then move to it with
\code{git checkout}.

Make a dummy change to the \code{MAINTAINERS} file, and commit your
change. Look at the difference between the master branch and your
branch (with \code{git log}, \code{git diff} and \code{gitk}).

Then, edit \code{init/main.c}. In the function \code{start_kernel()},
after the call to \code{printk()} to print the \code{linux_banner}
variable, add a call to \code{printk()} to print your own
message. Commit your change.

\section{Share your changes}

Generate the patch series corresponding to your two changes using
\code{git format-patch}.

Configure your SMTP server using:

\begin{verbatim}
git config --global sendemail.smtpserver smtp.company.com
\end{verbatim}

And then send the patches to yourself using \code{git send-email}.

\section{Tracking another tree}

Say you want to work on the realtime Linux tree, so we'll add this
tree to the trees you're tracking:

\small
\begin{verbatim}
git remote add realtime \
  git://git.kernel.org/pub/scm/linux/kernel/git/rostedt/linux-2.6-rt.git
\end{verbatim}
\normalsize

A \code{git fetch} will fetch the data for this tree. Of course, Git
will optimize the storage, and will no store everything that's common
between the two trees. This is the big advantage of having a single
local repository to track multiple remote trees, instead of having
multiple local repositories.

We can then switch to the master branch of the realtime tree:

\begin{verbatim}
git checkout realtime/master
\end{verbatim}

Or look at the difference between the scheduler code in the official
tree and in the realtime tree:

\begin{verbatim}
git diff master..realtime/master kernel/sched.c
\end{verbatim}

