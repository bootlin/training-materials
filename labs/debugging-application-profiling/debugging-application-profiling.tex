\subchapter
{Application profiling}
{Objectives:
  \begin{itemize}
    \item Visualizing application heap using {\em Massif}.
    \item Profiling an application with {\em Cachegrind}, {\em Callgrind} and
          {\em KCachegrind}.
    \item Analyzing application performance with {\em perf}.
  \end{itemize}
}

\section{Massif}

Massif is really helpful to understand what is going on with the memory allocation
side of an application. Compile the \code{heap_profile} example that we did provide
using the following command on your development host

\begin{bashinput}
$ cd /home/$USER/__SESSION_NAME__-labs/nfsroot/root/heap_profile
$ make
\end{bashinput}

Once compiled, run it on the target under massif using the following command:

\begin{bashinput}
$ cd /root/heap_profile
$ valgrind --tool=massif --time-unit=B ./heap_profile
\end{bashinput}

NOTE: we use \code{--time-unit=B} to set the X axis to be based on the allocated
size.

Once done, a \code{massif.out.<pid>} file will be created. This file can be
displayed with \code{ms_print}. Based on the result, can you answer the
following questions:
\begin{itemize}
  \item What is the peak allocation size of this program?
  \item How much memory was allocated during the program lifetime?
  \item Do we have memory leaks at the end of execution?
\end{itemize}

You can also visualize the data collected by Massif with a graphical
interface thanks to \code{massif-visualizer}. To do so, execute the
following commands on your development host:

\begin{bashinput}
$ cd /home/$USER/__SESSION_NAME__-labs/nfsroot/root/heap_profile
$ sudo apt install massif-visualizer
$ massif-visualizer massif.out.*
\end{bashinput}

Note: \code{heaptrack} is not available on buildroot but is available on debian.
You can try the same experience using heaptrack on your computer and visualizing
the results with \code{heaptrack_gui}.

\section{Cachegrind \& Callgrind}

Cachegrind and Callgrind allow to profile a userspace application by
simulating the processor that will run it. In order to analyze our application
and understand where time is spent, we are going to profile it with both
tools.

Let's start by profiling the application using the \code{cachegrind} tool. Our
program takes two file names as parameters: an input PNG image and an output
one. We provided a sample image in \code{tux_small.png} which can be used as an
input file. First let's compile it using the following commands on our development host:

\begin{bashinput}
$ cd /home/$USER/__SESSION_NAME__-labs/nfsroot/root/app_profiling
$ make
\end{bashinput}

We are going to profile cache usage using Cachegrind with the following
command on the target:

\begin{bashinput}
$ valgrind --tool=cachegrind --cache-sim=yes ./png_convert tux_small.png out.png
\end{bashinput}

The execution will take some time and a \code{cachegrind.out.<pid>} will be
generated. Once finished, on the host, fix the permissions on the
\code{cachegrind.out.*} file to be able to open it with \code{Kcachegrind}:

\begin{bashinput}
$ sudo chown $USER:$USER cachegrind.out.*
\end{bashinput}

Analyze the results with \code{Kcachegrind} in order to understand the
function that generates most of the D cache miss time.

Based on that result, modify the program to be more cache efficient. Run again
the cachegrind analysis to check that the modifications were actually effective.

We also profile the execution time using callgrind by running valgrind again on
the target but with a different tool:

\begin{bashinput}
$ valgrind --tool=callgrind ./png_convert tux_small.png out.png
\end{bashinput}

Again, on the host platform, fix the permissions of the file using:
\begin{bashinput}
$ sudo chown $USER:$USER callgrind.out.*
\end{bashinput}

Again, analyze the results using \code{Kcachegrind}. This time, the view is
different and allows to display all the call graphs.

Looking at the results, it seems like our conversion function is
actually taking a negligible time. However, valgrind simulate the program with
an "ideal" cache. In real life, the processor is often used by other
applications and the kernel also takes some time to execute which leads to cache
disturbance.

\section{Perf}

In order to have a better view of the performance of our program in a real
system, we will use \code{perf}. In order to gather performance counter from
the hardware, we will run our program using \code{perf stat}. We would like to
observe the number of L1 data cache store misses. In order to select the correct
event, use \code{perf list} on the target to find it amongst the cache events:

\begin{bashinput}
$ perf list cache
\end{bashinput}

\ifdefstring{\labboard}{beagleplay}{ You will quickly see that perf does
not expose L1 D-cache write misses ! That's because our processor's PMU
does not implement and expose such an event. So we are not really able to
count this very specific event. However, based on the nature of the issue
we have worked on so far, we should be able to see a benefit on the read
side as well, and fortunately, the processor is able to record the L1
D-cache read misses:}{Once found, execute the program on the
target using \code{perf stat} and specify that event using \code{-e}:}


\begin{bashinput}
$ perf stat -e %\ifdefstring{\labboard}{beagleplay}{L1-dcache-load-misses}{L1-dcache-store-misses}% ./png_convert tux.png out.png
\end{bashinput}


Revert the modifications that we did to invert the program loops and again,
measure the amount of misses.

After that, we will record our program execution using the \code{perf record}
command to obtain a callgrind like result.

\begin{bashinput}
$ perf record ./png_convert tux_small.png out.png
\end{bashinput}

Once recorded, a \code{perf.data} file will be generated. This file will
contain the traces that have been recorded. These traces can be analyzed using
\code{perf report} on the development host:

\begin{bashinput}
$ sudo chown $USER:$USER perf.data
$ perf report \
	--symfs=/home/$USER/__SESSION_NAME__-labs/buildroot/output/staging/ \
	-k /home/$USER/__SESSION_NAME__-labs/buildroot/output/build/linux-%\workingkernel%/vmlinux
\end{bashinput}

You will quickly notice that the output is not the same as valgrind because it
displays the time spent per function (excluding function calls inside them).
This allows to find which function takes most of the execution time. In order to
compare this output to the valgrind one, we can run perf and also record the
callgraph using the \code{--call-graph} option.

\begin{bashinput}
$ perf record --call-graph dwarf ./png_convert tux_small.png out.png
\end{bashinput}

We specify that we want to record the call graph using the DWARF information
that are contained in ELF file (compiled with \code{-g}). Once recorded, on the
desktop platform, display the results with \code{perf report} and compare them
with callgrind ones on your development platform.

{\em NOTE: the vmlinux file used for the \code{-k} option must match the kernel
build id that is running on the board or perf will not use it.}

{\em NOTE: in order to annotate the disassembled code and display the time spent
for each instruction, \code{CROSS_COMPILE} must be set.}
