\documentclass[a4paper,12pt,obeyspaces,spaces,hyphens]{article}

\usepackage{agenda}
\usepackage{colortbl}
\usepackage{xcolor}
\usepackage{calc}

\hypersetup{pdftitle={Graphics display and rendering with Linux training},
  pdfauthor={Bootlin}}

\renewcommand{\arraystretch}{2.0}

\begin{document}

\thispagestyle{fancy}

\setlength{\arrayrulewidth}{0.8pt}

\begin{center}
\LARGE
Graphics display and rendering with Linux training\\
\large
2-day session
\end{center}

\vspace{1cm}
{\bf Warning: this agenda can still undergo some changes.
The main topics will remain the same, but their order
and details may still change.}

\vspace{1cm}
\small
\newcolumntype{g}{>{\columncolor{fedarkblue}}m{4cm}}
\newcolumntype{h}{>{\columncolor{felightblue}}X}

\arrayrulecolor{lightgray} {
  \setlist[1]{itemsep=-5pt}
  \begin{tabularx}{\textwidth}{|g|h|}
    {\bf Title} & {\bf Graphics display and rendering with Linux training} \\
    \hline

    {\bf Overview} &
    Image and color representation \par
    Basic drawing \par
    Basic and advanced operations \par
    Hardware aspects overview \par
    Hardware for display \par
    Hardware for rendering \par
    Memory aspects \par
    Performance aspects \par
    Software aspects overview \par
    Kernel components in Linux \par
    Userspace components with Linux\\
    \hline
    {\bf Materials} &
     Materials for this course are still under development.
     They will be released under a free documentation license
     after the first session is delivered.\\
%    Check that the course contents correspond to your needs:
%    \newline \url{https://bootlin.com/doc/training/graphics}. \\
    \hline

    {\bf Duration} & {\bf Two} days - 16 hours (8 hours per day).
    \newline 75\% of lectures, 25\% of demos. \\
    \hline

    {\bf Trainer} & One of the engineers listed on:
    \newline \url{https://bootlin.com/training/trainers/}\\
    \hline

    {\bf Language} & Oral lectures: English or French.
    \newline Materials: English.\\
    \hline

    {\bf Audience} & People developing multimedia devices using the Linux kernel\\
    \hline

    {\bf Prerequisites} &
    {\bf Basic knowledge of concepts related to low-level hardware interaction
    (e.g. registers, interrupts), kernel-level system management (e.g. virtual
    memory mappings) and userspace interfaces (syscalls).
    Basic knowledge of concepts related to hardware interfaces
    (e.g. clocks, busses).}\\
    \hline

  \end{tabularx}

  \begin{tabularx}{\textwidth}{|g|h|}
    {\bf Required equipment} &
    {\bf For on-site sessions only}
    \newline Everything is supplied by Bootlin in public sessions.
    \begin{itemize}
    \item Video projector
    \item Large monitor
    \end{itemize}\\
    \hline

    {\bf Materials} & Print and electronic copies of presentations slides\\
    \hline

\end{tabularx}}
\normalsize

\section{Day 1 - Morning}

\feagendatwocolumn
{Lecture - Image and color representation}
{
  \begin{itemize}
  \item Pixels and quantization
  \item Frames, framebuffers and dimensions
  \item Color encoding and depth
  \item Colorspaces
  \item Sub-sampling
  \item Alpha component
  \item Pixel formats formalization
  \end{itemize}
  \vspace{0.5em}
  {\em Introducing the basic notions used for representing color images in graphics.}
}
{Lecture - Basic drawing and operations}
{
  \begin{itemize}
  \item Lines
  \item Circles, ellipses and arcs
  \item Gradients (linear and circular)
  \item Format and colorspace conversion
  \item Bit blitting
  \item Alpha blending
  \item Colorkeying
  \item Clipping
  \item Integer scaling
  \end{itemize}
  \vspace{0.5em}
  {\em Presenting how to draw basic shapes and perform basic operations on a framebuffer.}
}
\\
\feagendatwocolumn
{Lecture - Advanced operations}
{
  \begin{itemize}
  \item Filtering and convolution
  \item Blur
  \item Fractional scaling
  \item Anti-aliasing
  \item Dithering
  \end{itemize}
  \vspace{0.5em}
  {\em Providing basic notions about filtering, with very common examples of how it's used.}
}
{Demo - Drawing and operations}
{
  \begin{itemize}
  \item Drawing outlines for various shapes
  \item Filling shapes with solid colors
  \item Filling shapes with gradients
  \item Compositing shapes without alpha
  \item Compositing shapes with alpha
  \item Compositing shapes with a color-key
  \item RGB to YUV conversion
  \item Blur
  \end{itemize}
  \vspace{0.5em}
  {\em Illustrating the concepts presented so far.}
}

\section{Day 1 - Afternoon}

\feagendatwocolumn
{Lecture - Hardware components overview}
{
  \begin{itemize}
  \item Display hardware components
  \item 2D processing hardware components
  \item 2D rendering hardware components
  \item 3D rendering hardware components
  \item Video input hardware components
  \end{itemize}
  \vspace{0.5em}
  {\em Presenting the hardware involved in graphics pipelines.}
}
{Lecture - Display hardware}
{
  \begin{itemize}
  \item Frame streaming and timings
  \item Side channels and additional data
  \item Display interfaces
  \item Transcoders
  \end{itemize}
  \vspace{0.5em}
  {\em Presenting the inner workings of display hardware.}
}
\\

\feagendatwocolumn
{Lecture - Processing and rendering hardware}
{
  \begin{itemize}
  \item ISPs and DSPs overview
  \item Display engine processing
  \item Vector processors overview
  \item GPU architectures overview
  \end{itemize}
  \vspace{0.5em}
  {\em Describing the architecture of processing and rendering hardware.}
}
{Lecture - Memory management and constraints}
{
  \begin{itemize}
  \item Tearing, artifacts and double-buffering
  \item Alignment and tiling
  \item Contiguity and IOMMUs
  \item Cache coherency
  \end{itemize}
  \vspace{0.5em}
  {\em Presenting all things related to managing framebuffer memory.}
}
\\

\feagendatwocolumn
{Lecture - Performance aspects}
{
  \begin{itemize}
  \item Common bottlenecks
  \item DMA operations
  \item Zero-copy frame sharing
  \item Costs and benefits of offloading
  \end{itemize}
  \vspace{0.5em}
  {\em Presenting possible performance issues and how they are solved in hardware.}
}
{Demo - Pipelines debugging and performance}
{
  \begin{itemize}
  \item Misconfiguration patterns
  \item Tearing example
  \item CPU rendering performance
  \item Accelerated rendering performance
  \item Zero-copy performance
  \end{itemize}
  \vspace{0.5em}
  {\em Illustrating what can go wrong and what can be improved in a graphics pipeline.}
}

\section{Day 2 - Morning}

\feagendatwocolumn
{Lecture - Software aspects overview}
{
  \begin{itemize}
  \item Complete display pipelines
  \item Roles handled by the kernel
  \item Roles handled by userspace
  \end{itemize}
  \vspace{0.5em}
  {\em Showing what software components are required for modern computer graphics and how they are divided between kernel and userspace.}
}
{Lecture - Linux kernel components}
{
  \begin{itemize}
  \item DRM/KMS subsystem for display
  \item DRM/GEM subsystem for rendering
  \item 2D hardware support
  \item Legacy framebuffer interface
  \item Framebuffer console, VT switching
  \end{itemize}
  \vspace{0.5em}
  {\em How display and rendering is organized in the Linux kernel subsystems.}
}
\\
\feagendaonecolumn
{Demo - Setting up a display with DRM/KMS}
{
  \begin{itemize}
  \item Legacy framebuffer interface
  \item DRM initialization
  \item Legacy DRM interface
  \item Atomic DRM interface
  \item Using planes
  \end{itemize}
  \vspace{0.5em}
  {\em Interacting with the kernel subsystems directly and showing related code paths.}
}


\section{Day 2 - Afternoon}

\feagendaonecolumn
{Lecture - Userspace components}
{
  \begin{itemize}
  \item libdrm shim
  \item Display servers: functionalities, Xorg and Wayland architectures
  \item 3D acceleration with OpenGL and mesa
  \item 2D acceleration though OpenGL
  \item 2D rendering libraries: pixman, cairo
  \item UI toolkits overview: Qt, GTK/GDK, SDL
  \end{itemize}
  \vspace{0.5em}
  {\em Presenting each element involved in the userspace graphics stack.}
}
\\
\feagendaonecolumn
{Demo - Graphics libraries usage examples}
{
  \begin{itemize}
  \item Direct Xorg usage
  \item Direct Wayland usage
  \item Basic cairo rendering
  \item Basic OpenGL rendering
  \item Basic SDL usage
  \end{itemize}
  \vspace{0.5em}
  {\em Showing how to get started with userspace libraries.}
}

\end{document}

