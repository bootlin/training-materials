\documentclass[a4paper,12pt,obeyspaces,spaces,hyphens]{article}

\def \sessiontype{online}
\def \agendalanguage{english}

\input{agenda/boot-time.inc.tex}

\usepackage{agenda_old}

\begin{document}

\feshowtitle

\feshowinfo

\feagendatwocolumn
{Hardware}
{
  The hardware platform used for the practical demos of this training
  session is the {\bf BeagleBone Black} board, which features:

  \begin{itemize}
  \item An ARM AM335x processor from Texas Instruments (Cortex-A8
    based), 3D acceleration, etc.
  \item 512 MB of RAM
  \item 2 GB of on-board eMMC storage
        \newline(4 GB in Rev C)
  \item USB host and device
  \item HDMI output
  \item 2 x 46 pins headers, to access UARTs, SPI buses, I2C buses
    and more.
  \end{itemize}
}
{}
{
  \begin{center}
    \includegraphics[height=5cm]{../slides/shopping-list-beaglebone/beagleboneblack.png}
  \end{center}
}

\feagendaonecolumn
{Demos}
{
  The practical demos of this training session use the following
  hardware peripherals:

  \begin{itemize}
  \item A USB webcam
  \item An LCD and touchscreen cape connected to the
    BeagleBone Black board, to display the video captured by the webcam.
  \end{itemize}
}

\section{Half day 1}

\feagendatwocolumn
{Lecture - Principles}
{
  \begin{itemize}
  \item How to measure boot time
  \item Main ideas
  \end{itemize}
}
{Demo - Preparing the system}
{
 \begin{itemize}
 \item Downloading bootloader, kernel and Buildroot source code
 \item Board setup, setting up serial communication
 \item Configure Buildroot and build the system
 \item Configure and build the U-Boot bootloader. Prepare an SD card
       and boot the bootloader from it.
 \item Configure and build the kernel. Boot the system
 \end{itemize}
}

\feagendatwocolumn
{Lecture - Measuring time}
{
  \begin{itemize}
  \item Generic software techniques
  \item Hardware techniques
  \item Specific solutions for each stage
  \end{itemize}
}
{Demo - Measuring time - Software solution}
{
 \begin{itemize}
 \item Modify the system to measure time at various steps
 \item Timing messages on the serial console
 \item Timing the launching of the application
 \end{itemize}
}

\section{Half day 2}

\feagendaonecolumn
{Lecture - Toolchain optimizations}
{
  \begin{itemize}
  \item Introduction to toolchains
  \item C libraries
  \item Size information
  \item Measuring executable performance with \code{time}
  \end{itemize}
}

\feagendaonecolumn
{Demo - Toolchain optimizations}
{
  \begin{itemize}
  \item Measuring application execution time
  \item Switching to a Thumb2 toolchain
  \item Generate a Buildroot SDK to rebuild faster
  \end{itemize}
}

\feagendatwocolumn
{Lecture - Application optimization}
{
  \begin{itemize}
  \item Using \code{strace} and \code{ltrace}
  \item Other profiling techniques
  \end{itemize}
}
{Demo - Application optimization}
{
 \begin{itemize}
 \item Finding unnecessary configuration options in applications
 \item Modifying configuration options through Buildroot
 \item Experiments with \code{strace} to trace program execution
 \end{itemize}
}

\feagendatwocolumn
{Lecture - Optimizing system initialization}
{
  \begin{itemize}
  \item Using BusyBox \code{bootchartd}
  \item Optimizing init scripts
  \item Possibility to start your application directly
  \end{itemize}
}
{Demo - Optimizing system initialization}
{
 \begin{itemize}
 \item Using Buildroot to remove unnecessary scripts and commands
 \item Access-time based technique to identify  unused files
 \item Simplifying BusyBox
 \item Starting the application as the init program
 \end{itemize}
}

\section{Half day 3}

\feagendatwocolumn
{Lecture - Filesystem optimizations}
{
  \begin{itemize}
  \item Available filesystems, performance and boot time aspects
  \item Making UBIFS faster
  \item Tweaks for reducing boot time
  \item Booting on an initramfs
  \item Using static executables: licensing constraints
  \end{itemize}
}
{Demo - Filesystem optimizations}
{
 \begin{itemize}
 \item Trying and measuring two block filesystems: ext4 and SquashFS.
 \item Trying and measuring the initramfs solution. Constraints
       due to this solution.
 \end{itemize}
}

\feagendatwocolumn
{Lecture - Kernel optimizations}
{
  \begin{itemize}
  \item Using {\em Initcall debug} to generate a boot graph
  \item Compression and size features
  \item Reducing or suppressing console output
  \item Multiple tweaks to reduce boot time
  \end{itemize}
}
{Demo - Kernel optimizations}
{
 \begin{itemize}
 \item Generating and analyzing a boot graph for the kernel
 \item Find and eliminate unnecessary kernel features
 \item Find the best kernel compression solution for our system
 \end{itemize}
}

\section{Half day 4}

\feagendaonecolumn
{Demo - Kernel optimizations}
{
 Continued from the previous session
}

\feagendatwocolumn
{Lecture - Bootloader optimizations}
{
  \begin{itemize}
  \item Generic tips for reducing U-Boot's size and boot time
  \item Optimizing U-Boot scripts and kernel loading
  \item Skipping the bootloader - How to modify U-Boot to
        enable its {\em Falcon mode}
  \end{itemize}
}
{Lecture - U-Boot Falcon mode}
{
  \begin{itemize}
  \item Principles and goals
  \item The Device Tree preparation work that U-Boot does to prepare Linux kernel booting
  \item Using the \code{spl export} command to do this work in advance
  \item Modifying U-Boot's source code and configuring it for directly
        booting Linux and skipping the U-Boot second stage.
  \item Example instructions and setups for booting from MMC and NAND flash
  \item How to debug Falcon mode
  \item How to fall back to U-Boot
  \item Limitations
  \end{itemize}
}

\feagendaonecolumn
{Demo - Bootloader optimizations}
{
 \begin{itemize}
 \item Using the above techniques to make the bootloader
       as quick as possible.
 \item Switching to faster storage
 \item Configuring U-Boot for {\em Falcon mode} booting,
       skipping U-Boot's second stage.
 \end{itemize}
}

\feagendaonecolumn
{Wrap-up - Achieved results}
{
 \begin{itemize}
 \item Summary of results
 \item Questions and answers, experience sharing with the trainer
 \end{itemize}
}

\end{document}
