\documentclass[a4paper,12pt,obeyspaces,spaces,hyphens]{article}

\def \trainingtitle{Real-Time Linux with {\em PREEMPT\_RT} training}
\def \trainingduration{On-site training, 2 days}
\def \agendalanguage{english}
\def \training{preempt-rt}

\usepackage{agenda}

\begin{document}

\feshowtitle

\feagendasummaryitem{Title}{
  {\bf \trainingtitle{}}
}
\feagendasummaryitem{Training objectives}{
  \begin{itemize}
    \vspace{-0.5cm}
  \item Be able to understand the characteristics of a real-time
    operating system
  \item Be able to download, build and use the {\em PREEMPT\_RT} patch
  \item Be able to identify and benchmark the hardware platform in
    terms of real-time characteristics
  \item Be able to configure the Linux kernel for deterministic
    behavior.
  \item Be able to develop, trace and debug real-time user-space Linux
    applications.
    \vspace{-0.5cm}
  \end{itemize}
}
\feagendasummaryitem{Duration}{
  {\bf Two} days - 16 hours (8 hours per day).
}
\onsitepedagogics{50}{50}{preempt-rt}
\feagendasummaryitem{Trainer}{
  Maxime Chevallier
  \newline \url{https://bootlin.com/company/staff/maxime-chevallier/}
}
\feagendasummaryitem{Language}{
  Oral lectures: English
  \newline Materials: English.
}
\feagendasummaryitem{Audience}{
  Companies and engineers interested in writing and benchmarking
  real-time applications and drivers on an embedded Linux system.
}
\feagendasummaryitem{Prerequisites}{
  \begin{itemize}
    \prerequisitecommandline
    \prerequisiteembeddedlinux
    \prerequisiteenglish
  \end{itemize}
}
\ferequiredequipmentonsite{}
\certificate{}
\disabilities{}

\feagendatwocolumn
{Hardware in practical labs}
{
  The hardware platform used for the practical labs of this training
  session is the {\bf BeagleBone Black} board, which features:

  \begin{itemize}
  \item An ARM AM335x processor from Texas Instruments (Cortex-A8
    based), 3D acceleration, etc.
  \item 512 MB of RAM
  \item 2-4 GB of on-board eMMC storage
  \item USB host and device
  \item HDMI output
  \item 2 x 46 pins headers, to access UARTs, SPI buses, I2C buses
    and more.
  \end{itemize}
}
{}
{
  \begin{center}
  \includegraphics[width=5cm]{../slides/beagleboneblack-board/beagleboneblack.png}
  \end{center}
}

\section{Day 1}

\feagendaonecolumn
{Lecture - Introduction to Real-Time behaviour and determinism}
{
  \begin{itemize}
  \item Definition of a Real-Time Operating System
  \item Specificities of multi-task systems
  \item Common locking and prioritizing patterns
  \item Overview of existing Real-Time Operating Systems
  \item Approaches to bring Real-Time capabilities to Linux
  \end{itemize}
}

\feagendatwocolumn
{Lecture - The {\em PREEMPT\_RT} patch}
{
  \begin{itemize}
  \item History and future of the {\em PREEMPT\_RT} patch
  \item Real-Time improvements from {\em PREEMPT\_RT} in mainline Linux
  \item The internals of {\em PREEMPT\_RT}
  \item Interrupt handling: threaded interrupts, softirqs
  \item Locking primitives: mutexes and spinlocks, sleeping spinlocks
  \item Preemption models
  \end{itemize}
}
{Lab - Building a mainline Linux Kernel with the {\em PREEMPT\_RT} patch}
{
  \begin{itemize}
  \item Downloading the Linux Kernel, and applying the patch
  \item Configuring the Kernel
  \item Booting the Kernel on a BeagleBone Black
 \end{itemize}
}

\feagendaonecolumn
{Lecture - Hardware configuration and limitations for Real-Time}
{
  \begin{itemize}
  \item Interrupts and deep firmwares
  \item Interaction with power management features: CPU frequency
    scaling and sleep states
  \item DMA
  \end{itemize}
}

\feagendatwocolumn
{Lecture - Tools: Benchmarking, Stressing and Analyzing}
{
  \begin{itemize}
  \item Benchmarking with {\em cyclictest}
  \item System stressing with {\em stress-ng} and {\em hackbench}
  \item The Linux Kernel tracing infrastructure
  \item Latency and scheduling analysis with {\em ftrace}, {\em
      kernelshark} or {\em LTTng}
  \end{itemize}
}
{Lab - Tools: Benchmarking, Stressing and Analyzing}
{
  \begin{itemize}
  \item Usage of benchmarking and stress tools
  \item Common benchmarking techniques
  \item Benchmarking and configuring the BeagleBone Black Wireless
  \end{itemize}
}

\section{Day 2}

\feagendaonecolumn
{Lecture - Kernel infrastructures and configuration}
{
  \begin{itemize}
  \item Good practices when writing Linux kernel drivers
  \item Scheduling policies and priorities: {\em SCHED\_FIFO}, {\em
      SCHED\_RR}, {\em SCHED\_DEADLINE}
  \item CPU and IRQ Affinity
  \item Memory management
  \item CPU isolation with {\em isolcpus}
  \end{itemize}
}

\feagendatwocolumn
{Lecture - Real-Time Applications programming patterns}
{
  \begin{itemize}
  \item POSIX real-time API
  \item Thread management and configuration
  \item Memory management: memory allocation and memory locking, stack
  \item Locking patterns: mutexes, priority inheritance
  \item Inter-Process Communication
  \item Signaling
  \end{itemize}
}
{Lab - Debugging a demo application}
{
  \begin{itemize}
  \item Make a demo userspace application deterministic
  \item Use the tracing infrastructure to identify the cause of a latency
  \item Learn how to use the POSIX API to manage threads, locking and memory
  \item Learn how to use the CPU affinities and configure the scheduling policy
  \end{itemize}
}

\feagendaonecolumn
{Questions and Answers}
{
  \begin{itemize}
  \item Questions and answers with the audience about the course topics
  \item Extra presentations if time is left, according what most
        participants are interested in.
  \end{itemize}
}

\end{document}
