\documentclass[a4paper,12pt,obeyspaces,spaces,hyphens]{article}

\def \trainingtitle{Formation temps-réel sous Linux avec {\em PREEMPT\_RT}}
\def \trainingduration{Formation sur site, 2 jours}
\def \agendalanguage{french}
\def \training{preempt-rt}

\usepackage{agenda}

\begin{document}

\feshowtitle

\feagendasummaryitem{Title}{
  {\bf \trainingtitle{}}
}
\feagendasummaryitem{Objectifs\newline opérationnels}{
  \begin{itemize}
    \vspace{-0.5cm}
  \item Être capable de comprendre et de maîtriser les
    caractéristiques d'un système d'exploitation temps-réel
  \item Être capable de télécharger, compiler et utiliser le patch
    {\em PREEMPT\_RT}
  \item Être capable d'identifier et de benchmarker la plateforme
    matérielle en terme de caractéristiques temps-réel
  \item Être capable de configurer le noyau Linux pour un comportement
    déterministe
  \item Être capable de développer, de tracer et de débugger des
    applications user-space temps-réel.
    \vspace{-0.5cm}
  \end{itemize}
}
\feagendasummaryitem{Duration}{
  {\bf Deux} journées - 16 heures (8 h par jour)
}
\onsitepedagogics{50}{50}{preempt-rt}
\feagendasummaryitem{Formateur}{
  Maxime Chevallier
  \newline \url{https://bootlin.com/company/staff/maxime-chevallier/}
}
\feagendasummaryitem{Langue}{
  Présentations : Français
  \newline Supports : Anglais
}
\feagendasummaryitem{Audience}{
  Entreprises et ingénieurs intéressés dans le développement et le
  benchmarking d'applications et de drivers temps-réel pour un système
  Linux embarqué.
}
\feagendasummaryitem{Pré-requis}{
  \begin{itemize}
    \prerequisitecommandline
    \prerequisiteembeddedlinux
    \prerequisiteenglish
  \end{itemize}
}
\ferequiredequipmentonsite{}
\certificate{}
\disabilities{}

\feagendatwocolumn
{Matériel}
{
  La plateforme matérielle utilisée pendant les travaux pratiques de
  cette formation est la carte {\bf BeagleBone Black}, dont voici les
  caractéristiques :

  \begin{itemize}
  \item Un processeur ARM AM335x de Texas Instruments (à base de
    Cortex-A8), avec accélération 3D, etc.
  \item 512 Mo de RAM
  \item 2-4 Go de stockage eMMC
  \item USB hôte et device
  \item Sortie HDMI
  \item Connecteurs à 2 x 46 broches, pour accéder aux UARTs, aux
        bus SPI, aux bus I2C, et à d'autres entrées/sorties du
        processeur.
  \end{itemize}
}
{}
{
  \begin{center}
    \includegraphics[height=5cm]{../slides/beagleboneblack-board/beagleboneblack.png}
  \end{center}
}

\section{1\textsuperscript{ère} journée}

\feagendaonecolumn
{Cours - Introduction au comportement temps-réel et au déterminisme}
{
  \begin{itemize}
  \item Définition d'un système d'exploitation temps-réel
  \item Spécificigés des systèmes multi-tâches
  \item Principaux patterns de verrouillage et de gestion des priorités
  \item Aperçu des systèmes temps-réel existants
  \item Approches pour apporter un comportement temps-réel à Linux
  \end{itemize}
}

\feagendatwocolumn
{Cours - Le patch {\em PREEMPT\_RT}}
{
  \begin{itemize}
  \item Histoire et avenir du patch {\em PREEMPT\_RT}
  \item Améliorations temps-réel provenant de {\em PREEMPT\_RT} dans le noyau Linux officiel
  \item Fonctionnement interne de {\em PREEMPT\_RT}
  \item Gestion des interruptions: interruptions threadées, softirqs
  \item Primitives de verouillage: mutexes et spinlocks, spinlocks avec sommeil
  \item Modèles de préemption
  \end{itemize}
}
{TP - Compiler un noyau Linux avec {\em PREEMPT\_RT}}
{
  \begin{itemize}
  \item Télécharger le noyau Linux et appliquer le patch {\em PREEMPT\_RT}
  \item Configurer le noyau Linux
  \item Démarrer le kernel sur une plateforme matérielle, la
    BeagleBone Black Wireless
 \end{itemize}
}

\feagendaonecolumn
{Cours - Configuration et limites du matériel pour le temps-réel}
{
  \begin{itemize}
  \item Interruptions et firmware
  \item Interaction avec les fonctionnalités de gestion d'énergie:
    gestion dynamique de la fréquence du CPU et états de sommeil
  \item DMA
  \end{itemize}
}

\feagendatwocolumn
{Cours - Outils: Benchmarking, Stress et Analyse}
{
  \begin{itemize}
  \item Benchmarking avec {\em cyclictest}
  \item Stress du système avec {\em stress-ng} et {\em hackbench}
  \item L'infrastructure de {\em tracing} du noyau Linux
  \item Analyse de la latence et de l'ordonnancement avec {\em
      ftrace}, {\em kernelshark} ou {\em LTTng}
  \end{itemize}
}
{TP - Outils: Benchmarking, Stress et Analyse}
{
  \begin{itemize}
  \item Utilisation des outils de benchmark et de stress
  \item Techniques classiques de benchmarking
  \item Benchmarking et configuration de la BeagleBone Black Wireless
  \end{itemize}
}

\section{2\textsuperscript{ème} journée}

\feagendaonecolumn
{Cours - Infrastructures du noyau Linux et configuration}
{
  \begin{itemize}
  \item Bonnes pratiques pour le développement de drivers noyau Linux
    pour des systèmes temps-réel
  \item Politiques d'ordonnancement et priorités: {\em SCHED\_FIFO},
    {\em SCHED\_RR}, {\em SCHED\_DEADLINE}
  \item Affinité CPU et IRQ
  \item Gestion mémoire
  \item Isolution des CPUs avec {\em isolcpus}
  \end{itemize}
}

\feagendatwocolumn
{Cours - Patterns de développement d'applications temps-réel}
{
  \begin{itemize}
  \item API POSIX pour les applications temps-réel
  \item Gestion et configuration des threads
  \item Gestion mémoire: allocation mémoire et verouillage mémoire, gestion de la pile
  \item Patterns de verrouillage: mutexes, héritage de priorité
  \item Communication inter-processus (IPC)
  \item Signalisation
  \end{itemize}
}
{TP - Débugger une application de démonstration}
{
  \begin{itemize}
  \item Créer une application de démonstration déterministe
  \item Utiliser l'infrastructure de {\em tracing} pour identifier la source de latence
  \item Apprendre à utiliser l'API POSIX pour gérer les threads, le verouillage, la mémoire
  \item Apprendre à utiliser l'affinité CPU et configurer la politique d'ordonnancement
  \end{itemize}
}

\feagendaonecolumn
{Questions / réponses}
{
  \begin{itemize}
  \item Questions / réponses avec les participants autour du noyau Linux
  \item Des présentations supplémentaires s'il reste du temps, selon les sujets
	qui intéressent le plus les participants.
  \end{itemize}
}

\end{document}
