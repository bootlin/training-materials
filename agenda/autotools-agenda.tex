\documentclass[a4paper,12pt,obeyspaces,spaces,hyphens]{article}

\def \trainingtitle{Autotools training}
\def \trainingduration{On-site training, 1 day}
\def \agendalanguage{english}
\def \training{autotools}

\usepackage{agenda}

\begin{document}

\feshowtitle

\feagendasummaryitem{Title}{
  {\bf \trainingtitle{}}
}
\feagendasummaryitem{Training objectives}{
  \begin{itemize}
  \item Be able to understand the role of the {\em autotools}
  \item Be able to use the {\em autotools}
  \item Be able to set up a basic project with {\em autoconf} and {\em
      automake}
  \item Be able to use advanced {\em autoconf} features: configuration
    header, checking for functions, headers and libraries, writing
    custom tests, handling external software and optional features,
    pkg-config, etc.
  \item Be able to use advanced {\em automake} features:
    subdirectories, conditionals, shared libraries with {\em libtool},
    etc.
  \end{itemize}
}
\feagendasummaryitem{Duration}{
  {\bf One} day - 8 hours
  \newline 40\% of lectures, 60\% of practical labs.
}
\feagendasummaryitem{Trainer}{
  {\bf Thomas Petazzoni}. Thomas is a major
  Buildroot developer since 2009, an activity through which he has
  gained a good knowledge of {\em autoconf}, {\em automake} and {\em
    libtool}.
}
\feagendasummaryitem{Language}{
  Oral lectures: English, French.
  \newline Materials: English.
}
\feagendasummaryitem{Audience}{
  Companies already using or interested in using
  {\em autotools} to build their software components.
}
\feagendasummaryitem{Prerequisites}{
  \begin{itemize}
  \item {\bf Knowledge and practice of UNIX or GNU/Linux commands}:
    participants must be familiar with the Linux command
    line. Participants lacking experience on this topic should get
    trained by themselves, for example with our freely available
    on-line slides (\url{https://bootlin.com/blog/command-line/})
  \end{itemize}
}
\ferequiredequipmentonsite{}
\feagendasummaryitem{Materials}{
  Electronic copies of presentations and
  labs.
  \newline Electronic copy of lab files.
}

\section{Day 1 - Morning}

\feagendatwocolumn
{Lecture - Overview and usage of {\em autotools}}
{
  \begin{itemize}
  \item What the {\em autotools} are, what the alternatives are, and
    what they are useful for.
  \item Usage of an existing software component using the {\em
      autotools}: configuring and building the software component.
  \item Standard Makefile targets, filesystem hierarchy, configuration variables
  \item System types: build, host, target
  \item Cross-compilation
  \item Out of tree build
  \item Diverted installation
  \item Cache variables
  \item Using {\em autoreconf}
  \end{itemize}
}
{Lab - Usage of an existing software component using the {\em autotools}}
{
  \begin{itemize}
  \item First build of an {\em autotools} package
  \item Out-of-tree build and cross-compilation
  \item Overriding cache variables
  \item Using {\em autoreconf}
  \end{itemize}
}

\feagendaonecolumn
{Lecture - autoconf/automake: the basics}
{
  \begin{itemize}
  \item \code{configure.ac} language and basic macros
  \item \code{AC_CONFIG_FILES} and {\em output variables}
  \item Minimal \code{Makefile.am}
  \end{itemize}
}

\section{Day 1 - Afternoon}

\feagendaonecolumn
{Lab - autoconf/automake: the basics}
{
  \begin{itemize}
  \item Your first \code{configure.ac}
  \item Adding and building a program
  \item Going further: \code{autoscan} and \code{make dist}
  \end{itemize}
}

\feagendatwocolumn
{Lecture - Autoconf advanced}
{
  \begin{itemize}
  \item Configuration header
  \item Checking for functions, headers, libraries
  \item Custom tests
  \item Handling external software and optional features
  \item \code{pkg-config}
  \end{itemize}
}
{Lecture - Automake advanced}
{
  \begin{itemize}
  \item Subdirectories
  \item Conditionals
  \item Shared libraries
  \item Misc: variables, macro and auxiliarly directories, silent
    rules, etc.
  \end{itemize}
}

\feagendaonecolumn
{Lab - Implement more advanced options}
{
  \begin{itemize}
  \item Use \code{AC_ARG_ENABLE} and \code{config.h}
  \item Implement a shared library
  \item Switch to multiple directories
  \item Make the compilation of programs conditional
  \item Use \code{pkg-config}
  \end{itemize}
}

\end{document}
