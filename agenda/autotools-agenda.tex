\documentclass[a4paper,12pt,obeyspaces,spaces,hyphens]{article}

\usepackage{agenda}
\usepackage{colortbl}
\usepackage{xcolor}
\usepackage{palatino}
\usepackage{calc}

\hypersetup{pdftitle={Autotools training},
  pdfauthor={Free Electrons}}

\renewcommand{\arraystretch}{2.0}

\begin{document}

\thispagestyle{fancy}

\setlength{\arrayrulewidth}{0.8pt}

\begin{center}
\LARGE
{\em Autotools} training\\
\large
1-day session
\end{center}
\vspace{1cm}

\small
\newcolumntype{g}{>{\columncolor{fedarkblue}}m{4cm}}
\newcolumntype{h}{>{\columncolor{felightblue}}X}

\arrayrulecolor{lightgray} {
  \setlist[1]{itemsep=-5pt}
  \begin{tabularx}{\textwidth}{|g|h|}
    {\bf Title} & {\bf Autotools training} \\
    \hline

    {\bf Overview} &
    Understand the role of the {\em autotools} \par
    Comparison with other solutions \par
    Using {\em autoconf}, writing {\em configure.ac} files \par
    Using {\em automake}, writing {\em Makefile.am} files \par
    Using {\em libtool} to generate libraries \\
    \hline

    {\bf Duration} & {\bf One} day - 8 hours
    \newline 40\% of lectures, 60\% of practical labs. \\
    \hline

    {\bf Trainer} & {\bf Thomas Petazzoni}. Thomas is a major
    Buildroot developer since 2009, an activity through which he has
    gained a good knowledge of {\em autoconf}, {\em automake} and {\em
      libtool}.\\
    \hline

    {\bf Language} & Oral lectures: English, French.
    \newline Materials: English.\\
    \hline

    {\bf Audience} & Companies already using or interested in using
    {\em autotools} to build their software components.\\
    \hline

    {\bf Prerequisites} & {\bf Knowledge of embedded Linux} as covered
    in our embedded Linux course:
    \newline \url{http://free-electrons.com/training/embedded-linux/} \vspace{1em}
    \newline {\bf Knowledge and practice of Unix or GNU/Linux commands}
    \newline People lacking experience on this topic should get
    trained by themselves with our freely available on-line slides:
    \newline \url{http://free-electrons.com/docs/command-line/} \\
    \hline
  \end{tabularx}

  \begin{tabularx}{\textwidth}{|g|h|}
    {\bf Required equipment} &
    {\bf For on-site sessions only.}
    \newline Everything is supplied by Free Electrons in public
    sessions.
    \begin{itemize}
    \item Video projector
    \item PC computers with at least 2 GB of RAM, and Ubuntu Linux
    installed in a {\bf free partition of at least 20 GB. Using Linux
      in a virtual machine is not supported}, because of issues
    connecting to real hardware.
    \item We need Ubuntu Desktop 14.04 (32 or 64 bit, Xubuntu and
    Kubuntu variants are fine). We don't support other
    distributions, because we can't test all possible package versions.
    \item {\bf Connection to the Internet} (direct or through the
    company proxy).
    \item {\bf PC computers with valuable data must be backed up}
    before being used in our sessions.  Some people have already made
    mistakes during our sessions and damaged work data.
    \end{itemize}\\
    \hline

    {\bf Materials} & Print and electronic copies of presentations and
    labs.
    \newline Electronic copy of lab files.\\
    \hline

\end{tabularx}}
\normalsize

\section{Day 1 - Morning}

\feagendatwocolumn
{Lecture - Overview and usage of {\em autotools}}
{
  \begin{itemize}
  \item What the {\em autotools} are, what the alternatives are, and
    what they are useful for.
  \item General architecture of the {\em autotools}: role of
    \code{autoconf}, \code{automake} and \code{libtool}.
  \item Usage of an existing software component using the {\em
      autotools}: \code{autoreconf}, generated files, configuring and
    building the software component.
  \end{itemize}
}
{Lab - Usage of an existing software component using the {\em autotools}}
{
  \begin{itemize}
  \item Use \code{autoreconf}, and explore generated files
  \item Configure the software component, explore the newly generated
    files
  \item Build the software component
  \end{itemize}
}

\feagendatwocolumn
{Lecture - autoconf/automake: the basics}
{
  \begin{itemize}
  \item Brief introduction to the \code{m4} macro language, used in
    \code{autoconf}
  \item The basic \code{autoconf} macros, and writing a simple
    \code{configure.ac} file
  \item The basic \code{automake} macros, and writing simple
    \code{Makefile.am} files
  \item Interaction between \code{autoconf} and \code{automake}
  \item Using \code{config.h} in C code
  \item Documentation and resources about {\em autotools}
  \end{itemize}
}
{Lab - autoconf/automake: the basics}
{
  \begin{itemize}
  \item Experiment with \code{autoconf} and \code{automake} by creating
    the build system for a small project composed of several applications.
  \end{itemize}
}

\section{Day 1 - Afternoon}

\feagendatwocolumn
{Lecture - \code{libtool}: generating libraries}
{
  \begin{itemize}
  \item What \code{libtool} is
  \item How to use it with \code{autoconf} and \code{automake} to ease
    the building of static and shared libraries
  \end{itemize}
}
{Lab - Using \code{libtool}}
{
  \begin{itemize}
  \item Extend the project of the previous lab by adding libraries
    built using \code{libtool}
  \end{itemize}
}

\feagendatwocolumn
{Lecture - More advanced topics}
{
  \begin{itemize}
  \item Creating user-configurable options with \code{autoconf} and
    handle them using \code{automake}
  \item Detecting dependencies such as header files, functions,
    programs or libraries, usage of \code{pkg-config}
  \item Providing \code{pkg-config} files for the libraries being
    installed
  \end{itemize}
}
{Lab - Implement more advanced options}
{
  \begin{itemize}
  \item Extend the project of the previous lab by using more advanced
    \code{autoconf} and \code{automake} features.
  \end{itemize}
}

\end{document}
