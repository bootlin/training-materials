\documentclass[a4paper,12pt,obeyspaces,spaces,hyphens]{article}

\def \trainingtitle{Git training}
\def \trainingduration{On-site training, 1 day}
\def \agendalanguage{english}
\def \training{git}

\usepackage{agenda}

\begin{document}

\feshowtitle

\feagendasummaryitem{Title}{
  {\bf \trainingtitle{}}
}
\feagendasummaryitem{Training objectives}{
  \begin{itemize}
  \item Be able to understand the basic principles of version control
    and Git
  \item Be able to understand the terminology used by Git
  \item Be able to perform basic version control operations with Git
  \item Be able to create, use and manage branches with Git
  \item Be able to use {\em remote} repositories with Git
  \item Be able to use the main collaboration techniques with Git
  \item Be able to use different typical workflows used for version
    control with Git
  \end{itemize}
}
\feagendasummaryitem{Duration}{
  {\bf One} day - 8 hours.
  \newline 40\% of lectures, 60\% of practical labs.
}
\feagendasummaryitem{Language}{
  Oral lectures: English, French.
  \newline Materials: English.
}
\feagendasummaryitem{Audience}{
  Companies already using or interested in using
  Git to managed their source code.
}
\feagendasummaryitem{Prerequisites}{
  None
}
\ferequiredequipmentonsite{}
\feagendasummaryitem{Materials}{
  Electronic copies of presentations and labs.
  \newline Electronic copy of lab files.
}

\section{Morning}

\feagendatwocolumn
{Lecture - Source control}
{
  \begin{itemize}
  \item Version control systems: principles
  \item Centralized vs. distributed version control systems
  \item Git glossary and internal representation of data
  \item Git basics: installation and configuration, getting a repository, making commits, etc.
  \item Basic Git workflow: understanding {\em HEAD}, {\em index} and {\em working directory}
  \end{itemize}
}
{Lab - Basic Git usage}
{
  \begin{itemize}
  \item Installing and setting up Git
  \item Getting a repository
  \item Listing changes
  \item First commits
  \item Undoing changes
  \end{itemize}
}
\\
\feagendatwocolumn
{Lecture - Organizing sources}
{
  \begin{itemize}
  \item Navigating in sources and history
  \item Branches: creation, deletion, merging
  \item Rewriting history with {\em rebase}
  \item Tags
  \item Temporary storage of changes with {\em stash}
  \end{itemize}
}
{Lab - Everyday Git usage}
{
  \begin{itemize}
  \item Initializing a Git repository
  \item Creating and moving between branches
  \item Merging
  \item Rebasing
  \item Cleaning up a branch
  \item Tagging commits
  \end{itemize}
}

\section{Afternoon}

\feagendatwocolumn
{Lecture - Collaborating}
{
  \begin{itemize}
  \item Git {\em remotes}: interacting with other developers
  \item Importing changes: fetching changes from other developers, applying patches
  \item Sending changes: pushing commits and sending patches
  \item Getting more efficient
  \end{itemize}
}
{Lab - Collaborating}
{
  \begin{itemize}
  \item Working with multiple remotes
  \item Creating patches
  \item Applying patches
  \item Exporting a repository
  \item First Pull request
  \end{itemize}
}
\\
\feagendatwocolumn
{Lecture - Workflows}
{
  \begin{itemize}
  \item Centralized workflow, often used in internal company projects
  \item Linux Kernel workflow, often used in open-source communities
  \end{itemize}
}
{Lecture - Advanced source organization}
{
  \begin{itemize}
  \item Referencing Git repositories from another repository with {\em git submodules}
  \item Managing easily a large number of Git repositories with Google's {\em repo} tool
  \item Rewriting branches in an automated way with {\em filter-branch}
  \end{itemize}
}

\end{document}
