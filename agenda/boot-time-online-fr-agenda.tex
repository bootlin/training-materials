\documentclass[a4paper,12pt,obeyspaces,spaces,hyphens]{article}

\def \trainingtitle{Formation optimisation du temps de démarrage de Linux embarqué}
\def \trainingduration{Séminaire en ligne, 4 sessions de 4 heures}
\def \agendalanguage{french}
\def \training{boot-time}

\usepackage{agenda}

\begin{document}

\feshowtitle

\feagendasummaryitem{Titre}{
  {\bf \trainingtitle{}}
}
\feagendasummaryitem{Objectifs\newline opérationnels}{
  \begin{itemize}
  \item Être capable les outils et techniques pour mesurer le temps de
    démarrage d'un système embarqué.
  \item Être capable de réduire le temps de démarrage au niveau de
    l'initialisation de l'espace utilisateur Linux.
  \item Être capable de réduire le temps de démarrage au niveau de
    l'initialisation du noyau Linux.
  \item Être capable de réduire le temps de démarrage au niveau de
    l'initialisation du chargeur de démarrage.
  \item Être capable de mettre en oeuvre d'autres techniques avancées
    et alternatives d'optimisation du temps de démarrage.
  \end{itemize}
}
\feagendasummaryitem{Supports}{
  Vérifiez que le contenu de la formation correspond à vos besoins :
  \newline \url{https://bootlin.com/doc/training/boot-time}.
}
\feagendasummaryitem{Durée}{
  {\bf Quatre} demi-journées - 16 h (4 h par demi-journée)
  \newline 25\% de présentations et 75\% de démonstrations.
}
\feagendasummaryitem{Formateur}{
  Un des ingénieurs mentionnés sur :
  \newline \url{https://bootlin.com/training/trainers/}
}
\feagendasummaryitem{Langue}{
  Présentations : Français
  \newline Supports : Anglais
}
\feagendasummaryitem{Public visé}{
  Sociétés et ingénieurs développeurs de systèmes Linux embarqués.
  \newline Personnes offrant de l'assistance à de tels développeurs.
}
\feagendasummaryitem{Pré-requis}{
  \begin{itemize}
  \item {\bf Connaissance et pratique des commandes UNIX ou
      GNU/Linux}: les participants doivent être à l'aise avec
    l'utilisation de la ligne de commande Linux. Les participants
    manquant d'expérience sur ce sujet doivent se former par
    eux-mêmes, par exemple en utilisant nos supports de formation
    disponible à l'adresse
    \url{https://bootlin.com/blog/command-line/}.
  \item {\bf Expérience minimale en développement Linux embarqué}: les
    participants doivent avoir une compréhension minimal de
    l'architecture d'un système Linux embarqué: rôle du noyau Linux
    par rapport à l'espace utilisateur, développement d'applications
    espace utilisateur en C. Suivre la formation {\em Linux embarqué}
    de Bootlin, disponible sur
    \url{https://bootlin.com/fr/formation/linux-embarque/} permet de
    remplir ce pré-requis.
  \item {\bf Niveau minimal requis en anglais: B1}, d'après le {\em
      Common European Framework of References for Languages}, pour nos
    sessions animées en anglais.
  \end{itemize}
}
\feagendasummaryitem{Équipement nécessaire}{
  \begin{itemize}
  \item Ordinateur avec le système d'exploitation de votre choix, équipé du
    navigateur Google Chrome ou Chromium pour la conférence vidéo.
  \item Une webcam et un micro (de préférence un casque avec micro)
  \item Une connexion à Internet à haut débit
  \end{itemize}
}
\feagendasummaryitem{Supports}{
  Version électronique des présentations, des instructions
  et des données pour les démos.
}

\feagendatwocolumn
{Matériel}
{
  La plateforme matérielle utilisée pendant les démonstrations pratiques de
  cette formation est la carte {\bf BeagleBone Black}, dont voici les
  caractéristiques :

  \begin{itemize}
  \item Un processeur ARM AM335x de Texas Instruments (à base de
    Cortex-A8), avec accélération 3D, etc.
  \item 512 Mo de RAM
  \item 2 Go de stockage eMMC embarqué sur la carte
	\newline(4 Go avec la révision C)
  \item USB hôte et device
  \item Sortie HDMI
  \item Connecteurs à 2 x 46 broches, pour accéder aux UARTs, aux
        bus SPI, aux bus I2C, et à d'autres entrées/sorties du
        processeur.
  \end{itemize}
}
{}
{
  \begin{center}
    \includegraphics[height=5cm]{../slides/beagleboneblack-board/beagleboneblack.png}
  \end{center}
}

\feagendaonecolumn
{Démonstrations}
{
  Les démos de cette formation utiliseront les périphériques matériels suivants:

  \begin{itemize}
  \item Une webcam USB
  \item Une carte d'extension d'écran tactile LCD connectée à la carte
    BeagleBone Black, pour afficher la vidéo capturée par la webcam.
  \item Nous utiliserons également une carte Arduino comme moyen pour mesurer
    précisément le temps de démarrage, pour montrer comment mettre en place
    des techniques de mesure matérielles.
  \end{itemize}
}

\section{1\textsuperscript{ère} demi-journée}

\feagendatwocolumn
{Cours - Méthodes}
{
  \begin{itemize}
  \item Comment mesurer le temps de démarrage
  \item Principales approches
  \end{itemize}
}
{Démo - Construction du système}
{
 \begin{itemize}
 \item Téléchargement du code source du chargeur de démarrage, du noyau et de Buildroot
 \item Prise en main de la carte, mise en place de la communication série
 \item Configuration de Buildroot et génération du système
 \item Configuration et compilation du chargeur de démarrage U-Boot. Préparation d'une
       carte SD pour démarrer le système.
 \item Configuration et compilation du noyau. Démarrage du système.
 \end{itemize}
}

\feagendatwocolumn
{Cours - Mesure du temps}
{
  \begin{itemize}
  \item Techniques génériques par logiciel
  \item Techniques matérielles
  \item Solutions spécifiques à chaque étage du démarrage
  \end{itemize}
}
{Démo - Mesure du temps - Solution logicielle}
{
 \begin{itemize}
 \item Modification du système pour mesurer le temps au niveau des différentes étapes.
 \item Chronométrer les messages sur la console série
 \item Chronométrer le démarrage de l'application
 \end{itemize}
}

\feagendaonecolumn
{Démo - Mesure du temps - Solution matérielle}
{
  \begin{itemize}
  \item Mesure du temps total de démarrage en positionnant une GPIO
  \item Mise en oeuvre d'une carte Arduino
  \item Préparation d'un circuit de test avec un afficheur à 7 segments
  \item Modification du DTS pour configurer les broches de la Bone Black en tant que GPIOs
  \item Modification de l'application pour piloter les GPIOs personnalisées
  \end{itemize}
}

\section{2\textsuperscript{ème} demi-journée}

\feagendaonecolumn
{Cours - Optimisations des chaînes de compilation}
{
  \begin{itemize}
  \item Introduction aux chaînes de compilation
  \item Bibliothèques C
  \item Informations de taille
  \item Mesure de la performance d'un exécutable avec la commande \code{time}
  \end{itemize}
}

\feagendaonecolumn
{Démo - Optimisations des chaînes de compilation}
{
  \begin{itemize}
  \item Mesure du temps d'exécution de l'application
  \item Passage à une chaîne Thumb2
  \item Génération d'un SDK Buildroot pour recompiler plus vite
  \end{itemize}
}

\feagendatwocolumn
{Cours - Optimisation de l'application}
{
  \begin{itemize}
  \item Utilisation de \code{strace}
  \item Autres techniques de profiling
  \end{itemize}
}
{Démo - Optimisation de l'application}
{
 \begin{itemize}
 \item Rechercher d'options de configuration inutiles dans des applications
 \item Modification de ces options à travers Buildroot
 \item Expériences avec \code{strace} pour suivre l'exécution d'un programme
 \end{itemize}
}

\feagendatwocolumn
{Cours - Optimisation du démarrage du système}
{
  \begin{itemize}
  \item Utilisation de Bootchart
  \item Optimisation des scripts d'init
  \item Possibilité de démarrer directement votre application
  \end{itemize}
}
{Démo - Optimisation du démarrage du système}
{
 \begin{itemize}
 \item Utilisation de Buildroot pour supprimer scripts et commandes non nécessaires
 \item Une méthode pour identifier tous les fichiers inutilisés
 \item Simplification de BusyBox
 \item Démarrage de l'application en tant que programme init.
 \end{itemize}
}

\section{3\textsuperscript{ème} demi-journée}

\feagendatwocolumn
{Cours - Optimisations de systèmes de fichiers}
{
  \begin{itemize}
  \item Systèmes de fichiers disponibles, aspects de performance et de temps de démarrage
  \item Comment accélerer UBIFS
  \item Paramètres pour réduire le temps de démarrage
  \item Démarrer depuis un initramfs
  \item Utilisation d'exécutables statiques: contraintes de licence
  \end{itemize}
}
{Démo - Optimisations de systèmes de fichiers}
{
 \begin{itemize}
 \item Essayer et mesurer deux systèmes de fichiers bloc: ext4 et SquashFS.
 \item Essai et benchmark de la solution initramfs. Contraintes
       en rapport avec cette solution.
 \end{itemize}
}

\feagendatwocolumn
{Cours - Optimisations du noyau}
{
  \begin{itemize}
  \item Utilisation d'{\em Initcall debug} to générer un {\em boot graph}
  \item Options de compression et liées à la taille
  \item Réduction ou suppression de la sortie console
  \item Plusieurs réglages pour réduire le temps de démarrage
  \end{itemize}
}
{Démo - Optimisations du noyau}
{
 \begin{itemize}
 \item Génération et analyse d'un {\em boot graph} pour le noyau
 \item Identifier et éliminer les fonctionnalités du noyau non nécessaires
 \item Trouver la meilleure option de compression pour votre système
 \end{itemize}
}

\section{4\textsuperscript{ème} demi-journée}

\feagendaonecolumn
{Démo - Optimisations du noyau}
{
 \begin{itemize}
 \item Poursuite de la démo
 \end{itemize}
}

\feagendatwocolumn
{Cours - Optimisations du chargeur de démarrage}
{
  \begin{itemize}
  \item Compiler U-Boot avec moins de fonctionnalités
  \item Réglages de configuration d'U-Boot qui ont un impact
        sur le temps de démarrage
  \item Optimisation du chargement du noyau
  \item Sauter le chargeur de démarrage - Comment modifier U-Boot pour
	activer son {\em Falcon mode}
  \end{itemize}
}
{Démo - Optimisations du chargeur de démarrage}
{
 \begin{itemize}
 \item Utilisation des techniques ci-dessus pour rendre le chargeur
       de démarrage le plus rapide possible
 \item Passer à un stockage plus rapide
 \item Sauter le chargeur de démarrage avec le {\em Falcon mode} d'U-Boot
 \end{itemize}
}

\feagendaonecolumn
{Conclusion - Résultats obtenus}
{
 \begin{itemize}
 \item Résumé des résultats obtenus
 \item Questions / réponses, partage d'expérience avec le formateur
 \end{itemize}
}

\end{document}
