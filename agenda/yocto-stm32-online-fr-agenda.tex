\documentclass[a4paper,12pt,obeyspaces,spaces,hyphens]{article}

\usepackage{agenda}
\usepackage{colortbl}
\usepackage{xcolor}
\usepackage{calc}

\hypersetup{pdftitle={Formation développement Linux embarqué avec Yocto Project et OpenEmbedded},
  pdfauthor={Bootlin}}

\renewcommand{\arraystretch}{2.0}

\begin{document}

\thispagestyle{fancy}

\setlength{\arrayrulewidth}{0.8pt}

\begin{center}
\LARGE
Formation développement Linux embarqué avec Yocto Project et OpenEmbedded\\
\large
Séminaire de formation en ligne
\end{center}
\vspace{1cm}

\small
\newcolumntype{g}{>{\columncolor{fedarkblue}}m{4cm}}
\newcolumntype{h}{>{\columncolor{felightblue}}X}

\arrayrulecolor{lightgray} {
  \setlist[1]{itemsep=-5pt}
  \begin{tabularx}{\textwidth}{|g|h|}
    {\bf Titre} & {\bf Formation développement Linux embarqué avec Yocto Project et OpenEmbedded} \\
    \hline

    {\bf Aperçu} &
    Comprendre l'architecture de Yocto Project \par
    Utilisation pour compiler un système de fichiers et exécuter
    celui-ci sur votre plateforme matérielle \par
    Étendre des recettes ({\em recipes}) existantes et en écrire de nouvelles \par
    Création de {\em layers} \par
    Intégration de votre matériel dans un {\em BSP} \par
    Création d'images sur mesure \par
    Développement applicatif à l'aide du SDK de Yocto Project \\
    \hline
    {\bf Supports} &
    Vérifiez que le contenu de la formation correspond à vos besoins :
    \newline \url{https://bootlin.com/doc/training/yocto}. \\
    \hline

    {\bf Durée} & {\bf Quatre} demi-journées - 16 h (4 h par demi-journée)
    \newline 80\% de présentations et 20\% de démonstrations. \\
    \hline

    {\bf Formateur} & Un des ingénieurs mentionnés sur :
    \newline \url{https://bootlin.com/training/trainers/}\\
    \hline

    {\bf Langue} & Présentations : Français
    \newline Supports : Anglais\\
    \hline

    {\bf Public visé} & Sociétés et ingénieurs intéressés par l'utilisation de Yocto Project
    pour construire leur système Linux embarqué.\\
    \hline

    {\bf Pré-requis} & {\bf Connaissance de Linux embarqué}, sujet couvert par
    notre formation Linux embarqué (\url{https://bootlin.com/training/embedded-linux/}) \vspace{1em}
    \newline {\bf Connaissance et pratique des commandes UNIX ou
    GNU/Linux}
    \newline Les personnes n'ayant pas ces connaissances peuvent
    s'autoformer, par exemple en utilisant nos supports de formation
    disponibles en ligne :
    (\url{https://bootlin.com/blog/command-line/} \\
    \hline
  \end{tabularx}

  \begin{tabularx}{\textwidth}{|g|h|}
    {\bf Équipement nécessaire} &
    \begin{itemize}
    \item Ordinateur avec le système d'exploitation de votre choix, équipé du
          navigateur Google Chrome ou Chromium pour la conférence vidéo.
    \item Une webcam et un micro (de préférence un casque avec micro)
    \item Une connexion à Internet à haut débit
    \end{itemize}\\
    \hline

    {\bf Supports} & Version électronique des présentations, des instructions
   et des données pour les démos.\\
    \hline

\end{tabularx}}
\normalsize

\feagendatwocolumn
{Matériel}
{
  Carte STMicroelectronics STM32MP157D-DK1 Discovery
  \begin{itemize}
  \item Processeur STM32MP157D (double Cortex-A7) de STMicroelectronics
  \item Alimentée par USB
  \item 512 Mo DDR3L RAM
  \item Port Gigabit Ethernet port
  \item 4 ports hôte USB 2.0
  \item 1 port USB-C OTG
  \item 1 connecteur Micro SD
  \item Debugger ST-LINK/V2-1 sur la carte
  \item Connecteurs compatibles Arduino Uno v3
  \item Codec audio
  \item Divers: boutons, LEDs
  \end{itemize}
}
{}
{
  \includegraphics[width=5cm]{../slides/discovery-board-dk1/discovery-board-dk1.png}
}

\section{1\textsuperscript{ère} demi-journée}

\feagendaonecolumn
{Cours - Introduction aux outils de compilation de systèmes Linux embarqué}
{
  \begin{itemize}
  \item Vue d'ensemble de l'architecture d'un système Linux embarqué
  \item Méthodes pour compiler un système de fichiers
  \item Utilité des outils de compilation
  \end{itemize}
}

\feagendatwocolumn
{Cours - Vue d'ensemble de Yocto Project et du système de référence Poky}
{
  \begin{itemize}
  \item Organisation des sources du projet
  \item Création d'un système de fichiers avec Yocto Project
  \end{itemize}
}
{Démo - 1\textsuperscript{ère} compilation avec Yocto Project}
{
  \begin{itemize}
  \item Téléchargement du système de référence Poky
  \item Compilation d'une image système
 \end{itemize}
}

\feagendatwocolumn
{Cours - Utilisation de Yocto Project - Notions de base}
{
  \begin{itemize}
  \item Structure des fichiers générés
  \item Flasher et installer l'image du système
  \end{itemize}
}
{Démo - Flasher et booter}
{
  \begin{itemize}
  \item Flasher et booter l'image du système sur la carte
  \end{itemize}
}

\section{2\textsuperscript{ème} demi-journée}

\feagendatwocolumn
{Cours - Utilisation de Yocto Project - Utilisation avancée}
{
  \begin{itemize}
  \item Configuration de la compilation
  \item Personnalisation de la sélection de paquetages
  \end{itemize}
}
{Démo - Utilisation de NFS et configuration de la compilation}
{
  \begin{itemize}
  \item Configurer la carte pour démarrer via NFS
  \item Apprendre à utiliser le mécanisme \code{PREFERRED_PROVIDER}
  \end{itemize}
}
\\

\feagendatwocolumn
{Cours - Écriture de recettes - Fonctionnalités de base}
{
  \begin{itemize}
  \item Écriture d'une recette minimale
  \item Ajout de dépendances
  \item Organisation du développement avec {\em bitbake}
  \end{itemize}
}
{Démo - Ajouter la compilation d'une application}
{
  \begin{itemize}
  \item Création d'une recette pour {\em nInvaders}
  \item Ajout d'{\em nInvaders} à l'image finale
  \end{itemize}
}

\feagendaonecolumn
{Cours - Écriture de recettes - Fonctionnalités avancées}
{
  \begin{itemize}
  \item Extension et redéfinition de recettes
  \item Rajouter des étapes au processus de compilation
  \item Familiarisation avec les classes
  \item Analyse d'exemples
  \item Logs
  \item Mise au point des dépendances
  \end{itemize}
}

\section{3\textsuperscript{ème} demi-journée}

\feagendaonecolumn
{Démo - Apprendre à configurer les paquetages}
{
  \begin{itemize}
  \item Extension d'une recette pour ajouter des fichiers de configuration
  \item Utilisation de \code{ROOTFS_POSTPROCESS_COMMAND} pour modifier
        le système de fichier final
  \item Étude des dépendances entre paquetages
  \end{itemize}
}
\feagendatwocolumn
{Cours - Layers}
{
  \begin{itemize}
  \item Ce que sont les {\em layers}
  \item Où trouver les {\em layers}
  \item Création d'un {\em layer}
  \end{itemize}
}
{Démo - Écriture d'un layer}
{
  \begin{itemize}
  \item Apprendre à écrire un {\em layer}
  \item Ajouter le {\em layer} à la compilation
  \item Inclure {\em nInvaders} dans le nouveau {\em layer}
  \end{itemize}
}

\feagendatwocolumn
{Cours - Écriture d'un BSP}
{
  \begin{itemize}
  \item Extension d'un BSP existant
  \item Ajout d'une nouvelle machine
  \item Chargeurs de démarrage
  \item Linux et la recette linux-yocto
  \item Ajouter un type d'image personnalisé
  \end{itemize}
}
{Démo - Mise en oeuvre de modifications du noyau}
{
  \begin{itemize}
  \item Extension de la recette pour le noyau pour ajouter le pilote
        pour le Nunchuk
  \item Configurer le noyau pour compiler le pilote du Nunchuk
  \item Jouer à {\em nInvaders}
  \end{itemize}
}

\section{4\textsuperscript{ème} demi-journée}

\feagendatwocolumn
{Cours - Création d'une image sur mesure}
{
  \begin{itemize}
  \item Écriture d'une recette d'image
  \item Ajouter des utilisateurs et des groupes
  \item Ajouter une configuration personnalisée
  \item Écrire et utiliser des groupes de recettes de paquetages
  \end{itemize}
}
{Démo - Création d'une image sur mesure}
{
  \begin{itemize}
  \item Écrire une recette d'image personnalisée
  \item Ajouter {\em nInvaders} à l'image sur mesure
  \end{itemize}
}
\feagendatwocolumn
{Cours - Création et utilisation d'un SDK}
{
  \begin{itemize}
  \item Comprendre l'utilité d'un SDK pour le développeur d'applications
  \item Construire un SDK pour l'image sur mesure
  \end{itemize}
}
{Démo - Expérimentations avec le SDK}
{
  \begin{itemize}
  \item Construction d'un SDK
  \item Utilisation du SDK de Yocto Project
  \end{itemize}
}

\feagendaonecolumn
{Questions / réponses}
{
  \begin{itemize}
  \item Questions et réponses avec les participants à propos des sujets abordés.
  \item Présentations supplémentaires s'il reste du temps, en fonction des demandes
        de la majorité des participants.
  \end{itemize}
}

\end{document}

