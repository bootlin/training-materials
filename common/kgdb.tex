
\begin{frame}
  \frametitle{kgdb - A kernel debugger}
  \begin{itemize}
  \item \kconfig{CONFIG_KGDB} in {\em Kernel hacking}.
  \item The execution of the kernel is fully controlled by \code{gdb}
    from another machine, connected through a serial line.
  \item Can do almost everything, including inserting breakpoints in
    interrupt handlers.
  \item Feature supported for the most popular CPU architectures
  \item \kconfig{CONFIG_GDB_SCRIPTS} allows to build GDB python scripts that are
    provided by the kernel.
  \begin{itemize}
	  \item See \href{https://www.kernel.org/doc/html/next/dev-tools/kgdb.html}{dev-tools/gdb-kernel-debugging} for more information
  \end{itemize}
  \end{itemize}
\end{frame}

\ifthenelse{\equal{\training}{debugging}}{

\begin{frame}
  \frametitle{kgdb kernel config}
  \begin{itemize}
    \item \kconfigval{CONFIG_DEBUG_KERNEL}{y} to make KGDB support visible
    \item \kconfigval{CONFIG_KGDB}{y} to enable KGDB support
    \item \kconfigval{CONFIG_DEBUG_INFO}{y} to compile the kernel with debug info (\code{-g})
    \item \kconfigval{CONFIG_FRAME_POINTER}{y} to have more reliable
          stacktraces
    \item \kconfigval{CONFIG_KGDB_SERIAL_CONSOLE}{y} to enable KGDB support
          over serial
    \item \kconfigval{CONFIG_GDB_SCRIPTS}{y} to enable kernel GDB python
          scripts
    \item \kconfigval{CONFIG_RANDOMIZE_BASE}{n} to disable KASLR
    \item \kconfigval{CONFIG_WATCHDOG}{n} to disable watchdog
    \item \kconfigval{CONFIG_MAGIC_SYSRQ}{y} to enable Magic SysReq support
    \item \kconfigval{CONFIG_STRICT_KERNEL_RWX}{n} to disable memory protection
          on code section, thus allowing to put breakpoints
  \end{itemize}
\end{frame}

\begin{frame}
  \frametitle{kgdb pitfalls}
  \begin{itemize}
    \item KASLR should be disabled to avoid confusing gdb with randomized kernel
          addresses
    \begin{itemize}
      \item Disable \em{kaslr} mode using \code{nokaslr} command line parameter
            if enabled in your kernel.
    \end{itemize}
    \item Disable the platform watchdog to avoid rebooting while debugging.
    \begin{itemize}
      \item When interrupted by KGDB, all interrupts are disabled thus, the
            watchdog is not serviced.
      \item Sometimes, watchdog is enabled by upper boot levels. Make sure to
            disable the watchdog there too.
    \end{itemize}
    \item Can not interrupt kernel execution from gdb using \code{interrupt}
          command or \code{Ctrl + C}.
    \item Not possible to break everywhere (see \kconfig{CONFIG_KGDB_HONOUR_BLOCKLIST}).
    \item Need a console driver with polling support.
    \item Some architecture lacks functionalities (No watchpoints on arm32 for
          instance) and some instabilities might happen!
  \end{itemize}
\end{frame}
}{}

\begin{frame}
  \frametitle{Using kgdb (1/2)}
  \begin{itemize}
  \item Details available in the kernel documentation:
    \kdochtml{dev-tools/kgdb}
  \item You must include a kgdb I/O driver. One of them is \code{kgdb} over
    serial console (\code{kgdboc}: \code{kgdb} over console, enabled by
    \kconfig{CONFIG_KGDB_SERIAL_CONSOLE})
  \item Configure \code{kgdboc} at boot time by passing to the kernel:
    \begin{itemize}
    \item \code{kgdboc=<tty-device>,<bauds>}.
    \item For example: \code{kgdboc=ttyS0,115200}
    \end{itemize}
  \item Or at runtime using sysfs:
   \begin{itemize}
   \item \code{echo ttyS0 > /sys/module/kgdboc/parameters/kgdboc}
   \item If the console does not have polling support, this command will yield
         an error.
   \end{itemize}
  \end{itemize}
\end{frame}

\begin{frame}
  \frametitle{Using kgdb (2/2)}
  \begin{itemize}
  \item Then also pass \code{kgdbwait} to the kernel: it makes
    \code{kgdb} wait for a debugger connection.
  \item Boot your kernel, and when the console is initialized,
    interrupt the kernel with a break character and then \code{g}
    in the serial console (see our {\em Magic SysRq} explanations).
  \item On your workstation, start \code{gdb} as follows:
    \begin{itemize}
    \item \code{arm-linux-gdb ./vmlinux}
    \item \code{(gdb) set remotebaud 115200}
    \item \code{(gdb) target remote /dev/ttyS0}
    \end{itemize}
  \item Once connected, you can debug a kernel the way you would debug
    an application program.
  \item On GDB side, the first threads represent the CPU context (ShadowCPU<x>),
    then all the other threads represents a task.
  \end{itemize}
\end{frame}
