
\begin{frame}
  \frametitle{kgdb - A kernel debugger}
  \begin{itemize}
  \item \kconfig{CONFIG_KGDB} in {\em Kernel hacking}.
  \item The execution of the kernel is fully controlled by \code{gdb}
    from another machine, connected through a serial line.
  \item Can do almost everything, including inserting breakpoints in
    interrupt handlers.
  \item Feature supported for the most popular CPU architectures
  \item \kconfig{CONFIG_GDB_SCRIPTS} allows to build GDB python scripts that are
    provided by the kernel.
  \item {\em Note: to avoid confusing gdb with randomized kernel addresses, make
        sure \em{kaslr} is disabled using \code{nokaslr} command line
        parameter.}
  \begin{itemize}
    \item See \kdochtml{dev-tools/gdb-kernel-debugging} for more information
  \end{itemize}
  \end{itemize}
\end{frame}

\begin{frame}
  \frametitle{Using kgdb 1/2}
  \begin{itemize}
  \item Details available in the kernel documentation:
    \kdochtml{dev-tools/kgdb}
  \item Recommended to turn on \kconfig{CONFIG_FRAME_POINTER} to aid in
    producing more reliable stack backtraces in \code{gdb}.
  \item You must include a kgdb I/O driver. One of them is \code{kgdb} over
    serial console (\code{kgdboc}: \code{kgdb} over console, enabled by
    \kconfig{CONFIG_KGDB_SERIAL_CONSOLE})
  \item Configure \code{kgdboc} at boot time by passing to the kernel:
    \begin{itemize}
    \item \code{kgdboc=<tty-device>,<bauds>}.
    \item For example: \code{kgdboc=ttyS0,115200}
    \end{itemize}
  \item Or at runtime using sysfs:
   \begin{itemize}
   \item \code{echo ttyS0 > /sys/module/kgdboc/parameters/kgdboc}
   \end{itemize}
  \end{itemize}
\end{frame}

\begin{frame}
  \frametitle{Using kgdb 2/2}
  \begin{itemize}
  \item Then also pass \code{kgdbwait} to the kernel: it makes
    \code{kgdb} wait for a debugger connection.
  \item Boot your kernel, and when the console is initialized,
    interrupt the kernel with a break character and then \code{g}
    in the serial console (see our {\em Magic SysRq} explanations).
  \item On your workstation, start \code{gdb} as follows:
    \begin{itemize}
    \item \code{arm-linux-gdb ./vmlinux}
    \item \code{(gdb) set remotebaud 115200}
    \item \code{(gdb) target remote /dev/ttyS0}
    \end{itemize}
  \item Once connected, you can debug a kernel the way you would debug
    an application program.
  \item {\em Note: it might be needed to disable the platform watchdog to avoid
        rebooting while debugging.}
  \end{itemize}
\end{frame}