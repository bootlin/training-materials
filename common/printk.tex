\begin{frame}[fragile]{Debugging/tracing using logs 1/4}
  \begin{itemize}
  \item Good old \kfunc{printk}!
    \begin{itemize}
    \item Works in all contexts
    \item Can specify a log level ranging from \code{0} (emergency) to
      \code{7} (debug)
    \item Be careful of the delays introduced when logs are spitted out on a
      serial console at 115200 bauds
      \begin{itemize}
      \item A \code{*_ratelimited()} version exists to limit the amount of
        print if called too often
      \end{itemize}
    \item Not recommended for upstream contributions
    \end{itemize}
  \end{itemize}
  Example:
  \begin{minted}{c}
printk("in probe\n");
  \end{minted}
  Here's what you get in the kernel log:
\begin{verbatim}
[    1.878382] in probe
\end{verbatim}
All other logging facilities are based on it.
\end{frame}

\begin{frame}[fragile]{Debugging/tracing using logs 2/4}
  \begin{itemize}
  \item The \code{pr_*()} family of functions
    \begin{itemize}
    \item They include the log level in the name:\\
      \kfunc{pr_emerg}, \kfunc{pr_alert}, \kfunc{pr_crit},
      \kfunc{pr_err}, \kfunc{pr_warn}, \kfunc{pr_notice}, \kfunc{pr_info},
      \kfunc{pr_cont} and the special \kfunc{pr_debug} (see next pages)
    \end{itemize}
  \item Also defined in \kfile{include/linux/printk.h}
  \end{itemize}
  Example:
  \begin{minted}{c}
pr_info("in probe\n");
  \end{minted}
  Here's what you get in the kernel log:
\begin{verbatim}
[    1.878382] in probe
\end{verbatim}
\end{frame}

\begin{frame}[fragile]{Debugging/tracing using logs 3/4}
  \begin{itemize}
  \item The \code{dev_*()} family of functions
    \begin{itemize}
    \item They include a formatted prefix with the device
      name:\\ \kfunc{dev_emerg}, \kfunc{dev_alert}, \kfunc{dev_crit},
      \kfunc{dev_err}, \kfunc{dev_warn}, \kfunc{dev_notice}, \kfunc{dev_info}
      and the special \kfunc{dev_dbg} (see next pages)
    \item They additionally take a pointer to \kstruct{device} as first
      argument
    \item Defined in \kfile{include/linux/dev_printk.h}
    \item To be used in device drivers
    \end{itemize}
  \end{itemize}
  Example:
  \begin{minted}{c}
dev_info(&pdev->dev, "in probe\n");
  \end{minted}
  Here's what you get in the kernel log:
\begin{verbatim}
[    1.878382] serial 48024000.serial: in probe
[    1.884873] serial 481a8000.serial: in probe
\end{verbatim}
\end{frame}

\begin{frame}[fragile]{Debugging/tracing using logs 4/4}
  \begin{itemize}
  \item The kernel defines many more format specifiers than the standard
    \code{printf()} existing ones.
    \begin{itemize}
    % \path of package url directly supports percent characters, unless it is
    % used inside arguments of other macros which is the case of the \code macro
    \item {\codecolor \path{%p}}: Display the hashed value of pointer by default.
    \item {\codecolor \path{%px}}: Always display the address of a pointer (use
      carefully on non-sensitive addresses).
    \item {\codecolor \path{%pK}}: Display hashed pointer value, zeros or the
      pointer address depending on \code{kptr_restrict} sysctl value.
    \item {\codecolor \path{%pOF}}: Device-tree node format specifier.
    \item {\codecolor \path{%pr}}: Resource structure format specifier.
    \item {\codecolor \path{%pa}}: Physical address display (work on all architectures 32/64
      bits)
    \item {\codecolor \path{%pe}}: Error pointer (displays the string
      corresponding to the error number)
  \end{itemize}
  \item See \kdochtml{core-api/printk-formats} for an exhaustive list of format specifiers
  \item Also features a helper to dump entire buffers with a \code{hexdump} like display: \kfunc{print_hex_dump}
  \end{itemize}
\end{frame}

\begin{frame}
  \frametitle{pr\_debug() and dev\_dbg()}
  \begin{itemize}
  \item When the driver is compiled with \code{DEBUG} defined, all
    these messages are compiled and printed at the debug level.
    \code{DEBUG} can be defined by \codewithhash{\#define DEBUG} at the
    beginning of the driver, or using
    \code{ccflags-$(CONFIG_DRIVER) += -DDEBUG} in the \code{Makefile}
  \item When the kernel is compiled with \kconfig{CONFIG_DYNAMIC_DEBUG},
    then these messages can dynamically be enabled on a per-file,
    per-module or per-message basis, by writing commands to
    \code{/proc/dynamic_debug/control}. Note that messages are not enabled
    by default.
    \begin{itemize}
    \item Details in \kdochtml{admin-guide/dynamic-debug-howto}
    \item Very powerful feature to only get the debug messages you're
      interested in.
    \end{itemize}
  \item When neither \code{DEBUG} nor \kconfig{CONFIG_DYNAMIC_DEBUG} are
    used, these messages are not compiled in.
  \end{itemize}
\end{frame}

\IfBeginWith{\training}{debugging}{
\begin{frame}
  \frametitle{pr\_debug() and dev\_dbg() usage}
  \begin{itemize}
  \item Debug prints can be enabled using the
    \code{/proc/dynamic_debug/control} file.
  \begin{itemize}
    \item \code{cat /proc/dynamic_debug/control} will display all
      lines that can be enabled in the kernel
    \item Example: \code{init/main.c:1427 [main]run_init_process =p "    \%s\012"}
  \end{itemize}
  \item A syntax allows to enable individual print using lines, files or modules
  \begin{itemize}
    \item \code{echo "file drivers/pinctrl/core.c +p" > /proc/dynamic_debug/control}
      will enable all debug prints in \code{drivers/pinctrl/core.c}
    \item \code{echo "module pciehp +p" > /proc/dynamic_debug/control}
      will enable the debug print located in the \code{pciehp} module
    \item \code{echo "file init/main.c line 1427 +p" > /proc/dynamic_debug/control}
      will enable the debug print located at line 1247 of file \code{init/main.c}
    \item Replace \code{+p} with \code{-p} to disable the debug print
  \end{itemize}
  \end{itemize}
\end{frame}
}{}

\IfBeginWith{\training}{linux-kernel}{
\begin{frame}
  \frametitle{Configuring the priority}
  \begin{itemize}
  \item Each message is associated to a priority, as specified in
    \kfile{include/linux/kern_levels.h}.
  \item All the messages, regardless of their priority, are stored in
    the kernel log ring buffer
    \begin{itemize}
    \item Typically accessed using the \code{dmesg} command
    \end{itemize}
  \item Messages with a priority lower than the \code{loglevel} also
    appear on the console
  \item The \code{loglevel} can be changed:
    \begin{itemize}
    \item in the kernel configuration using \kconfig{CONFIG_CONSOLE_LOGLEVEL_DEFAULT})
    \item on the cmdline with \code{loglevel=} (see \kdochtml{admin-guide/kernel-parameters})
    \item at runtime through \code{/proc/sys/kernel/printk} (\kdochtml{admin-guide/sysctl/kernel})
    \end{itemize}
  \item Examples:
    \begin{itemize}
    \item \code{loglevel=0}: no message on the console (see also: \code{quiet})
    \item \code{loglevel=8}: all messages on the console (see also: \code{ignore_loglevel})
    \end{itemize}
  \end{itemize}
\end{frame}
}{}

\IfBeginWith{\training}{debugging}{
\begin{frame}
  \frametitle{Debug logs troubleshooting}
  \begin{itemize}
    \item When using dynamic debug, make sure that your debug call is enabled: it must be visible in
    \code{control} file in debugfs \textbf{and} be activated (\code{=p})
    \item Is your log output only in the kernel log buffer?
    \begin{itemize}
      \item You can see it thanks to \code{dmesg}
      \item You can lower the \code{loglevel} to output it to  the console directly
      \item You can also set \code{ignore_loglevel} in the kernel command line to
      force all kernel logs to console
    \end{itemize}
    \item If you are working on an out-of-tree module, you may prefer to define \code{DEBUG} in
    your module source or Makefile instead of using dynamic debug
    \item If configuration is done through the kernel command line, is it
    properly interpreted?
    \begin{itemize}
      \item Starting from 5.14, kernel will let you know about faulty
      command line:\\
      \code{Unknown kernel command line parameters loglevel, will be passed to user
      space.}
      \item You may need to take care of special characters escaping (e.g: quotes)
    \end{itemize}
    \item Be aware that a few subsystems bring their own logging infrastructure, with specific
    configuration/controls, eg: \code{drm.debug=0x1ff}
  \end{itemize}
\end{frame}
}{}
