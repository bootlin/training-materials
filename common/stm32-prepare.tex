\section{Prepare the STM32MP1 Discovery Kit 1}

The STM32MP1 Discovery Kit 1 is powered via a USB-C cable, which you
need to connect to the \code{CN6} (also labeled \code{PWR_IN})
connector.

In addition, to access the debug serial console, you need to use a
micro-USB cable connected to the \code{CN11} (also labeled
\code{ST-LINK}) connector.

Once your micro-USB cable is connected, a \code{/dev/ttyACM0} device
will apear on your PC. You can see this device appear by looking at
the output of \code{dmesg} on your workstation.

To communicate with the board through the serial port, install a
serial communication program, such as \code{picocom}:

\begin{bashinput}
sudo apt install picocom
\end{bashinput}

If you run \code{ls -l /dev/ttyACM0}, you can also see that only
\code{root} and users belonging to the \code{dialout} group have read
and write access to this file. Therefore, you need to add your user to
the \code{dialout} group:

\begin{bashinput}
sudo adduser $USER dialout
\end{bashinput}

{\bf Important}: for the group change to be effective, in Ubuntu 18.04, you have to
{\em completely reboot} the system \footnote{As explained on
\url{https://askubuntu.com/questions/1045993/after-adding-a-group-logoutlogin-is-not-enough-in-18-04/}.}.
A workaround is to run \code{newgrp dialout}, but it is not global.
You have to run it in each terminal.

Now, you can run \code{picocom -b 115200 /dev/ttyACM0}, to start
serial communication on \code{/dev/ttyACM0}, with a baudrate of
\code{115200}. If you wish to exit \code{picocom}, press
\code{[Ctrl][a]} followed by \code{[Ctrl][x]}.

There should be nothing on the serial line so far, as the board is not
powered up yet.
